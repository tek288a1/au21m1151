% \documentclass[10pt,t,presentation,ignorenonframetext,aspectratio=169]{beamer}
\documentclass[10pt,t,handout,ignorenonframetext,aspectratio=169]{beamer}
\usepackage[default]{lato}
\usepackage{tk_beamer1}
\input{tk_packages}
\input{tk_macros}
\input{tk_environ}
\input{tk_ximera}
\usepackage{wasysym}            % for smiley
\newcommand{\zoz}{$\mathbf{ \frac{0}{0} }$}

% some inverse trigs
\DeclareMathOperator{\arcsec}{arcsec}
\DeclareMathOperator{\arccot}{arccot}
\DeclareMathOperator{\arccsc}{arccsc}

%%%% META DATA
\newcommand{\semester}{Autumn 2021}
\newcommand{\course}{Math 1151}
\newcommand{\lecTitle}{Lecture 40: Working with Substitution (WWS)}

%%%% TITLE PAGE
\title[\course]{\lecTitle}
\institute[Ohio State]
{
  \medskip
}
\date[\week]{\semester}
\author{Tae Eun Kim, Ph.D.}

\begin{document}
\begin{frame}
  \titlepage
\end{frame}



\begin{frame}
  \frametitle{Substitution Procedures}
  Let's recall that in integrating a function which we suspect to be the
  derivative of another obtained by the chain rule:
  \begin{enumerate}
  \item Look for a candidate for the inner function; call it $u$.
  \item Rewrite the given function completely in terms of $u$ leaving
    no trace of the original variable.
  \item Integrate this new function of $u$. (If necessary, you may
    need to go back to Step 1 and modify your choice of $u$.)
  \item In dealing with an indefinite integral, make sure to replace
    $u$ by the equivalent expression of the original variable.
  \item Working with a definite integral, you may evaluate the result
    of Step 3 at the transformed bounds of $u$ or evaluate the
    antiderivate obtained in Step 4 at the original bounds.
  \end{enumerate}
  For the remainder of the lecture, we will work out practice examples.
\end{frame}

\begin{frame}
  \vs
  \begin{example}
    Compute:
    \begin{enumerate}
    \item $\ds \int_{2}^{3} \frac{1}{x\ln(x)} \d x$
    \item $\ds \int_0^{16} \sqrt{4 - \sqrt{x}} \d x$
    \end{enumerate}
  \end{example}
\end{frame}

\begin{frame}
  \vs
  \begin{example}
    Compute:
    \begin{enumerate}
    \item $\ds \int \frac{\sec(y) \tan(y) + \sec^2(y)}{\sec(y) + \tan(y)} \d y$
    \item $\ds \int \tan(x) \d x$
    \end{enumerate}
  \end{example}
\end{frame}

\begin{frame}
  \vs
  \begin{example}
    Compute:
    \begin{enumerate}
    \item $\ds \int \frac{u}{1-u^2} \d u$
    \item $\ds \int \frac{e^{2x}}{1 - e^{2x}} \d x$
    \end{enumerate}
  \end{example}
\end{frame}

\begin{frame}
  \vs
  \begin{example}
    Compute:
    \begin{enumerate}
    \item $\ds \int x^3\sqrt{1-x^2}\d x$
    \end{enumerate}
  \end{example}
\end{frame}


\begin{frame}
  \frametitle{Work}
  Suppose the force $F(s)$ is applied to an object and moves it from
  $s = s_0$ to $s = s_1$. Then the work done on the object over the
  course of motion is given by
  \[
    W = \int_{s_0}^{s_1} F(s) \d s .
  \]

  The international standard unit of force is a \textbf{Joule}, which
  is defined to be
  \[
    1\unit{J} = 1\unit{N}\cdot\unit{m}.
  \]

  In words, work measures the accumulated force over a distance. Note
  that we only accumulate force in the \textit{direction} or
  \textit{opposite direction} of motion.
\end{frame}


\begin{frame}
  \vs
  \begin{example}
    If an apple has a mass of $0.1\unit{kg}$, how much work is required
    to lift this 1 meter above the ground? Assume that the gravitational
    acceleration is $-9.8\unit{m}/\unit{s}^2$.
  \end{example}
\end{frame}


\begin{frame}
  \frametitle{Kinetic Energy}
  Now suppose that an object of mass $m$ is moving at velocity $v(t)$.
  The \textbf{kinetic energy} $E_k$ is the amount of \textit{energy}
  that an object possesses from its motion. It is defined by
  \[
    E_k = \frac{m  v^2}{2}.
  \]

  The SI unit of energy is also a \textbf{Joule} since
  $1\unit{N} = 1\unit{kg}\cdot\unit{m}/\unit{s}^2$:
  \[
    1\unit{J} = 1\unit{kg}\cdot \unit{m}^2/\unit{s}^2 .
  \]
\end{frame}


\begin{frame}
  \vs
  \begin{example}
    Now the apple is dropped from the height of 1 meter. How much
    kinetic energy is released when it hits the ground?
  \end{example}
\end{frame}

\begin{frame}
  \frametitle{Work-Energy Theorem}
  We observe that the work and the energy calculated above are the
  same with the same unit. This is not a coincidence and this
  phenomenon can be explained via the \textbf{Work-Energy Theorem}.

  \begin{thm}[Work-Energy Theorem]
    Suppose that an object of mass $m$ is moving along a straight
    line. If $s_0$ and $s_1$ are the the starting and ending positions,
    $v_0$ and $v_1$ are the the starting and ending velocities, and
    $F(s)$ is the force acting on the object at any given position,
    then
    \[
      W = \int_{s_0}^{s_1} F(s) \d s
      = \frac{m v_1^2}{2} - \frac{m v_0^2}{2}.
    \]
    In other words, the total work done by the force is equal to the net
    change in kinetic energy.
  \end{thm}
\end{frame}

\begin{frame}
  \frametitle{Explanation}
  \begingroup
  \footnotesize
  Let $t_{0,1}$ be the initial and terminal time of the motion
  respectively, so that we can write $s_{0,1} = s(t_{0,1})$
  respectively. Then the work formula can be written as
  \[
    W = \int_{s(t_0)}^{s(t_1)} F(s) \d s = \int_{t_0}^{t_1} F(s(t)) s'(t) \d t ,
  \]
  where the second equality is due to the substitution rule with
  $u = s(t)$. By Newton's second law of motion
  $F(s(t)) = m a(t)$, we can write
  \[
    \int_{t_0}^{t_1} F(s(t)) s'(t) \d t = \int_{t_0}^{t_1} m a(t) s'(t) \d t
    = m \int_{t_0}^{t_1} a(t) s'(t) \d t .
  \]
  Remembering that $a(t) = v'(t)$ and $s'(t) = v(t)$, we can write the
  integrand solely in terms of $v$ and its derivative, i.e.,
  \[
    m \int_{t_0}^{t_1} a(t) s'(t) \d t = m \int_{t_0}^{t_1} v(t) v'(t) \d t .
  \]
  Another substitution $u = v(t)$ with new bounds $v_0 = v(t_0)$ and
  $v_1 = v(t_1)$ yields the desired result.
  \[
    m \int_{t_0}^{t_1} v(t) v'(t) \d t = m \int_{v_0}^{v_1} u \d u
    = m \eval{ \frac{u^2}{2} }_{v_0}^{v_1}
    = \frac{m v_1^2}{2} - \frac{m v_0^2}{2}.
  \]
  \endgroup
\end{frame}

\end{document}
%%% Local Variables:
%%% mode: latex
%%% TeX-master: t
%%% End:
