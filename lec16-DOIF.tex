\documentclass[10pt,t,presentation,ignorenonframetext,aspectratio=169]{beamer}
% \documentclass[10pt,t,handout,ignorenonframetext,aspectratio=169]{beamer}
\usepackage[default]{lato}
\usepackage{tk_beamer1}
\input{tk_packages}
\input{tk_macros}
\input{tk_environ}
\input{tk_ximera}
\usepackage{wasysym}            % for smiley
\newcommand{\zoz}{$\mathbf{ \frac{0}{0} }$}

% some inverse trigs
\DeclareMathOperator{\arcsec}{arcsec}
\DeclareMathOperator{\arccot}{arccot}
\DeclareMathOperator{\arccsc}{arccsc}

%%%% META DATA
\newcommand{\semester}{Autumn 2021}
\newcommand{\course}{Math 1151}
\newcommand{\lecTitle}{Lecture 16: Derivatives of Inverse Functions (DOIF)}

%%%% TITLE PAGE
\title[\course]{\lecTitle}
\institute[Ohio State]
{
  \medskip
}
\date[\week]{\semester}
\author{Tae Eun Kim, Ph.D.}

\begin{document}
\begin{frame}
  \titlepage
\end{frame}

\begin{frame}
  \frametitle{The Derivatives of Inverse Trig Functions}
  \begin{thm}[Derivatives of inverse trigonometric functions]
    \hfill
    \begin{itemize}
    \item $\ds \frac{d}{dx} \arcsin (x) = \frac{d}{dx} \sin^{-1}(x) = \frac{1}{\sqrt{1-x^2}}$ for $|x| < 1$
    \item $\ds \frac{d}{dx} \arccos (x) = \frac{d}{dx} \cos^{-1}(x) = \frac{-1}{\sqrt{1-x^2}}$ for $|x| < 1$
    \item $\ds \frac{d}{dx} \arctan (x) = \frac{d}{dx} \tan^{-1}(x) = \frac{1}{1+x^2}$
    \item $\ds \frac{d}{dx} \arcsec (x) = \frac{d}{dx} \sec^{-1}(x) = \frac{1}{|x| \sqrt{x^2-1}}$ for $|x|>1$
    \item $\ds \frac{d}{dx} \arccsc (x) = \frac{d}{dx} \csc^{-1}(x) = \frac{-1}{|x|\sqrt{x^2-1}}$ for $|x|>1$
    \item $\ds \frac{d}{dx} \arccot (x) = \frac{d}{dx} \cot^{-1}(x) = \frac{-1}{1+x^2}$
    \end{itemize}
  \end{thm}
\end{frame}

\begin{frame}
  \vs
  \question{} Compute:
  \begin{enumerate}
  \item $\ds \ddx \tan^{-1} (\sqrt{x})$
    \vfill
  \item $\ds \ddx \sec^{-1} (3x)$
    \vfill
  \end{enumerate}
\end{frame}

\begin{frame}
  \frametitle{Explanation}
\end{frame}

\begin{frame}
  \frametitle{Remark}
  In the derivation of the above formulas, we repeatedly used the following form of implicit differentiation
  \[
    \ddx f(y) = f'(y) \cdot y' \,,
  \]
  which requires that the function \(y = f^{-1}(x)\) has a derivative. The differentiability of the inverse function is guaranteed by the following theorem.
\end{frame}

\begin{frame}
  \frametitle{Inverse Function Theorem}
  \begin{thm}[The inverse function theorem]
    Suppose $f$ is a differentiable function that is one-to-one near $a$ and $f'(a) \neq 0$ and let $b = f(a)$. Then
    \begin{enumerate}
    \item $f^{-1}(x)$ is \textbf{defined} for $x$ near $b$,
    \item $f^{-1}(x)$ is \textbf{differentiable} near $b$,
    \item last, but not least:
      \[
        \eval{\ddx f^{-1}(x)}_{x=b}
        = \frac{1}{f'(a)}\qquad\text{where}\qquad b = f(a) \,.
      \]
    \end{enumerate}
  \end{thm}
\end{frame}

\begin{frame}
  \frametitle{Illustration}
  Besides verifying the last result using implicit differentiation, convince yourselves by considering the following diagram of a function \(f\) and its inverse \(f^{-1}\):

  \begin{minipage}{0.8\linewidth}
    \begin{image}[0.7\linewidth]
      \begin{tikzpicture}
        \begin{axis}[
          xmin=-.5,xmax=5,ymin=-1,ymax=5,
          axis lines=center,
          xlabel=$x$, ylabel=$y$,
          every axis y label/.style={at=(current axis.above origin),anchor=south},
          every axis x label/.style={at=(current axis.right of origin),anchor=west},
          ]
          \addplot [very thick, penColor, smooth, domain=(-.5:6)] {e^x};
          \addplot [very thick, penColor2, samples=100, smooth, domain=(.002:6)] {ln(x)};
          \addplot [dashed, textColor, domain=(-.5:6)] {x};
          \node at (axis cs:1,4) [penColor] {$f$};
          \node at (axis cs:4,1) [penColor2] {$f^{-1}$};

          \addplot [draw=black,->] plot coordinates {(1,e) (1.5,e)};
          \addplot [draw=black,->] plot coordinates {(1.5,e) (1.5,1.5*e)};
          \addplot[color=penColor3,fill=penColor3,only marks,mark=*] coordinates{(1,e)};  %% closed hole
          \node at (axis cs:1.4,e) [below] {run($f$)};
          \node at (axis cs:1.5,3.4) [right] {rise($f$)};

          \addplot [draw=black,->] plot coordinates {(e,1) (e,1.5)};
          \addplot [draw=black,->] plot coordinates {(e,1.5) (1.5*e,1.5)};
          \addplot[color=penColor3,fill=penColor3,only marks,mark=*] coordinates{(e,1)};  %% closed hole
          \node at (axis cs:e,1.25) [left] {rise($f^{-1}$)};
          \node at (axis cs:3.4,1.5) [above] {run($f^{-1}$)};
        \end{axis}
      \end{tikzpicture}
    \end{image}
  \end{minipage}
  \hfill

\end{frame}

% Since the two graphs are symmetric about the 45-degree line \(y=x\), we see that the \emph{rises} and \emph{runs} are switched upon reflection; in particular,
% \[
%   \frac{ {\rm rise} (f^{-1}) }{ {\rm run} (f^{-1}) }
%   = \frac{ {\rm run} (f) }{ {\rm rise} (f) }
%   = \dfrac{ 1 }{ \dfrac{ {\rm rise} (f) }{ {\rm run} (f) } } \,.
% \]
% In the limit as \({\rm run}(f) \to 0\), we obtain the expression for the derivative of \(f^{-1}\).

\begin{frame}
  \vs
  \begin{question}
    Let $f$ be a differentiable function that has an inverse. In the
    table below we give several values for both $f$ and $f'$:
    \[
      \begin{array}{|c|c|c|}\hline
        x & f & f' \\ \hline \hline
        2 & 0 & 2  \\ \hline
        3 & 1 & 5  \\ \hline
        4 & 3 & 0  \\ \hline
      \end{array}
    \]
    Compute
    \[
      \ddx f^{-1}(x)\;\text{at $x=1$.}
    \]
  \end{question}
\end{frame}

% \begin{example}
%   Let $f$ be a differentiable function that has an inverse. In the
%   table below we give several values for both $f$ and $f'$:
%   \[
%     \begin{array}{|c|c|c|}\hline
%       x & f & f' \\ \hline \hline
%       2 & 0 & 2  \\ \hline
%       3 & 1 & 5   \\ \hline
%       4 & 3 & 0  \\ \hline
%     \end{array}
%   \]
%   Compute
%   \[
%     \left(f^{-1}\right)'(0) \,.
%   \]
% \end{example}

% \begin{example}
%   Let $f$ be a differentiable function that has an inverse. In the
%   table below we give several values for both $f$ and $f'$:
%   \[
%     \begin{array}{|c|c|c|}\hline
%       x & f & f' \\ \hline \hline
%       2 & 0 & 2  \\ \hline
%       3 & 1 & 4  \\ \hline
%       4 & 3 & 0  \\ \hline
%     \end{array}
%   \]
%   Compute
%   \[
%     \eval{\ddx f^{-1}(x)}_{x=3} \,.
%   \]
% \end{example}

\end{document}
%%% Local Variables:
%%% mode: latex
%%% TeX-master: t
%%% End:
