\documentclass[10pt,t,presentation,ignorenonframetext,aspectratio=169]{beamer}
% \documentclass[10pt,t,handout,ignorenonframetext,aspectratio=169]{beamer}
\usepackage[default]{lato}
\usepackage{tk_beamer1}
\input{tk_packages}
\input{tk_macros}
\input{tk_environ}
\input{tk_ximera}
\usepackage{wasysym}            % for smiley
\newcommand{\zoz}{$\mathbf{ \frac{0}{0} }$}

%%%% META DATA
\newcommand{\semester}{Autumn 2021}
\newcommand{\course}{Math 1151}
\newcommand{\lecTitle}{Lecture 13: Higher Order Derivatives and Graphs (HODAG)}

%%%% TITLE PAGE
\title[\course]{\lecTitle}
\institute[Ohio State]
{
  \medskip
}
\date[\week]{\semester}
\author{Tae Eun Kim, Ph.D.}

\begin{document}
\begin{frame}
  \titlepage
\end{frame}

\begin{frame}
  \frametitle{Higher-Order Derivatives}

  \begin{itemize}
  \item The derivative of a function is often called the \textbf{first derivative};
  \item The derivative of the derivative the \textbf{second derivative};
  \item The derivative of the second derivative the \textbf{third derivative}, and
    so on.
  \item Derivatives of derivatives are called \textbf{higher-order derivatives} with the following notations:
  \end{itemize}

  \[
    \begin{array}{rl}
      \text{First derivative:} & \dfrac{d}{dx} f(x) = f'(x) = f^{(1)}(x). \vspace{0.3em} \\
      \text{Second derivative:} & \dfrac{d^2}{dx^2} f(x) = f''(x) = f^{(2)}(x).\vspace{0.3em}  \\
      \text{Third derivative:} & \dfrac{d^3}{dx^3} f(x) = f'''(x) = f^{(3)}(x).
    \end{array}
  \]
\end{frame}

\begin{frame}
  \vs{}
  \question{} Compute the first, second, and third derivatives of:
  \begin{enumerate}
  \item $\ds f(x) = x^2 + 3x - 8$ \vfill
  \item $\ds g(x) = e^{2x}$ \vfill
  \item $\ds h(x) = \sin(x^2)$ \vfill
  \end{enumerate}
\end{frame}

\begin{frame}
  \frametitle{First Derivatives and Monotonicity}
  \begin{defn}[Monotonicity]
    \begin{itemize}
    \item We say that a function $f$ is \textbf{increasing} on an
      interval $I$ if $f(x_{1})<f(x_{2})$ for all pairs of numbers
      $x_{1}$, $x_{2}$ in $I$ such that $x_{1}<x_{2}$.
    \item We say that a function $f$ is \textbf{decreasing} on an
      interval $I$ if $f(x_{1})>f(x_{2})$ for all pairs of numbers
      $x_{1}$, $x_{2}$ in $I$ such that $x_{1}<x_{2}$.
    \end{itemize}
  \end{defn}

  \begin{itemize}
  \item We say that a function is \textbf{monotonic} on an interval if
    it is either increasing or decreasing there.
  \item The notion of monotonicity is closely related to the
    derivatives as they convey the slope information of tangent lines
    to the curve.
  \end{itemize}
\end{frame}

\begin{frame}
  \vs{}
  \begin{thm}
    A function $f$ is \textbf{increasing} on any interval $I$ where
    $f'(x)>0$, for all $x$ in $I$. A function $f$ is
    \textbf{decreasing} on any interval $I$ where $f'(x)<0$, for all
    $x$ in $I$.
  \end{thm}
  \vfill

  \textbf{Illustration.}
  Here we have graphs of $f(x) = \sin(x)$, $f'(x) = \cos(x)$, and
  $f''(x) = -\sin(x)$:
  \begin{image}[0.8\textwidth]
    \begin{tikzpicture}
      \begin{axis}[
        xmin=-6.75,xmax=6.75,ymin=-1.5,ymax=1.5,
        axis lines=center,
        ticks=none,
        width=6in,
        height=3in,
        every axis y label/.style={at=(current axis.above origin),anchor=south},
        every axis x label/.style={at=(current axis.right of origin),anchor=west},
        ]
        \addplot [very thick, penColor, samples=100,smooth,
        domain=(-6.75:6.75)] {sin(deg(x))};
        \addlegendentry{$f$};
        \addplot [very thick, dashed,penColor2, samples=100,smooth, domain=(-6.75:6.75)] {cos(deg(x))};
        \addlegendentry{$f'$};
        \addplot [very thick, dotted,penColor3, samples=100,smooth, domain=(-6.75:6.75)] {-sin(deg(x))};
        \addlegendentry{$f''$};
      \end{axis}
    \end{tikzpicture}
  \end{image}
\end{frame}

\begin{frame}
  \vs{}

  \question{} Below we have unlabeled graphs of $f$, $f'$, and
  $f''$. Identify each curve above as a graph of $f$, $f'$, or $f''$.

  \begin{image}[0.8\textwidth]
    \begin{tikzpicture}
      \begin{axis}[
        domain=-4:4,
        ticks=none,
        ymax=2, ymin=-2,
        xmax=4, xmin=-4,
        axis lines =middle,
        every axis y label/.style={at=(current axis.above origin),anchor=south},
        every axis x label/.style={at=(current axis.right of origin),anchor=west},
        width=6in,
        height=3in,
        ]
        \addplot [very thick, penColor,smooth,samples=100] {2/(.75*sqrt(2*pi))*exp(-((x)^2)/(2*.75^2)) *(-x)/(.75^2)};
        \addlegendentry{$A$};
        \addplot [very thick, dashed,penColor,smooth,samples=100] {2*gauss(0,.75)};
        \addlegendentry{$B$};
        \addplot [very thick, dotted,penColor,smooth,samples=100] {2/(.75*sqrt(2*pi))*exp(-((x)^2)/(2*.75^2)) *(-1)/(.75^2) +
          2/(.75*sqrt(2*pi))*exp(-((x)^2)/(2*.75^2)) *(x^2)/(.75^4)};
        \addlegendentry{$C$};
      \end{axis}
    \end{tikzpicture}
  \end{image}
\end{frame}

\begin{frame}
  \frametitle{Second Derivatives and Concavity}
  \begin{defn}[Concavity]
    Let $f$ be a function differentiable on an open interval $I$.
    \begin{itemize}
    \item We say that the graph of $f$ is \textbf{concave up} on $I$
      if $f'$ is \textbf{increasing} on $I$.
    \item We say that the graph of $f$ is \textbf{concave down} on $I$
      if $f'$ is \textbf{decreasing} on $I$.
    \end{itemize}
  \end{defn}
  \vfill
  \textbf{Illustration.}
  \begin{image}[0.7\textwidth]
    \begin{tikzpicture}
      \draw [penColor,  ultra thick,domain=180:270] plot ({2*cos(\x)+4},{9-2*sin(\x)});
      \draw [penColor, ultra thick,domain=270:360] plot ({2*cos(\x)+8}, {2*sin(\x)+11});
      \draw [color=red,thick, domain=1.92:2.2] plot({\x},{cot(193)*(\x-(2*cos(193)+4))+9-2*sin(193)});
      \draw [color=red, thick, domain=2.3:3.1] plot({\x},{cot(225)*(\x-(2*cos(225)+4))+9-2*sin(225)});
      \draw [color=red, thick, domain=3.19:4.2] plot({\x},{cot(260)*(\x-(2*cos(260)+4))+9-2*sin(260)});
      \draw [color=red, thick, domain=7.8:9.01] plot({\x},{-cot(280)*(\x-(2*cos(280)+8))+11+2*sin(280)});
      \draw [color=red, thick, domain=9.05:9.79] plot({\x},{-cot(315)*(\x-(2*cos(315)+8))+11+2*sin(315)});
      \draw [color=red, thick, domain=9.82:10.08] plot({\x},{-cot(347)*(\x-(2*cos(347)+8))+11+2*sin(347)});
      \node at (3,7.5) [text width=5cm] {
        The function $f$ is increasing, while the rate itself is decreasing.
        In this case the curve  $y=f(x)$ is \dfn{concave down}.};
      \node at (9,7.5) [text width=5cm] {
        The function $g$ is increasing, while the rate itself is increasing.
        In this case the curve  $y=g(x)$ is \dfn{concave up}.};
    \end{tikzpicture}
  \end{image}
\end{frame}

\begin{frame}
  \vs{}
  We know from the previous section that
  \begin{center}
    $f'$ is increasing/decreasing when its derivative $f''$ is
    positive/negative.
  \end{center}
  In other words, the second derivatives contain concavity information
  as summarized in the following theorem.

  \begin{thm}[Test for concavity]
    Let $I$ be an open interval.
    \begin{enumerate}
    \item If $f''(x)>0$ for all $x$ in $I$, then the graph of $f$ is  concave up on $I$.
    \item If $f''(x)<0$ for all $x$ in $I$, then the graph of $f$ is  concave down on $I$.
    \end{enumerate}
  \end{thm}
\end{frame}


\begin{frame}
  \frametitle{Summary: Derivatives and Graphs}
  % \vs{}
  % By considering signs of first and second derivatives of a function,
  % we can describe a general shape/behavior of its graph quite well as
  % summarized below.

  \begin{image}[0.7\textwidth]
    \begin{tikzpicture}
      \draw (0,0) -- (0,12);
      \draw (0,0) -- (12,0);
      \draw (6,0) -- (6,12);
      \draw (0,6) -- (12,6);
      \draw (12,0) -- (12,12);
      \draw (0,12) -- (12,12);

      \node at (-1.3,9) {\Large$f''(x)>0$};
      \node at (-1.3,3) {\Large$f''(x)<0$};
      \node at (3,12.4) {\Large$f'(x)<0$};
      \node at (9,12.4) {\Large$f'(x)>0$};

      \draw [penColor,ultra thick,domain=180:270] plot ({2*cos(\x)+4}, {2*sin(\x)+11});
      \draw [penColor,ultra thick,domain=270:360] plot ({2*cos(\x)+8}, {2*sin(\x)+11});
      \draw [penColor,ultra thick,domain=0:90] plot ({2*cos(\x)+2}, {2*sin(\x)+3});
      \draw [penColor,ultra thick,domain=180:90] plot ({2*cos(\x)+10}, {2*sin(\x)+3});

      \node at (3,7.5) [text width=5cm] {\large
        The function $f$ is decreasing, while the rate itself is increasing.
        In this case the curve $y=f(x)$ is \dfn{concave up}.};

      \node at (9,7.5) [text width=5cm] {\large
        The function $f$ is increasing, while the rate itself is increasing.
        In this case the curve $y=f(x)$ is \dfn{concave up}.};

      \node at (3,1.5) [text width=5cm] {\large
        The function $f$  is decreasing, while the rate itself is decreasing.
        In this case the curve  $y=f(x)$ is \dfn{concave down}.};

      \node at (9,1.5) [text width=5cm] {\large
        The function $f$ is increasing, while the rate itself is decreasing.
        In this case the curve $y=f(x)$ is \dfn{concave down}.};
    \end{tikzpicture}
  \end{image}
\end{frame}

\begin{frame}
  \vs{}

  \question{ } Find the intervals on which $f$ is increasing/decreasing
  and concave up/down and plot its graph.
  \[
    f(x) = x^3 - x^2 - 4x + 4 \,.
  \]

  \begin{image}[0.8\textwidth]
    \begin{tikzpicture}
      \begin{axis}[
        domain=-4:4,
        % ticks=none,
        ymax=10, ymin=-10,
        xmax=2.5, xmin=-2.5,
        axis lines =middle,
        every axis y label/.style={at=(current axis.above origin),anchor=south},
        every axis x label/.style={at=(current axis.right of origin),anchor=west},
        width=6in,
        height=4in,
        grid=major,
        grid style={dashed, gridColor},
        ]
        % \addplot [very thick, penColor,smooth,samples=100] {x^3 - x^2 -
        % 4*x + 4};
        % \addlegendentry{$f$};
        % \addplot [very thick, dashed,penColor2,smooth,samples=100] {3*x^2
        % - 2*x - 4};
        % \addlegendentry{$f'$};
        % \addplot [very thick, dotted,penColor3,smooth,samples=100] {6*x-2};
        % \addlegendentry{$f''$};
      \end{axis}
    \end{tikzpicture}
  \end{image}
  \note{
    \begin{image}[0.8\textwidth]
      \begin{tikzpicture}
        \begin{axis}[
          domain=-4:4,
          % ticks=none,
          ymax=10, ymin=-10,
          xmax=2.5, xmin=-2.5,
          axis lines =middle,
          every axis y label/.style={at=(current axis.above origin),anchor=south},
          every axis x label/.style={at=(current axis.right of origin),anchor=west},
          width=6in,
          height=3in,
          ]
          \addplot [very thick, penColor,smooth,samples=100] {x^3 - x^2 -
            4*x + 4};
          \addlegendentry{$f$};
          \addplot [very thick, dashed,penColor2,smooth,samples=100] {3*x^2
            - 2*x - 4};
          \addlegendentry{$f'$};
          \addplot [very thick, dotted,penColor3,smooth,samples=100] {6*x-2};
          \addlegendentry{$f''$};
        \end{axis}
      \end{tikzpicture}
    \end{image}
  }
\end{frame}

\begin{frame}
  \frametitle{Example from Physics}
  \begin{block}{Motion with constant acceleration}
    We know from physics that the motion of an object with constant
    acceleration $a$ is described by the following formulas:
    \begin{alignat*}{2}
      \text{position:}&\qquad& x(t) & = x_0 + v_0t + \frac{1}{2}at^2 \,, \\
      \text{velocity:}&\qquad& v(t) & = v_0 + at \,,
    \end{alignat*}
    where $x_0$ and $v_0$ are the initial position and velocity
    respectively.
  \end{block}
  In general,
  \begin{itemize}
  \item the derivative of the position function $x(t)$ is the velocity
    function $v(t)$, i.e., $v(t) = x'(t)$;
  \item the derivative of the velocity function $v(t)$ is the
    acceleration function $a(t)$, i.e. $a(t) = v'(t) = x''(t)$.
  \end{itemize}
\end{frame}

\begin{frame}
  \vs{}
  \question{}
  The position of a moving particle is given by
  \[
    s(t) = 36t^2 - 7t^3 \,.
  \]
  Find a formula for its acceleration.
\end{frame}


\end{document}
%%% Local Variables:
%%% mode: latex
%%% TeX-master: t
%%% End:
