%% Defining theorem environment
\theoremstyle{plain}
\newtheorem{prop}{Proposition}
\newtheorem{cor}[prop]{Corollary}
\newtheorem{lem}[prop]{Lemma}
\newtheorem{thm}[prop]{Theorem}
\newtheorem{cons}[prop]{Consequence}
\newtheorem{conv}[prop]{Convention}
\newtheorem{prob}[prop]{Problem}
\newtheorem{form}[prop]{Formulation}
\newtheorem{claim}[prop]{Claim}

\theoremstyle{definition}
\newtheorem{defn}{Definition}
\newtheorem{notn}[defn]{Notation}
\newtheorem{note}[defn]{Note}
\newtheorem{rmk}[defn]{Remark}
\newtheorem{exer}[defn]{Exercise}
\newtheorem{ex}[defn]{Example}


%% Listings Environments
\usepackage{courier}
\usepackage{listings}

% \definecolor{mygreen}{RGB}{28,172,0} % color values Red, Green, Blue
\definecolor{mygreen}{RGB}{0,100,0} % color values Red, Green, Blue
\definecolor{mylilas}{RGB}{170,55,241}

\lstdefinestyle{matlab}{language=Matlab,
% basicstyle=\small\ttfamily, % Use small true type font
% breaklines=true,%
% morekeywords={matlab2tikz},
% keywordstyle=\color{blue},%
% morekeywords=[2]{1}, keywordstyle=[2]{\color{black}},
% identifierstyle=\color{black},%
% stringstyle=\color{mylilas},
% commentstyle=\color{mygreen},%
% showstringspaces=false,%without this there will be a symbol in the places where there is a space
% numbers=left,%
% numberstyle={\tiny \color{black}},% size of the numbers
% numbersep=7pt, % this defines how far the numbers are from the text
% emph=[1]{for,end,break},emphstyle=[1]\color{red}, %some words to emphasise
% % emph=[2]{word1,word2}, emphstyle=[2]{style},
% frame=single,
basicstyle=\small\ttfamily,%
breaklines=true,%
morekeywords={matlab2tikz},%
keywordstyle=\color{blue},%
morekeywords=[2]{1},%
keywordstyle=[2]{\color{black}},%
identifierstyle=\color{black},%
stringstyle=\color{mylilas},
commentstyle=\color{mygreen},%
% moredelim=[il][\rmfamily]{//},%
morecomment=[n][\color{black}]{(*}{*)},%
showstringspaces=false,%
numbers=none,%
% numbers=left,%
% numberstyle={\tiny \color{black}},%
% numbersep=7pt,%
emph=[1]{for,end,break,if,while,mod,ones,randi,sind,cosd,tand},
emphstyle=[1]\color{blue}, %some words to emphasise
% emph=[2]{word1,word2}, emphstyle=[2]{style},
frame=single,%
framerule=0.7pt,%
mathescape=true,%
escapebegin=\color{mygreen},%
escapeend=,%
escapechar=`,
}
% ref: https://tex.stackexchange.com/questions/257938/how-to-include-matlab-code-into-latex-in-colour


%% Algorithm/Pseudocode Environment using `listings'
% https://tex.stackexchange.com/questions/111116/what-is-the-best-looking-pseudo-code-package
%
% \newcounter{nalg}[chapter] % defines algorithm counter for chapter-level
% \renewcommand{\thenalg}{\thechapter .\arabic{nalg}}

%defines appearance of the algorithm counter
\DeclareCaptionLabelFormat{algocaption}{Algorithm \thenalg} % defines a new caption label as Algorithm x.y

\lstnewenvironment{algorithm}[1][] %defines the algorithm listing environment
{
% \refstepcounter{nalg} %increments algorithm number
\captionsetup{labelformat=algocaption,labelsep=colon} %defines the caption setup for: it uses label format as the declared caption label above and makes label and caption text to be separated by a ':'
\lstset{language=,
basicstyle=\small,%
stringstyle=\small\ttfamily,%
breaklines=true,%
keywords={for, input, output, return, datatype, function, in, if, else, foreach, while, begin, end},
keywordstyle=\color{blue}\bfseries\ttfamily,%
numbers=none,%
frame=single,%
framerule=0.7pt,%
mathescape=true,%
escapechar=',
xleftmargin=.04\linewidth,
#1 % this is to add specific settings to an usage of this environment (for instance, the caption and referable label)
}
}
{}
