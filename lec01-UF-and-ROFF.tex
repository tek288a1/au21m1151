\documentclass[10pt,t,presentation,ignorenonframetext,aspectratio=169]{beamer}
% \documentclass[10pt,t,handout,ignorenonframetext,aspectratio=169]{beamer}
\usepackage[default]{lato}
\usepackage{tk_beamer1}
%% packages

\RequirePackage{lmodern} % math, rm, ss, tt
\RequirePackage[T1]{fontenc}
\RequirePackage[english]{babel}
\RequirePackage{enumerate}
\RequirePackage{etex}
\RequirePackage{color,xcolor,ucs}
\RequirePackage{graphicx}
\RequirePackage{amssymb}
\RequirePackage{amsmath}
\RequirePackage{subfig}
\RequirePackage{amsthm}
\RequirePackage{mathtools}
\RequirePackage{mathabx}        % required for \odiv; put this after mathtools, otherwise, over/underbrace get messed up
\RequirePackage{epsfig}
\RequirePackage{epstopdf}
\RequirePackage{float}
\RequirePackage{booktabs}
\RequirePackage{blkarray}
\RequirePackage{multirow}
\RequirePackage{hyperref}
\RequirePackage{verbatim}
\RequirePackage{lscape}
\RequirePackage[mathscr]{euscript}
\RequirePackage{movie15}
\RequirePackage{mwe,tikz}
\RequirePackage[percent]{overpic}
\RequirePackage{multicol}
\RequirePackage{pgfplots}
\pgfplotsset{compat=1.7}
\RequirePackage{relsize}
\RequirePackage{textcomp}
\RequirePackage{fancyvrb}
\RequirePackage{tcolorbox}
\RequirePackage{bm}


% %%%%%% for matlab listings
% \RequirePackage{courier}
% \RequirePackage{listings}
% % \newcommand{\matlab}{\textsc{MatLab}}
% % \definecolor{mygreen}{RGB}{28,172,0} % color values Red, Green, Blue
% \definecolor{mygreen}{RGB}{0,100,0} % color values Red, Green, Blue
% \definecolor{mylilas}{RGB}{170,55,241}

% \lstset{language=Matlab,
%   % basicstyle=\small\ttfamily, % Use small true type font
%   % breaklines=true,%
%   % morekeywords={matlab2tikz},
%   % keywordstyle=\color{blue},%
%   % morekeywords=[2]{1}, keywordstyle=[2]{\color{black}},
%   % identifierstyle=\color{black},%
%   % stringstyle=\color{mylilas},
%   % commentstyle=\color{mygreen},%
%   % showstringspaces=false,%without this there will be a symbol in the places where there is a space
%   % numbers=left,%
%   % numberstyle={\tiny \color{black}},% size of the numbers
%   % numbersep=7pt, % this defines how far the numbers are from the text
%   % emph=[1]{for,end,break},emphstyle=[1]\color{red}, %some words to emphasise
%   % % emph=[2]{word1,word2}, emphstyle=[2]{style},
%   % frame=single,
%   basicstyle=\small\ttfamily,%
%   breaklines=true,%
%   morekeywords={matlab2tikz},%
%   keywordstyle=\color{blue},%
%   morekeywords=[2]{1},%
%   keywordstyle=[2]{\color{black}},%
%   identifierstyle=\color{black},%
%   stringstyle=\color{mylilas},
%   commentstyle=\color{mygreen},%
%   % moredelim=[il][\rmfamily]{//},%
%   morecomment=[n][\color{black}]{(*}{*)},%
%   showstringspaces=false,%
%   numbers=none,%
%   % numbers=left,%
%   % numberstyle={\tiny \color{black}},%
%   % numbersep=7pt,%
%   emph=[1]{for,end,break,if,while},emphstyle=[1]\color{blue}, %some words to emphasise
%   % emph=[2]{word1,word2}, emphstyle=[2]{style},
%   frame=single,%
%   framerule=0.7pt,%
%   mathescape=true,%
%   escapebegin=\color{mygreen},%
%   escapeend=,%
% }
% % ref: https://tex.stackexchange.com/questions/257938/how-to-include-matlab-code-into-latex-in-colour



% % \usepackage{spalign}
% % \usepackage{enumitem}
% % \graphicspath{ {../codes/} }
% % \epstopdfsetup{outdir=../codes/}
% % \lstset{inputpath="../codes"}
% % \setlength{\fboxsep}{1.5pt}
% % \newcommand{\x}{\times}
% % \newcommand{\bigzero}{\makebox(0,0){\text{\LARGE 0}}}
% % \newcommand*{\bord}{\multicolumn{1}{c|}{}}

% %% Algorithm/Pseudocode Environment using `listings'
% % https://tex.stackexchange.com/questions/111116/what-is-the-best-looking-pseudo-code-package
% %
% % \newcounter{nalg}[chapter] % defines algorithm counter for chapter-level
% % \renewcommand{\thenalg}{\thechapter .\arabic{nalg}}

% %defines appearance of the algorithm counter
% \DeclareCaptionLabelFormat{algocaption}{Algorithm \thenalg} % defines a new caption label as Algorithm x.y

% \lstnewenvironment{algorithm}[1][] %defines the algorithm listing environment
% {
%     % \refstepcounter{nalg} %increments algorithm number
%     \captionsetup{labelformat=algocaption,labelsep=colon} %defines the caption setup for: it uses label format as the declared caption label above and makes label and caption text to be separated by a ':'
%     \lstset{ %this is the stype
%         mathescape=true,
%         % %frame=tB,
%         % frame=none,
%         % numbers=left,
%         % numberstyle=\tiny,
%         % commentstyle=,
%         % basicstyle=\small,
%         % stringstyle=\ttfamily,
%         basicstyle=,
%         keywordstyle=\color{black}\bfseries,
%         keywords={for, input, output, return, datatype, function, in, if, else, foreach, while, begin, end}, %add the keywords you want, or load a language as Rubens explains in his comment above.
%         % xleftmargin=.04\textwidth,
%         #1 % this is to add specific settings to an usage of this environment (for instnce, the caption and referable label)
%         }
% }
% {}

%%%% Custom macros
%%%% Macros

%% Greek letters

\newcommand{\al}{\alpha}
% \newcommand{\be}{\beta}
\newcommand{\g}{\gamma}
\newcommand{\de}{\delta}
\newcommand{\e}{\epsilon}
\newcommand{\eps}{\varepsilon}
\newcommand{\ka}{\kappa}
\newcommand{\la}{\lambda}
\newcommand{\sig}{\sigma}
\newcommand{\om}{\omega}
\newcommand{\Om}{\Omega}
\let\oldth\th %% \th is used for "thorn"
\renewcommand{\th}{\theta}

%% Tweak some Greek letters
\newcommand{\bchi}{\mbox{\raisebox{.4ex}{\begin{large}$\chi$\end{large}}}}
\newcommand{\Chi}{\mbox{\Large$\chi$}} % nicer looking Chi
\newcommand{\bzeta}{\boldsymbol{\zeta}} % Riemann zeta function
\newcommand{\bxi}{\boldsymbol{\xi}}
\newcommand{\balpha}{\boldsymbol{\alpha}}

%% Blackboard

\newcommand{\NN}{\mathbb{N}}
\newcommand{\ZZ}{\mathbb{Z}}
\newcommand{\QQ}{\mathbb{Q}}
\newcommand{\RR}{\mathbb{R}}
\newcommand{\CC}{\mathbb{C}}
\newcommand{\FF}{\mathbb{F}}
\newcommand{\TT}{\mathbb{T}}
\newcommand{\DD}{\mathbb{D}}
\newcommand{\HH}{\mathbb{H}}
\newcommand{\UU}{\mathbb{U}}
\newcommand{\1}{\mathbbm{1}}

%% Caligraphic

\newcommand{\cA}{\mathcal{A}}
\newcommand{\cB}{\mathcal{B}}
\newcommand{\cC}{\mathcal{C}}
\newcommand{\cD}{\mathcal{D}}
\newcommand{\cE}{\mathcal{E}}
\newcommand{\cF}{\mathcal{F}}
\newcommand{\cG}{\mathcal{G}}
\newcommand{\cH}{\mathcal{H}}
\newcommand{\cI}{\mathcal{I}}
\newcommand{\cJ}{\mathcal{J}}
\newcommand{\cK}{\mathcal{K}}
\newcommand{\cL}{\mathcal{L}}
\newcommand{\cM}{\mathcal{M}}
\newcommand{\cN}{\mathcal{N}}
\newcommand{\cO}{\mathcal{O}}
\newcommand{\cP}{\mathcal{P}}
\newcommand{\cQ}{\mathcal{Q}}
\newcommand{\cR}{\mathcal{R}}
\newcommand{\cS}{\mathcal{S}}
\newcommand{\cT}{\mathcal{T}}
\newcommand{\cU}{\mathcal{U}}
\newcommand{\cV}{\mathcal{V}}
\newcommand{\cW}{\mathcal{W}}


%% Roman, italic, boldface

\newcommand{\bA}{\mathbf{A}}
\newcommand{\bB}{\mathbf{B}}
\newcommand{\bC}{\mathbf{C}}
\newcommand{\bD}{\mathbf{D}}
\newcommand{\bE}{\mathbf{E}}
\newcommand{\bI}{\mathbf{I}}
\newcommand{\bK}{\mathbf{K}}
\newcommand{\bL}{\mathbf{L}}
\newcommand{\bM}{\mathbf{M}}
\newcommand{\bN}{\mathbf{N}}
\newcommand{\bP}{\mathbf{P}}
\newcommand{\bS}{\mathbf{S}}
\newcommand{\bT}{\mathbf{T}}
\newcommand{\bX}{\mathbf{X}}
\newcommand{\ba}{\mathbf{a}}
\newcommand{\bb}{\mathbf{b}}
\newcommand{\bc}{\mathbf{c}}
\newcommand{\bd}{\mathbf{d}}
\newcommand{\be}{\mathbf{e}}
\newcommand{\bg}{\mathbf{g}}
\newcommand{\bh}{\mathbf{h}}
\newcommand{\br}{\mathbf{r}}
\newcommand{\bx}{\mathbf{x}}
\newcommand{\by}{\mathbf{y}}
\newcommand{\bu}{\mathbf{u}}
\newcommand{\bv}{\mathbf{v}}
\newcommand{\bw}{\mathbf{w}}
\newcommand{\bz}{\mathbf{z}}
\newcommand{\bq}{\mathbf{q}}
\newcommand{\bzero}{\mathbf{0}}


%% Principal value integral
\def\Xint#1{\mathchoice
  {\XXint\displaystyle\textstyle{#1}}%
  {\XXint\textstyle\scriptstyle{#1}}%
  {\XXint\scriptstyle\scriptscriptstyle{#1}}%
  {\XXint\scriptscriptstyle\scriptscriptstyle{#1}}%
  \!\int}
\def\XXint#1#2#3{{\setbox0=\hbox{$#1{#2#3}{\int}$}
    \vcenter{\hbox{$#2#3$}}\kern-.5\wd0}}
\def\ddashint{\Xint=}
\def\pvint{\Xint-}

%% Arc over symbols
% reference: https://tex.stackexchange.com/questions/96680/a-better-notation-to-denote-arcs-for-an-american-high-school-textbook
\makeatletter
\DeclareFontFamily{U}{tipa}{}
\DeclareFontShape{U}{tipa}{m}{n}{<->tipa10}{}
\newcommand{\arc@char}{{\usefont{U}{tipa}{m}{n}\symbol{62}}}%
\newcommand{\arc}[1]{\mathpalette\arc@arc{#1}}
\newcommand{\arc@arc}[2]{%
  \sbox0{$\m@th#1#2$}%
  \vbox{
    \hbox{\resizebox{\wd0}{\height}{\arc@char}}
    \nointerlineskip
    \box0
  }%
}


%% colored boxes
\newsavebox{\astrutbox}
\sbox{\astrutbox}{\rule[-5pt]{0pt}{20pt}}
\newcommand{\astrut}{\usebox{\astrutbox}}
\newcommand{\rls}{\raisebox{2pt}{\tikz{\draw[red,solid,line width=0.9pt](0,0) -- (5mm,0);}}}
\newcommand{\rld}{\raisebox{2pt}{\tikz{\draw[red,dashed,line width=1.0pt](0,0) -- (5mm,0);}}}
\newcommand{\bls}{\raisebox{2pt}{\tikz{\draw[blue,solid,line width=0.9pt](0,0) -- (5mm,0);}}}
\newcommand{\bld}{\raisebox{2pt}{\tikz{\draw[blue,dashed,line width=1.0pt](0,0) -- (5mm,0);}}}
\newcommand{\gls}{\raisebox{2pt}{\tikz{\draw[green,solid,line width=0.9pt](0,0) -- (5mm,0);}}}
\newcommand{\gld}{\raisebox{2pt}{\tikz{\draw[green,dashed,line width=1.0pt](0,0) -- (5mm,0);}}}
\newcommand{\mls}{\raisebox{2pt}{\tikz{\draw[magenta,solid,line width=0.9pt](0,0) -- (5mm,0);}}}
\newcommand{\mld}{\raisebox{2pt}{\tikz{\draw[magenta,dashed,line width=1.0pt](0,0) -- (5mm,0);}}}
\newcommand{\cls}{\raisebox{2pt}{\tikz{\draw[cyan,solid,line width=0.9pt](0,0) -- (5mm,0);}}}
\newcommand{\cld}{\raisebox{2pt}{\tikz{\draw[cyan,dashed,line width=1.0pt](0,0) -- (5mm,0);}}}



%% Delimiters, accents, bars, etc

\newcommand{\abs}[1]{\left|#1\right|}
\newcommand{\ceil}[1]{\left\lceil#1\right\rceil}
\newcommand{\floor}[1]{\left\lfloor#1\right\rfloor}
\newcommand{\conj}[1]{\overline{#1}}
\newcommand{\norm}[1]{\left\|#1\right\|}
\newcommand{\Norm}[2]{\left\|#1\right\|_{#2}}
% Improvement of the above two
% example usage: \norm[2]{f} or \Norm[2]{f}{L^2}
\renewcommand{\norm}[2][0]{%
  \ifcase#1\relax
    \left\Vert #2 \right\Vert\or  % 0
    \lVert #2 \rVert\or           % 1
    \bigl\Vert #2 \bigr\Vert\or   % 2
    \Bigl\Vert #2 \Bigr\Vert\or   % 3
    \biggl\Vert #2 \biggr\Vert\or % 4
    \Biggl\Vert #2 \Biggr\Vert    % 5
  \fi}
\renewcommand{\Norm}[3][0]{%
  \ifcase#1\relax
    \left\Vert #2 \right\Vert_{#3}\or  % 0
    \lVert #2 \rVert_{#3}\or           % 1
    \bigl\Vert #2 \bigr\Vert_{#3}\or   % 2
    \Bigl\Vert #2 \Bigr\Vert_{#3}\or   % 3
    \biggl\Vert #2 \biggr\Vert_{#3}\or % 4
    \Biggl\Vert #2 \Biggr\Vert_{#3}    % 5
  \fi}
\newcommand{\avg}[1]{\langle#1\rangle}
\newcommand{\ds}{\displaystyle}


%% Mathematical operators

\DeclareMathOperator{\re}{Re}
\DeclareMathOperator{\im}{Im}
\DeclareMathOperator{\sgn}{sgn}
\DeclareMathOperator{\erf}{erf}
\DeclareMathOperator{\erfc}{erfc}
\DeclareMathOperator{\ii}{i}
\DeclareMathOperator{\dd}{\,d}
\DeclareMathOperator{\eu}{e}
\DeclareMathOperator{\Sp}{Sp}
\DeclareMathOperator{\acosh}{acosh}
\DeclareMathOperator{\asech}{asech}
\DeclareMathOperator{\atanh}{atanh}
\newcommand{\del}{\partial}
\newcommand{\tri}{\triangle}
\newcommand{\grad}{\nabla}
\newcommand{\dvg}{\nabla\cdot}
\newcommand{\curl}{\nabla\times}


%% colored texts

\newcommand{\red}[1]{{\color{red}{#1}}}
\newcommand{\blue}[1]{{\color{blue}{#1}}}
\newcommand{\green}[1]{{\color{green}{#1}}}


%%

\newcommand{\emptyframe}{\begin{frame}{}\end{frame}}
\newcommand{\question}{\textbf{Question.}}
\newcommand{\sqitem}{\item[$\square$]}


%% 06/12/18 addition

\newcommand{\vs}{\vspace{1em}}
\newcommand{\tp}{^{\rm T}}

%% 06/22/18 addition
% for 3607
\newcommand{\meps}{\fbox{eps}}
\newcommand{\flops}{\textit{flops}}
% \setlength{\fboxsep}{1.5pt}
\newcommand\x{\times}
\newcommand\bigzero{\makebox(0,0){\text{\LARGE 0}}}
\newcommand*{\bord}{\multicolumn{1}{c|}{}}


%% Continued numbering over multiple enumerate environments
% https://tex.stackexchange.com/questions/55000/continuing-enumerate-counters-in-beamer
\newcounter{saveenumi}
\newcommand{\seti}{\setcounter{saveenumi}{\value{enumi}}}
\newcommand{\conti}{\setcounter{enumi}{\value{saveenumi}}}


%% Extension to amsmath matrix environment
% improving matrix constructors
% source: http://texblog.net/latex-archive/maths/amsmath-matrix/
\makeatletter
\renewcommand*\env@matrix[1][*\c@MaxMatrixCols c]{%
  \hskip -\arraycolsep
  \let\@ifnextchar\new@ifnextchar
  \array{#1}}
\makeatother

% In order to make the column lines to look nicer:
\setlength\delimitershortfall{0pt}


%% 11/09/18 addition: vertical spaces
\newcommand{\vsone}{\vspace{\stretch{1}}}
\newcommand{\vstwo}{\vspace{\stretch{2}}}
\newcommand{\vsthree}{\vspace{\stretch{3}}}

%% 02/23/19 addition: wide hat and wide tilde
\newcommand{\wh}[1]{\widehat{#1}}
\newcommand{\wt}[1]{\widetilde{#1}}

%% Defining theorem environment
\theoremstyle{plain}
\newtheorem{prop}{Proposition}
\newtheorem{cor}[prop]{Corollary}
\newtheorem{lem}[prop]{Lemma}
\newtheorem{thm}[prop]{Theorem}
\newtheorem{cons}[prop]{Consequence}
\newtheorem{conv}[prop]{Convention}
\newtheorem{prob}[prop]{Problem}
\newtheorem{form}[prop]{Formulation}
\newtheorem{claim}[prop]{Claim}

\theoremstyle{definition}
\newtheorem{defn}{Definition}
\newtheorem{notn}[defn]{Notation}
\newtheorem{note}[defn]{Note}
\newtheorem{rmk}[defn]{Remark}
\newtheorem{exer}[defn]{Exercise}
\newtheorem{ex}[defn]{Example}


%% Listings Environments
\usepackage{courier}
\usepackage{listings}

% \definecolor{mygreen}{RGB}{28,172,0} % color values Red, Green, Blue
\definecolor{mygreen}{RGB}{0,100,0} % color values Red, Green, Blue
\definecolor{mylilas}{RGB}{170,55,241}

\lstdefinestyle{matlab}{language=Matlab,
% basicstyle=\small\ttfamily, % Use small true type font
% breaklines=true,%
% morekeywords={matlab2tikz},
% keywordstyle=\color{blue},%
% morekeywords=[2]{1}, keywordstyle=[2]{\color{black}},
% identifierstyle=\color{black},%
% stringstyle=\color{mylilas},
% commentstyle=\color{mygreen},%
% showstringspaces=false,%without this there will be a symbol in the places where there is a space
% numbers=left,%
% numberstyle={\tiny \color{black}},% size of the numbers
% numbersep=7pt, % this defines how far the numbers are from the text
% emph=[1]{for,end,break},emphstyle=[1]\color{red}, %some words to emphasise
% % emph=[2]{word1,word2}, emphstyle=[2]{style},
% frame=single,
basicstyle=\small\ttfamily,%
breaklines=true,%
morekeywords={matlab2tikz},%
keywordstyle=\color{blue},%
morekeywords=[2]{1},%
keywordstyle=[2]{\color{black}},%
identifierstyle=\color{black},%
stringstyle=\color{mylilas},
commentstyle=\color{mygreen},%
% moredelim=[il][\rmfamily]{//},%
morecomment=[n][\color{black}]{(*}{*)},%
showstringspaces=false,%
numbers=none,%
% numbers=left,%
% numberstyle={\tiny \color{black}},%
% numbersep=7pt,%
emph=[1]{for,end,break,if,while,mod,ones,randi,sind,cosd,tand},
emphstyle=[1]\color{blue}, %some words to emphasise
% emph=[2]{word1,word2}, emphstyle=[2]{style},
frame=single,%
framerule=0.7pt,%
mathescape=true,%
escapebegin=\color{mygreen},%
escapeend=,%
escapechar=`,
}
% ref: https://tex.stackexchange.com/questions/257938/how-to-include-matlab-code-into-latex-in-colour


%% Algorithm/Pseudocode Environment using `listings'
% https://tex.stackexchange.com/questions/111116/what-is-the-best-looking-pseudo-code-package
%
% \newcounter{nalg}[chapter] % defines algorithm counter for chapter-level
% \renewcommand{\thenalg}{\thechapter .\arabic{nalg}}

%defines appearance of the algorithm counter
\DeclareCaptionLabelFormat{algocaption}{Algorithm \thenalg} % defines a new caption label as Algorithm x.y

\lstnewenvironment{algorithm}[1][] %defines the algorithm listing environment
{
% \refstepcounter{nalg} %increments algorithm number
\captionsetup{labelformat=algocaption,labelsep=colon} %defines the caption setup for: it uses label format as the declared caption label above and makes label and caption text to be separated by a ':'
\lstset{language=,
basicstyle=\small,%
stringstyle=\small\ttfamily,%
breaklines=true,%
keywords={for, input, output, return, datatype, function, in, if, else, foreach, while, begin, end},
keywordstyle=\color{blue}\bfseries\ttfamily,%
numbers=none,%
frame=single,%
framerule=0.7pt,%
mathescape=true,%
escapechar=',
xleftmargin=.04\linewidth,
#1 % this is to add specific settings to an usage of this environment (for instance, the caption and referable label)
}
}
{}

%%%% ximera preamble extract

%% TikZ/PGFplot related

\usepackage{tkz-euclide}
\usepackage{tikz}
\usepackage{tikz-cd}
\usepackage{pgffor} %% required for integral for loops
\usetikzlibrary{arrows}
\tikzset{>=stealth,commutative diagrams/.cd,
  arrow style=tikz,diagrams={>=stealth}} %% cool arrow head
\tikzset{shorten <>/.style={ shorten >=#1, shorten <=#1 } } %% allows shorter vectors

\usetikzlibrary{backgrounds} %% for boxes around graphs
\usetikzlibrary{shapes,positioning}  %% Clouds and stars
\usetikzlibrary{matrix} %% for matrix
\usepgfplotslibrary{polar} %% for polar plots
% \usetkzobj{all}

% Gaussian function
\pgfmathdeclarefunction{gauss}{2}{% gives gaussian
  \pgfmathparse{1/(#2*sqrt(2*pi))*exp(-((x-#1)^2)/(2*#2^2))}%
}



%% colors

\colorlet{textColor}{black}
\colorlet{background}{white}
\colorlet{penColor}{blue!50!black} % Color of a curve in a plot
\colorlet{penColor2}{red!50!black}% Color of a curve in a plot
\colorlet{penColor3}{red!50!blue} % Color of a curve in a plot
\colorlet{penColor4}{green!50!black} % Color of a curve in a plot
\colorlet{penColor5}{orange!80!black} % Color of a curve in a plot
\colorlet{penColor6}{yellow!70!black} % Color of a curve in a plot
\colorlet{fill1}{penColor!20} % Color of fill in a plot
\colorlet{fill2}{penColor2!20} % Color of fill in a plot
\colorlet{fillp}{fill1} % Color of positive area
\colorlet{filln}{penColor2!20} % Color of negative area
\colorlet{fill3}{penColor3!20} % Fill
\colorlet{fill4}{penColor4!20} % Fill
\colorlet{fill5}{penColor5!20} % Fill
\colorlet{gridColor}{gray!50} % Color of grid in a plot
\newcommand{\surfaceColor}{violet}
\newcommand{\surfaceColorTwo}{redyellow}
\newcommand{\sliceColor}{greenyellow}



%% image environment
\usepackage{environ}
\NewEnviron{image}[1][3in]{%
  \begin{center}\resizebox{#1}{!}{\BODY}\end{center}% resize and center
}



%% more packages

\usepackage[makeroom]{cancel} %% for strike outs
\usepackage{multicol}
\usepackage{array}



%% lengths

\setlength{\extrarowheight}{+.1cm}


%% macros
% \newcommand{\dd}[2][]{\frac{\d #1}{\d #2}} % conflict
\newcommand{\pp}[2][]{\frac{\partial #1}{\partial #2}}
\newcommand{\ddx}{\frac{d}{\d x}}
\newcommand{\dfn}{\textbf}
\newcommand{\unit}{\mathop{}\!\mathrm}
\newcommand{\eval}[1]{\bigg[ #1 \bigg]}
\newcommand{\seq}[1]{\left( #1 \right)}
\renewcommand{\d}{\mathop{}\!d}
\renewcommand{\l}{\ell}
% \newcommand{\zeroOverZero}{\ensuremath{\boldsymbol{\tfrac{0}{0}}}}
% \newcommand{\inftyOverInfty}{\ensuremath{\boldsymbol{\tfrac{\infty}{\infty}}}}
% \newcommand{\zeroOverInfty}{\ensuremath{\boldsymbol{\tfrac{0}{\infty}}}}
% \newcommand{\zeroTimesInfty}{\ensuremath{\small\boldsymbol{0\cdot \infty}}}
% \newcommand{\inftyMinusInfty}{\ensuremath{\small\boldsymbol{\infty - \infty}}}
% \newcommand{\oneToInfty}{\ensuremath{\boldsymbol{1^\infty}}}
% \newcommand{\zeroToZero}{\ensuremath{\boldsymbol{0^0}}}
% \newcommand{\inftyToZero}{\ensuremath{\boldsymbol{\infty^0}}}
% \newcommand{\numOverZero}{\ensuremath{\boldsymbol{\tfrac{\#}{0}}}}

\usepackage{wasysym}            % for smiley

%%%% META DATA
\newcommand{\semester}{Autumn 2021}
\newcommand{\course}{Math 1151}
\newcommand{\lecTitle}{Lecture 1: Review of Precalculus}

%%%% TITLE PAGE
\title[\course]{\lecTitle}
\institute[Ohio State]
{
  \medskip
}
\date[\week]{\semester}
\author{Tae Eun Kim, Ph.D.}

\begin{document}
\begin{frame}
  \titlepage
\end{frame}

\begin{frame}
  \frametitle{Overview}
  \tableofcontents
\end{frame}

\section{Understanding Functions (UF)}
\label{sec:underst-funct-uf}
\begin{frame}
  \frametitle{``For each input, exactly on output''}
  \begin{definition}\index{function}
    \begin{itemize}
    \item \dfn{function}: a relation between sets where for each
      input, there is exactly one output
    \item \dfn{domain}: the set of the inputs of a function
    \item \dfn{range}: the set of the outputs of a function
    \end{itemize}
  \end{definition}
\end{frame}

\begin{frame}
  \frametitle{Representation of functions}
  \begin{itemize}
  \item \textbf{formula}: $f(x) = x^3$
  \item \textbf{table}: \\
    \begin{center}
      \begin{tabular}{c|c|c|c|c}
        input & 1 & -2 & 1.5 & $\cdots$ \\ \hline
        output & 1 & -8 & 3.375 & $\cdots$
      \end{tabular}
    \end{center}
  \item \textbf{graph}: \\
    \begin{center}
      \begin{tikzpicture}
        \begin{axis}[
          domain=-2:2,
          width=0.6*4in,
          height=0.6*3in,
          axis lines =middle, xlabel=$x$, ylabel=$y$,
          every axis y label/.style={at=(current axis.above origin),anchor=south},
          every axis x label/.style={at=(current axis.right of origin),anchor=west},
          ]
          \addplot [thick, penColor, smooth] {x^3};
        \end{axis}
      \end{tikzpicture}
    \end{center}
  \end{itemize}
\end{frame}

\begin{frame}
  \vs
  \begin{block}{Example: Greatest Integer Function}
    \begin{itemize}
    \item maps any real number $x$ to the greatest integer less than
      or equal to $x$.
    \item a.k.a. \textit{floor function}
    \item denoted by $\lfloor x \rfloor$
    \item many inputs to one output
    \end{itemize}
  \end{block}

  \vfill
  \begin{center}
    \begin{tikzpicture}
      \begin{axis}[
        domain=-2:4,
        width=0.6*4in,
        height=0.6*3in,
        axis lines =middle, xlabel=$x$, ylabel=$y$,
        every axis y label/.style={at=(current axis.above origin),anchor=south},
        every axis x label/.style={at=(current axis.right of origin),anchor=west},
        clip=false,
        % axis on top,
        ]
        \addplot [textColor, very thin, domain=(0:2.3)] {0}; % puts the axis back, axis on top clobbers our open holes
        \addplot [textColor, very thin] plot coordinates {(0,0) (0,2)}; % puts the axis back, axis on top clobbers our open holes
        \addplot [very thick, penColor, domain=(-2:-1)] {-2};
        \addplot [very thick, penColor, domain=(-1:0)] {-1};
        \addplot [very thick, penColor, domain=(0:1)] {0};
        \addplot [very thick, penColor, domain=(1:2)] {1};
        \addplot [very thick, penColor, domain=(2:3)] {2};
        \addplot [very thick, penColor, domain=(3:4)] {3};
        \addplot[color=penColor,fill=penColor,only marks,mark=*] coordinates{(-2,-2)};  %% closed hole
        \addplot[color=penColor,fill=penColor,only marks,mark=*] coordinates{(-1,-1)};  %% closed hole
        \addplot[color=penColor,fill=penColor,only marks,mark=*] coordinates{(0,0)};  %% closed hole
        \addplot[color=penColor,fill=penColor,only marks,mark=*] coordinates{(1,1)};  %% closed hole
        \addplot[color=penColor,fill=penColor,only marks,mark=*] coordinates{(2,2)};  %% closed hole
        \addplot[color=penColor,fill=penColor,only marks,mark=*] coordinates{(3,3)};  %% closed hole
        \addplot[color=penColor,fill=background,only marks,mark=*] coordinates{(-1,-2)};  %% open hole
        \addplot[color=penColor,fill=background,only marks,mark=*] coordinates{(0,-1)};  %% open hole
        \addplot[color=penColor,fill=background,only marks,mark=*] coordinates{(1,0)};  %% open hole
        \addplot[color=penColor,fill=background,only marks,mark=*] coordinates{(2,1)};  %% open hole
        \addplot[color=penColor,fill=background,only marks,mark=*] coordinates{(3,2)};  %% open hole
        \addplot[color=penColor,fill=background,only marks,mark=*] coordinates{(4,3)};  %% open hole
      \end{axis}
    \end{tikzpicture}
    %% \caption{A plot of $f(x)=\lfloor x\rfloor$. Here we can see that for each input (a
    %% value on the $x$-axis), there is exactly one output (a value on the
    %% $y$-axis).}
    %% \label{plot:greatest-integer fxn}
  \end{center}
\end{frame}

\begin{frame}
  \vs
  \begin{theorem}[Vertical line test]
    The curve $y=f(x)$ represents $y$ as a function of $x$ at $x=a$ if and
    only if the vertical line $x=a$ intersects the curve $y=f(x)$ at
    exactly one point. This is called the \dfn{vertical line test}.
  \end{theorem}
\end{frame}

\begin{frame}
  \vs
  \begin{block}{Distinguishing two functions}
    \begin{itemize}
    \item Do they have the same domain?
    \item Do they display the same relation?
    \end{itemize}
  \end{block}

  \vs
  \textbf{Question.} Determine if the two function are the same.
  \begin{enumerate}
  \item $f(x) = \sqrt{x^2}$ and  $g(x) = \abs{x}$ \vfill
  \item $\ds f(x) = \frac{x^2-3x+2}{x-2}$ and $g(x) = x-1$ \vfill
  \end{enumerate}
\end{frame}

\begin{frame}
  \frametitle{Composition of functions}
  \begin{block}{Composite functions}
    \begin{itemize}
    \item can be thought of as putting one function inside another
    \item \textbf{Notation:} $( f \circ g) (x) = f(g(x))$
    \item \textbf{Warning:} The range of inner function must be
      contained in the domain of outer function.
    \end{itemize}
  \end{block}
\end{frame}

\begin{frame}
  \vs
  \textbf{Question.} Study the composition $f \circ g$ where
  \begin{align*}
    f(x)&=x^2 &&\text{for $-\infty< x< \infty$,}\\
    g(x)&= \sqrt{x} &&\text{for $0\le x< \infty$.}
  \end{align*}
  \vfill
\end{frame}

\begin{frame}
  \vs
  \textbf{Question.} Study the composition $f \circ g$.
  \begin{align*}
    f(x)&=\sqrt{x} &&\text{for $0\le x< \infty$,}\\
    g(x)&= x^2 &&\text{for $-\infty< x< \infty$.}
  \end{align*}
  \vfill
\end{frame}

\begin{frame}
  \frametitle{Inverses of functions}
  \begin{definition}
    Let $f$ be a function with domain $A$ and range $B$:
    \[
      f : A \to B
    \]
    % \begin{image}
    %   \begin{tikzpicture}
    %     \node[star,star points=7,star point ratio=2.5,draw] at (0,0) {$A$};
    %     \node[cloud, draw,cloud puffs=10,cloud puff arc=120, aspect=2, inner ysep=1em] at (5,0) {$B$};
    %     \node at (2.25,.3) {$f$};
    %     \draw[->] (1.5,0) to (3,0);
    %   \end{tikzpicture}
    % \end{image}
    Let $g$ be a function with domain $B$ and range $A$:
    \[
      g : B \to A
    \]
    % \begin{image}
    %   \begin{tikzpicture}
    %     \node[cloud, draw,cloud puffs=10,cloud puff arc=120, aspect=2, inner ysep=1em] at (-.5,0) {$B$};
    %     \node[star,star points=7,star point ratio=2.5,draw] at (4.5,0) {$A$};
    %     \node at (2.25,.3) {$g$};
    %     \draw[->] (1.5,0) to (3,0);
    %   \end{tikzpicture}
    % \end{image}
    We say that $f$ and $g$ are \dfn{inverses} of each other if $f(g(b))
    = b$ for all $b$ in $B$, and also $g(f(a)) = a$ for all $a$ in $A$.
    Sometimes we write $g = f^{-1}$ in this case.
    % \begin{image}
    %   \begin{tikzpicture}
    %     \node[cloud, draw,cloud puffs=10,cloud puff arc=120, aspect=2, inner ysep=1em] at (-.5,0) {$B$};
    %     \node[cloud, draw,cloud puffs=10,cloud puff arc=120, aspect=2, inner ysep=1em] at (5,0) {$B$};
    %     \node at (2.25,.3) {$f\circ f^{-1}$};
    %     \draw[->] (1.5,0) to (3,0);
    %   \end{tikzpicture}
    % \end{image}
    % and
    % \begin{image}
    %   \begin{tikzpicture}
    %     \node[star,star points=7,star point ratio=2.5,draw] at (0,0) {$A$};
    %     \node[star,star points=7,star point ratio=2.5,draw] at (4.5,0) {$A$};
    %     \node at (2.25,.3) {$f^{-1}\circ f$};
    %     \draw[->] (1.5,0) to (3,0);
    %   \end{tikzpicture}
    % \end{image}
    % So, we could rephrase these conditions as
    % \[
    %   f(f^{-1}(x)) = x\qquad\text{and}\qquad f^{-1}(f(x)) = x.
    % \]
  \end{definition}
  \vfill
  We could rephrase these conditions as
  \[
    f(f^{-1}(x)) = x\qquad\text{and}\qquad f^{-1}(f(x)) = x.
  \]
\end{frame}

\begin{frame}
  \vs
  \begin{block}{Warning: notations}
    Pay attention to where we put the superscript:
    \begin{align*}
      f^{-1}(x) &= \text{the inverse function of $f(x)$.}\\
      f(x)^{-1} &= \text{the reciprocal of $f(x)$.}
    \end{align*}
  \end{block}
\end{frame}

\begin{frame}
  \vs
  \begin{definition}
    A function is called \dfn{one-to-one} if each output value corresponds
    to exactly one input value.
  \end{definition}
\end{frame}

\begin{frame}
  \vs
  \begin{theorem}[Horizontal line test]
    A function is one-to-one at $x=a$ if the horizontal line $y = f(a)$
    intersects the curve $y=f(x)$ in exactly one point. This is called
    the \dfn{horizontal line test}.
  \end{theorem}
\end{frame}

\begin{frame}
  \vs
  \textbf{Question.} Consider the graph of the function $f$ below:
  \begin{center}
    \begin{tikzpicture}
      \begin{axis}[
        ticks=none,
        domain=-2.5:2.5,
        width=0.6*6in,
        height=0.5*3in,
        xmin=-1.5, xmax=1.5,
        ymin=-1, ymax=1,
        axis lines =middle, xlabel=$x$, ylabel=$y$,
        every axis y label/.style={at=(current axis.above origin),anchor=south},
        every axis x label/.style={at=(current axis.right of origin),anchor=west},
        ]
        \addplot [very thick, penColor2, smooth, samples=100,domain=-2:2] {x^3-x-1/6};
        \node at (axis cs:1.3,1.5) [penColor2, anchor=west] {$f$};

        \addplot[color=penColor,fill=penColor,only marks,mark=*] coordinates{(-.9,0)};%%closed
        \addplot[color=penColor,fill=penColor,only marks,mark=*] coordinates{(-.58,0)};%%closed
        \addplot[color=penColor,fill=penColor,only marks,mark=*] coordinates{(-.17,0)};%%closed
        \addplot[color=penColor,fill=penColor,only marks,mark=*] coordinates{(.58,0)};%%closed
        \addplot[color=penColor,fill=penColor,only marks,mark=*] coordinates{(1.07,0)};%%closed

        \node at (axis cs:-1,-0.1) [penColor, anchor=west] {$A$};
        \node at (axis cs:-0.67,-0.1) [penColor, anchor=west] {$B$};
        \node at (axis cs:-0.33,-0.1) [penColor, anchor=west] {$C$};
        \node at (axis cs:0.47,-0.1) [penColor, anchor=west] {$D$};
        \node at (axis cs:1.03,-0.1) [penColor, anchor=west] {$E$};
        \node at (axis cs:1.15,0.6) [penColor2] {$f$};
      \end{axis}
    \end{tikzpicture}
  \end{center}
  On which of the following intervals is $f$ one-to-one?
  \begin{enumerate}
  \item $[A,B]$
  \item $[A,C]$
  \item $[B,D]$
  \item $[C,E]$
  \item $[C,D]$
  \end{enumerate}
\end{frame}

\section{Review of Famous Functions (ROFF)}
\begin{frame}[c]
  These are important functions for Math 1151:
  \begin{itemize}
  \item polynomial functions
  \item rational functions
  \item trigonometric functions and their inverses
  \item exponential and logarithmic functions
  \end{itemize}
\end{frame}

\begin{frame}
  \frametitle{Polynomial functions}
  \begin{definition}
    A \dfn{polynomial function} in the variable $x$ is a function
    which can be written in the form
    \[
      f(x) = a_nx^n + a_{n-1}x^{n-1} + \dots + a_1 x + a_0
    \]
    where the $a_i$'s are all constants (called the \dfn{coefficients})
    and $n$ is a whole number (called the \dfn{degree} when $n\ne
    0$). The domain of a polynomial function is $(-\infty,\infty)$.
  \end{definition}
\end{frame}

\begin{frame}
  \vs
  \textbf{Question.} Which of the following are polynomial functions?
  \vs
  \begin{enumerate}
  \item $f(x) = 7$ \vsone
  \item $f(x) = 3x+1$ \vsone
  \item $f(x) = x^{1/2}-x +8$ \vsone
  \item $f(x) = x^{-4}-3x^{-2}+7+12x^3$ \vsone
  \item $f(x) = (x+\pi)(x-\pi)+e^x - e^x $ \vsone
  \item $\ds f(x) = \frac{x^2 - 3x + 2}{x-2}$ \vsone
  \item $f(x) = x^7-32x^6-\pi x^3+3/7$ \vsone
  \end{enumerate}
\end{frame}

\begin{frame}
  \vs
  \textbf{Some possible graphs of polynomials.}
  \begin{image}[3in]
    \begin{tabular}{cc}
      \begin{tikzpicture}
        \begin{axis}[
          domain=-2:2,
          xmin=-2, xmax=2,
          ymin=-2, ymax=2,
          width=2.5in,
          axis lines =middle, xlabel=$x$, ylabel=$y$,
          every axis y label/.style={at=(current axis.above origin),anchor=south},
          every axis x label/.style={at=(current axis.right of origin),anchor=west},
          ]
	  \addplot [very thick, penColor, smooth] {5*x^5-5*x^4-5*x^3+5*x^2+.5*x -1};
          \node at (axis cs:1.2, 1 ) [penColor,anchor=west] {$A$};
        \end{axis}
      \end{tikzpicture}
      &
        \begin{tikzpicture}
          \begin{axis}[
            domain=-2:2,
            xmin=-2, xmax=2,
            ymin=-2, ymax=2,
            width=2.5in,
            axis lines =middle, xlabel=$x$, ylabel=$y$,
            every axis y label/.style={at=(current axis.above origin),anchor=south},
            every axis x label/.style={at=(current axis.right of origin),anchor=west},
            ]
            \addplot [very thick, penColor2, smooth] {-5*x^5+5*x^4+5*x^3-4.25*x^2-.3*x +.5};
            \node at (axis cs:1.2, 1 ) [penColor2,anchor=west] {$B$};
          \end{axis}
        \end{tikzpicture}\\
      \begin{tikzpicture}
        \begin{axis}[
          domain=-2:2,
          xmin=-2, xmax=2,
          ymin=-2, ymax=2,
          width=2.5in,
          axis lines =middle, xlabel=$x$, ylabel=$y$,
          every axis y label/.style={at=(current axis.above origin),anchor=south},
          every axis x label/.style={at=(current axis.right of origin),anchor=west},
          ]
	  \addplot [very thick, penColor3, smooth] {5*x^6-5*x^5-5*x^4+5*x^3+x^2 -.5};
          \node at (axis cs:1.2, 1 ) [penColor3,anchor=west] {$C$};
        \end{axis}
      \end{tikzpicture}
      &
        \begin{tikzpicture}
          \begin{axis}[
            domain=-2:2,
            xmin=-2, xmax=2,
            ymin=-2, ymax=2,
            width=2.5in,
            axis lines =middle, xlabel=$x$, ylabel=$y$,
            every axis y label/.style={at=(current axis.above origin),anchor=south},
            every axis x label/.style={at=(current axis.right of origin),anchor=west},
            ]
            \addplot [very thick, penColor4, smooth,samples=100] {-5*x^6+5*x^5+5*x^4-5*x^3-x^2+1.5*x+1};
            \node at (axis cs:1.2, 1 ) [penColor4,anchor=west] {$D$};
          \end{axis}
        \end{tikzpicture}
    \end{tabular}
  \end{image}
\end{frame}

\begin{frame}
  \frametitle{Rational functions}
  \begin{definition}
    A \dfn{rational function} in the variable $x$ is a function the form
    \[
      f(x) = \frac{p(x)}{q(x)}
    \]
    where $p$ and $q$ are polynomial functions. The domain of a rational
    function is all real numbers except for where the denominator is
    equal to zero.
  \end{definition}
\end{frame}

\begin{frame}
  \vs
  \question{} Which of the following are rational functions?
  \vs
  \begin{enumerate}
  \item $f(x) = 0$ \vsone
  \item $\ds f(x) = \frac{3x+1}{x^2-4x+5}$ \vsone
  \item $f(x)=e^x$ \vsone
  \item $\ds f(x)=\frac{\sin(x)}{\cos(x)}$ \vsone
  \item $f(x) = -4x^{-3}+5x^{-1}+7-18x^2$ \vsone
  \item $f(x) = x^{1/2}-x +8$ \vsone
  \item $\ds f(x)=\frac{\sqrt{x}}{x^3-x}$ \vsone
  \end{enumerate}
\end{frame}

\begin{frame}
  \vs
  \textbf{Some possible graphs of rational functions.}
  \begin{image}[3in]
    \begin{tabular}{cc}
      \begin{tikzpicture}
        \begin{axis}[
          xmin=-30,xmax=30,
          ymin=-30,ymax=30,
          domain=-2:2,
          width=2.5in,
          axis lines =middle, xlabel=$x$, ylabel=$y$,
          every axis y label/.style={at=(current axis.above origin),anchor=south},
          every axis x label/.style={at=(current axis.right of origin),anchor=west},
          ]
          \addplot [very thick, penColor, smooth, samples=100, domain=-30:-2.2] {(x^2-3*x+2)/(x+2)};
          \addplot [very thick, penColor, smooth, samples=100, domain=-1.8:30] {(x^2-3*x+2)/(x+2)};

          \node at (axis cs:10,15) [penColor,anchor=west] {$A$};
        \end{axis}
      \end{tikzpicture}
      &
        \begin{tikzpicture}
          \begin{axis}[
            xmin=-2,xmax=4,
            ymin=-3,ymax=3,
            width=2.5in,
            axis lines =middle, xlabel=$x$, ylabel=$y$,
            every axis y label/.style={at=(current axis.above origin),anchor=south},
            every axis x label/.style={at=(current axis.right of origin),anchor=west},
            ]
            \addplot [very thick, penColor2, domain=-2:.9] {1/(x-1)};
            \addplot [very thick, penColor2, domain=1.1:4] {1/(x-1)};
            \addplot[color=penColor2,fill=background,only marks,mark=*] coordinates{(2,1)};  %% open hole
            \node at (axis cs:2.5,1.3) [penColor2] {$B$};
          \end{axis}
        \end{tikzpicture}
      \\
      \begin{tikzpicture}
        \begin{axis}[
          xmin=-1,xmax=5,
          ymin=-30,ymax=30,
          width=2.5in,
          axis lines =middle, xlabel=$x$, ylabel=$y$,
          every axis y label/.style={at=(current axis.above origin),anchor=south},
          every axis x label/.style={at=(current axis.right of origin),anchor=west},
          ]
          \addplot [very thick, penColor3, smooth, samples=100, domain=-1:.95] {(x+2)/(x^2-3*x+2)};
          \addplot [very thick, penColor3, smooth, samples=100, domain=1.1:1.9]  {(x+2)/(x^2-3*x+2)};
          \addplot [very thick, penColor3, smooth, samples=100, domain=2.1:5]  {(x+2)/(x^2-3*x+2)};
          \node at (axis cs:3,7) [penColor3] {$C$};
        \end{axis}
      \end{tikzpicture}
      &
        \begin{tikzpicture}
          \begin{axis}[
            xmin=-2,xmax=4,
            ymin=-3,ymax=3,
            domain=-2:4,
            width=2.5in,
            axis lines =middle, xlabel=$x$, ylabel=$y$,
            every axis y label/.style={at=(current axis.above origin),anchor=south},
            every axis x label/.style={at=(current axis.right of origin),anchor=west},
            ]
            \addplot [very thick, penColor4] {x-1};
            \addplot[color=penColor4,fill=background,only marks,mark=*] coordinates{(2,1)};  %% open hole
            \node at (axis cs:3,1.5) [penColor4] {$D$};
          \end{axis}
        \end{tikzpicture}
    \end{tabular}
  \end{image}
\end{frame}

\begin{frame}
  \frametitle{Trigonometric functions}

  A \dfn{trigonometric function} is a function that relates a measure
  of an angle of a right triangle to a ratio of the triangle's sides.

  \vfill
  \begin{image}[2in]
    \begin{tikzpicture}
      \coordinate (C) at (0,2);
      \coordinate (D) at (4,2);
      \coordinate (E) at (4,4);
      \tkzMarkRightAngle(C,D,E)
      \tkzMarkAngle(D,C,E)
      \draw[decoration={brace,mirror,raise=.2cm},decorate,thin] (0,2)--(4,2);
      \draw[decoration={brace,mirror,raise=.2cm},decorate,thin] (4,2)--(4,4);
      \draw[decoration={brace,raise=.2cm},decorate,thin] (0,2)--(4,4);
      \draw[very thick] (D)--(E)--(C)--cycle;
      \node at (2,2-.5) {adjacent};
      \node[rotate=-90] at (4+.5,3) {opposite};
      \node[rotate=26.5] at (2-.2,3+.4) {hypotenuse};
      \node at (1.2,2.3) {$\theta$};
    \end{tikzpicture}
  \end{image}
  \vfill
\end{frame}

\begin{frame}
  \vs
  \begin{definition}
    The trigonometric functions are:
    \[
      \begin{aligned}
        \cos(\theta) &= \frac{\text{adj}}{\text{hyp}}\\
        \sec(\theta) &= \frac{1}{\cos(\theta)}
      \end{aligned}
      \qquad
      \begin{aligned}
        \sin(\theta) &= \frac{\text{opp}}{\text{hyp}}\\
        \csc(\theta) &= \frac{1}{\sin(\theta)}
      \end{aligned}
      \qquad
      \begin{aligned}
        \tan(\theta) &= \frac{\sin(\theta)}{\cos(\theta)} \\
        \cot(\theta) &= \frac{\cos(\theta)}{\sin(\theta)}
      \end{aligned}
    \]
    where the domain of sine and cosine is all real numbers, and the
    other are defined precisely when their
    denominators are nonzero.
  \end{definition}
\end{frame}

\begin{frame}
  \vs
  \textbf{The unit circle and trig functions}
  \begin{image}[2.2in]
    \begin{tikzpicture}
      \begin{axis}[
        xmin=-1.1,xmax=1.1,ymin=-1.1,ymax=1.1,
        axis lines=center,
        width=4in,
        xtick={-1,1},
        ytick={-1,1},
        clip=false,
        unit vector ratio*=1 1 1,
        xlabel=$x$, ylabel=$y$,
        every axis y label/.style={at=(current axis.above origin),anchor=south},
        every axis x label/.style={at=(current axis.right of origin),anchor=west},
        ]
        \addplot [dashed, smooth, domain=(0:360)] ({cos(x)},{sin(x)}); %% unit circle

        \addplot [textColor] plot coordinates {(0,0) (.766,.643)}; %% 40 degrees

        \addplot [ultra thick,penColor] plot coordinates {(.766,0) (.766,.643)}; %% 40 degrees
        \addplot [ultra thick,penColor2] plot coordinates {(0,0) (.766,0)}; %% 40 degrees

        % \addplot [ultra thick,penColor3] plot coordinates {(1,0) (1,.839)}; %% 40 degrees

        \addplot [textColor,smooth, domain=(0:40)] ({.15*cos(x)},{.15*sin(x)});
        % \addplot [very thick,penColor] plot coordinates {(0,0) (.766,.643)}; %% sector
        % \addplot [very thick,penColor] plot coordinates {(0,0) (1,0)}; %% sector
        % \addplot [very thick, penColor, smooth, domain=(0:40)] ({cos(x)},{sin(x)}); %% sector
        \node at (axis cs:.15,.07) [anchor=west] {$\theta$};
        \node[penColor, rotate=-90] at (axis cs:.84,.322) {$\sin(\theta)$};
        \node[penColor2] at (axis cs:.383,0) [anchor=north] {$\cos(\theta)$};
        % \node[penColor3, rotate=-90] at (axis cs:1.06,.322) {$\tan(\theta)$};
      \end{axis}
    \end{tikzpicture}
  \end{image}
\end{frame}


\begin{frame}
  \vs
  \textbf{Cosine and its inverse.}
  \begin{image}
    \begin{tikzpicture}
      \begin{axis}[
        xmin=-6.75,xmax=6.75,ymin=-1.5,ymax=1.5,
        axis lines=center,
        xtick={-6.28, -4.71, -3.14, -1.57, 0, 1.57, 3.142, 4.71, 6.28},
        xticklabels={$-2\pi$,$-3\pi/2$,$-\pi$, $-\pi/2$, $0$, $\pi/2$, $\pi$, $3\pi/2$, $2\pi$},
        ytick={-1,1},
        % ticks=none,
        width=6in,
        height=3in,
        unit vector ratio*=1 1.5 1,
        xlabel=$\theta$, ylabel=$x$,
        every axis y label/.style={at=(current axis.above origin),anchor=south},
        every axis x label/.style={at=(current axis.right of origin),anchor=west},
        ]
        \addplot [very thick, penColor2!20!background, samples=100,smooth, domain=(-6.75:0)] {cos(deg(x))};
        \addplot [very thick, penColor2!20!background, samples=100,smooth, domain=(3.14:6.75)] {cos(deg(x))};
        \addplot [very thick, penColor2, samples=100,smooth, domain=(0:3.14)] {cos(deg(x))};

        \addplot[color=penColor2,fill=penColor2,only marks,mark=*] coordinates{(0,1)};  %% closed hole
        \addplot[color=penColor2,fill=penColor2,only marks,mark=*] coordinates{(pi,-1)};  %% closed hole
        \node at (axis cs:-1.57,.75) [penColor2] {$\cos(\theta)$};
      \end{axis}
    \end{tikzpicture}
    %% \caption{The function $\cos(\theta)$ takes on all values between $-1$
    %% and $1$ exactly once on the interval $[0,\pi]$. If we restrict
    %% $\cos(\theta)$ to this interval, then this restricted function has
    %% an inverse.}
    %% \label{figure:cos-restricted}
    %% \end{figure*}
  \end{image}

  \begin{image}[2in]\index{arccosine}
    \begin{tikzpicture}
      \begin{axis}[
        xmin=-1.5,xmax=1.5,ymin=-.25,ymax=3.25,
        axis lines=center,
        ytick={0, 1.57,3.14},
        yticklabels={$0$, $\pi/2$,$\pi$},
        xtick={-1,1},
        unit vector ratio*=1.5 1 1,
        xlabel=$x$, ylabel=$\theta$,
        every axis y label/.style={at=(current axis.above origin),anchor=south},
        every axis x label/.style={at=(current axis.right of origin),anchor=west},
        ]
        \addplot [very thick, penColor5, samples=100,smooth, domain=(-1:1)] {acos(x)*pi/180};
        \addplot[color=penColor5,fill=penColor5,only marks,mark=*] coordinates{(1,0)};  %% closed hole
        \addplot[color=penColor5,fill=penColor5,only marks,mark=*] coordinates{(-1,pi)};  %% closed hole
      \end{axis}
      \node at (3,-1) [text width=2.75in] {Here we see a plot of $\arccos(x)$, the inverse function of
        $\cos(\theta)$ when the domain is restricted to the interval $[0,\pi]$.};
    \end{tikzpicture}
  \end{image}
\end{frame}

\begin{frame}
  \vs
  \textbf{Sine and its inverse.}
  \begin{image}
    \begin{tikzpicture}
      \begin{axis}[
        xmin=-6.75,xmax=6.75,ymin=-1.5,ymax=1.5,
        axis lines=center,
        xtick={-6.28, -4.71, -3.14, -1.57, 0, 1.57, 3.142, 4.71, 6.28},
        xticklabels={$-2\pi$,$-3\pi/2$,$-\pi$, $-\pi/2$, $0$, $\pi/2$, $\pi$, $3\pi/2$, $2\pi$},
        ytick={-1,1},
        % ticks=none,
        width=6in,
        height=3in,
        unit vector ratio*=1 1.5 1,
        xlabel=$\theta$, ylabel=$x$,
        every axis y label/.style={at=(current axis.above origin),anchor=south},
        every axis x label/.style={at=(current axis.right of origin),anchor=west},
        ]
        \addplot [very thick, penColor!20!background, samples=100,smooth, domain=(-6.75:-1.57)] {sin(deg(x))};
        \addplot [very thick, penColor!20!background, samples=100,smooth, domain=(1.57:6.75)] {sin(deg(x))};
        \addplot [very thick, penColor, samples=100,smooth, domain=(-1.57:1.57)] {sin(deg(x))};

        \addplot[color=penColor,fill=penColor,only marks,mark=*] coordinates{(-1.57,-1)};  %% closed hole
        \addplot[color=penColor,fill=penColor,only marks,mark=*] coordinates{(1.57,1)};  %% closed hole
        \node at (axis cs:3.14,.75) [penColor] {$\sin(\theta)$};
      \end{axis}
    \end{tikzpicture}
    %% \caption{The function $\sin(\theta)$ takes on all values between $-1$
    %% and $1$ exactly once on the interval $[-\pi/2,\pi/2]$. If we
    %% restrict $\sin(\theta)$ to this interval, then this restricted
    %% function has an inverse.}
    %% \label{figure:sin-restricted}
    %% \end{figure*}
  \end{image}

  \begin{image}[2in]\index{arcsine}
    \begin{tikzpicture}
      \begin{axis}[
        xmin=-1.5,xmax=1.5,ymin=-1.75,ymax=1.75,
        axis lines=center,
        ytick={-1.57, 0, 1.57},
        yticklabels={$-\pi/2$, $0$, $\pi/2$},
        xtick={-1,1},
        unit vector ratio*=1.5 1 1,
        xlabel=$x$, ylabel=$\theta$,
        every axis y label/.style={at=(current axis.above origin),anchor=south},
        every axis x label/.style={at=(current axis.right of origin),anchor=west},
        ]
        \addplot [very thick, penColor4, samples=100,smooth, domain=(-1:1)] {asin(x)*pi/180};

        \addplot[color=penColor4,fill=penColor4,only marks,mark=*] coordinates{(-1,-pi/2)};  %% closed hole
        \addplot[color=penColor4,fill=penColor4,only marks,mark=*] coordinates{(1,pi/2)};  %% closed hole
      \end{axis}
      \node at (3,-1) [text width=2.75in] {Here we see a plot of $\arcsin(x)$, the inverse function of
        $\sin(\theta)$ when the domain is restricted to the interval $[-\pi/2,\pi/2]$.};
    \end{tikzpicture}
  \end{image}
\end{frame}

\begin{frame}
  \vs
  \textbf{Tangent and its inverse.}
  \begin{image}
    \begin{tikzpicture}
      \begin{axis}[
        xmin=-6.75,xmax=6.75,ymin=-3,ymax=3,
        axis lines=center,
        width=6in,
        height=3in,
        xtick={-6.28, -4.71, -3.14, -1.57, 0, 1.57, 3.142, 4.71, 6.28},
        xticklabels={$-2\pi$,$-3\pi/2$,$-\pi$, $-\pi/2$, $0$, $\pi/2$, $\pi$, $3\pi/2$, $2\pi$},
        unit vector ratio*=1 1 1,
        xlabel=$\theta$, ylabel=$x$,
        every axis y label/.style={at=(current axis.above origin),anchor=south},
        every axis x label/.style={at=(current axis.right of origin),anchor=west},
        ]
        \addplot [very thick, penColor3, samples=100,smooth, domain=(-1.55:1.55)] {tan(deg(x))};
        \addplot [very thick, penColor3!30!background, samples=100,smooth, domain=(-4.69:-1.59)] {tan(deg(x))};
        \addplot [very thick, penColor3!30!background, samples=100,smooth, domain=(-6.75:-4.73)] {tan(deg(x))};
        \addplot [very thick, penColor3!30!background, samples=100,smooth, domain=(1.59:4.69)] {tan(deg(x))};
        \addplot [very thick, penColor3!30!background, samples=100,smooth, domain=(4.73:6.75)] {tan(deg(x))};

        \addplot [textColor,dashed] plot coordinates {(-4.71,-3) (-4.71,3)};
        \addplot [textColor,dashed] plot coordinates {(-1.57,-3) (-1.57,3)};
        \addplot [textColor,dashed] plot coordinates {(1.57,-3) (1.57,3)};
        \addplot [textColor,dashed] plot coordinates {(4.71,-3) (4.71,3)};

        \node at (axis cs:.4,1.25) [penColor3] {$\tan(\theta)$};
      \end{axis}
    \end{tikzpicture}
    %% \caption{The function $\tan(\theta)$ takes on all values in $\R$
    %% exactly once on the open interval $(-\pi/2,\pi/2)$. If we restrict
    %% $\tan(\theta)$ to this interval, then this restricted function has
    %% an inverse.}
    %% \end{figure*}
    %% \end{fullwidth}
  \end{image}

  \begin{image}
    \begin{tikzpicture}
      \begin{axis}[
        xmin=-6,xmax=6,ymin=-2,ymax=2,
        axis lines=center,
        ytick={0, -1.57,1.57},
        width=6in,
        height=3in,
        yticklabels={$0$, $-\pi/2$,$\pi/2$},
        xtick={0},
        unit vector ratio*=1 1 1,
        xlabel=$x$, ylabel=$\theta$,
        every axis y label/.style={at=(current axis.above origin),anchor=south},
        every axis x label/.style={at=(current axis.right of origin),anchor=west},
        ]
        \addplot [very thick, penColor3!20!penColor2, samples=100,smooth, domain=(-6:6)] {atan(x)*pi/180};
        \addplot [textColor,dashed] plot coordinates {(-6,-1.57) (6,-1.57)};
        \addplot [textColor,dashed] plot coordinates {(-6,1.57) (6,1.57)};
      \end{axis}
      \node at (7,-1) [text width=5in] {Here we see a plot of $\arctan(x)$, the inverse function of
        $\tan(\theta)$ when the domain is restricted to the interval $(-\pi/2,\pi/2)$.};
    \end{tikzpicture}
    %% \caption{Here we see a plot of $\arctan(y)$, the inverse function of
    %% $\tan(\theta)$ when it is restricted to the interval $(-\pi/2,\pi/2)$.}
    %% \end{figure*}
    %% \index{arctangent}
    %% \end{fullwidth}
  \end{image}
\end{frame}

\begin{frame}
  \vs
  \textbf{Pythagorean theorem and identities.}
  \begin{image}[0.5\textwidth]
    \begin{tikzpicture}
      \coordinate (C) at (0,2);
      \coordinate (D) at (4,2);
      \coordinate (E) at (4,4);
      \tkzMarkRightAngle(C,D,E)
      \tkzMarkAngle(D,C,E)
      \draw[decoration={brace,mirror,raise=.2cm},decorate,thin] (0,2)--(4,2);
      \draw[decoration={brace,mirror,raise=.2cm},decorate,thin] (4,2)--(4,4);
      \draw[decoration={brace,raise=.2cm},decorate,thin] (0,2)--(4,4);
      \draw[very thick] (D)--(E)--(C)--cycle;
      \node at (2,2-.5) {$b$};
      \node[] at (4+.5,3) {$a$};
      \node at (2-.2,3+.4) {$c$};
      \node at (1.2,2.3) {$\theta$};
    \end{tikzpicture}
  \end{image}
  \hfill
  \begin{minipage}[t]{0.45\textwidth}
    Pythagorean theorem:
    \begin{itemize}
    \item $a^2 + b^2 = c^2$
    \end{itemize}
  \end{minipage}
  \begin{minipage}[t]{0.45\textwidth}
    Pythagorean identities:
    \begin{itemize}
    \item $\cos^2\theta+\sin^2\theta = 1$
    \item $1 + \tan^2\theta = \sec^2\theta$
    \item $\cot^2\theta + 1 = \csc^2\theta$
    \end{itemize}
  \end{minipage}
  \hfill
\end{frame}


\begin{frame}
  \frametitle{Exponential and logarithmic functions}
  \begin{definition}
    An \dfn{exponential function} is a function of the form
    \[
      f(x) = b^x
    \]
    where  $b\ne 1$ is a positive real number. The domain of an
    exponential function is $(-\infty,\infty)$. (Special: $f(x) = e^x$.)
  \end{definition}

  \begin{image}[0.4\textwidth]
    \begin{tikzpicture}
      \begin{axis}[
        domain=-2:2,
        xmin=-2, xmax=2,
        ymin=-.5, ymax=4,
        axis lines =middle, xlabel=$x$, ylabel=$y$,
        every axis y label/.style={at=(current axis.above origin),anchor=south},
        every axis x label/.style={at=(current axis.right of origin),anchor=west},
        ]
	\addplot [very thick, penColor, smooth] {2^x};
        \addplot [very thick, penColor2, smooth] {2^(-x)};
        \node at (axis cs:-2.1, 2 ) [penColor2,anchor=west] {$0 < b < 1$};
        \node at (axis cs:1.2, 2 ) [penColor,anchor=west] {$b > 1$};
      \end{axis}
    \end{tikzpicture}
  \end{image}
\end{frame}

\begin{frame}
  \vs
  \begin{definition}
    A \dfn{logarithmic function} is a function defined as follows
    \[
      \log_b(x) = y \qquad\text{means that}\qquad b^y = x
    \]
    where  $b\ne 1$ is a positive real number. The domain of a
    logarithmic function is $(0,\infty)$. (Special: $f(x) = \ln (x)$.)
  \end{definition}
  \begin{image}[0.45\textwidth]
    \begin{tikzpicture}
      \begin{axis}[
        domain=0.05:4,
        xmin=-.5, xmax=4,
        ymin=-2, ymax=2,
        axis lines =middle, xlabel=$x$, ylabel=$y$,
        every axis y label/.style={at=(current axis.above origin),anchor=south},
        every axis x label/.style={at=(current axis.right of origin),anchor=west},
        ]
        \addplot [very thick, penColor2, smooth, samples=100]
        {ln(x)/ln(1/2))}; % 0 < b < 1
        \addplot [very thick, penColor, smooth] {ln(x)/ln(2)}; % b > 1

        \node at (axis cs:.5, 1.3 ) [penColor2,anchor=west] {$0 < b < 1$};
        \node at (axis cs:.5, -1.3 ) [penColor,anchor=west] {$b > 1$};
      \end{axis}
    \end{tikzpicture}
  \end{image}
\end{frame}

\begin{frame}
  \vs
  \begin{block}{Properties of exponents}
    Let $b$ be a positive real number with $b\ne 1$.
    \begin{itemize}
    \item $\ds b^m\cdot b^n = b^{m+n}$
    \item $\ds b^{-1} = \frac{1}{b}$
    \item $\ds \left(b^m\right)^n = b^{mn}$
    \end{itemize}
  \end{block}
  \vfill
  \begin{block}{Properties of logarithms}
    Let $b$ be a positive real number with $b\ne 1$.
    \begin{itemize}
    \item $\ds \log_b(m\cdot n) = \log_b(m) + \log_b(n)$
    \item $\ds \log_b(m^n) = n\cdot \log_b(m)$
    \item $\ds \log_b\left(1/m\right) = \log_b(m^{-1}) = -\log_b(m)$
    \item $\ds \log_a(m) = \frac{\log_b(m)}{\log_b(a)}$
    \end{itemize}
  \end{block}
  \vfill
\end{frame}

\end{document}


%%% Local Variables:
%%% mode: latex
%%% TeX-master: t
%%% End:
