% \documentclass[10pt,t,presentation,ignorenonframetext,aspectratio=169]{beamer}
\documentclass[10pt,t,handout,ignorenonframetext,aspectratio=169]{beamer}
\usepackage[default]{lato}
\usepackage{tk_beamer1}
\input{tk_packages}
\input{tk_macros}
\input{tk_environ}
\input{tk_ximera}
\usepackage{wasysym}            % for smiley
\newcommand{\zoz}{$\mathbf{ \frac{0}{0} }$}

% some inverse trigs
\DeclareMathOperator{\arcsec}{arcsec}
\DeclareMathOperator{\arccot}{arccot}
\DeclareMathOperator{\arccsc}{arccsc}

%%%% META DATA
\newcommand{\semester}{Autumn 2021}
\newcommand{\course}{Math 1151}
\newcommand{\lecTitle}{Lecture 21: Maximums and Minimums (MAM)}

%%%% TITLE PAGE
\title[\course]{\lecTitle}
\institute[Ohio State]
{
  \medskip
}
\date[\week]{\semester}
\author{Tae Eun Kim, Ph.D.}

\begin{document}
\begin{frame}
  \titlepage
\end{frame}


\begin{frame}
  \frametitle{Local Extrema and Critical Points}
  % Local \emph{extrema} of a function are points on its graph that are either
  % \emph{peaks} or \emph{troughs}. Precise definition reads as follows:

  \begin{defn}
    \begin{enumerate}
    \item A function $f$ has a \dfn{local maximum} at $a$, if $f(a)\ge
      f(x)$ for every $x$ in some {\it open interval\/} containing $a$.
    \item A function $f$ has a \dfn{local minimum} at $a$, if $f(a)\le
      f(x)$ for every $x$  in some {\it open interval\/} containing $a$.
    \end{enumerate}
    A \dfn{local extremum} is either a local maximum or a local minimum.
  \end{defn}

  % \textbf{Note} that the definition requires \(f(a)\) to be greater/less than \(f(x)\)
  % on an \emph{open interval} containing \(a\). This, in particular, rules out
  % boundary points or end points from local extrema.
\end{frame}

\begin{frame}
  \frametitle{Connection to Derivatives}
  When the function under consideration has a ``nice'' graph and has a
  peak or a trough, the tangent line at this local extremum must be
  horizontal.

  \begin{thm}[Fermat's Theorem]
    If $f$ has a local extremum at $a$ and $f$ is differentiable
    at $a$, then $f'(a)=0$.
  \end{thm}

  % According to the theorem, the condition \(f'(a)=0\) does not guarantee
  % that \(f\) must have a local extremum at \(a\). Yet, the condition is
  % necessary for the existence of a local minimum or
  % maximum. Furthermore, we need not neglect the situation in which
  % \(f'(a)\) is not defined but \(f\) still attains a local extreme value at
  % \(a\).
\end{frame}

\begin{frame}
  \frametitle{Example: horizontal tangent line}
  Consider the plots of $f(x) = x^3-4.5x^2+6x$ and $f'(x) = 3x^2-9x+6$.

  \begin{image}[2.5in]
    \begin{tikzpicture}
      \begin{axis}[
        domain=-3:3,
        ymax=3,
        ymin=-1.5,
        % samples=100,
        axis lines =middle, xlabel=$x$, ylabel=$y$,
        every axis y label/.style={at=(current axis.above origin),anchor=south},
        every axis x label/.style={at=(current axis.right of origin),anchor=west}
        ]
        \addplot [dashed, textColor, smooth] plot coordinates {(1,0) (1,5/2)}; %% {.451};
        \addplot [dashed, textColor, smooth] plot coordinates {(2,0) (2,2)}; %% axis{2.215};
        \addplot [very thick, penColor2, smooth] {3*x^2-9*x+6};
        \addplot [very thick, penColor, smooth] {x^3-(9/2)*x^2+6*x};
        \node at (axis cs:2.2,5/2) [anchor=west] {\color{penColor}$f$};
        \node at (axis cs:0.18,5/2) [anchor=west] {\color{penColor2}$f'$};
        \addplot[color=penColor2,fill=penColor2,only marks,mark=*] coordinates{(1,0)};  %% closed hole
        \addplot[color=penColor2,fill=penColor2,only marks,mark=*] coordinates{(2,0)};  %% closed hole
        \addplot[color=penColor,fill=penColor,only marks,mark=*] coordinates{(1,2.5)};  %% closed hole
        \addplot[color=penColor,fill=penColor,only marks,mark=*] coordinates{(2,2)};  %% closed hole
      \end{axis}
    \end{tikzpicture}
  \end{image}
\end{frame}

\begin{frame}
  \frametitle{Connection to Derivatives (cont'd)}
  \textbf{Question.} Is it still possible for a function to have a peak or a trough without having a horizontal tangent line there? If so, draw a graph.
\end{frame}

\begin{frame}
  \frametitle{Example: undefined derivative}
  Consider the plots of $f(x) = x^{2/3}$ and $\ds f'(x) = \frac{2}{3x^{1/3}}$:
  \begin{image}[2.5in]
    \begin{tikzpicture}
      \begin{axis}[
        domain=-3:3,
        ymax=2,
        ymin=-2,
        axis lines =middle, xlabel=$x$, ylabel=$y$,
        every axis y label/.style={at=(current axis.above origin),anchor=south},
        every axis x label/.style={at=(current axis.right of origin),anchor=west}
        ]
        \addplot [very thick, penColor2, samples=100, smooth,domain=(-3:-.01)] {-(2/3)*abs(x)^(-1/3)};
        \addplot [very thick, penColor2, samples=100, smooth,domain=(.01:3)] {(2/3)*abs(x)^(-1/3)};
        \addplot [very thick, penColor, smooth,domain=(-3:-0.001)] {(abs(x))^(2/3)};
        \addplot [very thick, penColor, smooth,domain=(0.0015:3)] {x^(2/3)};
        \node at (axis cs:-2,1.7) [anchor=west] {\color{penColor}$f$};
        \node at (axis cs:2,.7) [anchor=west] {\color{penColor2}$f'$};
      \end{axis}
    \end{tikzpicture}
    %% \caption{A plot of $f(x) = x^{2/3}$ and $f'(x) = \frac{2}{3x^{1/3}}$.}
    %% \label{figure:x^{2/3}}
  \end{image}
\end{frame}

\begin{frame}
  \frametitle{Critical Points}
  The following definition captures the two scenarios previously presented:

  \begin{defn}[Critical points]
    Assume that a function \(f\) is defined on an open
    interval \(I\) that contains a point \(a\). The function \(f\) has a
    \dfn{critical point} at \(a\) if
    \[
      f'(a) = 0 \qquad\text{or}\qquad \text{$f'(a)$ does not exist.}
    \]
  \end{defn}
\end{frame}

\begin{frame}
  \vs
  \textbf{Question.} Find all critical points of $\ds f(x) = e^{\frac{1}{3}x^3 - 4x + 5}$.
\end{frame}

\begin{frame}
  \vs
  \textbf{Question.} Find all critical points of $\ds g(x) = \abs{x-5}$.
\end{frame}


\begin{frame}
  \frametitle{\textit{Necessary but not sufficient}}
  \begin{exampleblock}{Scenario 1: $f'(a) = 0$ with no local extremum}
    Consider the plots of $f(x) = x^3$ and $f'(x) = 3x^2$.
    \begin{image}[0.4\linewidth]
      \begin{tikzpicture}
        \begin{axis}[
          domain=-3:3,
          ymax=3,
          ymin=-3,
          axis lines =middle, xlabel=$x$, ylabel=$y$,
          every axis y label/.style={at=(current axis.above origin),anchor=south},
          every axis x label/.style={at=(current axis.right of origin),anchor=west}
          ]
          \addplot [very thick, penColor2, smooth] {3*x^2};
          \addplot [very thick, penColor, smooth] {x^3};
          \node at (axis cs:1,.9) [anchor=west] {\color{penColor}$f$};
          \node at (axis cs:-.5,1) [anchor=west] {\color{penColor2}$f'$};
        \end{axis}
      \end{tikzpicture}
    \end{image}
    $x=0$ is a critical point since $f'(0) = 0$, yet $f$ has neither a
    local minimum or a local maximum at $0$.
  \end{exampleblock}
\end{frame}


\begin{frame}
  \frametitle{\textit{Necessary but not sufficient} (cont'd)}
  \begin{exampleblock}{$f'(a)$ DNE with no local extremum}
    Consider the plots of $f(x) = \sqrt[3]{x}$ and $f'(x) = \frac{1}{3} x^{-2/3}$.

    \begin{image}[0.4\linewidth]
      \begin{tikzpicture}
        \begin{axis}[
          domain=-3:3,
          ymax=2,
          ymin=-2,
          axis lines =middle, xlabel=$x$, ylabel=$y$,
          every axis y label/.style={at=(current axis.above origin),anchor=south},
          every axis x label/.style={at=(current axis.right of origin),anchor=west}
          ]
          \addplot [very thick, penColor2, samples=100, smooth,domain=(-3:-.01)] {(2/3)*abs(x)^(-1/3)};
          \addplot [very thick, penColor2, samples=100, smooth,domain=(.01:3)] {(2/3)*abs(x)^(-1/3)};
          \addplot [very thick, penColor, smooth,domain=(-3:-0.001)] {-(abs(x))^(2/3)};
          \addplot [very thick, penColor, smooth,domain=(0.0015:3)] {x^(2/3)};
          \node at (axis cs:2,1.7) [anchor=west] {\color{penColor}$f$};
          \node at (axis cs:2,.7) [anchor=west] {\color{penColor2}$f'$};
        \end{axis}
      \end{tikzpicture}
      %% \caption{A plot of $f(x) = x^{2/3}$ and $f'(x) = \frac{2}{3x^{1/3}}$.}
      %% \label{figure:x^{2/3}}
    \end{image}
    $x=0$ is a critical point of $f$ since $\lim_{x \to 0}f'(x) = \infty$ (vertical tangent), yet $f(0)$ is neither a local minimum nor a local maximum.
  \end{exampleblock}

\end{frame}

\begin{frame}
  \frametitle{The First Derivative Test}
  So the real question is how we classify a critical point. The
  following answer provides a method of finding relative extrema.

  \begin{thm}[First Derivative Test]
    Suppose that $f$ is continuous on an interval that contains a
    critical point $a$ and assume $f$ is differentiable on an interval
    containing $a$, except possibly at $a$.
    \begin{itemize}
    \item If $f'(x)>0$ to the left of $a$ and $f'(x)<0$ to the right of
      $a$, then $f$ has a \textbf{local maximum} at $a$.
    \item If $f'(x)<0$ to the left of $a$ and $f'(x)>0$ to the right of
      $a$, then $f$ has a \textbf{local minimum} at $a$.
    \item If $f'(x)$ has the same sign to the left and right of $a$,
      then $f$ has no local extreme value at $a$.
    \end{itemize}
  \end{thm}
\end{frame}

\begin{frame}
  \vs
  \begin{question}
    Find all local maximum and minimum points for the function
    $f(x)=x^3-x$.
  \end{question}
\end{frame}

\begin{frame}
  \frametitle{Concavity and Inflection Points}
  Recall that second derivatives carry concavity information:
  \begin{thm}[Test for Concavity]
    Suppose that $f''(x)$ exists on an interval.
    \begin{enumerate}
    \item If $f''(x)>0$ on an interval, then $f$ is concave up on that interval.
    \item If $f''(x)<0$ on an interval, then $f$ is concave down on that interval.
    \end{enumerate}
  \end{thm}

  And there are points at which the concavity changes from up to down or down to up.

  \begin{defn}[Inflection point]
    If \(f\) is continuous at \(x=a\) and its concavity changes either from up to down
    or down to up at \(x=a\), then \(f\) has an \dfn{inflection point} at
    \(a\).
  \end{defn}
\end{frame}

\begin{frame}
  \frametitle{Illustration: inflection points}
  \textbf{\small Examples}
  \begin{image}
    \begin{tikzpicture}
      \begin{axis}[
        % height=7cm,
        % width=2in,
        width=6in,
        height=2in,
        % ymax=8,
        % ymin=-1,
        axis lines=none,
        clip=false,
        ]
        \addplot [very thick, penColor, smooth, domain=(0:1)] {(x-1)^2+1};
        \addplot [very thick, penColor, smooth, domain=(1:2)] {-(x-1)^2+1};
        \addplot[color=penColor,fill=penColor,only marks,mark=*] coordinates{(1,1)};
        \node at (axis cs:1,-.5) [text width=2in] {This is an inflection point. The concavity changes from concave up to concave down.};

        \addplot [very thick, penColor, smooth,domain=(4:5)] {-sqrt(abs(1-(x-4)))+1};
        \addplot [very thick, penColor, smooth,domain=(5:6)] {sqrt((x-4)-1)+1};
        \addplot[color=penColor,fill=penColor,only marks,mark=*] coordinates{(5,1)};
        \node at (axis cs:5,-.5) [text width=2in] {This is an inflection point. The concavity changes from concave up to concave down.};
      \end{axis}
    \end{tikzpicture}
  \end{image}


  \vfill
  \textbf{\small Non-examples}
  \begin{image}
    \begin{tikzpicture}
      \begin{axis}[
        % height=7cm,
        % width=2in,
        width=6in,
        height=2in,
        % ymax=8,
        % ymin=-1,
        axis lines=none,
        clip=false,
        ]
        \addplot [very thick, penColor2, smooth, domain=(0:2)] {-(x-1)^2+1};
        \addplot[color=penColor2,fill=penColor2,only marks,mark=*] coordinates{(1,1)};
        \node at (axis cs:1,-.5) [text width=2in] {This is \textbf{not} an inflection point. The curve is concave down on either side of the point.};
        \addplot [very thick, penColor2, smooth,domain=(4:5)] {sqrt(abs(1-(x-4)))};
        \addplot [very thick, penColor2, smooth,domain=(5:6)] {sqrt(x-5)};
        \addplot[color=penColor2,fill=penColor2,only marks,mark=*] coordinates{(5,0)};
        \node at (axis cs:5,-.5) [text width=2in] {This is \textbf{not} an inflection point. The curve is concave down on either side of the point.};
      \end{axis}
    \end{tikzpicture}
  \end{image}

  {\footnotesize \textbf{\red{Warning.}} \blue{Even if \(f''\) vanishes at \(a\), the point \((a, f(a))\) may \textbf{not} be an inflection point.}}
\end{frame}

\begin{frame}
  \frametitle{The Second Derivative Test}
  \begin{thm}[Second Derivative Test]
    Suppose that $f''(x)$ is continuous on an open interval and that
    $f'(a)=0$ for some value of $a$ in that interval.
    \begin{itemize}
    \item If $f''(a) <0$, then $f$ has a local maximum at $a$.
    \item If $f''(a) >0$, then $f$ has a local minimum at $a$.
    \item If $f''(a) =0$, then the test is inconclusive. In this case,
      $f$ may or may not have a local extremum at $x=a$.
    \end{itemize}
  \end{thm}
\end{frame}

\begin{frame}
  \vs
  \begin{question}
    Consider the function
    \[
      f(x) = \frac{x^4}{4}+\frac{x^3}{3}-x^2
    \]
    Using the second derivative test, find the intervals on which $f$ is increasing and decreasing and
    identify the local extrema of $f$.
  \end{question}
\end{frame}

\end{document}
%%% Local Variables:
%%% mode: latex
%%% TeX-master: t
%%% End:
