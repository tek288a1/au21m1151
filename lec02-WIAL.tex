\documentclass[10pt,t,presentation,ignorenonframetext,aspectratio=169]{beamer}
% \documentclass[10pt,t,handout,ignorenonframetext,aspectratio=169]{beamer}
\usepackage[default]{lato}
\usepackage{tk_beamer1}
\input{tk_packages}
\input{tk_macros}
\input{tk_environ}
\input{tk_ximera}
\usepackage{wasysym}            % for smiley

%%%% META DATA
\newcommand{\semester}{Autumn 2021}
\newcommand{\course}{Math 1151}
\newcommand{\lecTitle}{Lecture 2: What Is A Limit (WIAL)}

%%%% TITLE PAGE
\title[\course]{\lecTitle}
\institute[Ohio State]
{
  \medskip
}
\date[\week]{\semester}
\author{Tae Eun Kim, Ph.D.}

\begin{document}
\begin{frame}
  \titlepage
\end{frame}

\section{What Is a Limit? (WIAL)}
\begin{frame}
  \frametitle{What is a limit?}
  \textbf{Basic idea.} Consider the function
  \[
    f(x) = \frac{\sin(x)}{x}.
  \]
  \begin{image}[4in]
    \begin{tikzpicture}
      \begin{axis}[
        xmin=-6.75,xmax=6.75,ymin=-1.5,ymax=1.5,
        axis lines=center,
        xtick={-6.28, -4.71, -3.14, -1.57, 0, 1.57, 3.142, 4.71, 6.28},
        xticklabels={$-2\pi$,$-3\pi/2$,$-\pi$, $-\pi/2$, $0$, $\pi/2$, $\pi$, $3\pi/2$, $2\pi$},
        ytick={-1,1},
        % ticks=none,
        width=6in,
        height=3in,
        unit vector ratio*=1 1 1,
        xlabel=$x$, ylabel=$y$,
        every axis y label/.style={at=(current axis.above origin),anchor=south},
        every axis x label/.style={at=(current axis.right of origin),anchor=west},
        ]
        \addplot [very thick, penColor, samples=100,smooth, domain=(-6.75:6.75)] {sin(deg(x))/x};

        \addplot[color=penColor,fill=white,only marks,mark=*] coordinates{(0,1)};  %% open hole
      \end{axis}
    \end{tikzpicture}
  \end{image}

  \question{}
  \begin{itemize}
  \item Is $f$ defined at $x = 0$?
  \item Where is $f(x)$ approaching as $x$ gets closer to 0?
  \end{itemize}
\end{frame}

\begin{frame}
  \vs
  \begin{definition}
    Intuitively, we say that
    \begin{center}
      the \dfn{limit} of $f(x)$ as $x$ approaches $a$ is $L$,
    \end{center}
    written
    \[
      \lim_{x\to a} f(x) = L,
    \]
    if the value of $f(x)$ can be made as close as one wishes to $L$ for
    all $x$ sufficiently close, but not equal to, $a$.
  \end{definition}
\end{frame}

\begin{frame}
  \vs
  \begin{definition}
    Intuitively,
    \begin{center}
      the \dfn{limit from the right} of $f$ as $x$ approaches $a$ is
      $L$,
    \end{center}
    written
    \[
      \lim_{x\to a^+} f(x) = L,
    \]
    if the value of $f(x)$ can be made as close as one wishes to $L$ for
    all $x>a$ sufficiently close, but not equal to, $a$.\\

    Similarly,
    \begin{center}
      the \dfn{limit from the left} of $f(x)$ as $x$ approaches $a$ is
      $L$,
    \end{center}
    written
    \[
      \lim_{x\to a^-} f(x) = L,
    \]
    if the value of $f(x)$ can be made as close as one wishes to $L$ for
    all $x<a$ sufficiently close, but not equal to, $a$.
  \end{definition}
\end{frame}

\begin{frame}
  \vs
  \begin{theorem}\index{limit}\index{one-sided limit}
    A limit
    \[
      \lim_{x \to a} f(x)
    \]
    exists if and only if
    \begin{itemize}
    \item $\lim_{x \to a^-} f(x)$ exists
    \item $\lim_{x \to a^+} f(x)$ exists
    \item $\lim_{x \to a^-} f(x) = \lim_{x \to a^+} f(x)$
    \end{itemize}
    In this case, $\lim_{x \to a} f(x)$ is equal to the common
    value of the two one sided limits.
  \end{theorem}
\end{frame}

\begin{frame}
  \vs
  \question{} Study limits of the following graph at various points.
  \begin{image}[2.2in]
    \begin{tikzpicture}
      \begin{axis}[
        domain=-4:6, xmin=-4, xmax=6, ymin=-3,ymax=10,
        unit vector ratio*=1 1 1,
        axis lines =middle, xlabel=$x$, ylabel=$y$,
        every axis y label/.style={at=(current axis.above origin),anchor=south},
        every axis x label/.style={at=(current axis.right of origin),anchor=west},
        xtick={-4,...,6}, ytick={-3,...,10},
        xticklabels={-4,,-2,,0,,2,,4,,6}, yticklabels={,-2,,0,,2,,4,,6,,8,,10},
        grid=major,width=4in,
        grid style={dashed, gridColor},
        ]
        \addplot [very thick, penColor, smooth, domain=(-4:-2)] {6};
        \addplot [very thick, penColor, smooth, domain=(-2:0)] {x^2-2};
        \addplot [very thick, penColor, smooth, domain=(0:2)] {(x-1)^3+3*(x-1)+3};
        \addplot [very thick, penColor, smooth, domain=(2:6)] {(x-4)^3+8};
        \addplot[color=penColor,fill=background,only marks,mark=*] coordinates{(-2,6)};  %% open hole
        \addplot[color=penColor,fill=background,only marks,mark=*] coordinates{(-2,2)};  %% open hole
        \addplot[color=penColor,fill=background,only marks,mark=*] coordinates{(0,-2)};  %% open hole
        \addplot[color=penColor,fill=background,only marks,mark=*] coordinates{(0,-1)};  %% open hole
        \addplot[color=penColor,fill=background,only marks,mark=*] coordinates{(2,0)};  %% open hole
        \addplot[color=penColor,fill=penColor,only marks,mark=*] coordinates{(-2,8)};  %% closed hole
        \addplot[color=penColor,fill=penColor,only marks,mark=*] coordinates{(0,-1.5)};  %% closed hole
        \addplot[color=penColor,fill=penColor,only marks,mark=*] coordinates{(2,7)};  %% closed hole
      \end{axis}
    \end{tikzpicture}
  \end{image}
\end{frame}

\begin{frame}
  \frametitle{Continuity}
  \begin{definition}
    A function $f$ is \dfn{continuous at a point} $a$ if
    \[
      \lim_{x\to a}f(x) = f(a).
    \]
  \end{definition}

  \vfill
  We can unpack the single equation above as:
  \begin{enumerate}
  \item $f(a)$ is defined.
  \item $\ds \lim_{x\to a} f(x)$ exists.
  \item $\ds \lim_{x\to a} f(x) = f(a)$.
  \end{enumerate}

  \vfill
  \question{} How can a function be discontinuous at a point?
\end{frame}

\begin{frame}
  \vs
  \question{} Find the discontinuities.

  \begin{image}[2.5in]
    \begin{tikzpicture}
      \begin{axis}[
        domain=0:10,
        ymax=5,
        ymin=0,
        % samples=100,
        axis lines =middle, xlabel=$x$, ylabel=$y$,
        every axis y label/.style={at=(current axis.above origin),anchor=south},
        every axis x label/.style={at=(current axis.right of origin),anchor=west},
        %% ytick={0.5,1,1.5,2},
        %% yticklabels={$0.5$,$1$,$1.5$,$2$},
        %% xtick={0.5,1.0,1.5,2},
        %% xticklabels={$0.5$,$1$,$1.5$,$2$},
        grid = major
        ]
        \addplot [very thick, penColor, smooth, domain=(4:10)] {3 + sin(deg(x*2))/(x-1)};
        \addplot [very thick, penColor, smooth, domain=(0:4)] {1};
        \addplot[color=penColor,fill=background,only marks,mark=*] coordinates{(4,3.30)};  %% open hole
        \addplot[color=penColor,fill=background,only marks,mark=*] coordinates{(2,1)};  %% open hole
        \addplot[color=penColor,fill=background,only marks,mark=*] coordinates{(6,2.893)};  %% open hole
        \addplot[color=penColor,fill=penColor,only marks,mark=*] coordinates{(4,1)};  %% closed hole
        \addplot[color=penColor,fill=penColor,only marks,mark=*] coordinates{(6,2)};  %% closed hole

        \addplot[color=penColor,fill=penColor,only marks,mark=*] coordinates{(0,1)};  %% closed hole
        \addplot[color=penColor,fill=penColor,only marks,mark=*] coordinates{(10,3.1)};  %% closed hole
      \end{axis}
    \end{tikzpicture}
    %% \caption{A plot of a function with discontinuities at $x=4$ and $x=6$.}
    %% \label{plot:discontinuous-function}
  \end{image}
\end{frame}

\begin{frame}
  \vs
  \begin{definition}
    \begin{itemize}
    \item A function $f$ is \dfn{left continuous} at a point $a$ if
      $\lim_{x\to a^-} f(x) = f(a)$.
    \item A function $f$ is \dfn{right continuous} at a point $a$ if
      $\lim_{x\to a^+} f(x) = f(a)$.
    \end{itemize}
  \end{definition}
  \vspace{1cm}
  We can talk about continuity on intervals now.
  \begin{definition}
    A function $f$ is
    \begin{itemize}
    \item \dfn{continuous on an open interval} $(a,b)$ if
      $\lim_{x\to c} f(x) = f(c)$ for all $c$ in $(a,b)$;
    \item \dfn{continuous on a closed interval} $[a,b]$ if $f$ is
      continuous on $(a,b)$, right continuous at $a$, and left
      continuous at $b$.
    \end{itemize}
  \end{definition}
\end{frame}

\begin{frame}
  \vs
  \begin{block}{Continuity of Famous Functions}\index{continuity of famous functions}\label{theorem:continuity}
    The following functions are continuous on their natural domains, for $k$ a real number and $b$ a positive real number:
    \begin{itemize}
    \item \textbf{Constant function}  $f(x) =k$
    \item \textbf{Identity function}  $f(x) = x$
    \item \textbf{Power function}  $f(x)=x^b$
    \item \textbf{Exponential function}  $f(x)=b^x$
    \item \textbf{Logarithmic function}  $f(x)=\log_b(x)$
    \item \textbf{Sine and cosine functions}  $f(x)=\sin(x)$ and $f(x)=\cos(x)$
    \end{itemize}
  \end{block}
\end{frame}

\begin{frame}
  \vs
  \question{} (Revisiting the previous graph) What are the \textit{largest intervals} of continuity?
  \begin{image}[2.5in]
    \begin{tikzpicture}
      \begin{axis}[
        domain=0:10,
        ymax=5,
        ymin=0,
        % samples=100,
        axis lines =middle, xlabel=$x$, ylabel=$y$,
        every axis y label/.style={at=(current axis.above origin),anchor=south},
        every axis x label/.style={at=(current axis.right of origin),anchor=west},
        %% ytick={0.5,1,1.5,2},
        %% yticklabels={$0.5$,$1$,$1.5$,$2$},
        %% xtick={0.5,1.0,1.5,2},
        %% xticklabels={$0.5$,$1$,$1.5$,$2$},
        grid = major
        ]
        \addplot [very thick, penColor, smooth, domain=(4:10)] {3 + sin(deg(x*2))/(x-1)};
        \addplot [very thick, penColor, smooth, domain=(0:4)] {1};
        \addplot[color=penColor,fill=background,only marks,mark=*] coordinates{(4,3.30)};  %% open hole
        \addplot[color=penColor,fill=background,only marks,mark=*] coordinates{(2,1)};  %% open hole
        \addplot[color=penColor,fill=background,only marks,mark=*] coordinates{(6,2.893)};  %% open hole
        \addplot[color=penColor,fill=penColor,only marks,mark=*] coordinates{(4,1)};  %% closed hole
        \addplot[color=penColor,fill=penColor,only marks,mark=*] coordinates{(6,2)};  %% closed hole

        \addplot[color=penColor,fill=penColor,only marks,mark=*] coordinates{(0,1)};  %% closed hole
        \addplot[color=penColor,fill=penColor,only marks,mark=*] coordinates{(10,3.1)};  %% closed hole
      \end{axis}
    \end{tikzpicture}
    %% \caption{A plot of a function with discontinuities at $x=4$ and $x=6$.}
    %% \label{plot:discontinuous-function}
  \end{image}
\end{frame}
\end{document}


%%% Local Variables:
%%% mode: latex
%%% TeX-master: t
%%% End:
