% \documentclass[10pt,t,presentation,ignorenonframetext,aspectratio=169]{beamer}
\documentclass[10pt,t,handout,ignorenonframetext,aspectratio=169]{beamer}
\usepackage[default]{lato}
\usepackage{tk_beamer1}
%% packages

\RequirePackage{lmodern} % math, rm, ss, tt
\RequirePackage[T1]{fontenc}
\RequirePackage[english]{babel}
\RequirePackage{enumerate}
\RequirePackage{etex}
\RequirePackage{color,xcolor,ucs}
\RequirePackage{graphicx}
\RequirePackage{amssymb}
\RequirePackage{amsmath}
\RequirePackage{subfig}
\RequirePackage{amsthm}
\RequirePackage{mathtools}
\RequirePackage{mathabx}        % required for \odiv; put this after mathtools, otherwise, over/underbrace get messed up
\RequirePackage{epsfig}
\RequirePackage{epstopdf}
\RequirePackage{float}
\RequirePackage{booktabs}
\RequirePackage{blkarray}
\RequirePackage{multirow}
\RequirePackage{hyperref}
\RequirePackage{verbatim}
\RequirePackage{lscape}
\RequirePackage[mathscr]{euscript}
\RequirePackage{movie15}
\RequirePackage{mwe,tikz}
\RequirePackage[percent]{overpic}
\RequirePackage{multicol}
\RequirePackage{pgfplots}
\pgfplotsset{compat=1.7}
\RequirePackage{relsize}
\RequirePackage{textcomp}
\RequirePackage{fancyvrb}
\RequirePackage{tcolorbox}
\RequirePackage{bm}


% %%%%%% for matlab listings
% \RequirePackage{courier}
% \RequirePackage{listings}
% % \newcommand{\matlab}{\textsc{MatLab}}
% % \definecolor{mygreen}{RGB}{28,172,0} % color values Red, Green, Blue
% \definecolor{mygreen}{RGB}{0,100,0} % color values Red, Green, Blue
% \definecolor{mylilas}{RGB}{170,55,241}

% \lstset{language=Matlab,
%   % basicstyle=\small\ttfamily, % Use small true type font
%   % breaklines=true,%
%   % morekeywords={matlab2tikz},
%   % keywordstyle=\color{blue},%
%   % morekeywords=[2]{1}, keywordstyle=[2]{\color{black}},
%   % identifierstyle=\color{black},%
%   % stringstyle=\color{mylilas},
%   % commentstyle=\color{mygreen},%
%   % showstringspaces=false,%without this there will be a symbol in the places where there is a space
%   % numbers=left,%
%   % numberstyle={\tiny \color{black}},% size of the numbers
%   % numbersep=7pt, % this defines how far the numbers are from the text
%   % emph=[1]{for,end,break},emphstyle=[1]\color{red}, %some words to emphasise
%   % % emph=[2]{word1,word2}, emphstyle=[2]{style},
%   % frame=single,
%   basicstyle=\small\ttfamily,%
%   breaklines=true,%
%   morekeywords={matlab2tikz},%
%   keywordstyle=\color{blue},%
%   morekeywords=[2]{1},%
%   keywordstyle=[2]{\color{black}},%
%   identifierstyle=\color{black},%
%   stringstyle=\color{mylilas},
%   commentstyle=\color{mygreen},%
%   % moredelim=[il][\rmfamily]{//},%
%   morecomment=[n][\color{black}]{(*}{*)},%
%   showstringspaces=false,%
%   numbers=none,%
%   % numbers=left,%
%   % numberstyle={\tiny \color{black}},%
%   % numbersep=7pt,%
%   emph=[1]{for,end,break,if,while},emphstyle=[1]\color{blue}, %some words to emphasise
%   % emph=[2]{word1,word2}, emphstyle=[2]{style},
%   frame=single,%
%   framerule=0.7pt,%
%   mathescape=true,%
%   escapebegin=\color{mygreen},%
%   escapeend=,%
% }
% % ref: https://tex.stackexchange.com/questions/257938/how-to-include-matlab-code-into-latex-in-colour



% % \usepackage{spalign}
% % \usepackage{enumitem}
% % \graphicspath{ {../codes/} }
% % \epstopdfsetup{outdir=../codes/}
% % \lstset{inputpath="../codes"}
% % \setlength{\fboxsep}{1.5pt}
% % \newcommand{\x}{\times}
% % \newcommand{\bigzero}{\makebox(0,0){\text{\LARGE 0}}}
% % \newcommand*{\bord}{\multicolumn{1}{c|}{}}

% %% Algorithm/Pseudocode Environment using `listings'
% % https://tex.stackexchange.com/questions/111116/what-is-the-best-looking-pseudo-code-package
% %
% % \newcounter{nalg}[chapter] % defines algorithm counter for chapter-level
% % \renewcommand{\thenalg}{\thechapter .\arabic{nalg}}

% %defines appearance of the algorithm counter
% \DeclareCaptionLabelFormat{algocaption}{Algorithm \thenalg} % defines a new caption label as Algorithm x.y

% \lstnewenvironment{algorithm}[1][] %defines the algorithm listing environment
% {
%     % \refstepcounter{nalg} %increments algorithm number
%     \captionsetup{labelformat=algocaption,labelsep=colon} %defines the caption setup for: it uses label format as the declared caption label above and makes label and caption text to be separated by a ':'
%     \lstset{ %this is the stype
%         mathescape=true,
%         % %frame=tB,
%         % frame=none,
%         % numbers=left,
%         % numberstyle=\tiny,
%         % commentstyle=,
%         % basicstyle=\small,
%         % stringstyle=\ttfamily,
%         basicstyle=,
%         keywordstyle=\color{black}\bfseries,
%         keywords={for, input, output, return, datatype, function, in, if, else, foreach, while, begin, end}, %add the keywords you want, or load a language as Rubens explains in his comment above.
%         % xleftmargin=.04\textwidth,
%         #1 % this is to add specific settings to an usage of this environment (for instnce, the caption and referable label)
%         }
% }
% {}

%%%% Custom macros
%%%% Macros

%% Greek letters

\newcommand{\al}{\alpha}
% \newcommand{\be}{\beta}
\newcommand{\g}{\gamma}
\newcommand{\de}{\delta}
\newcommand{\e}{\epsilon}
\newcommand{\eps}{\varepsilon}
\newcommand{\ka}{\kappa}
\newcommand{\la}{\lambda}
\newcommand{\sig}{\sigma}
\newcommand{\om}{\omega}
\newcommand{\Om}{\Omega}
\let\oldth\th %% \th is used for "thorn"
\renewcommand{\th}{\theta}

%% Tweak some Greek letters
\newcommand{\bchi}{\mbox{\raisebox{.4ex}{\begin{large}$\chi$\end{large}}}}
\newcommand{\Chi}{\mbox{\Large$\chi$}} % nicer looking Chi
\newcommand{\bzeta}{\boldsymbol{\zeta}} % Riemann zeta function
\newcommand{\bxi}{\boldsymbol{\xi}}
\newcommand{\balpha}{\boldsymbol{\alpha}}

%% Blackboard

\newcommand{\NN}{\mathbb{N}}
\newcommand{\ZZ}{\mathbb{Z}}
\newcommand{\QQ}{\mathbb{Q}}
\newcommand{\RR}{\mathbb{R}}
\newcommand{\CC}{\mathbb{C}}
\newcommand{\FF}{\mathbb{F}}
\newcommand{\TT}{\mathbb{T}}
\newcommand{\DD}{\mathbb{D}}
\newcommand{\HH}{\mathbb{H}}
\newcommand{\UU}{\mathbb{U}}
\newcommand{\1}{\mathbbm{1}}

%% Caligraphic

\newcommand{\cA}{\mathcal{A}}
\newcommand{\cB}{\mathcal{B}}
\newcommand{\cC}{\mathcal{C}}
\newcommand{\cD}{\mathcal{D}}
\newcommand{\cE}{\mathcal{E}}
\newcommand{\cF}{\mathcal{F}}
\newcommand{\cG}{\mathcal{G}}
\newcommand{\cH}{\mathcal{H}}
\newcommand{\cI}{\mathcal{I}}
\newcommand{\cJ}{\mathcal{J}}
\newcommand{\cK}{\mathcal{K}}
\newcommand{\cL}{\mathcal{L}}
\newcommand{\cM}{\mathcal{M}}
\newcommand{\cN}{\mathcal{N}}
\newcommand{\cO}{\mathcal{O}}
\newcommand{\cP}{\mathcal{P}}
\newcommand{\cQ}{\mathcal{Q}}
\newcommand{\cR}{\mathcal{R}}
\newcommand{\cS}{\mathcal{S}}
\newcommand{\cT}{\mathcal{T}}
\newcommand{\cU}{\mathcal{U}}
\newcommand{\cV}{\mathcal{V}}
\newcommand{\cW}{\mathcal{W}}


%% Roman, italic, boldface

\newcommand{\bA}{\mathbf{A}}
\newcommand{\bB}{\mathbf{B}}
\newcommand{\bC}{\mathbf{C}}
\newcommand{\bD}{\mathbf{D}}
\newcommand{\bE}{\mathbf{E}}
\newcommand{\bI}{\mathbf{I}}
\newcommand{\bK}{\mathbf{K}}
\newcommand{\bL}{\mathbf{L}}
\newcommand{\bM}{\mathbf{M}}
\newcommand{\bN}{\mathbf{N}}
\newcommand{\bP}{\mathbf{P}}
\newcommand{\bS}{\mathbf{S}}
\newcommand{\bT}{\mathbf{T}}
\newcommand{\bX}{\mathbf{X}}
\newcommand{\ba}{\mathbf{a}}
\newcommand{\bb}{\mathbf{b}}
\newcommand{\bc}{\mathbf{c}}
\newcommand{\bd}{\mathbf{d}}
\newcommand{\be}{\mathbf{e}}
\newcommand{\bg}{\mathbf{g}}
\newcommand{\bh}{\mathbf{h}}
\newcommand{\br}{\mathbf{r}}
\newcommand{\bx}{\mathbf{x}}
\newcommand{\by}{\mathbf{y}}
\newcommand{\bu}{\mathbf{u}}
\newcommand{\bv}{\mathbf{v}}
\newcommand{\bw}{\mathbf{w}}
\newcommand{\bz}{\mathbf{z}}
\newcommand{\bq}{\mathbf{q}}
\newcommand{\bzero}{\mathbf{0}}


%% Principal value integral
\def\Xint#1{\mathchoice
  {\XXint\displaystyle\textstyle{#1}}%
  {\XXint\textstyle\scriptstyle{#1}}%
  {\XXint\scriptstyle\scriptscriptstyle{#1}}%
  {\XXint\scriptscriptstyle\scriptscriptstyle{#1}}%
  \!\int}
\def\XXint#1#2#3{{\setbox0=\hbox{$#1{#2#3}{\int}$}
    \vcenter{\hbox{$#2#3$}}\kern-.5\wd0}}
\def\ddashint{\Xint=}
\def\pvint{\Xint-}

%% Arc over symbols
% reference: https://tex.stackexchange.com/questions/96680/a-better-notation-to-denote-arcs-for-an-american-high-school-textbook
\makeatletter
\DeclareFontFamily{U}{tipa}{}
\DeclareFontShape{U}{tipa}{m}{n}{<->tipa10}{}
\newcommand{\arc@char}{{\usefont{U}{tipa}{m}{n}\symbol{62}}}%
\newcommand{\arc}[1]{\mathpalette\arc@arc{#1}}
\newcommand{\arc@arc}[2]{%
  \sbox0{$\m@th#1#2$}%
  \vbox{
    \hbox{\resizebox{\wd0}{\height}{\arc@char}}
    \nointerlineskip
    \box0
  }%
}


%% colored boxes
\newsavebox{\astrutbox}
\sbox{\astrutbox}{\rule[-5pt]{0pt}{20pt}}
\newcommand{\astrut}{\usebox{\astrutbox}}
\newcommand{\rls}{\raisebox{2pt}{\tikz{\draw[red,solid,line width=0.9pt](0,0) -- (5mm,0);}}}
\newcommand{\rld}{\raisebox{2pt}{\tikz{\draw[red,dashed,line width=1.0pt](0,0) -- (5mm,0);}}}
\newcommand{\bls}{\raisebox{2pt}{\tikz{\draw[blue,solid,line width=0.9pt](0,0) -- (5mm,0);}}}
\newcommand{\bld}{\raisebox{2pt}{\tikz{\draw[blue,dashed,line width=1.0pt](0,0) -- (5mm,0);}}}
\newcommand{\gls}{\raisebox{2pt}{\tikz{\draw[green,solid,line width=0.9pt](0,0) -- (5mm,0);}}}
\newcommand{\gld}{\raisebox{2pt}{\tikz{\draw[green,dashed,line width=1.0pt](0,0) -- (5mm,0);}}}
\newcommand{\mls}{\raisebox{2pt}{\tikz{\draw[magenta,solid,line width=0.9pt](0,0) -- (5mm,0);}}}
\newcommand{\mld}{\raisebox{2pt}{\tikz{\draw[magenta,dashed,line width=1.0pt](0,0) -- (5mm,0);}}}
\newcommand{\cls}{\raisebox{2pt}{\tikz{\draw[cyan,solid,line width=0.9pt](0,0) -- (5mm,0);}}}
\newcommand{\cld}{\raisebox{2pt}{\tikz{\draw[cyan,dashed,line width=1.0pt](0,0) -- (5mm,0);}}}



%% Delimiters, accents, bars, etc

\newcommand{\abs}[1]{\left|#1\right|}
\newcommand{\ceil}[1]{\left\lceil#1\right\rceil}
\newcommand{\floor}[1]{\left\lfloor#1\right\rfloor}
\newcommand{\conj}[1]{\overline{#1}}
\newcommand{\norm}[1]{\left\|#1\right\|}
\newcommand{\Norm}[2]{\left\|#1\right\|_{#2}}
% Improvement of the above two
% example usage: \norm[2]{f} or \Norm[2]{f}{L^2}
\renewcommand{\norm}[2][0]{%
  \ifcase#1\relax
    \left\Vert #2 \right\Vert\or  % 0
    \lVert #2 \rVert\or           % 1
    \bigl\Vert #2 \bigr\Vert\or   % 2
    \Bigl\Vert #2 \Bigr\Vert\or   % 3
    \biggl\Vert #2 \biggr\Vert\or % 4
    \Biggl\Vert #2 \Biggr\Vert    % 5
  \fi}
\renewcommand{\Norm}[3][0]{%
  \ifcase#1\relax
    \left\Vert #2 \right\Vert_{#3}\or  % 0
    \lVert #2 \rVert_{#3}\or           % 1
    \bigl\Vert #2 \bigr\Vert_{#3}\or   % 2
    \Bigl\Vert #2 \Bigr\Vert_{#3}\or   % 3
    \biggl\Vert #2 \biggr\Vert_{#3}\or % 4
    \Biggl\Vert #2 \Biggr\Vert_{#3}    % 5
  \fi}
\newcommand{\avg}[1]{\langle#1\rangle}
\newcommand{\ds}{\displaystyle}


%% Mathematical operators

\DeclareMathOperator{\re}{Re}
\DeclareMathOperator{\im}{Im}
\DeclareMathOperator{\sgn}{sgn}
\DeclareMathOperator{\erf}{erf}
\DeclareMathOperator{\erfc}{erfc}
\DeclareMathOperator{\ii}{i}
\DeclareMathOperator{\dd}{\,d}
\DeclareMathOperator{\eu}{e}
\DeclareMathOperator{\Sp}{Sp}
\DeclareMathOperator{\acosh}{acosh}
\DeclareMathOperator{\asech}{asech}
\DeclareMathOperator{\atanh}{atanh}
\newcommand{\del}{\partial}
\newcommand{\tri}{\triangle}
\newcommand{\grad}{\nabla}
\newcommand{\dvg}{\nabla\cdot}
\newcommand{\curl}{\nabla\times}


%% colored texts

\newcommand{\red}[1]{{\color{red}{#1}}}
\newcommand{\blue}[1]{{\color{blue}{#1}}}
\newcommand{\green}[1]{{\color{green}{#1}}}


%%

\newcommand{\emptyframe}{\begin{frame}{}\end{frame}}
\newcommand{\question}{\textbf{Question.}}
\newcommand{\sqitem}{\item[$\square$]}


%% 06/12/18 addition

\newcommand{\vs}{\vspace{1em}}
\newcommand{\tp}{^{\rm T}}

%% 06/22/18 addition
% for 3607
\newcommand{\meps}{\fbox{eps}}
\newcommand{\flops}{\textit{flops}}
% \setlength{\fboxsep}{1.5pt}
\newcommand\x{\times}
\newcommand\bigzero{\makebox(0,0){\text{\LARGE 0}}}
\newcommand*{\bord}{\multicolumn{1}{c|}{}}


%% Continued numbering over multiple enumerate environments
% https://tex.stackexchange.com/questions/55000/continuing-enumerate-counters-in-beamer
\newcounter{saveenumi}
\newcommand{\seti}{\setcounter{saveenumi}{\value{enumi}}}
\newcommand{\conti}{\setcounter{enumi}{\value{saveenumi}}}


%% Extension to amsmath matrix environment
% improving matrix constructors
% source: http://texblog.net/latex-archive/maths/amsmath-matrix/
\makeatletter
\renewcommand*\env@matrix[1][*\c@MaxMatrixCols c]{%
  \hskip -\arraycolsep
  \let\@ifnextchar\new@ifnextchar
  \array{#1}}
\makeatother

% In order to make the column lines to look nicer:
\setlength\delimitershortfall{0pt}


%% 11/09/18 addition: vertical spaces
\newcommand{\vsone}{\vspace{\stretch{1}}}
\newcommand{\vstwo}{\vspace{\stretch{2}}}
\newcommand{\vsthree}{\vspace{\stretch{3}}}

%% 02/23/19 addition: wide hat and wide tilde
\newcommand{\wh}[1]{\widehat{#1}}
\newcommand{\wt}[1]{\widetilde{#1}}

%% Defining theorem environment
\theoremstyle{plain}
\newtheorem{prop}{Proposition}
\newtheorem{cor}[prop]{Corollary}
\newtheorem{lem}[prop]{Lemma}
\newtheorem{thm}[prop]{Theorem}
\newtheorem{cons}[prop]{Consequence}
\newtheorem{conv}[prop]{Convention}
\newtheorem{prob}[prop]{Problem}
\newtheorem{form}[prop]{Formulation}
\newtheorem{claim}[prop]{Claim}

\theoremstyle{definition}
\newtheorem{defn}{Definition}
\newtheorem{notn}[defn]{Notation}
\newtheorem{note}[defn]{Note}
\newtheorem{rmk}[defn]{Remark}
\newtheorem{exer}[defn]{Exercise}
\newtheorem{ex}[defn]{Example}


%% Listings Environments
\usepackage{courier}
\usepackage{listings}

% \definecolor{mygreen}{RGB}{28,172,0} % color values Red, Green, Blue
\definecolor{mygreen}{RGB}{0,100,0} % color values Red, Green, Blue
\definecolor{mylilas}{RGB}{170,55,241}

\lstdefinestyle{matlab}{language=Matlab,
% basicstyle=\small\ttfamily, % Use small true type font
% breaklines=true,%
% morekeywords={matlab2tikz},
% keywordstyle=\color{blue},%
% morekeywords=[2]{1}, keywordstyle=[2]{\color{black}},
% identifierstyle=\color{black},%
% stringstyle=\color{mylilas},
% commentstyle=\color{mygreen},%
% showstringspaces=false,%without this there will be a symbol in the places where there is a space
% numbers=left,%
% numberstyle={\tiny \color{black}},% size of the numbers
% numbersep=7pt, % this defines how far the numbers are from the text
% emph=[1]{for,end,break},emphstyle=[1]\color{red}, %some words to emphasise
% % emph=[2]{word1,word2}, emphstyle=[2]{style},
% frame=single,
basicstyle=\small\ttfamily,%
breaklines=true,%
morekeywords={matlab2tikz},%
keywordstyle=\color{blue},%
morekeywords=[2]{1},%
keywordstyle=[2]{\color{black}},%
identifierstyle=\color{black},%
stringstyle=\color{mylilas},
commentstyle=\color{mygreen},%
% moredelim=[il][\rmfamily]{//},%
morecomment=[n][\color{black}]{(*}{*)},%
showstringspaces=false,%
numbers=none,%
% numbers=left,%
% numberstyle={\tiny \color{black}},%
% numbersep=7pt,%
emph=[1]{for,end,break,if,while,mod,ones,randi,sind,cosd,tand},
emphstyle=[1]\color{blue}, %some words to emphasise
% emph=[2]{word1,word2}, emphstyle=[2]{style},
frame=single,%
framerule=0.7pt,%
mathescape=true,%
escapebegin=\color{mygreen},%
escapeend=,%
escapechar=`,
}
% ref: https://tex.stackexchange.com/questions/257938/how-to-include-matlab-code-into-latex-in-colour


%% Algorithm/Pseudocode Environment using `listings'
% https://tex.stackexchange.com/questions/111116/what-is-the-best-looking-pseudo-code-package
%
% \newcounter{nalg}[chapter] % defines algorithm counter for chapter-level
% \renewcommand{\thenalg}{\thechapter .\arabic{nalg}}

%defines appearance of the algorithm counter
\DeclareCaptionLabelFormat{algocaption}{Algorithm \thenalg} % defines a new caption label as Algorithm x.y

\lstnewenvironment{algorithm}[1][] %defines the algorithm listing environment
{
% \refstepcounter{nalg} %increments algorithm number
\captionsetup{labelformat=algocaption,labelsep=colon} %defines the caption setup for: it uses label format as the declared caption label above and makes label and caption text to be separated by a ':'
\lstset{language=,
basicstyle=\small,%
stringstyle=\small\ttfamily,%
breaklines=true,%
keywords={for, input, output, return, datatype, function, in, if, else, foreach, while, begin, end},
keywordstyle=\color{blue}\bfseries\ttfamily,%
numbers=none,%
frame=single,%
framerule=0.7pt,%
mathescape=true,%
escapechar=',
xleftmargin=.04\linewidth,
#1 % this is to add specific settings to an usage of this environment (for instance, the caption and referable label)
}
}
{}

%%%% ximera preamble extract

%% TikZ/PGFplot related

\usepackage{tkz-euclide}
\usepackage{tikz}
\usepackage{tikz-cd}
\usepackage{pgffor} %% required for integral for loops
\usetikzlibrary{arrows}
\tikzset{>=stealth,commutative diagrams/.cd,
  arrow style=tikz,diagrams={>=stealth}} %% cool arrow head
\tikzset{shorten <>/.style={ shorten >=#1, shorten <=#1 } } %% allows shorter vectors

\usetikzlibrary{backgrounds} %% for boxes around graphs
\usetikzlibrary{shapes,positioning}  %% Clouds and stars
\usetikzlibrary{matrix} %% for matrix
\usepgfplotslibrary{polar} %% for polar plots
% \usetkzobj{all}

% Gaussian function
\pgfmathdeclarefunction{gauss}{2}{% gives gaussian
  \pgfmathparse{1/(#2*sqrt(2*pi))*exp(-((x-#1)^2)/(2*#2^2))}%
}



%% colors

\colorlet{textColor}{black}
\colorlet{background}{white}
\colorlet{penColor}{blue!50!black} % Color of a curve in a plot
\colorlet{penColor2}{red!50!black}% Color of a curve in a plot
\colorlet{penColor3}{red!50!blue} % Color of a curve in a plot
\colorlet{penColor4}{green!50!black} % Color of a curve in a plot
\colorlet{penColor5}{orange!80!black} % Color of a curve in a plot
\colorlet{penColor6}{yellow!70!black} % Color of a curve in a plot
\colorlet{fill1}{penColor!20} % Color of fill in a plot
\colorlet{fill2}{penColor2!20} % Color of fill in a plot
\colorlet{fillp}{fill1} % Color of positive area
\colorlet{filln}{penColor2!20} % Color of negative area
\colorlet{fill3}{penColor3!20} % Fill
\colorlet{fill4}{penColor4!20} % Fill
\colorlet{fill5}{penColor5!20} % Fill
\colorlet{gridColor}{gray!50} % Color of grid in a plot
\newcommand{\surfaceColor}{violet}
\newcommand{\surfaceColorTwo}{redyellow}
\newcommand{\sliceColor}{greenyellow}



%% image environment
\usepackage{environ}
\NewEnviron{image}[1][3in]{%
  \begin{center}\resizebox{#1}{!}{\BODY}\end{center}% resize and center
}



%% more packages

\usepackage[makeroom]{cancel} %% for strike outs
\usepackage{multicol}
\usepackage{array}



%% lengths

\setlength{\extrarowheight}{+.1cm}


%% macros
% \newcommand{\dd}[2][]{\frac{\d #1}{\d #2}} % conflict
\newcommand{\pp}[2][]{\frac{\partial #1}{\partial #2}}
\newcommand{\ddx}{\frac{d}{\d x}}
\newcommand{\dfn}{\textbf}
\newcommand{\unit}{\mathop{}\!\mathrm}
\newcommand{\eval}[1]{\bigg[ #1 \bigg]}
\newcommand{\seq}[1]{\left( #1 \right)}
\renewcommand{\d}{\mathop{}\!d}
\renewcommand{\l}{\ell}
% \newcommand{\zeroOverZero}{\ensuremath{\boldsymbol{\tfrac{0}{0}}}}
% \newcommand{\inftyOverInfty}{\ensuremath{\boldsymbol{\tfrac{\infty}{\infty}}}}
% \newcommand{\zeroOverInfty}{\ensuremath{\boldsymbol{\tfrac{0}{\infty}}}}
% \newcommand{\zeroTimesInfty}{\ensuremath{\small\boldsymbol{0\cdot \infty}}}
% \newcommand{\inftyMinusInfty}{\ensuremath{\small\boldsymbol{\infty - \infty}}}
% \newcommand{\oneToInfty}{\ensuremath{\boldsymbol{1^\infty}}}
% \newcommand{\zeroToZero}{\ensuremath{\boldsymbol{0^0}}}
% \newcommand{\inftyToZero}{\ensuremath{\boldsymbol{\infty^0}}}
% \newcommand{\numOverZero}{\ensuremath{\boldsymbol{\tfrac{\#}{0}}}}

\usepackage{wasysym}            % for smiley
\newcommand{\zoz}{$\mathbf{ \frac{0}{0} }$}

% some inverse trigs
\DeclareMathOperator{\arcsec}{arcsec}
\DeclareMathOperator{\arccot}{arccot}
\DeclareMathOperator{\arccsc}{arccsc}

%%%% META DATA
\newcommand{\semester}{Autumn 2021}
\newcommand{\course}{Math 1151}
\newcommand{\lecTitle}{Lecture 38: Applications of Integration (AOI)}

%%%% TITLE PAGE
\title[\course]{\lecTitle}
\institute[Ohio State]
{
  \medskip
}
\date[\week]{\semester}
\author{Tae Eun Kim, Ph.D.}

\begin{document}
\begin{frame}
  \titlepage
\end{frame}



\subsection{Net Change and Future Value}
\begin{frame}
  \frametitle{Net change and future value}
  Recall the following formulas for net change and future value:
  \begin{itemize}
  \item \textbf{Net change:} $\ds \int_a^b Q'(s) \d s = Q(b) - Q(a)$
  \item \textbf{Future value:} $\ds Q(t) = Q(0) + \int_0^t Q'(s) \d s$
  \end{itemize}

  Equipped with this, we will now refine our understanding of the
  relationship between position and velocity.
\end{frame}

\begin{frame}
  \frametitle{Velocity and displacement}
  Let $v(t)$ be the \dfn{velocity} of an object at time $t$. This
  represents the ``rate of change in position'' at time $t$. Let $s(t)$ be
  the \dfn{position} of an object at time $t$. This gives location
  with respect to the origin.
  \begin{itemize}
  \item If we can assume that
    $s(a) = 0$, then by the future value formula
    \[
      s(t) = \int_a^t v(x) \d x.
    \]
  \item $s(b) -s(a)$ is the \dfn{displacement}, the distance between
    the starting and finishing locations.
  \end{itemize}
\end{frame}

\begin{frame}
  \frametitle{Speed and distance}
  Velocity and displacement are values containing not only information
  about {``magnitude''} but also of {``direction''} that is relative
  to some fixed point. On the other hand, there are values without
  {``direction''} information. For instance:
  \begin{itemize}
  \item $|v(t)|$ is the \dfn{speed}.
  \item $\ds \int_a^b |v(t)| \d t$ is the \dfn{distance} traveled.
  \end{itemize}
\end{frame}

\begin{frame}
  \vs
  \begingroup
  \small
  \begin{example}
    Consider a particle whose velocity at time $t$ is given by $v(t) = \sin(t)$.
    \begin{enumerate}
    \item What is the displacement of the particle from $t = 0$ to
      $t = \pi$?
    \item What is the displacement of the particle from $t = 0$ to
      $t = 2\pi$?
    \item What is the distance traveled by the particle from $t = 0$ to
      $t = \pi$?
    \item What is the distance traveled by the particle from $t = 0$ to
      $t = 2\pi$?
    \end{enumerate}
  \end{example}
  \endgroup
  \vs
  \hfill
  \begin{minipage}{0.5\linewidth}
    \begin{image}[0.99\linewidth]
      \begin{tikzpicture}[
        declare function = {f(\x) = sin(deg(\x));}]
        \begin{axis}[
          domain=-.2:7, xmin =-.2,xmax=7,ymax=1.2,ymin=-1.2,
          width=4in,
          height=2in,xtick={3.141,6.282},
          xticklabels={$\pi$, $2\pi$},
          ytick style={draw=none},
          yticklabels={},
          axis lines=center, xlabel=$t$, ylabel=${y=\sin(t)}$,
          every axis y label/.style={at=(current axis.above origin),anchor=south},
          every axis x label/.style={at=(current axis.right of origin),anchor=west},
          axis on top,
          ]
          \addplot [very thick,penColor, smooth] {f(x)};
        \end{axis}
      \end{tikzpicture}
    \end{image}
  \end{minipage}
\end{frame}

\begin{frame}
  \vs
  \begingroup
  \small
  \begin{example}
    An experiment is conducted in which a culture of bacteria is grown
    in a controlled lab environment. The initial population was
    estimated at 100 cells. The growth rate of the population $P(t)$ is
    estimated to be $P'(t) = 4/(1+t^2)$ cells per day.
    \begin{enumerate}
    \item By how much has the population grown during the first day of
      the experiment?
    \item Find the population at any time $t\ge0$.
    \end{enumerate}
  \end{example}
  \endgroup
\end{frame}

\subsection{Average Value}

\begin{frame}
  \frametitle{Alternate interpretation of definite integrals}
  Our framework of choice in interpreting definite integrals was
  {``signed area''} between a curve and the horizontal axis, which
  also provides a very good visualization of integration process.  An
  alternate way to understand integrals is to relate them to
  \textit{average values}.

  \begin{itemize}
  \item Recall that the average of $n$ discrete data
    $\{f_1, f_2, f_3, \ldots, f_n\}$ is given by
    \[
      \frac{f_1 + f_2 + \cdots + f_n}{n} = \frac{1}{n} \sum_{k=1}^n f_k \,.
      \tag{average; discrete}
    \]
  \item Now suppose you want to find the average value of a certain quantity
    that changes {``continuously''} over some interval, e.g., the
    temperature of water in my electric kettle from 6 a.m. to noon.
  \item In general terms, we want to find the average value of the function
    $f(t)$ over $[a,b]$.
  \end{itemize}
\end{frame}

\begin{frame}
  \frametitle{Approximate average value of a function}
  A natural way to approximate the average value $f(t)$ over $[a,b]$
  is:
  \begin{enumerate}
  \item partition the domain into $n$ equal subintervals,
    \[
      a = t_0 < t_1 < \cdots < t_n = b \,,
    \]
  \item collect data $f(t_1), f(t_2), \cdots, f(t_n)$ at the end of
    each subinterval,
  \item take the average of this $n$ data:
    \[
      \frac{1}{n} \sum_{k=1}^n f(x_k) \,.
    \]
  \end{enumerate}

  If $n$ is small, we get a coarse estimate of the true average;
  sampling more frequently, i.e, increasing $n$, we get a better
  estimate. When $n$ approaches $\infty$, we will have the true
  average value. But before we send this to limit as $n \to \infty$,
  we need a small touch-up on the previous expression.

  \note{
    Multiplying by $1 = (b-a)/(b-a)$ and shuffling some terms, the
    previous approximate average becomes
    \[
      \frac{1}{b-a} \sum_{k=1}^n f(x_k) \underbrace{\frac{b-a}{n}}_{= \Delta x} \,,
    \]
    a right Riemann sum of $f$ on $[a,b]$ divided by $b-a$! Passing to
    the limit, we obtain the following average formula:

  }
\end{frame}

\begin{frame}
  \frametitle{Average value of a function}
  \begin{defn}
    Let $f$ be continuous on $[a,b]$. The \textbf{average value} of $f$
    on $[a,b]$ is given by
    \[
      \frac{1}{b-a} \int_a^b	f(x) \; dx \,.
    \]
  \end{defn}
\end{frame}

\begin{frame}
  \vs
  \begin{example}
    Find the average height of points on the upper-half unit circle.
  \end{example}
  \vs
  \hfill
  \begin{minipage}{0.4\linewidth}
    \begin{image}[0.9\linewidth]
      \begin{tikzpicture}[
        declare function = {f(\x) = sin(deg(\x));}]
        \begin{axis}[
          domain=-3:3, xmin =-1.2,xmax=1.2,ymax=1.2,ymin=-.2,
          width=3in,
          height=2in,xtick={-1,1},
          xticklabels={$-1$, $1$},
          ytick style={draw=none},
          yticklabels={},
          axis lines=center, xlabel=$x$, ylabel=$y$,
          every axis y label/.style={at=(current axis.above origin),anchor=south},
          every axis x label/.style={at=(current axis.right of origin),anchor=west},
          axis on top,
          ]
          \addplot [penColor, very thick, smooth, domain=(0:180)] ({cos(x)},{sin(x)});
        \end{axis}
      \end{tikzpicture}
    \end{image}
  \end{minipage}
\end{frame}


\subsection{Mean Value Theorem for Integrals}
\begin{frame}
  \frametitle{Another MVT}
  Having seen how the average value (that is, the mean value) of a
  function is calculated using a definite integral, we are now ready
  for the following version of mean value theorem.

  \begin{thm}[The Mean Value Theorem for integrals]
    Let $f$ be continuous on $[a,b]$. There exists a value $c$ in $[a,b]$
    such that
    \[
      \int_a^bf(x)\d x = f(c)(b-a).
    \]
  \end{thm}

  \textbf{Note.} This is an \textit{existence} theorem just as its
  differential counterpart. It states that
  \begin{quote}
    \textbf{The average value of a continuous function falls within the
      range of the function.}
  \end{quote}
\end{frame}

\begin{frame}
  \frametitle{Illustration}
  The following is a visual description of the integral MVT:
  \begin{image}
    \begin{tikzpicture}[
      declare function = {f(\x) = -sin(deg(\x)) + 3;}]
      \begin{axis}[
        domain=-.2:7, xmin =-.2,xmax=7,ymax=5,ymin=-.7,
        width=4in,
        height=2in,xtick={1,6},
        xticklabels={$a$,$b$},
        ytick style={draw=none},
        yticklabels={},
        axis lines=center, xlabel=$x$, ylabel=$y$,
        every axis y label/.style={at=(current axis.above origin),anchor=south},
        every axis x label/.style={at=(current axis.right of origin),anchor=west},
        axis on top,
        ]
        \foreach \rectnumber in {1}
        {
          \addplot [draw=penColor,fill=fillp] plot coordinates
          {({1+(\rectnumber - 1) * 5},{3.08})
            ({1+(\rectnumber) * 5},{3.08})} \closedcycle;
        };
        \addplot [very thick,penColor, smooth] {f(x)};
        \node[black] at (axis cs:3.5,1.5) {$\ds \int_a^b f(x) \d x$};
      \end{axis}
    \end{tikzpicture}
  \end{image}
\end{frame}

\begin{frame}
  \vs
  \begin{example}
    Consider $\ds \int_0^\pi \sin x \d x$. Find a value $c$
    guaranteed by the Mean Value Theorem.
  \end{example}
\end{frame}

\end{document}
%%% Local Variables:
%%% mode: latex
%%% TeX-master: t
%%% End:
