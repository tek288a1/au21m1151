% \documentclass[10pt,t,presentation,ignorenonframetext,aspectratio=169]{beamer}
\documentclass[10pt,t,handout,ignorenonframetext,aspectratio=169]{beamer}
\usepackage[default]{lato}
\usepackage{tk_beamer1}
\input{tk_packages}
\input{tk_macros}
\input{tk_environ}
\input{tk_ximera}
\usepackage{wasysym}            % for smiley
\newcommand{\zoz}{$\mathbf{ \frac{0}{0} }$}

% some inverse trigs
\DeclareMathOperator{\arcsec}{arcsec}
\DeclareMathOperator{\arccot}{arccot}
\DeclareMathOperator{\arccsc}{arccsc}

%%%% META DATA
\newcommand{\semester}{Autumn 2021}
\newcommand{\course}{Math 1151}
\newcommand{\lecTitle}{Lecture 38: Applications of Integration (AOI)}

%%%% TITLE PAGE
\title[\course]{\lecTitle}
\institute[Ohio State]
{
  \medskip
}
\date[\week]{\semester}
\author{Tae Eun Kim, Ph.D.}

\begin{document}
\begin{frame}
  \titlepage
\end{frame}



\subsection{Net Change and Future Value}
\begin{frame}
  \frametitle{Net change and future value}
  Recall the following formulas for net change and future value:
  \begin{itemize}
  \item \textbf{Net change:} $\ds \int_a^b Q'(s) \d s = Q(b) - Q(a)$
  \item \textbf{Future value:} $\ds Q(t) = Q(0) + \int_0^t Q'(s) \d s$
  \end{itemize}

  Equipped with this, we will now refine our understanding of the
  relationship between position and velocity.
\end{frame}

\begin{frame}
  \frametitle{Velocity and displacement}
  Let $v(t)$ be the \dfn{velocity} of an object at time $t$. This
  represents the ``rate of change in position'' at time $t$. Let $s(t)$ be
  the \dfn{position} of an object at time $t$. This gives location
  with respect to the origin.
  \begin{itemize}
  \item If we can assume that
    $s(a) = 0$, then by the future value formula
    \[
      s(t) = \int_a^t v(x) \d x.
    \]
  \item $s(b) -s(a)$ is the \dfn{displacement}, the distance between
    the starting and finishing locations.
  \end{itemize}
\end{frame}

\begin{frame}
  \frametitle{Speed and distance}
  Velocity and displacement are values containing not only information
  about {``magnitude''} but also of {``direction''} that is relative
  to some fixed point. On the other hand, there are values without
  {``direction''} information. For instance:
  \begin{itemize}
  \item $|v(t)|$ is the \dfn{speed}.
  \item $\ds \int_a^b |v(t)| \d t$ is the \dfn{distance} traveled.
  \end{itemize}
\end{frame}

\begin{frame}
  \vs
  \begingroup
  \small
  \begin{example}
    Consider a particle whose velocity at time $t$ is given by $v(t) = \sin(t)$.
    \begin{enumerate}
    \item What is the displacement of the particle from $t = 0$ to
      $t = \pi$?
    \item What is the displacement of the particle from $t = 0$ to
      $t = 2\pi$?
    \item What is the distance traveled by the particle from $t = 0$ to
      $t = \pi$?
    \item What is the distance traveled by the particle from $t = 0$ to
      $t = 2\pi$?
    \end{enumerate}
  \end{example}
  \endgroup
  \vs
  \hfill
  \begin{minipage}{0.5\linewidth}
    \begin{image}[0.99\linewidth]
      \begin{tikzpicture}[
        declare function = {f(\x) = sin(deg(\x));}]
        \begin{axis}[
          domain=-.2:7, xmin =-.2,xmax=7,ymax=1.2,ymin=-1.2,
          width=4in,
          height=2in,xtick={3.141,6.282},
          xticklabels={$\pi$, $2\pi$},
          ytick style={draw=none},
          yticklabels={},
          axis lines=center, xlabel=$t$, ylabel=${y=\sin(t)}$,
          every axis y label/.style={at=(current axis.above origin),anchor=south},
          every axis x label/.style={at=(current axis.right of origin),anchor=west},
          axis on top,
          ]
          \addplot [very thick,penColor, smooth] {f(x)};
        \end{axis}
      \end{tikzpicture}
    \end{image}
  \end{minipage}
\end{frame}

\begin{frame}
  \vs
  \begingroup
  \small
  \begin{example}
    An experiment is conducted in which a culture of bacteria is grown
    in a controlled lab environment. The initial population was
    estimated at 100 cells. The growth rate of the population $P(t)$ is
    estimated to be $P'(t) = 4/(1+t^2)$ cells per day.
    \begin{enumerate}
    \item By how much has the population grown during the first day of
      the experiment?
    \item Find the population at any time $t\ge0$.
    \end{enumerate}
  \end{example}
  \endgroup
\end{frame}

\subsection{Average Value}

\begin{frame}
  \frametitle{Alternate interpretation of definite integrals}
  Our framework of choice in interpreting definite integrals was
  {``signed area''} between a curve and the horizontal axis, which
  also provides a very good visualization of integration process.  An
  alternate way to understand integrals is to relate them to
  \textit{average values}.

  \begin{itemize}
  \item Recall that the average of $n$ discrete data
    $\{f_1, f_2, f_3, \ldots, f_n\}$ is given by
    \[
      \frac{f_1 + f_2 + \cdots + f_n}{n} = \frac{1}{n} \sum_{k=1}^n f_k \,.
      \tag{average; discrete}
    \]
  \item Now suppose you want to find the average value of a certain quantity
    that changes {``continuously''} over some interval, e.g., the
    temperature of water in my electric kettle from 6 a.m. to noon.
  \item In general terms, we want to find the average value of the function
    $f(t)$ over $[a,b]$.
  \end{itemize}
\end{frame}

\begin{frame}
  \frametitle{Approximate average value of a function}
  A natural way to approximate the average value $f(t)$ over $[a,b]$
  is:
  \begin{enumerate}
  \item partition the domain into $n$ equal subintervals,
    \[
      a = t_0 < t_1 < \cdots < t_n = b \,,
    \]
  \item collect data $f(t_1), f(t_2), \cdots, f(t_n)$ at the end of
    each subinterval,
  \item take the average of this $n$ data:
    \[
      \frac{1}{n} \sum_{k=1}^n f(x_k) \,.
    \]
  \end{enumerate}

  If $n$ is small, we get a coarse estimate of the true average;
  sampling more frequently, i.e, increasing $n$, we get a better
  estimate. When $n$ approaches $\infty$, we will have the true
  average value. But before we send this to limit as $n \to \infty$,
  we need a small touch-up on the previous expression.

  \note{
    Multiplying by $1 = (b-a)/(b-a)$ and shuffling some terms, the
    previous approximate average becomes
    \[
      \frac{1}{b-a} \sum_{k=1}^n f(x_k) \underbrace{\frac{b-a}{n}}_{= \Delta x} \,,
    \]
    a right Riemann sum of $f$ on $[a,b]$ divided by $b-a$! Passing to
    the limit, we obtain the following average formula:

  }
\end{frame}

\begin{frame}
  \frametitle{Average value of a function}
  \begin{defn}
    Let $f$ be continuous on $[a,b]$. The \textbf{average value} of $f$
    on $[a,b]$ is given by
    \[
      \frac{1}{b-a} \int_a^b	f(x) \; dx \,.
    \]
  \end{defn}
\end{frame}

\begin{frame}
  \vs
  \begin{example}
    Find the average height of points on the upper-half unit circle.
  \end{example}
  \vs
  \hfill
  \begin{minipage}{0.4\linewidth}
    \begin{image}[0.9\linewidth]
      \begin{tikzpicture}[
        declare function = {f(\x) = sin(deg(\x));}]
        \begin{axis}[
          domain=-3:3, xmin =-1.2,xmax=1.2,ymax=1.2,ymin=-.2,
          width=3in,
          height=2in,xtick={-1,1},
          xticklabels={$-1$, $1$},
          ytick style={draw=none},
          yticklabels={},
          axis lines=center, xlabel=$x$, ylabel=$y$,
          every axis y label/.style={at=(current axis.above origin),anchor=south},
          every axis x label/.style={at=(current axis.right of origin),anchor=west},
          axis on top,
          ]
          \addplot [penColor, very thick, smooth, domain=(0:180)] ({cos(x)},{sin(x)});
        \end{axis}
      \end{tikzpicture}
    \end{image}
  \end{minipage}
\end{frame}


\subsection{Mean Value Theorem for Integrals}
\begin{frame}
  \frametitle{Another MVT}
  Having seen how the average value (that is, the mean value) of a
  function is calculated using a definite integral, we are now ready
  for the following version of mean value theorem.

  \begin{thm}[The Mean Value Theorem for integrals]
    Let $f$ be continuous on $[a,b]$. There exists a value $c$ in $[a,b]$
    such that
    \[
      \int_a^bf(x)\d x = f(c)(b-a).
    \]
  \end{thm}

  \textbf{Note.} This is an \textit{existence} theorem just as its
  differential counterpart. It states that
  \begin{quote}
    \textbf{The average value of a continuous function falls within the
      range of the function.}
  \end{quote}
\end{frame}

\begin{frame}
  \frametitle{Illustration}
  The following is a visual description of the integral MVT:
  \begin{image}
    \begin{tikzpicture}[
      declare function = {f(\x) = -sin(deg(\x)) + 3;}]
      \begin{axis}[
        domain=-.2:7, xmin =-.2,xmax=7,ymax=5,ymin=-.7,
        width=4in,
        height=2in,xtick={1,6},
        xticklabels={$a$,$b$},
        ytick style={draw=none},
        yticklabels={},
        axis lines=center, xlabel=$x$, ylabel=$y$,
        every axis y label/.style={at=(current axis.above origin),anchor=south},
        every axis x label/.style={at=(current axis.right of origin),anchor=west},
        axis on top,
        ]
        \foreach \rectnumber in {1}
        {
          \addplot [draw=penColor,fill=fillp] plot coordinates
          {({1+(\rectnumber - 1) * 5},{3.08})
            ({1+(\rectnumber) * 5},{3.08})} \closedcycle;
        };
        \addplot [very thick,penColor, smooth] {f(x)};
        \node[black] at (axis cs:3.5,1.5) {$\ds \int_a^b f(x) \d x$};
      \end{axis}
    \end{tikzpicture}
  \end{image}
\end{frame}

\begin{frame}
  \vs
  \begin{example}
    Consider $\ds \int_0^\pi \sin x \d x$. Find a value $c$
    guaranteed by the Mean Value Theorem.
  \end{example}
\end{frame}

\end{document}
%%% Local Variables:
%%% mode: latex
%%% TeX-master: t
%%% End:
