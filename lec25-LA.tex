% \documentclass[10pt,t,presentation,ignorenonframetext,aspectratio=169]{beamer}
\documentclass[10pt,t,handout,ignorenonframetext,aspectratio=169]{beamer}
\usepackage[default]{lato}
\usepackage{tk_beamer1}
\input{tk_packages}
\input{tk_macros}
\input{tk_environ}
\input{tk_ximera}
\usepackage{wasysym}            % for smiley
\newcommand{\zoz}{$\mathbf{ \frac{0}{0} }$}

% some inverse trigs
\DeclareMathOperator{\arcsec}{arcsec}
\DeclareMathOperator{\arccot}{arccot}
\DeclareMathOperator{\arccsc}{arccsc}

%%%% META DATA
\newcommand{\semester}{Autumn 2021}
\newcommand{\course}{Math 1151}
\newcommand{\lecTitle}{Lecture 25: Linear Approximation (LA)}

%%%% TITLE PAGE
\title[\course]{\lecTitle}
\institute[Ohio State]
{
  \medskip
}
\date[\week]{\semester}
\author{Tae Eun Kim, Ph.D.}

\begin{document}
\begin{frame}
  \titlepage
\end{frame}

\begin{frame}
  \frametitle{Linear Approximation}

  \begin{itemize}
  \item Recall that the derivative contains the slope information of
    tangent lines to a given curve.
  \item Using this, we spent a lot of time calculating the equations
    of lines tangent to curves.
  \end{itemize}
  Let's formalize this discussion.

  \begin{defn}
    If $f$ is a function differentiable at $x=a$, then a \textbf{linear
      approximation} for $f$ at $x=a$ is given by
    \[
      L(x) = f(a) + f'(a) (x-a) \,.
    \]
    Note that the graph of $L$ is simply the tangent line to the curve
    $y = f(x)$ at $x=a$.
  \end{defn}
\end{frame}

\begin{frame}
  \frametitle{Illustration}
  \begin{image}
    \begin{tikzpicture}
      \begin{axis}[
        xmin=0,xmax=2,ymin=0,ymax=2,
        axis lines=center,
        ticks=none,
        width=3in,
        height=2in,
        unit vector ratio*=1.5 1 1,
        xlabel=$x$, ylabel=$y$,
        every axis y label/.style={at=(current axis.above origin),anchor=south},
        every axis x label/.style={at=(current axis.right of origin),anchor=west},
        ]
        \addplot [thick, penColor, smooth, domain=(0:3)] {3*sqrt(x)-2};
        \addplot [thick, penColor2,smooth] {(3/2)*x+3/2-2};
        \node at (axis cs:1.7,1.5) [penColor] {$y=f(x)$};
        \node at (axis cs:1,1.5) [penColor2] {$y=L(x)$};
        \node at (axis cs:1.3,1) [penColor2] {$(a,f(a))$};
        \addplot[color=penColor3,fill=penColor3,only marks,mark=*] coordinates{(1,1)};  %% closed hole
      \end{axis}
    \end{tikzpicture}
  \end{image}
\end{frame}


\begin{frame}
  \frametitle{Examples}
  One major advantage of linear approximation is in computation of
  various mathematical functions.
  \vs{}

  \question{} Use a linear approximation of $f(x) =\sqrt[3]{x}$ at
  $a=64$ to approximate $\sqrt[3]{50}$. What would happen if we chose
  $a=27$ instead?
\end{frame}


\begin{frame}
  \frametitle{Examples}
  \question{} Use a linear approximation of $f(x) =\sin(x)$ at $a=0$
  to approximate $\sin(0.3)$.
\end{frame}

\begin{frame}
  \frametitle{[REVIEW] Over or Under?}
  At times, we need to determine whether a linear approximation is an overestimate or an underestimate without any access to a graph. The following may be useful.
  \vs
  \begin{itemize}
  \item If $f''(a)>0$, then $L(x) \le f(x)$ for $x$ \textit{near} $a$, i.e., $L(x)$ is an underestimate of $f(x)$.
  \item If $f''(a)>0$, then $L(x) \le f(x)$ for $x$ \textit{near} $a$, i.e., $L(x)$ is an underestimate of $f(x)$.
  \end{itemize}
  \vs
  The phrase ``$x$ near $a$'' is vague. The following statements are less ambiguous.
  \vs
  \begin{itemize}
  \item If $f''>0$ on an interval $I$ containing $a$, then $L(x) \le f(x)$ on $I$, that is, $L(x)$ is an \textit{underestimate} of $f(x)$.
  \item If $f''<0$ on an interval $I$ containing $a$, then $L(x) \ge f(x)$ on $I$, that is, $L(x)$ is an \textit{overestimate} of $f(x)$.
  \end{itemize}
\end{frame}

\begin{frame}
  \frametitle{Differentials}
  \begin{defn}
    Let $f$ be a differentiable function.
    We define  $\d f$, a differential of $f$,  at a point $x$ by
    \[
      \d f=f'(x) \, \d x
    \]
    where $\d x$ is an independent variable that  is called a \textbf{differential of $x$}.

  \end{defn}
  \vs{}
  Geometrically, differentials can be interpreted via the diagram below.
  \begin{image}[0.4\linewidth]
    \begin{tikzpicture}
      \begin{axis}[
        xmin=1, xmax=2, range=0:6,ymax=6,ymin=0,
        axis lines =left, xlabel=$x$, ylabel=$y$,
        every axis y label/.style={at=(current axis.above origin),anchor=south},
        every axis x label/.style={at=(current axis.right of origin),anchor=west},
        ticks=none,
        axis on top,
        ]
        \addplot [draw=black,dashed,->,>=stealth'] plot coordinates {(1.4,10/6) (1.7,10/6)};
        \addplot [draw=black,dashed,->,>=stealth'] plot coordinates {(1.7,10/6) (1.7,10/6 +.3/.36)};
        \addplot [very thick,penColor, smooth,samples=100,domain=(0:1.833)] {-1/(x-2)};
        \addplot [very thick,penColor2, smooth,samples=100,domain=(0:1.833)] {1/0.6+(1/0.36)*(x-1.4)};
        \addplot[color=penColor,fill=penColor,only marks,mark=*] coordinates{(1.7,1/0.3)};  %% closed hole
        \addplot[color=penColor3,fill=penColor3,only marks,mark=*] coordinates{(1.7,10/6 +.3/.36)};  %% closed hole
        \addplot[color=penColor3,fill=penColor3,only marks,mark=*] coordinates{(1.4,10/6)};  %% closed hole
        \node at (axis cs:1.55,1.67) [below] {$\d x$};
        \node at (axis cs:1.7,2.08) [right] {$\d f$};
        \node at (axis cs:1.15,1.85) [below] {$f$};
        \node at (axis cs:1.05,0.55) [right] {$L$};
        \node at (axis cs:1.24,2.0) [right] {$(x,f(x))$};
        \node at (axis cs:1.26,3.6) [right] {$(x+\d x,f(x+\d x))$};
      \end{axis}
    \end{tikzpicture}
  \end{image}
\end{frame}

\begin{frame}
  \frametitle{Remark}
  \begin{itemize}
  \item It is important to remember that the notations $\d x$ and
    $\d y = \d f$ represent variables.
  \item Observe from the picture that
    \[
      f(x+\d x) \approx f(x) + \d f \,.
    \]
  \item Noting that the right-hand side is identical to $L(x)$ given
    above, we once again confirm the validity of the linear
    approximation formula.
  \end{itemize}
\end{frame}

\begin{frame}
  \vs{}
  \question{} Use differentials to approximate $\sqrt[3]{50}$ and $\sin(0.3)$.
\end{frame}

\begin{frame}
  \frametitle{Error Approximation}
  \begin{itemize}
  \item We now consider how \textit{linear approximation} or
    \textit{differentials} can be used to estimate errors in various
    computations.
  \item The quantity that we are interested in is the difference or
    \textit{error} $E(x)$ between the quantity $f(x)$ and the quantity
    $f(x+\d x)$ with a slightly perturbed input $x+\d x$, i.e., for
    small $\d x$.
  \item Using differentials, we can estimate this error by
    \[
      E(x) = f(x+\d x) - f(x)
      \approx f(x) + \d f - f(x) = \d f = f'(x) \d x \,.
    \]
  \end{itemize}
\end{frame}


\begin{frame}
  \frametitle{Example}
  The cross-section of a $250$ ml glass can be modeled by the function
  $r(x) = x^4/3$:
  \vs{}

  \begin{minipage}[c]{0.4\linewidth}
    At $16.8$ cm from the base of the glass, there is a mark
    indicating when the glass is filled to $250$ ml. If the glass is
    filled within $\pm 2$ millimeters of the mark, what are the bounds
    on the volume? The volume in milliliters, as a function of the
    height of water in centimeters, $y$, is given by
    \[
      V(y) = \frac{2\pi y^{3/2}}{\sqrt{3}}.
    \]
  \end{minipage}
  \hfill
  \begin{minipage}[c]{0.55\linewidth}
    \begin{image}[0.95\linewidth]
      \begin{tikzpicture}[declare function = {f(\x) = (1/3)* pow(\x,4);} ]
        \begin{axis}[
          xmin =-4,xmax=4,ymax=23,ymin=-.2,
          axis lines=center, xlabel=$x$, ylabel=$y$,
          every axis y label/.style={at=(current axis.above origin),anchor=south},
          every axis x label/.style={at=(current axis.right of origin),anchor=west},
          axis on top,
          ]
          \addplot [draw=none,fill=fillp!50!white,domain=-2.65:2.65, smooth] {.8*sqrt(2.65^2-x^2)+16.8} \closedcycle;
          \addplot [draw=none,fill=fillp,domain=-2.65:2.65, smooth] {-.8*sqrt(2.65^2-x^2)+16.8} \closedcycle;
          \addplot [draw=none,fill=white,domain=-2.7:2.7, smooth] {f(x)} \closedcycle;
          \addplot [ultra thick,penColor, smooth,domain=-2.75:2.75] {f(x)};
          \draw[penColor,very thick] (axis cs:0,16.8) ellipse (265 and 20);
          \draw[penColor,very thick] (axis cs:0,19) ellipse (275 and 20);
          \node[black] at (axis cs:2.5,3) {$y=\frac{x^4}{3}$};
        \end{axis}
      \end{tikzpicture}
    \end{image}
  \end{minipage}
\end{frame}


\end{document}
%%% Local Variables:
%%% mode: latex
%%% TeX-master: t
%%% End:
