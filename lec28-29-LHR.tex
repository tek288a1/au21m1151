% \documentclass[10pt,t,presentation,ignorenonframetext,aspectratio=169]{beamer}
\documentclass[10pt,t,handout,ignorenonframetext,aspectratio=169]{beamer}
\usepackage[default]{lato}
\usepackage{tk_beamer1}
\input{tk_packages}
\input{tk_macros}
\input{tk_environ}
\input{tk_ximera}
\usepackage{wasysym}            % for smiley
\newcommand{\zoz}{$\mathbf{ \frac{0}{0} }$}

% some inverse trigs
\DeclareMathOperator{\arcsec}{arcsec}
\DeclareMathOperator{\arccot}{arccot}
\DeclareMathOperator{\arccsc}{arccsc}

%%%% META DATA
\newcommand{\semester}{Autumn 2021}
\newcommand{\course}{Math 1151}
\newcommand{\lecTitle}{Lecture 28-29: L'H\^{o}pital's Rule (LHR)}

%%%% TITLE PAGE
\title[\course]{\lecTitle}
\institute[Ohio State]
{
  \medskip
}
\date[\week]{\semester}
\author{Tae Eun Kim, Ph.D.}

\begin{document}
\begin{frame}
  \titlepage
\end{frame}

\begin{frame}
  \frametitle{Basic Ideas}
  This is our final application of derivatives: using derivatives to
  calculate difficult limits. Enter L'H\^{o}pital's rule.

  \begin{thm}[L'H\^{o}pital's Rule]
    Let $f(x)$ and $g(x)$ be functions that are differentiable near $a$.  If
    \[
      \lim_{x \to a} f(x) = \lim_{x \to a}g(x) = 0 \qquad \text{or } \pm \infty,
    \]
    and $\lim_{x \to a} \frac{f'(x)}{g'(x)}$ exists, and $g'(x) \neq 0$
    for all $x$ near $a$, then
    \[
      \lim_{x \to a} \frac{f(x)}{g(x)} = \lim_{x \to a} \frac{f'(x)}{g'(x)}.
    \]
  \end{thm}
\end{frame}


\begin{frame}
  \frametitle{List of Indeterminate Forms}
  \begin{itemize}
  \item $\boldsymbol{\frac{0}{0}}$
  \item $\boldsymbol{\frac{\infty}{\infty}}$
  \item $\boldsymbol{0 \cdot \infty}$
  \item $\boldsymbol{\infty - \infty}$
  \item $\boldsymbol{1^\infty}$
  \item $\boldsymbol{0^0}$
  \item $\boldsymbol{\infty^0}$
  \end{itemize}

  In each of these cases, the value of the limit is \textbf{not} immediately
  obvious. Hence, a careful analysis is required!
\end{frame}


\begin{frame}
  \frametitle{Examples: Basic Indeterminate Forms}
  \begin{question}
    Compute $\ds \lim_{x\to 0} \frac{\sin(x)}{x}$.
  \end{question}
\end{frame}

\begin{frame}
  \vs
  \begin{question}
    Compute $\ds \lim_{x\to \pi/2^+} \frac{\sec(x)}{\tan(x)}$.
  \end{question}
\end{frame}

\begin{frame}
  \vs The following $\boldsymbol{0 \cdot \infty}$ can be reduced to one of the
  two previous ones. For instance: \\
  \begin{question}
    Compute $\ds \lim_{x\to 0^+} x\ln x$.
  \end{question}
\end{frame}

\begin{frame}
  \frametitle{Examples: Indeterminate Forms Involving Subtraction}

  The name of the game once again is reduction. We will transform
  differences into either quotients or products then apply
  L'H\^{o}pital's rule on the basic forms. \\

  \begin{question}
    Compute $\ds \lim_{x\to 0} \left(\cot(x) - \csc(x)\right)$.
  \end{question}
\end{frame}

\begin{frame}
  \vs
  \begin{question}
    Compute $\ds \lim_{x\to\infty}\left(\sqrt{x^2+x}-x\right)$.
  \end{question}
\end{frame}

\begin{frame}
  \frametitle{Examples: Exponential Indeterminate Form}
  This pertains to the forms
  \[
    \boldsymbol{1^\infty}, \quad \boldsymbol{0^0}, \quad \boldsymbol{\infty^0}
  \]
  Suppose we have functions $u(x)$ and $v(x)$ such that
  \[
    \lim_{x \to a} u(x)^{v(x)}
  \]
  falls into one of the forms described above. We use the inverse relation between $\exp$ and $\log$ functions to rewrite the limit as
  \[
    \lim_{x\to a}e^{v(x) \ln(u(x))} \,.
  \]
  Using the fact that the exponential function is continuous, the limit equals to
  \[
    \exp[ \lim_{x\to a} v(x) \ln (u(x)) ] \,.
  \]
  Note that the limit now is in one of the previously presented forms.
\end{frame}

\begin{frame}
  \vs

  \begin{question}
    First determine the form of the limit, then compute the limit.
    \[
      \lim_{x\to \infty}\left(1 + \frac{1}{x}\right)^x.
    \]
  \end{question}
\end{frame}



\end{document}
%%% Local Variables:
%%% mode: latex
%%% TeX-master: t
%%% End:
