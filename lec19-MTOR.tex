% \documentclass[10pt,t,presentation,ignorenonframetext,aspectratio=169]{beamer}
\documentclass[10pt,t,handout,ignorenonframetext,aspectratio=169]{beamer}
\usepackage[default]{lato}
\usepackage{tk_beamer1}
\input{tk_packages}
\input{tk_macros}
\input{tk_environ}
\input{tk_ximera}
\usepackage{wasysym}            % for smiley
\newcommand{\zoz}{$\mathbf{ \frac{0}{0} }$}

% some inverse trigs
\DeclareMathOperator{\arcsec}{arcsec}
\DeclareMathOperator{\arccot}{arccot}
\DeclareMathOperator{\arccsc}{arccsc}

%%%% META DATA
\newcommand{\semester}{Autumn 2021}
\newcommand{\course}{Math 1151}
\newcommand{\lecTitle}{Lecture 19: More Than One Rate (MTOR)}

%%%% TITLE PAGE
\title[\course]{\lecTitle}
\institute[Ohio State]
{
  \medskip
}
\date[\week]{\semester}
\author{Tae Eun Kim, Ph.D.}

\begin{document}
\begin{frame}
  \titlepage
\end{frame}

\begin{frame}
  \frametitle{Related rates problems}

  \begin{itemize}
  \item<2-> Suppose two variables $x$ and $y$ are both dependent on time
    $t$.
  \item<3-> Moreover, assume that these two are related to each other.
  \item<4-> In this context, the rate of change of $y$ with respect to
    time is expected to be {\it related\/} to that of $x$;
  \item<5-> when one of the rates is known and the other is to be found,
    we have a {\bf related rates} problem.
  \end{itemize}

  \vfill
  \onslide<6->{\small
  \begin{block}{\bf \small Key idea.}
    \onslide<6->{If $y$ is written in terms of $x$ and we are given
    $\ds \frac{dx}{dt} = x'(t)$, then we can find $\ds \frac{dy}{dt} = y'(t)$
    using the chain rule:}

    \onslide<7->{\[
      \frac{dy}{dt} =y'(x(t))\cdot x'(t) \,.
    \]}
  \end{block}
  }
\end{frame}

\begin{frame}
  \frametitle{Problem-solving strategies}

  \begin{block}{General procedure}
    \vspace{0.1in}
    \begin{enumerate}
    \item<2-> \textbf{Draw a picture.} If possible, draw a schematic
      picture with all the relevant information.
      \vspace{0.1in}
    \item<3-> \textbf{Find equations.} We want equations that relate all
      relevant functions.
      \vspace{0.1in}
    \item<4-> \textbf{Differentiate the equations.} Here we will often use
      implicit differentiation.
      \vspace{0.1in}
    \item<5-> \textbf{Evaluate.} Evaluate each quantity at the relevant
      moment.
      \vspace{0.1in}
    \item<6-> \textbf{Solve.} Solve for the relevant rate at the relevant
      moment.
      \vspace{0.1in}
    \end{enumerate}
  \end{block}

\end{frame}

\begin{frame}[t]
  \frametitle{}
  \vs
  \begin{block}{Example 1. (Circular geometry)}<1->
    Imagine an expanding circle. If we know that the perimeter is
    expanding at a rate of $4$ m/s, what rate is the area changing
    when the radius is $3$ meters?
  \end{block}

  \onslide<2->
  \begin{figure}
    \hfill
    \begin{tikzpicture}[scale=0.7, every node/.style={transform shape}]
      % \draw [penColor, very thick] (0,0) circle [radius=2];
      % \draw [penColor] (0,0) -- (2,0);
      % \node [below,penColor] at (1,0) {$r=3$ m};
      % \node [penColor,left] at (-1.5,1.42) {$\dfrac{d}{dt}P(t) = 4$ m/s};
      % \node [penColor, right] at (1.5,-1.42) {$A = \pi\cdot r^2$};
      \draw [thick] (0,0) circle [radius=2];
      \draw [] (0,0) -- (2,0);
      \node [below] at (1,0) {$r=3$ m};
      \node [left] at (-1.4,1.7) {$\dfrac{d}{dt}P(t) = 4$ m/s};
      \node [right] at (1.5,-1.42) {$A = \pi\cdot r^2$};
    \end{tikzpicture}
  \end{figure}
\end{frame}
\begin{frame}\end{frame}
% \emptyframe

\begin{frame}[t]
  \frametitle{}
  \vs
  \begin{block}{Example 2. (Right triangles)}<1->
    Imagine an expanding right triangle. If one leg has a fixed length
    of $3$ m, one leg is increasing with a rate of $2$ m/s, and the
    hypotenuse is expanding to accommodate the expanding leg, at what
    rate is the hypotenuse expanding when both legs are $3$ m long?
  \end{block}

  \onslide<2->
  \begin{figure}
    \hfill
    \begin{tikzpicture}[scale=0.7, every node/.style={transform shape}]
      \coordinate (A) at (0,2);
      \coordinate (B) at (0,5);
      \coordinate (C) at (6.5,2);
      \tkzMarkRightAngle(C,A,B)
      \tkzDefMidPoint(A,B) \tkzGetPoint{a}
      \tkzDefMidPoint(A,C) \tkzGetPoint{b}
      \tkzDefMidPoint(B,C) \tkzGetPoint{c}
      \draw (A)--(B)--(C)--cycle;
      \tkzLabelPoints[above](c)
      \tkzLabelPoints[above](b)
      %\tkzLabelPoints[left](a)
      \node [left] at (a) {$a = 3$};
      \node [below] at (b) {$b'(t) = 2$};
    \end{tikzpicture}
  \end{figure}
\end{frame}
\begin{frame}\end{frame}
% \emptyframe

\begin{frame}[t]
  \frametitle{}
  \vs
  \begin{block}{Example 3. (Angular rates)}<1->
    Imagine an expanding right triangle. If one leg has a fixed length
    of $3$ m, one leg is increasing with a rate of $2$ m/s, and the
    hypotenuse is expanding to accommodate the expanding leg, at what
    rate is the angle opposite the fixed leg changing when both legs
    are $3$ m long?
  \end{block}

  \onslide<2->
  \begin{figure}
    \hfill
    \begin{tikzpicture}[scale=0.7, every node/.style={transform shape}]
      \coordinate (A) at (0,2);
      \coordinate (B) at (0,5);
      \coordinate (C) at (6.5,2);
      \tkzMarkRightAngle(C,A,B)
      \tkzMarkAngle[size=1.2cm,thin](B,C,A)
      \tkzLabelAngle[pos = 1](B,C,A){$\theta$}
      \tkzDefMidPoint(A,B) \tkzGetPoint{a}
      \tkzDefMidPoint(A,C) \tkzGetPoint{b}
      \tkzDefMidPoint(B,C) \tkzGetPoint{c}
      \draw (A)--(B)--(C)--cycle;
      \tkzLabelPoints[above](c)
      \tkzLabelPoints[above](b)
      %\tkzLabelPoints[left](a)
      \node [left] at (a) {$a = 3$};
      \node [below] at (b) {$b'(t) = 2$};
    \end{tikzpicture}
  \end{figure}
\end{frame}
\begin{frame}\end{frame}
% \emptyframe

\begin{frame}[t]
  \frametitle{}
  \vs
  \begin{block}{Example 4. (Similar triangles)}<1->
    Imagine two right triangles that share an angle. If $x$ is growing
    from the vertex with a rate of $3$ m/s, what rate is the area of
    the smaller triangle changing when $x = 5$ m?
  \end{block}

  \onslide<2->
  \begin{figure}
    \hfill
    \begin{tikzpicture}[scale=0.7, every node/.style={transform shape}]
      \coordinate (A) at (6,2);
      \coordinate (B) at (6,5);
      \coordinate (C) at (0,2);
      \coordinate (D) at (4,2);
      \coordinate (E) at (4,4);
      \tkzMarkRightAngle(C,A,B)
      \tkzMarkRightAngle(C,D,E)
      \tkzDefMidPoint(A,B) \tkzGetPoint{a}
      \tkzDefMidPoint(A,C) \tkzGetPoint{b}
      \tkzDefMidPoint(D,C) \tkzGetPoint{x}
      \draw[decoration={brace,mirror,raise=.2cm},decorate,thin] (0,2)--(6,2);
      \draw[decoration={brace,mirror,raise=.2cm},decorate,thin] (6,2)--(6,5);
      \draw[dashed] (A)--(B)--(C)--cycle;
      \draw[very thick] (D)--(E)--(C)--cycle;
      \tkzLabelPoints[above](x)
      \node at (3,2-.7) {$6$};
      \node at (6+.7,3.5) {$3$};
      \node at (2,2-.9) {$x'(t) = 3$};
    \end{tikzpicture}
  \end{figure}
\end{frame}
\begin{frame}\end{frame}
% \emptyframe

\end{document}
%%% Local Variables:
%%% mode: latex
%%% TeX-master: t
%%% End:
