\documentclass[10pt,t,presentation,ignorenonframetext,aspectratio=169]{beamer}
% \documentclass[10pt,t,handout,ignorenonframetext,aspectratio=169]{beamer}
\usepackage[default]{lato}
\usepackage{tk_beamer1}
\input{tk_packages}
\input{tk_macros}
\input{tk_environ}
\input{tk_ximera}
\usepackage{wasysym}            % for smiley
\newcommand{\zoz}{$\mathbf{ \frac{0}{0} }$}

%%%% META DATA
\newcommand{\semester}{Autumn 2021}
\newcommand{\course}{Math 1151}
\newcommand{\lecTitle}{Lecture 6: Continuity and the Intermediate Value Theorem (CATIVT)}

%%%% TITLE PAGE
\title[\course]{\lecTitle}
\institute[Ohio State]
{
  \medskip
}
\date[\week]{\semester}
\author{Tae Eun Kim, Ph.D.}

\begin{document}
\begin{frame}
  \titlepage
\end{frame}

\begin{frame}
  \frametitle{Recall \ldots}
  Recall that $f$ is said to be continuous at $x =
  a$ if
  \[
    \lim_{x \to a} f(x) = f(a) .
  \]
  \vfill
  Below is our up-to-date library of functions that are continuous on their natural
  domains:
  \begin{itemize}
  \item Constant functions
  \item Power functions
  \item Polynomial functions
  \item Rational functions
  \item Exponential functions
  \item Logarithmic functions
  \item Trigonometric functions
  \item Inverse trigonometric functions
  \end{itemize}
  \vfill
\end{frame}

\begin{frame}
  \frametitle{Continuity of piecewise functions}
  Today, we will consider continuity of functions obtained by patching
  two or more functions. In doing so, pay attention to
  \begin{itemize}
  \item any possible discontinuities of individual \textit{pieces}
  \item (dis)continuity at junctions where two pieces are joined
  \end{itemize}
\end{frame}

\begin{frame}
  \vs
  \begin{question}
    Consider the function defined piecewise as
    \[
      f(x) =
      \begin{dcases}
        \frac{x}{x-1} &\text{if $x<0$,}\\
        e^{-x} + c &\text{if $x\ge 0$}.
      \end{dcases}
    \]
    Find $c$ so that $f$ is continuous at $x = 0$.
  \end{question}
\end{frame}

\begin{frame}
  \vs
  \begin{question}
    Consider the following piecewise defined function
    \[
      f(x) =
      \begin{dcases}
        x+4 &\text{if $x<1$,}\\
        ax^2+bx+2 &\text{if $1\le x< 3$,}\\
        6x+a-b &\text{if $x\ge 3$.}
      \end{dcases}
    \]
    Find $a$ and $b$ so that $f$ is continuous at both $x=1$ and $x=3$.
  \end{question}
\end{frame}

\begin{frame}[fragile]
  \frametitle{For those interested \ldots}
  You can create interactive plots of such functions in
  \textbf{Mathematica}. For example:

  % \begin{tcolorbox}
  %   \vspace{-3mm}
  %   \begin{lstlisting}[numbers=none,frame=none]
  %     Manipulate[
  %     Plot[
  %     Piecewise[{
  %     {x/(x - 1), x < 0},
  %     {Exp[-x] + c, x >= 0}
  %   }],
  %     {x, -2, 2}
  %     ],
  %     {c, -3, 1}
  %     ]
  %   \end{lstlisting}
  %   \vspace{-3mm}
  % \end{tcolorbox}

  \begin{tcolorbox}
    \vspace{-0mm}
    \begin{lstlisting}[numbers=none,frame=none,basicstyle=\ttfamily]
Manipulate[Plot[
    Piecewise[{ {x + 4, x < 1},
                {a*x^2 + b*x + 2, 1 <= x < 3},
                {6*x + a - b, x >= 3} }],
    {x, 0, 4}], {a, 0, 2}, {b, 1, 3}]
    \end{lstlisting}
  \end{tcolorbox}
\end{frame}

\begin{frame}
  \frametitle{The Intermediate Value Theorem}
  \begin{thm}[Intermediate Value Theorem]
    If
    \begin{itemize}
    \item $f$ is a continuous function for all $x$ in $[a,b]$ and
    \item $d$ is between $f(a)$ and $f(b)$,
    \end{itemize}
    then there is a number $c$ in $[a,b]$ such that
    \[
      f(c) = d \,.
    \]
  \end{thm}
\end{frame}

\begin{frame}
  \frametitle{Illustration}
  \begin{minipage}[t]{0.45\textwidth}
    \textbf{Case 1:} $f$ is continuous on $[a, b]$
    \begin{image}[0.98\textwidth]
      \begin{tikzpicture}
        \begin{axis}[
          domain=0:6, ymin=0, ymax=2.2,xmax=6,
          axis lines =left, xlabel=$x$, ylabel=$y$,
          every axis y label/.style={at=(current axis.above origin),anchor=south},
          every axis x label/.style={at=(current axis.right of origin),anchor=west},
          xtick={1,3.597,5}, ytick={.203,1,1.679},
          xticklabels={$a$,$c$,$b$}, yticklabels={$f(a)$,$f(c)=d$,$f(b)$},
          axis on top,
          ]
          \addplot [draw=none, fill=fill1, domain=(0:7)] {1.679} \closedcycle;
          \addplot [draw=none, fill=background, domain=(0:7)] {.203} \closedcycle;
          \addplot [textColor,dashed] plot coordinates {(0,1.679) (6,1.679)};
          \addplot [textColor,dashed] plot coordinates {(0,.203) (6,.203)};
          \addplot [textColor,dashed] plot coordinates {(5,0) (5,1.679)};
          \addplot [textColor,dashed] plot coordinates {(1,0) (1,.203)};
          \addplot [textColor,dashed] plot coordinates {(3.587,0) (3.597,1)};
          \addplot [penColor2,domain=(0:6)] {1};
          \addplot [very thick,penColor, smooth,domain=(0:2.5)] {sin(deg((x - 4)/2)) + 1.2};
          \addplot [very thick,penColor, smooth,domain=(4:6)] {sin(deg((x - 4)/2)) + 1.2};
          \addplot [very thick,dashed,penColor!50!background, smooth,domain=(2.5:4)] {sin(deg((x - 4)/2)) + 1.2};
          \addplot [color=penColor!50!background,fill=penColor!50!background,only marks,mark=*] coordinates{(3.587,1)};  %% closed hole
          \addplot [color=penColor,fill=penColor,only marks,mark=*] coordinates{(1,.203)};  %% closed hole
          \addplot [color=penColor,fill=penColor,only marks,mark=*] coordinates{(5,1.679)};  %% closed hole
        \end{axis}
      \end{tikzpicture}
    \end{image}
  \end{minipage}
  \hfill
  \begin{minipage}[t]{0.45\textwidth}
    \textbf{Case 2:} $f$ is continuous on $[a, b)$
    \begin{image}[0.9\textwidth]
      \begin{tikzpicture}
        \begin{axis}[
          domain=0:6, ymin=0, ymax=2.2,xmax=6,
          axis lines =left, xlabel=$x$, ylabel=$y$,
          every axis y label/.style={at=(current axis.above origin),anchor=south},
          every axis x label/.style={at=(current axis.right of origin),anchor=west},
          xtick={1,5}, ytick={.2,1,1.7},
          xticklabels={$a$,$b$}, yticklabels={$f(a)$,$d$,$f(b)$},
          axis on top,
          ]
          \addplot [draw=none, fill=fill1, domain=(0:7)] {1.7} \closedcycle;
          \addplot [draw=none, fill=background, domain=(0:7)] {.2} \closedcycle;
          \addplot [textColor,dashed] plot coordinates {(0,1.7) (6,1.7)};
          \addplot [textColor,dashed] plot coordinates {(0,.2) (6,.2)};
          \addplot [textColor,dashed] plot coordinates {(5,0) (5,1.7)};
          \addplot [textColor,dashed] plot coordinates {(1,0) (1,.2)};
          \addplot [penColor2,domain=(0:6)] {1};
          \addplot [very thick,penColor] plot coordinates {(1,.2) (5,.7)};
          \addplot [color=penColor,fill=penColor,only marks,mark=*] coordinates{(1,.2)};  %% closed hole
          \addplot [color=penColor,fill=fill1,only marks,mark=*] coordinates{(5,.7)};  %% open hole
          \addplot [color=penColor,fill=penColor,only marks,mark=*] coordinates{(5,1.7)};  %% closed hole
        \end{axis}
      \end{tikzpicture}
    \end{image}
  \end{minipage}
\end{frame}

\begin{frame}
  \vs
  \begin{question}
    Demonstrate, using the IVT, that the function
    \[
      f(x)=x^3 + 3x^2+x-2
    \]
    has a root~\footnote{It can be shown that
      $f(x) = (x+2)(x^2+x-1)$ and so we know precisely that $f$ has three roots
      \[
        -2\,,
        \frac{-1 + \sqrt{5}}{2}\,, \text{ and }
        \frac{-1 - \sqrt{5}}{2}\,.
      \]} between $0$ and $1$.
  \end{question}
\end{frame}

\begin{frame}
  \vs
  \begin{question}
    Explain why the functions
    \begin{align*}
      f(x) &= x^2\ln(x)\\
      g(x) &= 2x\cos(\ln(x))
    \end{align*}
    intersect on the interval $[1,e]$.
  \end{question}
\end{frame}

\end{document}
%%% Local Variables:
%%% mode: latex
%%% TeX-master: t
%%% End:
