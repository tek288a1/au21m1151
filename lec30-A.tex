% \documentclass[10pt,t,presentation,ignorenonframetext,aspectratio=169]{beamer}
\documentclass[10pt,t,handout,ignorenonframetext,aspectratio=169]{beamer}
\usepackage[default]{lato}
\usepackage{tk_beamer1}
\input{tk_packages}
\input{tk_macros}
\input{tk_environ}
\input{tk_ximera}
\usepackage{wasysym}            % for smiley
\newcommand{\zoz}{$\mathbf{ \frac{0}{0} }$}

% some inverse trigs
\DeclareMathOperator{\arcsec}{arcsec}
\DeclareMathOperator{\arccot}{arccot}
\DeclareMathOperator{\arccsc}{arccsc}

%%%% META DATA
\newcommand{\semester}{Autumn 2021}
\newcommand{\course}{Math 1151}
\newcommand{\lecTitle}{Lecture 30: Antiderivatives (A)}

%%%% TITLE PAGE
\title[\course]{\lecTitle}
\institute[Ohio State]
{
  \medskip
}
\date[\week]{\semester}
\author{Tae Eun Kim, Ph.D.}

\begin{document}
\begin{frame}
  \titlepage
\end{frame}


\begin{frame}
  \frametitle{Basic Antiderivatives}

  \textbf{Antidifferentiation} is a process where we undo differentiation. Precisely:
  \begin{defn}
    A function $F$ is called an \dfn{antiderivative} of $f$ on an
    interval if
    \[
      F'(x) = f(x)
    \]
    for all $x$ in the interval.
  \end{defn}

\end{frame}

\begin{frame}
  \vs
  Since the derivative of a constant is zero, we can add it to any
  antiderivative of $f$ and it will still be an antiderivative.


  \begin{thm}[The family of antiderivatives]
    If $F$ is an antiderivative of $f$, then the function $f$ has a
    whole \textbf{family of antiderivatives}. Each antiderivative of $f$
    is the sum of $F$ and some constant $C$. The family of \textit{all}
    antiderivatives of $f$ is denoted by
    \[
      \int f(x) \; dx \,.
    \]
    This is called the \textbf{indefinite integral} of $f$.
  \end{thm}

  It follows that
  \[
    \int f(x) \d x =F(x)+C,
  \]
  where $F$ is any antiderivative of $f$ and $C$ is an arbitrary constant.

\end{frame}

\begin{frame}
  \frametitle{Basic Indefinite Integrals}
  \begin{itemize}
  \item $\int k \d x= k x+C$
  \item $\int \frac{1}{x} \d x= \ln|x|+C$
  \item $\int x^n \d x= \frac{x^{n+1}}{n+1}+C\qquad(n\ne-1)$
  \item $\int e^x \d x= e^x + C$
  \item $\int a^x \d x= \frac{a^x}{\ln(a)}+C$
  \item $\int \cos(x) \d x = \sin(x) + C$
  \item $\int \sin(x) \d x = -\cos(x) + C$
  \item $\int \sec^2(x) \d x = \tan(x) + C$
  \item $\int \csc^2(x) \d x = -\cot(x) + C$
  \item $\int \sec(x)\tan(x) \d x = \sec(x) + C$
  \item $\int \csc(x)\cot(x) \d x = -\csc(x) + C$
  \item $\int \frac{1}{x^2+1}\d x = \arctan{x} + C$
  \item $\int \frac{1}{\sqrt{1-x^2}}\d x= \arcsin{x}+C$
  \end{itemize}
\end{frame}


\begin{frame}
  \frametitle{Basic Antiderivative Rules}
  We have the following rules that mirror basic derivative rules.
  \begin{thm}
    If $F$ is an antiderivative of $f$ and $G$ is an antiderivative of
    $g$, then $F+G$ is an antiderivative of $f+g$. Moreover, for any
    constant $k$, $kF$ is an antiderivative of $kf$.

    We can write equivalently, using indefinite integrals,
    \begin{gather}
      \int \left(f(x)+g(x)\right) \d x
      = \int f(x)\d x +\int g(x) \d x \,, \tag{sum rule}\\
      \int kf(x) \d x= k\int f(x)\d x \,. \tag{constant multiple rule}
    \end{gather}
  \end{thm}

\end{frame}

\begin{frame}
  \vs

  \begin{question}
    Compute:
    \[
      \int \left(x^4 + 5x^2 - \cos(x)\right) \d x
    \]
  \end{question}

\end{frame}

\begin{frame}
  \vs
  \begin{question}
    A student claims that $\int 2x \cos(x) \d x = x^2 \sin(x) +C$.  Determine whether the student is correct or incorrect.
  \end{question}


\end{frame}

\begin{frame}
  \frametitle{Guessing Antiderivatives}
  \begin{question}
    Compute:
    \[
      \int \frac{\sqrt{x}+1+x}{x} \d x
    \]
  \end{question}

\end{frame}


\begin{frame}
  \vs
  \begin{question}
    Compute:
    \[
      \int 3x^2\sin{(x^3 -6)} \d x
    \]
  \end{question}


\end{frame}


\begin{frame}
  \vs

  \begin{question}
    Compute:
    \[
      \int \frac{2x^2}{7x^3 + 3} \d x
    \]
  \end{question}

\end{frame}



\begin{frame}
  \frametitle{Differential Equations}
  \begin{itemize}
  \item A \textit{differential equation}\index{differential equation} is
    simply an equation with a derivative in it. Here is an example:
    \[
      a f'(x)+ b f(x) = g(x).
    \]
  \item Differential equations show you relationships between rates of
    functions.
  \item The theory of differential equation is a very important
    branch of mathematics with vast real-life applications.
  \end{itemize}

\end{frame}

\begin{frame}
  \frametitle{What Does It Mean To Solve A Differential Equation?}
  When a mathematician solves a differential equation, they are finding
  \textit{functions} satisfying the equation. For example, consider the
  following differential equation:
  \[
    f'(x) = f(x) \,.
  \]

  \begin{itemize}
  \item It turns out that the complete solution to this differential equation
    is $Ce^x$, i.e., all the solutions of this differential equation have
    this form.
  \item Showing that any function $y=Ce^x$ is a solution of this differential
    equation is easy,
  \item but showing that \textbf{all} of the solutions have
    this form is beyond the scope of this course.
  \end{itemize}

\end{frame}

\begin{frame}
  \frametitle{General Solution and Initial Value Problems}
  \begin{itemize}
  \item In the previous example, a function $Ce^x$ is called a \dfn{general
      solution} of the differential equation.
  \item Since there are infinitely many solutions to a differential equation,
    we can impose additional condition, called an \dfn{initial condition},
    e.g. $f(0)=1$.
  \end{itemize}
  The problem now is to find a function $f$ that satisfies both the
  differential equation (DE) and the initial condition (IC).

  \[
    f'(x) = f(x) \tag{DE}
  \]
  \[
    f(0) =1   \tag{IC}
  \]
  This is called an \dfn{initial value problem} (IVP).
\end{frame}

\begin{frame}
  \frametitle{Example: IVP and A Falling Object}
  Here is a classical example of IVP arising in simple physics. \\

  \vs
  \begin{question}
    A ball is tossed into the air with an initial velocity of $15$
    m/s. What is the velocity of the ball after 1 second? How about
    after 2 seconds?
  \end{question}

  \vfill
  \begin{question}
    A ball is tossed into the air with an initial velocity of $15$ m/s
    from a height of 2 meters. When does the ball hit the ground?
  \end{question}
  \vfill
\end{frame}

\end{document}
%%% Local Variables:
%%% mode: latex
%%% TeX-master: t
%%% End:
