% \documentclass[10pt,t,presentation,ignorenonframetext,aspectratio=169]{beamer}
\documentclass[10pt,t,handout,ignorenonframetext,aspectratio=169]{beamer}
\usepackage[default]{lato}
\usepackage{tk_beamer1}
\input{tk_packages}
\input{tk_macros}
\input{tk_environ}
\input{tk_ximera}
\usepackage{wasysym}            % for smiley
\newcommand{\zoz}{$\mathbf{ \frac{0}{0} }$}

% some inverse trigs
\DeclareMathOperator{\arcsec}{arcsec}
\DeclareMathOperator{\arccot}{arccot}
\DeclareMathOperator{\arccsc}{arccsc}

%%%% META DATA
\newcommand{\semester}{Autumn 2021}
\newcommand{\course}{Math 1151}
\newcommand{\lecTitle}{Lecture 22-23: Graphing Functions (COGF \& CFGF)}

%%%% TITLE PAGE
\title[\course]{\lecTitle}
\institute[Ohio State]
{
  \medskip
}
\date[\week]{\semester}
\author{Tae Eun Kim, Ph.D.}

\begin{document}
\begin{frame}
  \titlepage
\end{frame}

\begin{frame}
  \frametitle{Calculus and Graphs}
  Let's put together all the tools we've learned so far with graphical
  implications:
  \begin{itemize}
  \item \textbf{Infinite limits} indicate the presence of vertical asymptotes.
  \item \textbf{Limits at infinity} describe "far-field" behavior of the function, e.g., horizontal asymptotes.
  \item \textbf{The sign of the first derivative} tells us about the monotonicity, i.e., whether the graph is increasing or decreasing.
  \item \textbf{The sign of the second derivative} conveys the concavity information, i.e., whether it is concave up or down.
  \end{itemize}
\end{frame}

\begin{frame}
  \frametitle{On Monotonicity and Concavity}
  Combining two possible monotonicity and two possible concavity
  modes, we came up with the following four signature of curves:
  \begin{itemize}
  \item $f'>0$ and $f''>0$: increasing and concave up
  \item $f'>0$ and $f''<0$: increasing and concave down
  \item $f'<0$ and $f''>0$: decreasing and concave up
  \item $f'<0$ and $f''<0$: decreasing and concave down
  \end{itemize}
\end{frame}

\begin{frame}
  \vs{}
  Recall the following table from couple weeks ago.
  \begin{image}[0.7\textwidth]
    \begin{tikzpicture}
      \draw (0,0) -- (0,12);
      \draw (0,0) -- (12,0);
      \draw (6,0) -- (6,12);
      \draw (0,6) -- (12,6);
      \draw (12,0) -- (12,12);
      \draw (0,12) -- (12,12);

      \node at (-1.3,9) {\Large$f''(x)>0$};
      \node at (-1.3,3) {\Large$f''(x)<0$};
      \node at (3,12.4) {\Large$f'(x)<0$};
      \node at (9,12.4) {\Large$f'(x)>0$};

      \draw [penColor,ultra thick,domain=180:270] plot ({2*cos(\x)+4}, {2*sin(\x)+11});
      \draw [penColor,ultra thick,domain=270:360] plot ({2*cos(\x)+8}, {2*sin(\x)+11});
      \draw [penColor,ultra thick,domain=0:90] plot ({2*cos(\x)+2}, {2*sin(\x)+3});
      \draw [penColor,ultra thick,domain=180:90] plot ({2*cos(\x)+10}, {2*sin(\x)+3});

      \node at (3,7.5) [text width=5cm] {\large
        The function $f$ is decreasing, while the rate itself is increasing.
        In this case the curve $y=f(x)$ is \dfn{concave up}.};

      \node at (9,7.5) [text width=5cm] {\large
        The function $f$ is increasing, while the rate itself is increasing.
        In this case the curve $y=f(x)$ is \dfn{concave up}.};

      \node at (3,1.5) [text width=5cm] {\large
        The function $f$  is decreasing, while the rate itself is decreasing.
        In this case the curve  $y=f(x)$ is \dfn{concave down}.};

      \node at (9,1.5) [text width=5cm] {\large
        The function $f$ is increasing, while the rate itself is decreasing.
        In this case the curve $y=f(x)$ is \dfn{concave down}.};
    \end{tikzpicture}
  \end{image}
\end{frame}

\begin{frame}
  \frametitle{On Critical and Inflection Points}
  I have several important remarks on \textbf{critical points} and
  \textbf{inflection points}:
  \begin{itemize}
  \item \textbf{Critical points} are interior points.
  \item There are two types of critical points -- one at which $f'=0$
    (\textbf{the nice ones}) and the other at which $f'$ is not
    defined (\textbf{the exotic ones}). Do not neglect the second
    kind.
  \item Being a critical point is merely a requirement to be a local
    extremum. It is not guaranteed that a critical point must be a
    local minimum or a local maximum.
  \item An \textbf{inflection point} is a point at which
    \begin{itemize}
    \item $f$ is continuous AND
    \item $f$ changes concavity from concave down to up or up to down.
    \end{itemize}
  \end{itemize}
\end{frame}


\begin{frame}
  \frametitle{On Derivative Tests}
  Lastly, on the derivative tests:
  \begingroup
  \small
  \begin{itemize}
  \item These are used to classify critical points into local maxima or local minima. Once again, understand that a critical point may be neither one of them.
  \item The key idea of these derivative tests is as follows: \\
    Suppose $c$ is a critical points of $f$.
    \begin{itemize}
    \item If a graph shifts from an increasing to a decreasing phase about $c$, \\ then it is a local maximum.
    \item If a graph shifts from a decreasing to an increasing phase about $c$, \\ then it is a local minimum.
    \end{itemize}
  \item In the 1st Derivative Test, we look out for the change in sign of $f'$ \textbf{about} $c$.
  \item In the 2nd Derivative Test, we look out for the sign of $f''$ \textbf{at} $c$.
  \end{itemize}
  \endgroup
\end{frame}

\begin{frame}
  \vs{}
  \begin{example}
    Sketch the graph of a function $f$ which has the following properties:
  \end{example}
  \begingroup
  \footnotesize
  \begin{minipage}[t]{0.4\linewidth}
    \begin{itemize}
    \item $f(0)=0$
    \item $\ds \lim_{x \to 10^+} f(x) = +\infty$
    \item $\ds \lim_{x \to 10^-} f(x) = -\infty$
    \end{itemize}
  \end{minipage}
  \hfill
  \begin{minipage}[t]{0.58\linewidth}
    \begin{itemize}
    \item $f'(x)<0$ on $(-\infty,0) \cup (6,10) \cup (10,14)$
    \item $f'(x)>0$ on $(0,6) \cup (14,\infty)$
    \item $f''(x)<0$ on $(4,10)$
    \item $f''(x)>0$ on $(-\infty,4) \cup (10,\infty)$
    \end{itemize}
  \end{minipage}
  \endgroup
\end{frame}

\begin{frame}
  \vs{}
  \begin{example}
    The graph of $f'$ (the derivative of $f$ ) is shown below. Assume
    $f$ is continuous for all real numbers.
  \end{example}
  \vs{}
  \begin{minipage}[t]{0.5\linewidth}
    \begingroup
    \small
    \begin{enumerate}
    \item On which of the following intervals is $f$ increasing? \vspace{1.5ex}
    \item Which of the following are critical points of $f$? \vspace{1.5ex}
    \item Where do the local maxima occur? \vspace{1.5ex}
    \item Where does a point of inflection occur? \vspace{1.5ex}
    \item On which of the following intervals is $f$ concave down? \vspace{1.5ex}
    \end{enumerate}
    \endgroup
  \end{minipage}
  \hfill
  \begin{minipage}[t]{0.45\linewidth}
    \begin{image}[0.85\linewidth]
      \begin{tikzpicture}[baseline=(current bounding box.north)]
        \begin{axis}[
          domain=-5:5,
          ymax=5,
          ymin=-5,
          xtick = {-4,...,4},
          axis lines =middle, xlabel=$x$, ylabel=$y$,
          every axis y label/.style={at=(current axis.above origin),anchor=south},
          every axis x label/.style={at=(current axis.right of origin),anchor=west}
          ]
          \addplot [very thick, penColor2, smooth, domain = -5:3, samples=100] {(abs(x-1)-1)*(1+x^2)/(1+0.4*x^2)};
          \addplot [very thick, penColor2, smooth, domain = 3:5] {-10/x};
          \addplot[color=penColor2,fill=background, only marks,mark=*] coordinates{(3,2.173)};
          \addplot[color=penColor2,fill=background, only marks,mark=*] coordinates{(3,-10/3)};
        \end{axis}
      \end{tikzpicture}
    \end{image}
  \end{minipage}
\end{frame}


\begin{frame}
  \vs{}
  \begin{example}
    Let $\ds f(x) = \frac{1}{1+x^2}$. Find the following for $f$:
  \end{example}
  \vs{}
  \begin{enumerate}
  \item $f'$ and $f''$ \vsone
  \item Critical points \vsone
  \item Local extrema \vsone
  \item Inflection points \vsone
  \end{enumerate}
\end{frame}



\begin{frame}
  \vs{}
  \begin{example}
    Sketch the plot of $2x^3-3x^2-12x$.
  \end{example}

  \note{
    \begin{image}[0.8\linewidth]
      \begin{tikzpicture}
        \begin{axis}[
          axis on top=true,
          domain=-2:4,
          xmin=-2,
          xmax=4,
          ymax=25,
          ymin=-25,
          axis lines =middle, xlabel=$x$, ylabel=$y$,
          every axis y label/.style={at=(current axis.above origin),anchor=south},
          every axis x label/.style={at=(current axis.right of origin),anchor=west}
          ]
          \addplot [->, line width=10, penColor!10!background] plot coordinates {(-2,-25) (-1,7)};
          \addplot [->, line width=10, penColor!10!background] plot coordinates {(-1,7) (2,-20)};
          \addplot [->, line width=10, penColor!10!background] plot coordinates {(2,-20) (4,25)};
          \addplot [dashed, penColor2] plot coordinates {(-1,-25) (-1,25)};
          \addplot [dashed, penColor2] plot coordinates {(2,-25) (2,25)};
          \addplot [dashed, penColor4] plot coordinates {(1/2,-25) (1/2,25)};
          \addplot [color=penColor,fill=penColor,only marks,mark=*] coordinates{(1/2,-6.5)};  %% closed hole
          \addplot [color=penColor,fill=penColor,only marks,mark=*] coordinates{(0,0)};  %% closed hole
          \addplot [color=penColor,fill=penColor,only marks,mark=*] coordinates{(-1,7)};  %% closed hole
          \addplot [color=penColor,fill=penColor,only marks,mark=*] coordinates{(2,-20)};  %% closed hole
          \addplot [color=penColor,fill=penColor,only marks,mark=*] coordinates{(-1.812,0)};  %% closed hole
          \addplot [color=penColor,fill=penColor,only marks,mark=*] coordinates{(3.312,0)};  %% closed hole
          \addplot [very thick, penColor, samples=100, smooth,domain=(-2:4)] {2*x^3-3*x^2-12*x};
        \end{axis}
      \end{tikzpicture}
    \end{image}
  }

\end{frame}


\begin{frame}
  \vs{}
  \begin{example}
    Sketch the plot of
    \[
      f(x) = \begin{cases} xe^x+2 &\text{if $x<0$} \\
        x^4-x^2+3 &\text{if $x \geq 0$}.
      \end{cases}
    \]
  \end{example}

  \note{
    \begin{image}
      \begin{tikzpicture}
        \begin{axis}[
          domain=-5:2,
          xmin=-5,
          xmax=2,
          ymax=5,
          ymin=-1,
          axis lines =middle, xlabel=$x$, ylabel=$y$,
          every axis y label/.style={at=(current axis.above origin),anchor=south},
          every axis x label/.style={at=(current axis.right of origin),anchor=west}
          ]
          \addplot [->, line width=4, penColor!10!background] plot coordinates {(-5,2) (-1,1.632)};
          \addplot [->, line width=4, penColor!10!background] plot coordinates {(-1,1.632) (0,2)};
          \addplot [->, line width=4, penColor!10!background] plot coordinates {(0,3) (.7,2.75)};
          \addplot [->, line width=4, penColor!10!background] plot coordinates {(.7,2.75) (2,5)};
          \addplot[color=penColor,fill = background, only marks,mark=*] coordinates{(0,2)};
          \addplot[color=penColor,fill=penColor,only marks,mark=*] coordinates{(0,3)};  %% closed hole
          \addplot [dashed, penColor2] plot coordinates {(-5,2) (0,2)};
          \addplot [dashed, penColor2] plot coordinates {(-1,-1) (-1,5)};
          \addplot [dashed, penColor2] plot coordinates {(0.70710678118,-1) (0.70710678118,5)};
          \addplot [dashed, penColor4] plot coordinates {(-2,-1) (-2,5)};
          \addplot [dashed, penColor4] plot coordinates {(0.40824829046,-1) (0.40824829046,5)};
          \addplot [very thick, penColor, samples=100, smooth,domain=(-5:0)] {x*e^x+2};
          \addplot [very thick, penColor, samples=100, smooth,domain=(0:2)] {x^4-x^2+3};
        \end{axis}
      \end{tikzpicture}
    \end{image}
  }
\end{frame}


\begin{frame}
  \frametitle{Summary}
  \begingroup
  \footnotesize
  The following is the list of all the tools at our finger tips to
  sketch the graph of $y = f(x)$
  \vs{}
  \begin{itemize}
    \only<1>{
    \item Compute $f'$ and $f''$.
    \item Find the $y$-intercept, this is the point $(0,f(0))$. Place this point on your graph.
    \item Find any vertical asymptotes, these are points $x=a$ where $f(x)$ goes to infinity as $x$ goes to $a$ (from the right, left, or both).
    \item If possible, find the $x$-intercepts, the points where $f(x) = 0$. Place these points on your graph.
    \item Analyze end behavior: as $x \to \pm \infty$, what happens to the graph of $f$?  Does it have horizontal asymptotes, increase or decrease without bound, or have some other kind of behavior?
    }
    \only<2>{
    \item Find the critical points (the points where $f'(x) = 0$ or $f'(x)$ is undefined).
    \item Use either the first or second derivative test to identify local extrema and/or find the intervals where your function is increasing/decreasing.
    \item Find the candidates for inflection points, the points where $f''(x) = 0$ or $f''(x)$ is undefined.
    \item Identify inflection points and concavity.
    \item Determine an interval that shows all relevant behavior.
    }
  \end{itemize}
  \endgroup
\end{frame}

\end{document}
%%% Local Variables:
%%% mode: latex
%%% TeX-master: t
%%% End:
