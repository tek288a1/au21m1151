% \documentclass[10pt,t,presentation,ignorenonframetext,aspectratio=169]{beamer}
\documentclass[10pt,t,handout,ignorenonframetext,aspectratio=169]{beamer}
\usepackage[default]{lato}
\usepackage{tk_beamer1}
\input{tk_packages}
\input{tk_macros}
\input{tk_environ}
\input{tk_ximera}
\usepackage{wasysym}            % for smiley
\newcommand{\zoz}{$\mathbf{ \frac{0}{0} }$}

% some inverse trigs
\DeclareMathOperator{\arcsec}{arcsec}
\DeclareMathOperator{\arccot}{arccot}
\DeclareMathOperator{\arccsc}{arccsc}

%%%% META DATA
\newcommand{\semester}{Autumn 2021}
\newcommand{\course}{Math 1151}
\newcommand{\lecTitle}{Lecture 39: The Idea of Substitution (TIOS)}

%%%% TITLE PAGE
\title[\course]{\lecTitle}
\institute[Ohio State]
{
  \medskip
}
\date[\week]{\semester}
\author{Tae Eun Kim, Ph.D.}

\begin{document}
\begin{frame}
  \titlepage
\end{frame}

\begin{frame}
  \frametitle{Two Sides of a Coin}
  \begingroup
  \small

  Recall that from the chain rule that
  \[
    \ddx f(g(x)) = f'(g(x)) g'(x).
  \]

  So by the fundamental theorem of calculus, we have
  \begin{equation*}
    \int_a^b f'(g(x))g'(x) \d x
    = \eval{f(g(x))}_a^b
    = f(g(b)) - f(g(a)).
  \end{equation*}

  Using the fundamental theorem in reverse direction once again, the
  last line can be thought of as
  \[
    \eval{f(u)}_{g(a)}^{g(b)} = \int_{g(a)}^{g(b)}f'(u) \d u.
  \]

  This is the gist of the integration technique known as
  \textbf{substitution rule} or
  \textbf{$u$-substitution}\footnote{{\footnotesize This name is due to a popular and customary
    choice of substitution variable $u$. The choice, however, is not
    an absolute rule written on a stone. Any variable of your choice
    such as $v$ or $\smiley$ works if used consistently.}}.
  \endgroup
\end{frame}


\begin{frame}
  \frametitle{Substitution Rule}
  \begin{thm}[Integral Substitution Formula]
    If $g$ is differentiable on the interval $[a,b]$ and $f$ is
    differentiable on the interval $[g(a),g(b)]$, then
    \[
      \int_a^b f'(g(x)) g'(x) \d x =\int_{g(a)}^{g(b)} f'(u) \d u.
    \]
  \end{thm}
  In sum, the substitution rule is the integral counterpart of
  differential chain rule and the fundamental theorem of calculus
  serves as a bridge between the two.
\end{frame}

\begin{frame}
  \frametitle{Procedures}
  In integrating a function which we suspect to be the derivative of
  another obtained by the chain rule:
  \begin{enumerate}
  \item Look for a candidate for the inner function; call it $u$.
  \item Rewrite the given function completely in terms of $u$ leaving
    no trace of the original variable.
  \item Integrate this new function of $u$. (If necessary, you may
    need to go back to Step 1 and modify your choice of $u$.)
  \item In dealing with an indefinite integral, make sure to replace
    $u$ by the equivalent expression of the original variable.
  \item Working with a definite integral, you may evaluate the result
    of Step 3 at the transformed bounds of $u$ or evaluate the
    antiderivate obtained in Step 4 at the original bounds.
  \end{enumerate}
\end{frame}

\begin{frame}
  \vs
  \begin{question}
    Compute $\ds \int_1^3 x \cos(x^2) \d x$.
  \end{question}
  % \vs
  % \begin{example}
  %   Compute the following integrals.
  %   \begin{enumerate}
  %   \item $\ds \int_1^3 x \cos(x^2) \d x$ \vfill
  %   \item $\ds \int \sec^2(x) \tan(x) \d x$ \vfill
  %   \end{enumerate}
  % \end{example}
\end{frame}

\begin{frame}
  \vs
  \begin{question}
    Compute $\ds \int \sec^2(x) \tan(x) \d x$.
  \end{question}
\end{frame}

% \begin{frame}
%   \vs
%   \begin{example}
%     Compute the following integrals.
%     \begin{enumerate}
%     \item $\ds \int x^4(x^5+1)^9 \d x$ \vfill
%     \item $\ds \int_{\pi/3}^{\pi/2} \sin(x) \sec^2(\cos(x)) \d x$ \vfill
%     \end{enumerate}
%   \end{example}
% \end{frame}

\begin{frame}
  \vs
  \begin{question}
    Compute $\ds \int x^4(x^5+1)^9 \d x$.
  \end{question}
\end{frame}

\begin{frame}
  \vs
  \begin{question}
    Compute $\ds \int_{\pi/3}^{\pi/2} \sin(x) \sec^2(\cos(x)) \d x$.
  \end{question}
\end{frame}

% \begin{frame}
%   \vs
%   \begin{example}
%     Compute the following integrals.
%     \begin{enumerate}
%     \item $\ds \int_{-2}^1 t^2 \sin(t^3) \d t$ \vfill
%     \item $\ds \int_0^{1/2} \frac{13e^x}{3e^x-5} \d x$ \vfill
%     \end{enumerate}
%   \end{example}
% \end{frame}

\begin{frame}
  \vs
  \begin{question}
    Compute $\ds \int_{-2}^1 t^2 \sin(t^3) \d t$.
  \end{question}
\end{frame}

\begin{frame}
  \vs
  \begin{question}
    Compute $\ds \int_0^{1/2} \frac{13e^x}{3e^x-5} \d x$.
  \end{question}
\end{frame}


\end{document}
%%% Local Variables:
%%% mode: latex
%%% TeX-master: t
%%% End:
