\documentclass[10pt,t,presentation,ignorenonframetext,aspectratio=169]{beamer}
% \documentclass[10pt,t,handout,ignorenonframetext,aspectratio=169]{beamer}
\usepackage[default]{lato}
\usepackage{tk_beamer1}
\input{tk_packages}
\input{tk_macros}
\input{tk_environ}
\input{tk_ximera}
\usepackage{wasysym}            % for smiley

%%%% META DATA
\newcommand{\semester}{Autumn 2021}
\newcommand{\course}{Math 1151}
\newcommand{\lecTitle}{Lecture 3: Limit Laws (LL)}

%%%% TITLE PAGE
\title[\course]{\lecTitle}
\institute[Ohio State]
{
  \medskip
}
\date[\week]{\semester}
\author{Tae Eun Kim, Ph.D.}

\begin{document}
\begin{frame}
  \titlepage
\end{frame}

% \begin{frame}
%   \frametitle{Weekly Overview}
%   \tableofcontents
% \end{frame}

% \section{Limit Laws (LL)}
\begin{frame}
  \frametitle{The limit laws}
  \begin{itemize}
  \item Recall the definition of \textbf{continuity}: $f$ is continuous at $a$ if
    \[
      \lim_{x \to a} f(x) = f(a) \,.
    \]
  \item In other words, continuity of a function allows us to calculate its
    limits simply by function evaluation.
  \item In addition, we learned that many famous functions are
    continuous on their natural domains.
  \item Today, using limit laws, we can expand the library of
    continuous functions even further.
  \end{itemize}
\end{frame}

\begin{frame}
  \vs
  \begin{theorem}[Limit laws]
    Suppose that $\ds \lim_{x \to a} f(x) = L$ and
    $\ds \lim_{x \to a} g(x) = M$, i.e., these limits exist.
    \begin{itemize}
    \item \textbf{Sum/Difference Law}: $\ds \lim_{x \to a} \left( f(x) \pm
        g(x) \right) = L \pm M$.
    \item \textbf{Product Law}: $\ds \lim_{x \to a} \left( f(x) g(x)
      \right) = L M$.
    \item \textbf{Quotient Law}: $\ds \lim_{x \to a} \frac{f(x)}{g(x)} = \frac{L}{M}$,
      provided that $M \neq 0$.
    \end{itemize}
  \end{theorem}
  \vfill
  \begin{rmk}
    Using these laws, we can show that polynomial and rational
    functions are also continuous on their natural domains.
  \end{rmk}
\end{frame}

\begin{frame}
  \vs
  \question{} Compute the following limit using limit laws:
  \[
    \lim_{x \to 2} (5x^2 + 3x - 2)
  \]
\end{frame}

\begin{frame}
  \vs
  \question{} Compute the following limit using limit laws:
  \[
    \lim_{x\to 1} \frac{x^2-3x+2}{x-2}
  \]
  Where is $\ds f(x) = \frac{x^2 -3x + 2}{x - 2}$ continuous?
  \vfill
\end{frame}

\begin{frame}
  \vs
  \begin{thm}[Composition limit law]
    If $f(x)$ is continuous at $\ds b = \lim_{x \to a} g(x)$, then
    \[
      \lim_{x \to a} f(g(x)) = f( \lim_{x \to a} g(x) ) \,.
    \]
    Consequently, if $g$ is continuous at $x=a$, and if $f$ is
    continuous at $g(a)$, then $f\circ g$ is continuous at $x=a$.
  \end{thm}
  \vs
  \question{} Compute the following limit using limit laws:
  \[
    \lim_{x \to 0} \sqrt{\cos(x)}
  \]
\end{frame}

\begin{frame}
  \vs
  \question{} Determine if the following limits can be directly
  computed using limit laws.
  \begin{enumerate}
  \item $\ds \lim_{x \to 2} \frac{x^2 - 3x + 2}{x - 2}$ \vfill
  \item $\ds \lim_{x \to 0} \frac{2^x - 1}{3^{x-1}}$ \vfill
  \item $\ds \lim_{x \to 0} x \sin(1/x)$ \vfill
  \item $\ds \lim_{x \to 0} \cot (x^3)$ \vfill
  \item $\ds \lim_{x \to 0} (1 + x)^{1/x}$ \vfill
  \end{enumerate}
\end{frame}

\begin{frame}
  \frametitle{The Squeeze Theorem}
  \begin{thm}[The Squeeze Theorem]
    Suppose that
    \[
      g(x) \le f(x) \le h(x)
    \]
    for all $x$ close to $a$ but not necessarily equal to $a$. If
    \[
      \lim_{x \to a} g(x) = L = \lim_{x \to a} h(x) \,,
    \]
    then $\lim_{x \to a} f(x) = L$.
  \end{thm}
  \begin{itemize}
  \item This theorem is often called the \textbf{sandwich theorem}.
  \end{itemize}
\end{frame}

\begin{frame}
  \vs
  \question{} Suppose we have a function $f(x)$ defined for all $x$ in
  the open interval $(-2, 2)$ and all I know about $f$ is that
  \[
    0 \le f(x) \le x^2 \,,
  \]
  in the interval. Can I say anything about $\lim_{x \to 0} f(x)$ with
  this limited knowledge?
\end{frame}

\begin{frame}
  \vs
  \question{} Consider the three functions, $g$, $f$, and $h$,
  defined on the interval $(-2, 2)$. Given that
  \[
    g(x) = \cos(\pi x), \quad
    h(x) = x^{2} + 1 \, \quad \text{and} \;
    g(x) \le f(x) \le h(x),
  \]

  \begin{minipage}[t]{0.4\linewidth}
    \begin{enumerate}
    \item Sketch and label the graph of $g$ and $h$, and a possible
      graph of $f$.
    \item Use the Squeeze Theorem to evaluate $\ds \lim_{x \to 0} f(x)$.
    \end{enumerate}
  \end{minipage}
  \hfill
  \begin{minipage}[t]{0.55\linewidth}
    \begin{image}[2in]
      \begin{tikzpicture}
        \begin{axis}[
          xmin=-2.5, xmax=2.5, ymin=-1.2,ymax=2.2,
          unit vector ratio*=1 1 1,
          axis lines=middle, xlabel=$x$, ylabel=$y$,
          every axis y label/.style={at=(current axis.above origin),anchor=south},
          every axis x label/.style={at=(current axis.right of origin),anchor=west},
          xtick={-2, -1, 0, 1, 2},
          ytick={-1, 1, 2},
          grid=major,
          width=3in,
          grid style={dashed, gridColor},
          ]
        \end{axis}
      \end{tikzpicture}
    \end{image}
  \end{minipage}
\end{frame}

\begin{frame}
  \vs
  \question{} Compute $\ds \lim_{\theta \to 0}
  \frac{\sin(\theta)}{\theta}$.

  \vs
  \begin{itemize}
  \item The answer is $1$!
  \item Please read the textbook for a detailed solution.
  \item Later in the course, we will learn an alternate method to
    calculate this limit.
  \end{itemize}

\end{frame}
\end{document}


%%% Local Variables:
%%% mode: latex
%%% TeX-master: t
%%% End:
