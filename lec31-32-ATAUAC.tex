% \documentclass[10pt,t,presentation,ignorenonframetext,aspectratio=169]{beamer}
\documentclass[10pt,t,handout,ignorenonframetext,aspectratio=169]{beamer}
\usepackage[default]{lato}
\usepackage{tk_beamer1}
\input{tk_packages}
\input{tk_macros}
\input{tk_environ}
\input{tk_ximera}
\usepackage{wasysym}            % for smiley
\newcommand{\zoz}{$\mathbf{ \frac{0}{0} }$}

% some inverse trigs
\DeclareMathOperator{\arcsec}{arcsec}
\DeclareMathOperator{\arccot}{arccot}
\DeclareMathOperator{\arccsc}{arccsc}

%%%% META DATA
\newcommand{\semester}{Autumn 2021}
\newcommand{\course}{Math 1151}
\newcommand{\lecTitle}{Lecture 31-32: Approximating the Area \\ Under a Curve (ATAUAC)}

%%%% TITLE PAGE
\title[\course]{\lecTitle}
\institute[Ohio State]
{
  \medskip
}
\date[\week]{\semester}
\author{Tae Eun Kim, Ph.D.}

\begin{document}
\begin{frame}
  \titlepage
\end{frame}


\begin{frame}
  \frametitle{Introduction to Sigma Notation}
  \begingroup
  \small
  \textit{Sigma notation} is a way of writing a sum of many terms in a
  concise form. For example,
  \[
    \sum_{k=1}^{5} 3k \,.
  \]
  \begin{itemize}
  \item The $\Sigma$ (sigma) indicates that a sum is being taken.
  \item The variable $k$ is called the (summation) \textit{index}.
  \item The numbers at the top and bottom of the $\Sigma$ are called the
    \textit{upper and lower limits} of the summation.
  \item The expression following $\Sigma$ is called the \textit{summand} formula
    which gives a recipe for the terms to be added up.
  \item The notation altogether means that we will take every integer
    value of $k$ between 1 and 5, plug them each into $3k$, and then
    add them all up:
    \[
      \sum_{k=1}^{5} 3k = 3\cdot 1 + 3\cdot 2 + 3\cdot 3 + 3\cdot 4 + 3\cdot 5 = 45 \,.
    \]
  \item The index does not have to be $k$. Popular choices include
    $i, j, k, m$, and $n$. For instance:
  \end{itemize}
  \endgroup
\end{frame}

\begin{frame}
  \vs
  \begin{question}
    Write out what is meant by the following:
    \[
      \sum_{k=0}^{3} \frac{1}{k+1} \,.
    \]
  \end{question}
  \vfill
  \begin{question}
    Write out what is meant by the following:
    \[
      \sum_{i=1}^{8} \left(-1\right)^i \,.
    \]
  \end{question}
\end{frame}

\begin{frame}
  \vs
  \begingroup
  \small
  Now let's work backward.

  \vs
  \begin{question}
    Write the following sum in sigma notation.
    \[
      2 + 4 + 6 + 8 + \ldots + 22 + 24
    \]
  \end{question}
  \vfill
  \begin{question}
    Write the following sum in sigma notation.
    \[
      1 - \frac{1}{2} + \frac{1}{4} - \frac{1}{8} +
      \cdots + \frac{1}{64} - \frac{1}{128}
    \]
  \end{question}
  \vfill
  \begin{question}
    Write the following sum in sigma notation.
    \[
      \frac{1}{\sqrt{2}} - \frac{2}{\sqrt{3}} + \frac{3}{\sqrt{4}} -
      \frac{4}{\sqrt{5}}+ \ldots + \frac{51}{\sqrt{52}} -
      \frac{52}{\sqrt{53}}
    \]
  \end{question}
  \vfill
  \endgroup
\end{frame}

\begin{frame}
  \frametitle{Calculating with Sigma Notation}
  The following formulas written in terms of sigma notation will be useful.

  \begin{description}
  \item \textbf{Formula 1.} $\displaystyle \sum_{k=1}^n C = nC$.
  \item \textbf{Formula 2.} $\displaystyle \sum_{k=1}^n k = \frac{n(n+1)}{2}$.
  \item \textbf{Formula 3.} $\displaystyle \sum_{k=1}^n k^2 = \frac{n(n+1)(2n+1)}{6}$.
  \item \textbf{Formula 4.} $\displaystyle \sum_{k=1}^{n} (a_k + b_k) = \sum_{k=1}^n a_k + \sum_{k=1}^n b_k$.
  \item \textbf{Formula 5.} $\displaystyle \sum_{k=1}^{n} c \cdot a_k = c \sum_{k=1}^n a_k$.
  \end{description}
\end{frame}

\begin{frame}
  \vs
  \begin{question}
    Find the value of the sum
    $\displaystyle \sum_{k=1}^{10} \left(2k^2+5\right)$.
  \end{question}

  \vfill
  \begin{question}
    Find the value of the sum
    $\displaystyle \sum_{k=1}^{200} \left(-6k^2+3\right)$.
  \end{question}
\end{frame}



\begin{frame}
  \frametitle{Rectangles and Areas}

  Our goal here is to compute the area between the curve $y = f(x)$
  and the horizontal axis on the interval $[a,b]$.

  \begin{image}[0.45\linewidth]
    \begin{tikzpicture}[
      declare function = {f(\x) = -sin(deg(\x)) + 3;}]
      \begin{axis}[
        domain=-.2:7, xmin =-.2,xmax=7,ymax=5,ymin=-.2,
        width=6in,
        height=3in,
        xtick={1,6},
        xticklabels={$a$,$b$},
        ytick style={draw=none},
        yticklabels={},
        axis lines=center, xlabel=$x$, ylabel=$y$,
        every axis y label/.style={at=(current axis.above origin),anchor=south},
        every axis x label/.style={at=(current axis.right of origin),anchor=west},
        axis on top,
        ]
        \addplot [draw=none,fill=fillp,domain=1:6, smooth] {f(x)}
        \closedcycle; \addplot [very thick,penColor, smooth] {f(x)};
      \end{axis}
    \end{tikzpicture}
  \end{image}

  We approximate this area with $n$ rectangles of equal width
  $\Delta x = (b-a)/n$ as follows:

  \begin{image}[0.45\linewidth]
    \begin{tikzpicture}[
      declare function = {f(\x) = -sin(deg(\x)) + 3;}]
      \begin{axis}[
        domain=-.2:7, xmin =-.2,xmax=7,ymax=5,ymin=-.7,
        width=6in,
        height=3in,xtick={1,6},
        xticklabels={$a$,$b$},
        ytick style={draw=none},
        yticklabels={},
        axis lines=center, xlabel=$x$, ylabel=$y$,
        every axis y label/.style={at=(current axis.above origin),anchor=south},
        every axis x label/.style={at=(current axis.right of origin),anchor=west},
        axis on top,
        ]
        \foreach \rectnumber in {1,2,...,5}
        {
          \addplot [draw=penColor,fill=fillp] plot coordinates
          {({1+(\rectnumber - 1) * 1},{f(1+\rectnumber * 1)})
            ({1+(\rectnumber) * 1},{f(1+\rectnumber * 1) })} \closedcycle;

          \addplot[decoration={brace,mirror,raise=.2cm},decorate,thin] plot coordinates
          {(\rectnumber+.1,0)
            (1+\rectnumber-.1,0)};
        };
        \addplot [very thick,penColor, smooth] {f(x)};
        \node at (axis cs:1.5,-.5) {$\Delta x$};
        \node at (axis cs:2.5,-.5) {$\Delta x$};
        \node at (axis cs:3.5,-.5) {$\Delta x$};
        \node at (axis cs:4.5,-.5) {$\Delta x$};
        \node at (axis cs:5.5,-.5) {$\Delta x$};
      \end{axis}
    \end{tikzpicture}
  \end{image}
\end{frame}

\begin{frame}
  \frametitle{More Rectangles}

  As we divide the interval $[a,b]$ into finer and finer subintervals
  and construct more approximating rectangles, we are more closely
  approximating the exact area we are interested in:

  \begin{image}
    \begin{tikzpicture}[
      declare function = {f(\x) = -sin(deg(\x)) + 3;}]
      \begin{axis}[
        domain=-.2:7, xmin =-.2,xmax=7,ymax=5,ymin=-.7,
        width=6in,
        height=3in,xtick={1,6},
        xticklabels={$a$,$b$},
        ytick style={draw=none},
        yticklabels={},
        axis lines=center, xlabel=$x$, ylabel=$y$,
        every axis y label/.style={at=(current axis.above origin),anchor=south},
        every axis x label/.style={at=(current axis.right of origin),anchor=west},
        axis on top,
        ]
        \foreach \rectnumber in {1,2,...,30}
        {
          \addplot [draw=penColor,fill=fillp] plot coordinates
          {({1+(\rectnumber - 1) * 5/30},{f(1+\rectnumber * 5/30)})
            ({1+(\rectnumber) * 5/30},{f(1+\rectnumber * 5/30) })} \closedcycle;
        };
        \addplot [very thick,penColor, smooth] {f(x)};
      \end{axis}
    \end{tikzpicture}
  \end{image}
\end{frame}


\begin{frame}
  \frametitle{Key Idea}
  We could find the area exactly if we could compute the
  limit as the width of the rectangles goes to zero and the number of
  rectangles goes to infinity.

  In order to make this argument precise, we need to establish some
  notations to help with calculation.

  \begin{itemize}
  \item When approximating an area with $n$ rectangles, the \textbf{grid points}
    \[
      x_0, x_1, x_2, \ldots, x_n
    \]
    are the $x$-coordinates that determine the edges of the
    rectangles. In particular,
    \[
      x_k = a + k \Delta x \,.
    \]
  \item When approximating an area with rectangles, a \dfn{sample point} is
    the $x$-coordinate that determines the height of $k^{th}$
    rectangle. For $k=1,..., n$, we denote a sample point as $x_k^*$
    and the value $f( x_k^*)$ is the height of the $k^{th}$
    rectangle.
  \item We will use either left-endpoints, right-endpoints, or midpoints as
    sample points.
  \end{itemize}
\end{frame}


\begin{frame}
  \frametitle{Riemann Sums and Approximating Area}
  We now have a general formula for approximation of area with $n$ rectangles:
  \begin{align*}
    \text{Area} &\approx \sum_{k=1}^n (\text{height of $k$th rectangle})\times(\text{width of $k$th rectangle}) \\
                &=  f(x_1^*)\Delta x +  f(x_2^*)\Delta x +   f(x_3^*)\Delta x + \dots +   f(x_n^*)\Delta x
                  =\sum_{k=1}^n  f(x_k^*)\Delta x \,.
  \end{align*}

  This approximating sum is called a \textbf{Riemann sum} of $f$ on the
  interval $[a, b]$.
\end{frame}

\begin{frame}
  \frametitle{Schematic}
  The following is a schematic of a left Riemann sum
  where sample points are collected from left-endpoints of each
  subinterval:

  \begin{image}
    \begin{tikzpicture}[
      declare function = {f(\x) = -sin(deg(\x)) + 3;}]
      \begin{axis}[
        domain=-.2:7, xmin =-.2,xmax=7,ymax=5,ymin=-.7,
        width=6in,
        height=3in,
        xtick style={draw=none},
        xticklabels={},
        ytick style={draw=none},
        yticklabels={},
        axis lines=center, xlabel=$x$, ylabel=$y$,
        every axis y label/.style={at=(current axis.above origin),anchor=south},
        every axis x label/.style={at=(current axis.right of origin),anchor=west},
        axis on top,
        ]
        \foreach \rectnumber in {1,2,...,5}
        {
          \addplot [draw=penColor,fill=fillp] plot coordinates
          {({1+\rectnumber-1) * 5/5},{f(1+(\rectnumber-1) * 5/5)})
            ({(1+\rectnumber) * 5/5},{f(1+(\rectnumber-1) * 5/5) })} \closedcycle;
          \addplot[decoration={brace,mirror,raise=.2cm},decorate,thin] plot coordinates
          {(\rectnumber+.1,0)
            (1+\rectnumber-.1,0)};
        };
        \addplot [very thick,penColor, smooth] {f(x)};
        \addplot[decoration={brace,raise=.2cm},decorate,thin] plot coordinates
        {(.95,.1) (.95,{f(1)-.1})};
        \addplot[decoration={brace,mirror,raise=.2cm},decorate,thin] plot coordinates
        {(6.05,.1) (6.05,{f(5)-.1})};
        \node at (axis cs:1.5,-.5) {$\Delta x$};
        \node at (axis cs:2.5,-.5) {$\Delta x$};
        \node at (axis cs:3.5,-.5) {$\Delta x$};
        \node at (axis cs:4.5,-.5) {$\Delta x$};
        \node at (axis cs:5.5,-.5) {$\Delta x$};

        \node at (axis cs:1.5,1) {\large$f(x_1^*)\Delta x$};
        \node at (axis cs:2.5,1) {\large$f(x_2^*)\Delta x$};
        \node at (axis cs:3.5,1) {\large$f(x_3^*)\Delta x$};
        \node at (axis cs:4.5,1) {\large$f(x_4^*)\Delta x$};
        \node at (axis cs:5.5,1) {\large$f(x_5^*)\Delta x$};

        \node[anchor=east] at (axis cs:.8,{f(1)/2}) {$f(x_1^*)$};
        \node[anchor=west] at (axis cs:6.2,{f(5)/2}) {$f(x_5^*)$};

        \node[anchor=south] at (axis cs:1,{f(1)}) {$f(x_1^*)$};
        \node[anchor=south] at (axis cs:2,{f(2)}) {$f(x_2^*)$};
        \node[anchor=south] at (axis cs:3,{f(3)}) {$f(x_3^*)$};
        \node[anchor=south] at (axis cs:4,{f(4)}) {$f(x_4^*)$};
        \node[anchor=south] at (axis cs:5,{f(5)}) {$f(x_5^*)$};
      \end{axis}
    \end{tikzpicture}
  \end{image}

  The associated Riemann sum is
  \[
    \sum_{k=1}^5  f(x_k^*)\Delta x  = f(x_1^*)\Delta x +  f(x_2^*)\Delta x +   f(x_3^*)\Delta x + f(x_4^*)\Delta x + f(x_5^*)\Delta x.
  \]
\end{frame}


\end{document}
%%% Local Variables:
%%% mode: latex
%%% TeX-master: t
%%% End:
