\documentclass[10pt,t,presentation,ignorenonframetext,aspectratio=169]{beamer}
% \documentclass[10pt,t,handout,ignorenonframetext,aspectratio=169]{beamer}
\usepackage[default]{lato}
\usepackage{tk_beamer1}
\input{tk_packages}
\input{tk_macros}
\input{tk_environ}
\input{tk_ximera}
\usepackage{wasysym}            % for smiley
\newcommand{\zoz}{$\mathbf{ \frac{0}{0} }$}

%%%% META DATA
\newcommand{\semester}{Autumn 2021}
\newcommand{\course}{Math 1151}
\newcommand{\lecTitle}{Lecture 9: Derivatives as Functions (DAF)}

%%%% TITLE PAGE
\title[\course]{\lecTitle}
\institute[Ohio State]
{
  \medskip
}
\date[\week]{\semester}
\author{Tae Eun Kim, Ph.D.}

\begin{document}
\begin{frame}
  \titlepage
\end{frame}

\begin{frame}
  \frametitle{The derivative as a function}
  \begin{itemize}
  \item Recall from last time that the \textbf{derivative} of a
    function $f$ at a point $a$ is given by
    \[
      f'(a) = \lim_{h \to 0} \frac{f(a+h)-f(a)}{h} \,.
    \]
  \item If we replace $a$ by a variable $x$, we now have the following function:
    \[
      f'(x) = \lim_{h \to 0} \frac{f(x+h)-f(x)}{h} \,.
    \]
  \item This function gives us the \textit{instantaneous rate of change} at
    any variable point $x$. Calculation of this derived function is
    called \textbf{differentiation}.
  \item \textbf{Notation.}
    \[
      f'(x) = \ddx f(x) .
    \]
  \end{itemize}
\end{frame}


\begin{frame}
  \frametitle{Differentiability implies continuity}
  Differentiability is a property that is stronger than continuity.

  \begin{thm}[Differentiability implies continuity]
    If $f$ is a differentiable function at $a$, then $f$ is continuous at $a$.
  \end{thm}

  \vfill
  \textbf{Notes.}
  \begin{itemize}
  \item The contrapositive of the theorem is stated as follows:
    \begin{quote}
      If $f$ is not continuous at $a$, then $f$ is not
      differentiable at $a$.
    \end{quote}
  \item Consequently, all differentiable functions are continuous, but
    not all continuous functions are differentiable.
    % \item A famous and important example of continuous but non-differentiable function is the absolute value function, $f(x) = |x|$.
  \end{itemize}
  % Assuming that $f'(a)$ exists, we need to show that $f$ is continuous at $a$, i.e., we need to show that
  % \[
  %   \lim_{x \to a} f(x) = f(a) \,.
  % \]
  % To this end, we will show that $\lim_{x \to a} (f(x) - f(a)) = 0$. (Why?)
  % \begin{align*}
  %   \lim_{x\to a} \left(f(x) - f(a)\right)
  %   &= \lim_{x\to a} \left((x-a)\frac{f(x) - f(a)}{x-a}\right) \\
  %   &= \left(\lim_{x\to a} (x-a) \right) \left(\lim_{x\to a}\frac{f(x) - f(a)}{x-a}\right) &\text{Limit Law.} \\
  %   &= 0 \cdot f'(a) = 0 \,.
  % \end{align*}
\end{frame}


\begin{frame}
  \vs
  \question{}
  Which of the following functions are continuous but not
  differentiable on $\RR$?
  \begin{multicols}{2}
    \begin{enumerate}
    \item {$x^2$}
    \item {$\lfloor x \rfloor$}
    \item {$|x|$}
    \item {$\ds \frac{\sin(x)}{x}$}
    \end{enumerate}
  \end{multicols}
\end{frame}

\begin{frame}
  \vs
  \begin{question}
    Consider
    \[
      f(x) = \begin{dcases}
        x^2 &\text{if $x<3$,}\\
        mx+b &\text{if $x\ge 3$.}
      \end{dcases}
    \]
    What values of $m$ and $b$ make $f$ differentiable at $x=3$?
  \end{question}
\end{frame}

\end{document}
%%% Local Variables:
%%% mode: latex
%%% TeX-master: t
%%% End:
