\documentclass[10pt,t,presentation,ignorenonframetext,aspectratio=169]{beamer}
% \documentclass[10pt,t,handout,ignorenonframetext,aspectratio=169]{beamer}
\usepackage[default]{lato}
\usepackage{tk_beamer1}
\input{tk_packages}
\input{tk_macros}
\input{tk_environ}
\input{tk_ximera}
\usepackage{wasysym}            % for smiley
\newcommand{\zoz}{$\mathbf{ \frac{0}{0} }$}

%%%% META DATA
\newcommand{\semester}{Autumn 2021}
\newcommand{\course}{Math 1151}
\newcommand{\lecTitle}{Lecture 5: Using Limites to Detect Asymptotes (ULTDA)}

%%%% TITLE PAGE
\title[\course]{\lecTitle}
\institute[Ohio State]
{
  \medskip
}
\date[\week]{\semester}
\author{Tae Eun Kim, Ph.D.}

\begin{document}
\begin{frame}
  \titlepage
\end{frame}

% \begin{frame}
%   \frametitle{Weekly Overview}
%   \tableofcontents
% \end{frame}

% \section{Using Limits to Detect Asymptotes (ULTDA)}
\begin{frame}
  \frametitle{Vertical asymptotes -- infinite limits} Consider the
  graphs of the following two functions near $x = 1$:
  \[
    f(x) = \frac{1}{(x-1)^2}
    \quad \text{and} \quad
    g(x) = \frac{1}{x-1} \,.
  \]

  \begin{image}[0.75\textwidth]
    \begin{tabular}{cc}
      \begin{tikzpicture}
        \begin{axis}[
          domain=-1:2,
          ymax=100,
          samples=100,
          axis lines =middle, xlabel=$x$, ylabel=$y$,
          every axis y label/.style={at=(current axis.above origin),anchor=south},
          every axis x label/.style={at=(current axis.right of origin),anchor=west}
          ]
          \addplot [very thick, penColor, smooth, domain=(-1:0.9)] {1/(x-1)^2};
          \addplot [very thick, penColor, smooth, domain=(1.1:2)] {1/(x-1)^2};
          \addplot [textColor, dashed] plot coordinates {(1,0) (1,100)};
        \end{axis}
      \end{tikzpicture}
      &
        \begin{tikzpicture}
          \begin{axis}[
            domain=-1:2,
            ymax=50,
            ymin=-50,
            samples=100,
            axis lines =middle, xlabel=$x$, ylabel=$y$,
            every axis y label/.style={at=(current axis.above origin),anchor=south},
            every axis x label/.style={at=(current axis.right of origin),anchor=west}
            ]
            \addplot [very thick, penColor, smooth, domain=(1.02:2)] {1/(x-1)};
            \addplot [very thick, penColor, smooth, domain=(-1:.98)] {1/(x-1)};
            \addplot [textColor, dashed] plot coordinates {(1,-50) (1,50)};
          \end{axis}
        \end{tikzpicture}
    \end{tabular}
  \end{image}

  \begin{itemize}
  \item In both cases, the graphs get closer and closer to the
    vertical line $x = 1$, but they never touch it.
  \item  Such a line is called a vertical asymptote.
  \end{itemize}
\end{frame}

\begin{frame}
  \vs
  \begin{defn}
    If at least one of the following holds:
    \begin{itemize}
    \item $\lim_{x\to a} f(x) = \pm\infty$,
    \item $\lim_{x\to a^+} f(x) = \pm\infty$,
    \item $\lim_{x\to a^-} f(x) = \pm\infty$,
    \end{itemize}
    then the line $x=a$ is a \dfn{vertical asymptote} of $f$.
  \end{defn}
\end{frame}

\begin{frame}
  \vs
  \begin{question}
    Find the vertical asymptotes of
    \[
      f(x) = \frac{ x^2-9x+14 }{ x^2-5x+6 } \,.
    \]
  \end{question}
\end{frame}

\begin{frame}
  \vs
  \begin{question}
    Find the vertical asymptotes of
    \[
      f(x) = \frac{ \sqrt{x^2-3x+2} }{ x-2 }
      \,, \quad x > 2 \,.
    \]
  \end{question}
\end{frame}

\begin{frame}
  \frametitle{Horizontal asymptotes -- limits at infinity}

  % To determine whether a function has horizontal asymptotes, we need
  % to investigate its behavior as $x$ becomes large \textit{in magnitude},
  % which can be compactly expressed as $x \to \pm \infty$.

  \begin{defn}
    \begin{itemize}
    \item If $f(x)$ becomes arbitrarily close to a specific value $L$
      by making $x$ sufficiently large, we write
      \[
        \lim_{x \to \infty} f(x) = L
      \]
      and we say that the \textbf{limit at infinity} of $f(x)$ is $L$.
    \item If $f(x)$ becomes arbitrarily close to a specific value $L$
      by making $x$ sufficiently large and negative, we write
      \[
        \lim_{x \to -\infty} f(x) = L
      \]
      and we say that the \textbf{limit at negative infinity} of $f(x)$ is $L$.
    \end{itemize}
  \end{defn}
\end{frame}

\begin{frame}
  \vs

  \textbf{Illustration.} The function $f(x) = 1/(x-1)$ once again
  provides us with a valuable insight:
  % \[
  %   \lim_{x \to \pm\infty} \frac{1}{x-1} = 0 \,.
  % \]

  \begin{image}[0.7\textwidth]
    \begin{tikzpicture}
      \begin{axis}[
        domain=-1:2,
        ymax=50,
        ymin=-50,
        samples=100,
        axis lines =middle, xlabel=$x$, ylabel=$y$,
        every axis y label/.style={at=(current axis.above origin),anchor=south},
        every axis x label/.style={at=(current axis.right of origin),anchor=west}
        ]
        \addplot [very thick, penColor, smooth, domain=(1.02:2)] {1/(x-1)};
        \addplot [very thick, penColor, smooth, domain=(-1:.98)] {1/(x-1)};
        \addplot [textColor, dashed] plot coordinates {(1,-50)
          (1,50)};
        \node at (axis cs:1.6,10) [penColor2] {\scriptsize $\ds\lim_{x\to \infty}
          f(x) = 0$};
        \node at (axis cs:-0.5,10) [penColor2] {\scriptsize $\ds\lim_{x\to -\infty}
          f(x) = 0$};
      \end{axis}
    \end{tikzpicture}
  \end{image}
\end{frame}

\begin{frame}
  \vs
  The graph suggests that having finite limits at infinity has a lot
  to do with horizontal asymptotes, thus the following definition:

  \begin{defn}
    If
    \[
      \lim_{x\to \infty} f(x) = L
      \qquad\text{or}\qquad
      \lim_{x\to -\infty} f(x) = L,
    \]
    then the line $y=L$ is a \textbf{horizontal asymptote} of $f(x)$.
  \end{defn}
\end{frame}

\begin{frame}
  \vs
  \begin{question}
    Find the horizontal asymptotes of
    \[
      f(x) = \frac{6x-9}{x-1}.
    \]
  \end{question}
\end{frame}

\begin{frame}
  \vs
  \begin{question}
    Find the horizontal asymptotes of
    \[
      f(x) = \frac{x^3+1}{\sqrt{x^6+6}}.
    \]
  \end{question}
\end{frame}

\begin{frame}
  \vs
  \begin{question}
    Compute
    \[
      \lim_{x\to \infty} \frac{\sin(7x)+4x}{x}.
    \]
  \end{question}
  \textit{Hint}. Use the squeeze theorem.
\end{frame}

\begin{frame}
  \vs
  For your viewing pleasure:
  \vfill
  \begin{image}[0.7\textwidth]
    \begin{tikzpicture}
      \begin{axis}[
        domain=2:20,
        ymax=5,
        ymin=3,
        samples=100,
        axis lines =middle, xlabel=$x$, ylabel=$y$,
        every axis y label/.style={at=(current axis.above origin),anchor=south},
        every axis x label/.style={at=(current axis.right of origin),anchor=west}
        ]
        \addplot [very thick, penColor, smooth] {(1/x) * sin(deg(7*x))+4};
        \addplot [textColor, dashed] plot coordinates {(0,4) (20,4)};
      \end{axis}
    \end{tikzpicture}
  \end{image}
\end{frame}


\end{document}


%%% Local Variables:
%%% mode: latex
%%% TeX-master: t
%%% End:
