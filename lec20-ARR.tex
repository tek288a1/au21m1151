% \documentclass[10pt,t,presentation,ignorenonframetext,aspectratio=169]{beamer}
\documentclass[10pt,t,handout,ignorenonframetext,aspectratio=169]{beamer}
\usepackage[default]{lato}
\usepackage{tk_beamer1}
\input{tk_packages}
\input{tk_macros}
\input{tk_environ}
\input{tk_ximera}
\usepackage{wasysym}            % for smiley
\newcommand{\zoz}{$\mathbf{ \frac{0}{0} }$}

% some inverse trigs
\DeclareMathOperator{\arcsec}{arcsec}
\DeclareMathOperator{\arccot}{arccot}
\DeclareMathOperator{\arccsc}{arccsc}

%%%% META DATA
\newcommand{\semester}{Autumn 2021}
\newcommand{\course}{Math 1151}
\newcommand{\lecTitle}{Lecture 20: Applied Related Rates (ARR)}

%%%% TITLE PAGE
\title[\course]{\lecTitle}
\institute[Ohio State]
{
  \medskip
}
\date[\week]{\semester}
\author{Tae Eun Kim, Ph.D.}

\begin{document}
\begin{frame}
  \titlepage
\end{frame}

\begin{frame}[t]
  \frametitle{Applied Related Rates Problems}
  \begin{block}{General procedures}
    \begin{enumerate}
    \item \textbf{Introduce variables and identify the given and unknown
        rates.} Assign a variable to each quantity that changes in time.
    \item \textbf{Draw a picture.} If possible, draw a schematic picture
      with all the relevant information.
    \item \textbf{Find equations.} Write equations that relate all
      relevant variables.
    \item \textbf{Differentiate with respect to time t.} Here we will
      often use implicit differentiation and obtain an equation that
      relates the given rate and the unknown rate.
    \item \textbf{Evaluate.} Evaluate each quantity at the relevant
      moment.
    \item \textbf{Solve.} Solve for the unknown rate at that moment.
    \end{enumerate}
  \end{block}
\end{frame}

%%%% Example 1
\begin{frame}[t]
  \vs
  \begin{block}{Example 1. (Cylindrical geometry)}
    A hand-tossed pizza crust starts off as a ball of dough with a volume of $400\pi\, \text{cm}^3$. First, the cook stretches the dough to the shape of a cylinder of radius $12$ cm. Next the cook tosses the dough.

    If during tossing, the dough maintains the shape of a cylinder and the radius is increasing at a rate of $15$ cm/min, how fast is its thickness changing when the radius is $20$ cm?
  \end{block}

  \begin{figure} \hfill
    \begin{tikzpicture}[scale=0.7, every node/.style={transform shape}]
      \draw[penColor,very thick] (0,0) ellipse (2.5 and .8);
      \draw[very thick,penColor] (-2.5,-.5) arc (180:360:2.5 and .8);% bottom
      \draw[penColor] (0,0) -- (2.5,0);
      \draw[penColor,very thick] (-2.5,0) -- (-2.5,-.5);
      \draw[penColor,very thick] (2.5,0) -- (2.5,-.5);
      \node[above,penColor] at (1,0) {$r$};
      \node[below,penColor] at (1,0) {$r' = 15$};
      \draw[decoration={brace,raise=.2cm},decorate,thin] (2.5,0)--(2.5,-.5);
      \node [penColor,right] at (3,-.25) {$h$};
      \node [penColor,left] at (-0.5,1.2) {$V = 400\pi$ cm$^3$};
      \node [penColor, right] at (0.5,-1.6) {$V = \pi\cdot r^2 \cdot h$ cm$^3$};
      \end{tikzpicture}
  \end{figure}
\end{frame}
% \emptyframe
\begin{frame}\end{frame}

%%%% Example 2
\begin{frame}[t]
  \vs
  \begin{block}{Example 2. (Spherical geometry)}
    Consider a melting snowball. We will assume that the rate at which the
    snowball is melting is proportional to its surface area. Show that
    the radius of the snowball is changing at a constant rate.
  \end{block}

  \begin{figure} \hfill
    \begin{tikzpicture}[scale=0.7, every node/.style={transform shape}]
      %\draw[penColor!50!background,very thick] (0,0) ellipse (2 and 1);
      \draw[very thick,penColor!20!background] (2,0) arc (0:180:2 and .7);% top half of ellipse
      \draw [penColor, very thick] (0,0) circle [radius=2];
      \draw[penColor] (0,0) -- (2,0);
      \node [below,penColor] at (1,0) {$r$ cm};
      \draw[very thick,penColor] (-2,0) arc (180:360:2 and .7);% bottom half of ellipse
      \node [penColor,right] at (1.3,1.65) {$V = \frac{4}{3}\cdot \pi \cdot r^3$};
      \node [penColor,right] at (1.3,-1.65) {$A = 4\cdot \pi \cdot r^2$};
    \end{tikzpicture}
  \end{figure}
\end{frame}
% \emptyframe
\begin{frame}\end{frame}

%%%% Example 3
\begin{frame}[t]
  \vs
  \begin{block}{Example 3. (Right triangles)}
    A road running north to south crosses a road going east to west at the
    point $P$.  Cyclist $A$ is riding north along the first road, and
    cyclist $B$ is riding east along the second road.  At a particular
    time, cyclist $A$ is $3$ kilometers to the north of $P$ and traveling
    at $20$ km/hr, while cyclist $B$ is $4$ kilometers to the east of $P$
    and traveling at $15$ km/hr.  How fast is the distance between the two
    cyclists changing at that time?
  \end{block}

  \begin{figure} \hfill
    \begin{tikzpicture}[scale=0.7, every node/.style={transform shape}]
      \draw[->,penColor!50!background, very thick] (-1,0) -- (4,0);
      \draw[->,penColor!50!background, very thick] (0,-1) -- (0,4);
      \draw[->,penColor, very thick] (0,3) -- (0,4);
      \draw[->,penColor, very thick] (3,0) -- (4,0);
      \draw [penColor, fill] (0,0) circle [radius=.07];
      \draw [penColor, fill] (3,0) circle [radius=.07];
      \draw [penColor, fill] (0,3) circle [radius=.07];
      \draw[dashed,penColor2, very thick] (3,0) -- (0,3);

      %\node[penColor,rotate=90,right] at (.5,3) {\scalebox{-2} \Bicycle};
      \node[penColor,right] at (0,.2) {$P$};
      \node[penColor,left] at (0,1.5) {$a(t)$ };
      \node[penColor,left] at (-.005,3) {$A$ };
      \node[penColor,below] at (1.5,0) {$b(t)$ };
      \node[penColor,below] at (3,0) {B};
      \node[penColor2,above] at (1.6,1.6) {$c(t)$};
      %\node[penColor,right,above] at (3.5,0) {\scalebox{-2}[2] \Bicycle};
    \end{tikzpicture}
  \end{figure}
\end{frame}
% \emptyframe
\begin{frame}\end{frame}


%%%% Example 4
\begin{frame}[t]
  \vs
  \begin{block}{Example 4. (Right triangles)}
    A plane is flying at an altitude of $3$ miles directly away from you at $500$ mph. How fast is the plane's distance from you increasing at the moment when the plane is flying over a point on the ground $4$ miles from you?
  \end{block}

  \begin{figure} \hfill
    \begin{tikzpicture}[scale=0.7, every node/.style={transform shape}]
      \draw[penColor2, dashed, very thick] (0,0) -- (5,4);
      %\draw[penColor, dashed, very thick] (0,0) -- (0,4);
      \draw[penColor, dashed, very thick] (0,0) -- (0,4);
      \draw[penColor, dashed, very thick] (0,0) -- (5,0);
      \draw[->,penColor, very thick] (0,4) -- (6,4);
      \draw [penColor, fill] (5,4) circle [radius=.07];
      %\node [left,penColor] at (0,0) {\scalebox{3} \Ladiesroom};
      %\node [right,penColor] at (6,4) {\scalebox{3}{\ding{40}}};
      \node [right,penColor] at (0,2) {$3$ miles};
      \node [above,penColor] at (2.6,4) {$p$ };
      \node [above,penColor] at (5,4) {$plane$ };
      \node [above,penColor] at (2.6,3.1) {$\frac{dp}{dt}=500$mph};
      \node [below,penColor] at (2.5,0) {ground};
      \node [left,penColor2] at (3.3,2) {$s$ };
      \draw [penColor, fill] (0,0) circle [radius=.07];
      \node [left,penColor] at (-.1,0) {You};
    \end{tikzpicture}
  \end{figure}
\end{frame}
% \emptyframe
\begin{frame}\end{frame}


%%%% Example 5
\begin{frame}[t]
  \vs
  \begin{block}{Example 5. (Angular rates)}
    A plane is flying at an altitude of $3$ miles directly away from
    you at $500$ mph .  Let $\theta$ be the \textbf{angle of elevation} of
    the plane, i.e., the angle between the ground and your line of
    sight to the plane. How fast (in radians per second) is the angle
    $\theta$ decreasing at the moment when the plane is flying over a point
    on the ground $4$ miles from you?
  \end{block}

  \begin{figure} \hfill
    \begin{tikzpicture}[scale=0.7, every node/.style={transform shape}]
      \draw[penColor2, dashed, very thick] (0,0) -- (5,4);
      %\draw[penColor, dashed, very thick] (0,0) -- (0,4);
      \draw[penColor, dashed, very thick] (0,0) -- (0,4);
      \draw[penColor, dashed, very thick] (5,0) -- (5,4);
      \draw[penColor, dashed, very thick] (0,0) -- (5,0);
      \draw[->,penColor, very thick] (0,4) -- (6,4);
      \draw [penColor, fill] (5,4) circle [radius=.07];
      \coordinate (A) at (3.3,2.6);
              \coordinate (B) at (0,0);
              \coordinate (C) at (4,0);
      \tkzMarkAngle[size=2cm,thin](C,B,A)
              \tkzLabelAngle[pos = 1.3](C,B,A){$\theta$}
      %\node [left,penColor] at (0,0) {\scalebox{3} \Ladiesroom};
      %\node [right,penColor] at (6,4) {\scalebox{3}{\ding{40}}};
      \node [right,penColor] at (5,2) {$3$ miles};
      \node [above,penColor] at (2.6,4) {$p$ };
      \node [above,penColor] at (5,4) {plane};
      \node [above,penColor] at (2.18,3.1) {$\frac{dp}{dt}=\frac{500}{60\cdot60}$mi/s};
      \node [below,penColor] at (2.5,0) {ground};
      \draw [penColor, fill] (0,0) circle [radius=.07];
      \node [left,penColor] at (-.1,0) {You};
    \end{tikzpicture}
  \end{figure}
\end{frame}
\begin{frame}\end{frame}
% \emptyframe


%%%% Example 6
\begin{frame}[t]
  \vs
  \begin{block}{Example 6. (Similar triangles)}
    It is night. Someone who is $6$ feet tall is walking away from a
    street light at a rate of $3$ feet per second.  The street light is
    $15$ feet tall.  The person casts a shadow on the ground in front of
    them. How fast is the length of the shadow growing when the person
    is $7$ feet from the street light?
  \end{block}

  \begin{figure} \hfill
    \begin{tikzpicture}[scale=0.7, every node/.style={transform shape}]
      \coordinate (A) at (6,2);
      \coordinate (B) at (0,5);
      \coordinate (C) at (0,2);
      \coordinate (D) at (2,2);
      \coordinate (E) at (2,4);
      \tkzMarkRightAngle(A,C,B)
      \tkzMarkRightAngle(A,D,E)
      \tkzDefMidPoint(A,B) \tkzGetPoint{a}
      \tkzDefMidPoint(A,C) \tkzGetPoint{b}
      \tkzDefMidPoint(D,C) \tkzGetPoint{x}
      \draw[decoration={brace,mirror,raise=.2cm},decorate,thin] (.2,2)--(1.8,2);
      \draw[decoration={brace,mirror,raise=.2cm},decorate,thin] (2.2,2)--(5.8,2);
      \draw[decoration={brace,raise=.2cm},decorate,thin] (0,2)--(0,5);
      \draw[dashed] (A)--(B)--(C)--cycle;
      \draw[very thick] (D)--(E);
      \draw[very thick] (D)--(A);
      \draw[very thick] (B)--(C);
      \node[left] at (2,3) {$6$};
      \node at (1,2-.7) {$p$};
      \node at (4,2-.7) {$s$};
      \node at (0-.7,3.5) {$15$};
      \draw [fill] (0,5) circle [radius=.07];
    \end{tikzpicture}
  \end{figure}
\end{frame}
\begin{frame}\end{frame}
% \emptyframe


%%%% Example 7
\begin{frame}[t]
  \vs
  \begin{block}{Example 7. (Similar triangles)}
    Water is poured into a conical container at the rate of 10 cm${}^3$/s.  The cone points directly down, and it has a height of 30 cm and a base radius of 10 cm.  How fast is the water level rising when the water is 4 cm deep?
  \end{block}

  \begin{figure} \hfill
    \begin{tikzpicture}[scale=0.6, every node/.style={transform shape}]
      \draw[penColor,very thick] (0,4) ellipse (4 and 1);
      \draw[very thick,penColor!20!background] (2,2) arc (0:180:2 and .5);% top half of ellipse
      \draw[very thick,penColor] (-2,2) arc (180:360:2 and .5);% bottom half of ellipse
      \draw[penColor, very thick] (3.97,3.85) -- (0,0);
      \draw[penColor, very thick] (-3.97,3.85) -- (0,0);
      \draw[penColor, very thick] (0,4) -- (4,4);
      \draw[penColor!50!background, very thick] (0,2) -- (2,2);
      \draw[->,line width=0.4cm, penColor!20!background] (0,6) -- (0,4.25);
      \draw[dashed, penColor2, very thick] (2.1,0) -- (2.1,2);
      \draw[dashed, penColor, very thick] (-4.1,0) -- (-4.1,4);
      \node[right, penColor] at (.7,5.6) {$\frac{dV}{dt} = 10$ cm$^3$/sec};
      \node[below, penColor] at (2,4) {$10$ cm};
      \node[above, penColor] at (1,2) {$r$ cm};
      \node[right, penColor2] at (2.1,1) {$h$ cm};
      \node[left, penColor] at (-4.1,2) {$30$ cm};
    \end{tikzpicture}
  \end{figure}
\end{frame}
\begin{frame}\end{frame}
% \emptyframe

\end{document}
%%% Local Variables:
%%% mode: latex
%%% TeX-master: t
%%% End:
