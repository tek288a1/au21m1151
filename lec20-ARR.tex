% \documentclass[10pt,t,presentation,ignorenonframetext,aspectratio=169]{beamer}
\documentclass[10pt,t,handout,ignorenonframetext,aspectratio=169]{beamer}
\usepackage[default]{lato}
\usepackage{tk_beamer1}
%% packages

\RequirePackage{lmodern} % math, rm, ss, tt
\RequirePackage[T1]{fontenc}
\RequirePackage[english]{babel}
\RequirePackage{enumerate}
\RequirePackage{etex}
\RequirePackage{color,xcolor,ucs}
\RequirePackage{graphicx}
\RequirePackage{amssymb}
\RequirePackage{amsmath}
\RequirePackage{subfig}
\RequirePackage{amsthm}
\RequirePackage{mathtools}
\RequirePackage{mathabx}        % required for \odiv; put this after mathtools, otherwise, over/underbrace get messed up
\RequirePackage{epsfig}
\RequirePackage{epstopdf}
\RequirePackage{float}
\RequirePackage{booktabs}
\RequirePackage{blkarray}
\RequirePackage{multirow}
\RequirePackage{hyperref}
\RequirePackage{verbatim}
\RequirePackage{lscape}
\RequirePackage[mathscr]{euscript}
\RequirePackage{movie15}
\RequirePackage{mwe,tikz}
\RequirePackage[percent]{overpic}
\RequirePackage{multicol}
\RequirePackage{pgfplots}
\pgfplotsset{compat=1.7}
\RequirePackage{relsize}
\RequirePackage{textcomp}
\RequirePackage{fancyvrb}
\RequirePackage{tcolorbox}
\RequirePackage{bm}


% %%%%%% for matlab listings
% \RequirePackage{courier}
% \RequirePackage{listings}
% % \newcommand{\matlab}{\textsc{MatLab}}
% % \definecolor{mygreen}{RGB}{28,172,0} % color values Red, Green, Blue
% \definecolor{mygreen}{RGB}{0,100,0} % color values Red, Green, Blue
% \definecolor{mylilas}{RGB}{170,55,241}

% \lstset{language=Matlab,
%   % basicstyle=\small\ttfamily, % Use small true type font
%   % breaklines=true,%
%   % morekeywords={matlab2tikz},
%   % keywordstyle=\color{blue},%
%   % morekeywords=[2]{1}, keywordstyle=[2]{\color{black}},
%   % identifierstyle=\color{black},%
%   % stringstyle=\color{mylilas},
%   % commentstyle=\color{mygreen},%
%   % showstringspaces=false,%without this there will be a symbol in the places where there is a space
%   % numbers=left,%
%   % numberstyle={\tiny \color{black}},% size of the numbers
%   % numbersep=7pt, % this defines how far the numbers are from the text
%   % emph=[1]{for,end,break},emphstyle=[1]\color{red}, %some words to emphasise
%   % % emph=[2]{word1,word2}, emphstyle=[2]{style},
%   % frame=single,
%   basicstyle=\small\ttfamily,%
%   breaklines=true,%
%   morekeywords={matlab2tikz},%
%   keywordstyle=\color{blue},%
%   morekeywords=[2]{1},%
%   keywordstyle=[2]{\color{black}},%
%   identifierstyle=\color{black},%
%   stringstyle=\color{mylilas},
%   commentstyle=\color{mygreen},%
%   % moredelim=[il][\rmfamily]{//},%
%   morecomment=[n][\color{black}]{(*}{*)},%
%   showstringspaces=false,%
%   numbers=none,%
%   % numbers=left,%
%   % numberstyle={\tiny \color{black}},%
%   % numbersep=7pt,%
%   emph=[1]{for,end,break,if,while},emphstyle=[1]\color{blue}, %some words to emphasise
%   % emph=[2]{word1,word2}, emphstyle=[2]{style},
%   frame=single,%
%   framerule=0.7pt,%
%   mathescape=true,%
%   escapebegin=\color{mygreen},%
%   escapeend=,%
% }
% % ref: https://tex.stackexchange.com/questions/257938/how-to-include-matlab-code-into-latex-in-colour



% % \usepackage{spalign}
% % \usepackage{enumitem}
% % \graphicspath{ {../codes/} }
% % \epstopdfsetup{outdir=../codes/}
% % \lstset{inputpath="../codes"}
% % \setlength{\fboxsep}{1.5pt}
% % \newcommand{\x}{\times}
% % \newcommand{\bigzero}{\makebox(0,0){\text{\LARGE 0}}}
% % \newcommand*{\bord}{\multicolumn{1}{c|}{}}

% %% Algorithm/Pseudocode Environment using `listings'
% % https://tex.stackexchange.com/questions/111116/what-is-the-best-looking-pseudo-code-package
% %
% % \newcounter{nalg}[chapter] % defines algorithm counter for chapter-level
% % \renewcommand{\thenalg}{\thechapter .\arabic{nalg}}

% %defines appearance of the algorithm counter
% \DeclareCaptionLabelFormat{algocaption}{Algorithm \thenalg} % defines a new caption label as Algorithm x.y

% \lstnewenvironment{algorithm}[1][] %defines the algorithm listing environment
% {
%     % \refstepcounter{nalg} %increments algorithm number
%     \captionsetup{labelformat=algocaption,labelsep=colon} %defines the caption setup for: it uses label format as the declared caption label above and makes label and caption text to be separated by a ':'
%     \lstset{ %this is the stype
%         mathescape=true,
%         % %frame=tB,
%         % frame=none,
%         % numbers=left,
%         % numberstyle=\tiny,
%         % commentstyle=,
%         % basicstyle=\small,
%         % stringstyle=\ttfamily,
%         basicstyle=,
%         keywordstyle=\color{black}\bfseries,
%         keywords={for, input, output, return, datatype, function, in, if, else, foreach, while, begin, end}, %add the keywords you want, or load a language as Rubens explains in his comment above.
%         % xleftmargin=.04\textwidth,
%         #1 % this is to add specific settings to an usage of this environment (for instnce, the caption and referable label)
%         }
% }
% {}

%%%% Custom macros
%%%% Macros

%% Greek letters

\newcommand{\al}{\alpha}
% \newcommand{\be}{\beta}
\newcommand{\g}{\gamma}
\newcommand{\de}{\delta}
\newcommand{\e}{\epsilon}
\newcommand{\eps}{\varepsilon}
\newcommand{\ka}{\kappa}
\newcommand{\la}{\lambda}
\newcommand{\sig}{\sigma}
\newcommand{\om}{\omega}
\newcommand{\Om}{\Omega}
\let\oldth\th %% \th is used for "thorn"
\renewcommand{\th}{\theta}

%% Tweak some Greek letters
\newcommand{\bchi}{\mbox{\raisebox{.4ex}{\begin{large}$\chi$\end{large}}}}
\newcommand{\Chi}{\mbox{\Large$\chi$}} % nicer looking Chi
\newcommand{\bzeta}{\boldsymbol{\zeta}} % Riemann zeta function
\newcommand{\bxi}{\boldsymbol{\xi}}
\newcommand{\balpha}{\boldsymbol{\alpha}}

%% Blackboard

\newcommand{\NN}{\mathbb{N}}
\newcommand{\ZZ}{\mathbb{Z}}
\newcommand{\QQ}{\mathbb{Q}}
\newcommand{\RR}{\mathbb{R}}
\newcommand{\CC}{\mathbb{C}}
\newcommand{\FF}{\mathbb{F}}
\newcommand{\TT}{\mathbb{T}}
\newcommand{\DD}{\mathbb{D}}
\newcommand{\HH}{\mathbb{H}}
\newcommand{\UU}{\mathbb{U}}
\newcommand{\1}{\mathbbm{1}}

%% Caligraphic

\newcommand{\cA}{\mathcal{A}}
\newcommand{\cB}{\mathcal{B}}
\newcommand{\cC}{\mathcal{C}}
\newcommand{\cD}{\mathcal{D}}
\newcommand{\cE}{\mathcal{E}}
\newcommand{\cF}{\mathcal{F}}
\newcommand{\cG}{\mathcal{G}}
\newcommand{\cH}{\mathcal{H}}
\newcommand{\cI}{\mathcal{I}}
\newcommand{\cJ}{\mathcal{J}}
\newcommand{\cK}{\mathcal{K}}
\newcommand{\cL}{\mathcal{L}}
\newcommand{\cM}{\mathcal{M}}
\newcommand{\cN}{\mathcal{N}}
\newcommand{\cO}{\mathcal{O}}
\newcommand{\cP}{\mathcal{P}}
\newcommand{\cQ}{\mathcal{Q}}
\newcommand{\cR}{\mathcal{R}}
\newcommand{\cS}{\mathcal{S}}
\newcommand{\cT}{\mathcal{T}}
\newcommand{\cU}{\mathcal{U}}
\newcommand{\cV}{\mathcal{V}}
\newcommand{\cW}{\mathcal{W}}


%% Roman, italic, boldface

\newcommand{\bA}{\mathbf{A}}
\newcommand{\bB}{\mathbf{B}}
\newcommand{\bC}{\mathbf{C}}
\newcommand{\bD}{\mathbf{D}}
\newcommand{\bE}{\mathbf{E}}
\newcommand{\bI}{\mathbf{I}}
\newcommand{\bK}{\mathbf{K}}
\newcommand{\bL}{\mathbf{L}}
\newcommand{\bM}{\mathbf{M}}
\newcommand{\bN}{\mathbf{N}}
\newcommand{\bP}{\mathbf{P}}
\newcommand{\bS}{\mathbf{S}}
\newcommand{\bT}{\mathbf{T}}
\newcommand{\bX}{\mathbf{X}}
\newcommand{\ba}{\mathbf{a}}
\newcommand{\bb}{\mathbf{b}}
\newcommand{\bc}{\mathbf{c}}
\newcommand{\bd}{\mathbf{d}}
\newcommand{\be}{\mathbf{e}}
\newcommand{\bg}{\mathbf{g}}
\newcommand{\bh}{\mathbf{h}}
\newcommand{\br}{\mathbf{r}}
\newcommand{\bx}{\mathbf{x}}
\newcommand{\by}{\mathbf{y}}
\newcommand{\bu}{\mathbf{u}}
\newcommand{\bv}{\mathbf{v}}
\newcommand{\bw}{\mathbf{w}}
\newcommand{\bz}{\mathbf{z}}
\newcommand{\bq}{\mathbf{q}}
\newcommand{\bzero}{\mathbf{0}}


%% Principal value integral
\def\Xint#1{\mathchoice
  {\XXint\displaystyle\textstyle{#1}}%
  {\XXint\textstyle\scriptstyle{#1}}%
  {\XXint\scriptstyle\scriptscriptstyle{#1}}%
  {\XXint\scriptscriptstyle\scriptscriptstyle{#1}}%
  \!\int}
\def\XXint#1#2#3{{\setbox0=\hbox{$#1{#2#3}{\int}$}
    \vcenter{\hbox{$#2#3$}}\kern-.5\wd0}}
\def\ddashint{\Xint=}
\def\pvint{\Xint-}

%% Arc over symbols
% reference: https://tex.stackexchange.com/questions/96680/a-better-notation-to-denote-arcs-for-an-american-high-school-textbook
\makeatletter
\DeclareFontFamily{U}{tipa}{}
\DeclareFontShape{U}{tipa}{m}{n}{<->tipa10}{}
\newcommand{\arc@char}{{\usefont{U}{tipa}{m}{n}\symbol{62}}}%
\newcommand{\arc}[1]{\mathpalette\arc@arc{#1}}
\newcommand{\arc@arc}[2]{%
  \sbox0{$\m@th#1#2$}%
  \vbox{
    \hbox{\resizebox{\wd0}{\height}{\arc@char}}
    \nointerlineskip
    \box0
  }%
}


%% colored boxes
\newsavebox{\astrutbox}
\sbox{\astrutbox}{\rule[-5pt]{0pt}{20pt}}
\newcommand{\astrut}{\usebox{\astrutbox}}
\newcommand{\rls}{\raisebox{2pt}{\tikz{\draw[red,solid,line width=0.9pt](0,0) -- (5mm,0);}}}
\newcommand{\rld}{\raisebox{2pt}{\tikz{\draw[red,dashed,line width=1.0pt](0,0) -- (5mm,0);}}}
\newcommand{\bls}{\raisebox{2pt}{\tikz{\draw[blue,solid,line width=0.9pt](0,0) -- (5mm,0);}}}
\newcommand{\bld}{\raisebox{2pt}{\tikz{\draw[blue,dashed,line width=1.0pt](0,0) -- (5mm,0);}}}
\newcommand{\gls}{\raisebox{2pt}{\tikz{\draw[green,solid,line width=0.9pt](0,0) -- (5mm,0);}}}
\newcommand{\gld}{\raisebox{2pt}{\tikz{\draw[green,dashed,line width=1.0pt](0,0) -- (5mm,0);}}}
\newcommand{\mls}{\raisebox{2pt}{\tikz{\draw[magenta,solid,line width=0.9pt](0,0) -- (5mm,0);}}}
\newcommand{\mld}{\raisebox{2pt}{\tikz{\draw[magenta,dashed,line width=1.0pt](0,0) -- (5mm,0);}}}
\newcommand{\cls}{\raisebox{2pt}{\tikz{\draw[cyan,solid,line width=0.9pt](0,0) -- (5mm,0);}}}
\newcommand{\cld}{\raisebox{2pt}{\tikz{\draw[cyan,dashed,line width=1.0pt](0,0) -- (5mm,0);}}}



%% Delimiters, accents, bars, etc

\newcommand{\abs}[1]{\left|#1\right|}
\newcommand{\ceil}[1]{\left\lceil#1\right\rceil}
\newcommand{\floor}[1]{\left\lfloor#1\right\rfloor}
\newcommand{\conj}[1]{\overline{#1}}
\newcommand{\norm}[1]{\left\|#1\right\|}
\newcommand{\Norm}[2]{\left\|#1\right\|_{#2}}
% Improvement of the above two
% example usage: \norm[2]{f} or \Norm[2]{f}{L^2}
\renewcommand{\norm}[2][0]{%
  \ifcase#1\relax
    \left\Vert #2 \right\Vert\or  % 0
    \lVert #2 \rVert\or           % 1
    \bigl\Vert #2 \bigr\Vert\or   % 2
    \Bigl\Vert #2 \Bigr\Vert\or   % 3
    \biggl\Vert #2 \biggr\Vert\or % 4
    \Biggl\Vert #2 \Biggr\Vert    % 5
  \fi}
\renewcommand{\Norm}[3][0]{%
  \ifcase#1\relax
    \left\Vert #2 \right\Vert_{#3}\or  % 0
    \lVert #2 \rVert_{#3}\or           % 1
    \bigl\Vert #2 \bigr\Vert_{#3}\or   % 2
    \Bigl\Vert #2 \Bigr\Vert_{#3}\or   % 3
    \biggl\Vert #2 \biggr\Vert_{#3}\or % 4
    \Biggl\Vert #2 \Biggr\Vert_{#3}    % 5
  \fi}
\newcommand{\avg}[1]{\langle#1\rangle}
\newcommand{\ds}{\displaystyle}


%% Mathematical operators

\DeclareMathOperator{\re}{Re}
\DeclareMathOperator{\im}{Im}
\DeclareMathOperator{\sgn}{sgn}
\DeclareMathOperator{\erf}{erf}
\DeclareMathOperator{\erfc}{erfc}
\DeclareMathOperator{\ii}{i}
\DeclareMathOperator{\dd}{\,d}
\DeclareMathOperator{\eu}{e}
\DeclareMathOperator{\Sp}{Sp}
\DeclareMathOperator{\acosh}{acosh}
\DeclareMathOperator{\asech}{asech}
\DeclareMathOperator{\atanh}{atanh}
\newcommand{\del}{\partial}
\newcommand{\tri}{\triangle}
\newcommand{\grad}{\nabla}
\newcommand{\dvg}{\nabla\cdot}
\newcommand{\curl}{\nabla\times}


%% colored texts

\newcommand{\red}[1]{{\color{red}{#1}}}
\newcommand{\blue}[1]{{\color{blue}{#1}}}
\newcommand{\green}[1]{{\color{green}{#1}}}


%%

\newcommand{\emptyframe}{\begin{frame}{}\end{frame}}
\newcommand{\question}{\textbf{Question.}}
\newcommand{\sqitem}{\item[$\square$]}


%% 06/12/18 addition

\newcommand{\vs}{\vspace{1em}}
\newcommand{\tp}{^{\rm T}}

%% 06/22/18 addition
% for 3607
\newcommand{\meps}{\fbox{eps}}
\newcommand{\flops}{\textit{flops}}
% \setlength{\fboxsep}{1.5pt}
\newcommand\x{\times}
\newcommand\bigzero{\makebox(0,0){\text{\LARGE 0}}}
\newcommand*{\bord}{\multicolumn{1}{c|}{}}


%% Continued numbering over multiple enumerate environments
% https://tex.stackexchange.com/questions/55000/continuing-enumerate-counters-in-beamer
\newcounter{saveenumi}
\newcommand{\seti}{\setcounter{saveenumi}{\value{enumi}}}
\newcommand{\conti}{\setcounter{enumi}{\value{saveenumi}}}


%% Extension to amsmath matrix environment
% improving matrix constructors
% source: http://texblog.net/latex-archive/maths/amsmath-matrix/
\makeatletter
\renewcommand*\env@matrix[1][*\c@MaxMatrixCols c]{%
  \hskip -\arraycolsep
  \let\@ifnextchar\new@ifnextchar
  \array{#1}}
\makeatother

% In order to make the column lines to look nicer:
\setlength\delimitershortfall{0pt}


%% 11/09/18 addition: vertical spaces
\newcommand{\vsone}{\vspace{\stretch{1}}}
\newcommand{\vstwo}{\vspace{\stretch{2}}}
\newcommand{\vsthree}{\vspace{\stretch{3}}}

%% 02/23/19 addition: wide hat and wide tilde
\newcommand{\wh}[1]{\widehat{#1}}
\newcommand{\wt}[1]{\widetilde{#1}}

%% Defining theorem environment
\theoremstyle{plain}
\newtheorem{prop}{Proposition}
\newtheorem{cor}[prop]{Corollary}
\newtheorem{lem}[prop]{Lemma}
\newtheorem{thm}[prop]{Theorem}
\newtheorem{cons}[prop]{Consequence}
\newtheorem{conv}[prop]{Convention}
\newtheorem{prob}[prop]{Problem}
\newtheorem{form}[prop]{Formulation}
\newtheorem{claim}[prop]{Claim}

\theoremstyle{definition}
\newtheorem{defn}{Definition}
\newtheorem{notn}[defn]{Notation}
\newtheorem{note}[defn]{Note}
\newtheorem{rmk}[defn]{Remark}
\newtheorem{exer}[defn]{Exercise}
\newtheorem{ex}[defn]{Example}


%% Listings Environments
\usepackage{courier}
\usepackage{listings}

% \definecolor{mygreen}{RGB}{28,172,0} % color values Red, Green, Blue
\definecolor{mygreen}{RGB}{0,100,0} % color values Red, Green, Blue
\definecolor{mylilas}{RGB}{170,55,241}

\lstdefinestyle{matlab}{language=Matlab,
% basicstyle=\small\ttfamily, % Use small true type font
% breaklines=true,%
% morekeywords={matlab2tikz},
% keywordstyle=\color{blue},%
% morekeywords=[2]{1}, keywordstyle=[2]{\color{black}},
% identifierstyle=\color{black},%
% stringstyle=\color{mylilas},
% commentstyle=\color{mygreen},%
% showstringspaces=false,%without this there will be a symbol in the places where there is a space
% numbers=left,%
% numberstyle={\tiny \color{black}},% size of the numbers
% numbersep=7pt, % this defines how far the numbers are from the text
% emph=[1]{for,end,break},emphstyle=[1]\color{red}, %some words to emphasise
% % emph=[2]{word1,word2}, emphstyle=[2]{style},
% frame=single,
basicstyle=\small\ttfamily,%
breaklines=true,%
morekeywords={matlab2tikz},%
keywordstyle=\color{blue},%
morekeywords=[2]{1},%
keywordstyle=[2]{\color{black}},%
identifierstyle=\color{black},%
stringstyle=\color{mylilas},
commentstyle=\color{mygreen},%
% moredelim=[il][\rmfamily]{//},%
morecomment=[n][\color{black}]{(*}{*)},%
showstringspaces=false,%
numbers=none,%
% numbers=left,%
% numberstyle={\tiny \color{black}},%
% numbersep=7pt,%
emph=[1]{for,end,break,if,while,mod,ones,randi,sind,cosd,tand},
emphstyle=[1]\color{blue}, %some words to emphasise
% emph=[2]{word1,word2}, emphstyle=[2]{style},
frame=single,%
framerule=0.7pt,%
mathescape=true,%
escapebegin=\color{mygreen},%
escapeend=,%
escapechar=`,
}
% ref: https://tex.stackexchange.com/questions/257938/how-to-include-matlab-code-into-latex-in-colour


%% Algorithm/Pseudocode Environment using `listings'
% https://tex.stackexchange.com/questions/111116/what-is-the-best-looking-pseudo-code-package
%
% \newcounter{nalg}[chapter] % defines algorithm counter for chapter-level
% \renewcommand{\thenalg}{\thechapter .\arabic{nalg}}

%defines appearance of the algorithm counter
\DeclareCaptionLabelFormat{algocaption}{Algorithm \thenalg} % defines a new caption label as Algorithm x.y

\lstnewenvironment{algorithm}[1][] %defines the algorithm listing environment
{
% \refstepcounter{nalg} %increments algorithm number
\captionsetup{labelformat=algocaption,labelsep=colon} %defines the caption setup for: it uses label format as the declared caption label above and makes label and caption text to be separated by a ':'
\lstset{language=,
basicstyle=\small,%
stringstyle=\small\ttfamily,%
breaklines=true,%
keywords={for, input, output, return, datatype, function, in, if, else, foreach, while, begin, end},
keywordstyle=\color{blue}\bfseries\ttfamily,%
numbers=none,%
frame=single,%
framerule=0.7pt,%
mathescape=true,%
escapechar=',
xleftmargin=.04\linewidth,
#1 % this is to add specific settings to an usage of this environment (for instance, the caption and referable label)
}
}
{}

%%%% ximera preamble extract

%% TikZ/PGFplot related

\usepackage{tkz-euclide}
\usepackage{tikz}
\usepackage{tikz-cd}
\usepackage{pgffor} %% required for integral for loops
\usetikzlibrary{arrows}
\tikzset{>=stealth,commutative diagrams/.cd,
  arrow style=tikz,diagrams={>=stealth}} %% cool arrow head
\tikzset{shorten <>/.style={ shorten >=#1, shorten <=#1 } } %% allows shorter vectors

\usetikzlibrary{backgrounds} %% for boxes around graphs
\usetikzlibrary{shapes,positioning}  %% Clouds and stars
\usetikzlibrary{matrix} %% for matrix
\usepgfplotslibrary{polar} %% for polar plots
% \usetkzobj{all}

% Gaussian function
\pgfmathdeclarefunction{gauss}{2}{% gives gaussian
  \pgfmathparse{1/(#2*sqrt(2*pi))*exp(-((x-#1)^2)/(2*#2^2))}%
}



%% colors

\colorlet{textColor}{black}
\colorlet{background}{white}
\colorlet{penColor}{blue!50!black} % Color of a curve in a plot
\colorlet{penColor2}{red!50!black}% Color of a curve in a plot
\colorlet{penColor3}{red!50!blue} % Color of a curve in a plot
\colorlet{penColor4}{green!50!black} % Color of a curve in a plot
\colorlet{penColor5}{orange!80!black} % Color of a curve in a plot
\colorlet{penColor6}{yellow!70!black} % Color of a curve in a plot
\colorlet{fill1}{penColor!20} % Color of fill in a plot
\colorlet{fill2}{penColor2!20} % Color of fill in a plot
\colorlet{fillp}{fill1} % Color of positive area
\colorlet{filln}{penColor2!20} % Color of negative area
\colorlet{fill3}{penColor3!20} % Fill
\colorlet{fill4}{penColor4!20} % Fill
\colorlet{fill5}{penColor5!20} % Fill
\colorlet{gridColor}{gray!50} % Color of grid in a plot
\newcommand{\surfaceColor}{violet}
\newcommand{\surfaceColorTwo}{redyellow}
\newcommand{\sliceColor}{greenyellow}



%% image environment
\usepackage{environ}
\NewEnviron{image}[1][3in]{%
  \begin{center}\resizebox{#1}{!}{\BODY}\end{center}% resize and center
}



%% more packages

\usepackage[makeroom]{cancel} %% for strike outs
\usepackage{multicol}
\usepackage{array}



%% lengths

\setlength{\extrarowheight}{+.1cm}


%% macros
% \newcommand{\dd}[2][]{\frac{\d #1}{\d #2}} % conflict
\newcommand{\pp}[2][]{\frac{\partial #1}{\partial #2}}
\newcommand{\ddx}{\frac{d}{\d x}}
\newcommand{\dfn}{\textbf}
\newcommand{\unit}{\mathop{}\!\mathrm}
\newcommand{\eval}[1]{\bigg[ #1 \bigg]}
\newcommand{\seq}[1]{\left( #1 \right)}
\renewcommand{\d}{\mathop{}\!d}
\renewcommand{\l}{\ell}
% \newcommand{\zeroOverZero}{\ensuremath{\boldsymbol{\tfrac{0}{0}}}}
% \newcommand{\inftyOverInfty}{\ensuremath{\boldsymbol{\tfrac{\infty}{\infty}}}}
% \newcommand{\zeroOverInfty}{\ensuremath{\boldsymbol{\tfrac{0}{\infty}}}}
% \newcommand{\zeroTimesInfty}{\ensuremath{\small\boldsymbol{0\cdot \infty}}}
% \newcommand{\inftyMinusInfty}{\ensuremath{\small\boldsymbol{\infty - \infty}}}
% \newcommand{\oneToInfty}{\ensuremath{\boldsymbol{1^\infty}}}
% \newcommand{\zeroToZero}{\ensuremath{\boldsymbol{0^0}}}
% \newcommand{\inftyToZero}{\ensuremath{\boldsymbol{\infty^0}}}
% \newcommand{\numOverZero}{\ensuremath{\boldsymbol{\tfrac{\#}{0}}}}

\usepackage{wasysym}            % for smiley
\newcommand{\zoz}{$\mathbf{ \frac{0}{0} }$}

% some inverse trigs
\DeclareMathOperator{\arcsec}{arcsec}
\DeclareMathOperator{\arccot}{arccot}
\DeclareMathOperator{\arccsc}{arccsc}

%%%% META DATA
\newcommand{\semester}{Autumn 2021}
\newcommand{\course}{Math 1151}
\newcommand{\lecTitle}{Lecture 20: Applied Related Rates (ARR)}

%%%% TITLE PAGE
\title[\course]{\lecTitle}
\institute[Ohio State]
{
  \medskip
}
\date[\week]{\semester}
\author{Tae Eun Kim, Ph.D.}

\begin{document}
\begin{frame}
  \titlepage
\end{frame}

\begin{frame}[t]
  \frametitle{Applied Related Rates Problems}
  \begin{block}{General procedures}
    \begin{enumerate}
    \item \textbf{Introduce variables and identify the given and unknown
        rates.} Assign a variable to each quantity that changes in time.
    \item \textbf{Draw a picture.} If possible, draw a schematic picture
      with all the relevant information.
    \item \textbf{Find equations.} Write equations that relate all
      relevant variables.
    \item \textbf{Differentiate with respect to time t.} Here we will
      often use implicit differentiation and obtain an equation that
      relates the given rate and the unknown rate.
    \item \textbf{Evaluate.} Evaluate each quantity at the relevant
      moment.
    \item \textbf{Solve.} Solve for the unknown rate at that moment.
    \end{enumerate}
  \end{block}
\end{frame}

%%%% Example 1
\begin{frame}[t]
  \vs
  \begin{block}{Example 1. (Cylindrical geometry)}
    A hand-tossed pizza crust starts off as a ball of dough with a volume of $400\pi\, \text{cm}^3$. First, the cook stretches the dough to the shape of a cylinder of radius $12$ cm. Next the cook tosses the dough.

    If during tossing, the dough maintains the shape of a cylinder and the radius is increasing at a rate of $15$ cm/min, how fast is its thickness changing when the radius is $20$ cm?
  \end{block}

  \begin{figure} \hfill
    \begin{tikzpicture}[scale=0.7, every node/.style={transform shape}]
      \draw[penColor,very thick] (0,0) ellipse (2.5 and .8);
      \draw[very thick,penColor] (-2.5,-.5) arc (180:360:2.5 and .8);% bottom
      \draw[penColor] (0,0) -- (2.5,0);
      \draw[penColor,very thick] (-2.5,0) -- (-2.5,-.5);
      \draw[penColor,very thick] (2.5,0) -- (2.5,-.5);
      \node[above,penColor] at (1,0) {$r$};
      \node[below,penColor] at (1,0) {$r' = 15$};
      \draw[decoration={brace,raise=.2cm},decorate,thin] (2.5,0)--(2.5,-.5);
      \node [penColor,right] at (3,-.25) {$h$};
      \node [penColor,left] at (-0.5,1.2) {$V = 400\pi$ cm$^3$};
      \node [penColor, right] at (0.5,-1.6) {$V = \pi\cdot r^2 \cdot h$ cm$^3$};
      \end{tikzpicture}
  \end{figure}
\end{frame}
% \emptyframe
\begin{frame}\end{frame}

%%%% Example 2
\begin{frame}[t]
  \vs
  \begin{block}{Example 2. (Spherical geometry)}
    Consider a melting snowball. We will assume that the rate at which the
    snowball is melting is proportional to its surface area. Show that
    the radius of the snowball is changing at a constant rate.
  \end{block}

  \begin{figure} \hfill
    \begin{tikzpicture}[scale=0.7, every node/.style={transform shape}]
      %\draw[penColor!50!background,very thick] (0,0) ellipse (2 and 1);
      \draw[very thick,penColor!20!background] (2,0) arc (0:180:2 and .7);% top half of ellipse
      \draw [penColor, very thick] (0,0) circle [radius=2];
      \draw[penColor] (0,0) -- (2,0);
      \node [below,penColor] at (1,0) {$r$ cm};
      \draw[very thick,penColor] (-2,0) arc (180:360:2 and .7);% bottom half of ellipse
      \node [penColor,right] at (1.3,1.65) {$V = \frac{4}{3}\cdot \pi \cdot r^3$};
      \node [penColor,right] at (1.3,-1.65) {$A = 4\cdot \pi \cdot r^2$};
    \end{tikzpicture}
  \end{figure}
\end{frame}
% \emptyframe
\begin{frame}\end{frame}

%%%% Example 3
\begin{frame}[t]
  \vs
  \begin{block}{Example 3. (Right triangles)}
    A road running north to south crosses a road going east to west at the
    point $P$.  Cyclist $A$ is riding north along the first road, and
    cyclist $B$ is riding east along the second road.  At a particular
    time, cyclist $A$ is $3$ kilometers to the north of $P$ and traveling
    at $20$ km/hr, while cyclist $B$ is $4$ kilometers to the east of $P$
    and traveling at $15$ km/hr.  How fast is the distance between the two
    cyclists changing at that time?
  \end{block}

  \begin{figure} \hfill
    \begin{tikzpicture}[scale=0.7, every node/.style={transform shape}]
      \draw[->,penColor!50!background, very thick] (-1,0) -- (4,0);
      \draw[->,penColor!50!background, very thick] (0,-1) -- (0,4);
      \draw[->,penColor, very thick] (0,3) -- (0,4);
      \draw[->,penColor, very thick] (3,0) -- (4,0);
      \draw [penColor, fill] (0,0) circle [radius=.07];
      \draw [penColor, fill] (3,0) circle [radius=.07];
      \draw [penColor, fill] (0,3) circle [radius=.07];
      \draw[dashed,penColor2, very thick] (3,0) -- (0,3);

      %\node[penColor,rotate=90,right] at (.5,3) {\scalebox{-2} \Bicycle};
      \node[penColor,right] at (0,.2) {$P$};
      \node[penColor,left] at (0,1.5) {$a(t)$ };
      \node[penColor,left] at (-.005,3) {$A$ };
      \node[penColor,below] at (1.5,0) {$b(t)$ };
      \node[penColor,below] at (3,0) {B};
      \node[penColor2,above] at (1.6,1.6) {$c(t)$};
      %\node[penColor,right,above] at (3.5,0) {\scalebox{-2}[2] \Bicycle};
    \end{tikzpicture}
  \end{figure}
\end{frame}
% \emptyframe
\begin{frame}\end{frame}


%%%% Example 4
\begin{frame}[t]
  \vs
  \begin{block}{Example 4. (Right triangles)}
    A plane is flying at an altitude of $3$ miles directly away from you at $500$ mph. How fast is the plane's distance from you increasing at the moment when the plane is flying over a point on the ground $4$ miles from you?
  \end{block}

  \begin{figure} \hfill
    \begin{tikzpicture}[scale=0.7, every node/.style={transform shape}]
      \draw[penColor2, dashed, very thick] (0,0) -- (5,4);
      %\draw[penColor, dashed, very thick] (0,0) -- (0,4);
      \draw[penColor, dashed, very thick] (0,0) -- (0,4);
      \draw[penColor, dashed, very thick] (0,0) -- (5,0);
      \draw[->,penColor, very thick] (0,4) -- (6,4);
      \draw [penColor, fill] (5,4) circle [radius=.07];
      %\node [left,penColor] at (0,0) {\scalebox{3} \Ladiesroom};
      %\node [right,penColor] at (6,4) {\scalebox{3}{\ding{40}}};
      \node [right,penColor] at (0,2) {$3$ miles};
      \node [above,penColor] at (2.6,4) {$p$ };
      \node [above,penColor] at (5,4) {$plane$ };
      \node [above,penColor] at (2.6,3.1) {$\frac{dp}{dt}=500$mph};
      \node [below,penColor] at (2.5,0) {ground};
      \node [left,penColor2] at (3.3,2) {$s$ };
      \draw [penColor, fill] (0,0) circle [radius=.07];
      \node [left,penColor] at (-.1,0) {You};
    \end{tikzpicture}
  \end{figure}
\end{frame}
% \emptyframe
\begin{frame}\end{frame}


%%%% Example 5
\begin{frame}[t]
  \vs
  \begin{block}{Example 5. (Angular rates)}
    A plane is flying at an altitude of $3$ miles directly away from
    you at $500$ mph .  Let $\theta$ be the \textbf{angle of elevation} of
    the plane, i.e., the angle between the ground and your line of
    sight to the plane. How fast (in radians per second) is the angle
    $\theta$ decreasing at the moment when the plane is flying over a point
    on the ground $4$ miles from you?
  \end{block}

  \begin{figure} \hfill
    \begin{tikzpicture}[scale=0.7, every node/.style={transform shape}]
      \draw[penColor2, dashed, very thick] (0,0) -- (5,4);
      %\draw[penColor, dashed, very thick] (0,0) -- (0,4);
      \draw[penColor, dashed, very thick] (0,0) -- (0,4);
      \draw[penColor, dashed, very thick] (5,0) -- (5,4);
      \draw[penColor, dashed, very thick] (0,0) -- (5,0);
      \draw[->,penColor, very thick] (0,4) -- (6,4);
      \draw [penColor, fill] (5,4) circle [radius=.07];
      \coordinate (A) at (3.3,2.6);
              \coordinate (B) at (0,0);
              \coordinate (C) at (4,0);
      \tkzMarkAngle[size=2cm,thin](C,B,A)
              \tkzLabelAngle[pos = 1.3](C,B,A){$\theta$}
      %\node [left,penColor] at (0,0) {\scalebox{3} \Ladiesroom};
      %\node [right,penColor] at (6,4) {\scalebox{3}{\ding{40}}};
      \node [right,penColor] at (5,2) {$3$ miles};
      \node [above,penColor] at (2.6,4) {$p$ };
      \node [above,penColor] at (5,4) {plane};
      \node [above,penColor] at (2.18,3.1) {$\frac{dp}{dt}=\frac{500}{60\cdot60}$mi/s};
      \node [below,penColor] at (2.5,0) {ground};
      \draw [penColor, fill] (0,0) circle [radius=.07];
      \node [left,penColor] at (-.1,0) {You};
    \end{tikzpicture}
  \end{figure}
\end{frame}
\begin{frame}\end{frame}
% \emptyframe


%%%% Example 6
\begin{frame}[t]
  \vs
  \begin{block}{Example 6. (Similar triangles)}
    It is night. Someone who is $6$ feet tall is walking away from a
    street light at a rate of $3$ feet per second.  The street light is
    $15$ feet tall.  The person casts a shadow on the ground in front of
    them. How fast is the length of the shadow growing when the person
    is $7$ feet from the street light?
  \end{block}

  \begin{figure} \hfill
    \begin{tikzpicture}[scale=0.7, every node/.style={transform shape}]
      \coordinate (A) at (6,2);
      \coordinate (B) at (0,5);
      \coordinate (C) at (0,2);
      \coordinate (D) at (2,2);
      \coordinate (E) at (2,4);
      \tkzMarkRightAngle(A,C,B)
      \tkzMarkRightAngle(A,D,E)
      \tkzDefMidPoint(A,B) \tkzGetPoint{a}
      \tkzDefMidPoint(A,C) \tkzGetPoint{b}
      \tkzDefMidPoint(D,C) \tkzGetPoint{x}
      \draw[decoration={brace,mirror,raise=.2cm},decorate,thin] (.2,2)--(1.8,2);
      \draw[decoration={brace,mirror,raise=.2cm},decorate,thin] (2.2,2)--(5.8,2);
      \draw[decoration={brace,raise=.2cm},decorate,thin] (0,2)--(0,5);
      \draw[dashed] (A)--(B)--(C)--cycle;
      \draw[very thick] (D)--(E);
      \draw[very thick] (D)--(A);
      \draw[very thick] (B)--(C);
      \node[left] at (2,3) {$6$};
      \node at (1,2-.7) {$p$};
      \node at (4,2-.7) {$s$};
      \node at (0-.7,3.5) {$15$};
      \draw [fill] (0,5) circle [radius=.07];
    \end{tikzpicture}
  \end{figure}
\end{frame}
\begin{frame}\end{frame}
% \emptyframe


%%%% Example 7
\begin{frame}[t]
  \vs
  \begin{block}{Example 7. (Similar triangles)}
    Water is poured into a conical container at the rate of 10 cm${}^3$/s.  The cone points directly down, and it has a height of 30 cm and a base radius of 10 cm.  How fast is the water level rising when the water is 4 cm deep?
  \end{block}

  \begin{figure} \hfill
    \begin{tikzpicture}[scale=0.6, every node/.style={transform shape}]
      \draw[penColor,very thick] (0,4) ellipse (4 and 1);
      \draw[very thick,penColor!20!background] (2,2) arc (0:180:2 and .5);% top half of ellipse
      \draw[very thick,penColor] (-2,2) arc (180:360:2 and .5);% bottom half of ellipse
      \draw[penColor, very thick] (3.97,3.85) -- (0,0);
      \draw[penColor, very thick] (-3.97,3.85) -- (0,0);
      \draw[penColor, very thick] (0,4) -- (4,4);
      \draw[penColor!50!background, very thick] (0,2) -- (2,2);
      \draw[->,line width=0.4cm, penColor!20!background] (0,6) -- (0,4.25);
      \draw[dashed, penColor2, very thick] (2.1,0) -- (2.1,2);
      \draw[dashed, penColor, very thick] (-4.1,0) -- (-4.1,4);
      \node[right, penColor] at (.7,5.6) {$\frac{dV}{dt} = 10$ cm$^3$/sec};
      \node[below, penColor] at (2,4) {$10$ cm};
      \node[above, penColor] at (1,2) {$r$ cm};
      \node[right, penColor2] at (2.1,1) {$h$ cm};
      \node[left, penColor] at (-4.1,2) {$30$ cm};
    \end{tikzpicture}
  \end{figure}
\end{frame}
\begin{frame}\end{frame}
% \emptyframe

\end{document}
%%% Local Variables:
%%% mode: latex
%%% TeX-master: t
%%% End:
