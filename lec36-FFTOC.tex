% \documentclass[10pt,t,presentation,ignorenonframetext,aspectratio=169]{beamer}
\documentclass[10pt,t,handout,ignorenonframetext,aspectratio=169]{beamer}
\usepackage[default]{lato}
\usepackage{tk_beamer1}
\input{tk_packages}
\input{tk_macros}
\input{tk_environ}
\input{tk_ximera}
\usepackage{wasysym}            % for smiley
\newcommand{\zoz}{$\mathbf{ \frac{0}{0} }$}

% some inverse trigs
\DeclareMathOperator{\arcsec}{arcsec}
\DeclareMathOperator{\arccot}{arccot}
\DeclareMathOperator{\arccsc}{arccsc}

%%%% META DATA
\newcommand{\semester}{Autumn 2021}
\newcommand{\course}{Math 1151}
\newcommand{\lecTitle}{Lecture 36: First Fundamental Theorem \\ of Calculus (FFTOC)}

%%%% TITLE PAGE
\title[\course]{\lecTitle}
\institute[Ohio State]
{
  \medskip
}
\date[\week]{\semester}
\author{Tae Eun Kim, Ph.D.}

\begin{document}
\begin{frame}
  \titlepage
\end{frame}


\subsection{Accumulation Function}
\begin{frame}
  \frametitle{Accumulation functions}
  \begin{defn}
    Given a function $f$, an \textbf{accumulation function} is
    \[
      F(x) = \int_a^x f(t) \; dt
    \]
  \end{defn}

  \begin{itemize}
  \item It calculates the signed area of the region between $y = f(t)$ and $t$-axis over the interval $[a, x]$ where the location of right-endpoint is now a variable.
  \end{itemize}
\end{frame}

\begin{frame}
  \vs
  \begin{example}[Rectangles]
    Let $\ds F(x) = \int_{-3}^x 4 \; dt$. What is $F(5)$? What is
    $F(-5)$? What is $F(x)$?
  \end{example}
\end{frame}

\begin{frame}
  \vs
  \begin{example}[Trapezoid]
    Let $\ds F(x) = \int_0^x (2t+1) \; dt$. Find $F(x)$.
  \end{example}
\end{frame}

\begin{frame}
  \vs
  \begin{example}[Monotonicity of accumulation function]
    Let $\ds F(x) = \int_{-1}^x t^3 \; dt$. On the interval $[-1, 1]$,

    \begin{enumerate}
    \item Where is $F$ increasing/decreasing?
    \item When does $F$ have local
      extrema?
    \item Answer the same questions with the interval replaced by
      $(-\infty, \infty)$.
    \end{enumerate}
  \end{example}
\end{frame}



\subsection{The First Fundamental Theorem of Calculus}

\begin{frame}
  \frametitle{Motivation}
  Let $f$ be a continuous function on the real numbers and consider
  \[
    F(x)= \int_a^x f(t)\; dt \,.
  \]
  We know that
  \begin{itemize}
  \item $F$ is increasing when $f$ is positive;
  \item $F$ is decreasing when $f$ is negative.
  \end{itemize}

  It is also clear that
  \begin{itemize}
  \item $F$ is concave up when $f'$ is positive;
  \item $F$ is concave down when $f'$ is negative.
  \end{itemize}

  There must be a deep connection between $F'$ and $f$.
\end{frame}

\begin{frame}
  \frametitle{The First Fundamental Theorem of Calculus}
  \begin{thm}[First Fundamental Theorem of Calculus, FTC1]
    Suppose that $f$ is continuous on the real numbers and let
    \[
      F(x) = \int_a^x f(t) \; dt \,.
    \]
    Then
    \[
      F'(x) = \ddx \int_a^x f(t) \; dt = f(x) \,.
    \]
  \end{thm}

  \textbf{Interpretation.}
  \begin{itemize}
  \item An accumulation function of $f$ is an antiderivative of $f$.
  \item The rate at which the accumulated area under a curve grows is precisely described by the curve itself.
  \end{itemize}
\end{frame}

\begin{frame}
  \vs
  \red{\textit{The idea of proof}.}
  Assume $h>0$. Note that $F(x+h) - F(x)$ is the net area of the region whose base
  is $[x, x+h]$ since
  \[
    F(x+h) - F(x) = \int_x^{x+h} f(t) \; dt \,.
  \]
  For sufficiently small $h$, the region is approximately rectagular
  and so this region is approximately $f(x) h$, i.e.,
  \[
    F(x+h) - F(x) \approx f(x) h \,.
  \]
  Upon division by $h$, we obtain
  \[
    \frac{F(x+h) - F(x)}{h} \approx f(x) \,,
  \]
  which, in the limit as $h \to 0$, yields
  \[
    F'(x) = \lim_{h\to0} \frac{F(x+h) - F(x)}{h} = f(x) \,,
  \]
  as required. \qed{}
\end{frame}

\begin{frame}
  \frametitle{Derivatives of composed accumulation functions}

  The following variation of the FTC1 is noteworthy:
  \[
    \ddx \int_a^{g(x)} f(t) \; dt = f(g(x)) g'(x) \,.
  \]
\end{frame}

\begin{frame}
  \vs
  \question{} Find the derivative of
  \begin{enumerate}
  \item  $\ds \int_2^{x^2} \ln t \; dt$.
    \vfill
  \item $\ds \int_{\cos x}^{5} t^3 \; dt$.
    \vfill
  \item $\ds \int_{x^2}^x f(t) \; dt$.
    \vfill
  \end{enumerate}
\end{frame}



\end{document}
%%% Local Variables:
%%% mode: latex
%%% TeX-master: t
%%% End:
