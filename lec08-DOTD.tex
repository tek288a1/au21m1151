\documentclass[10pt,t,presentation,ignorenonframetext,aspectratio=169]{beamer}
% \documentclass[10pt,t,handout,ignorenonframetext,aspectratio=169]{beamer}
\usepackage[default]{lato}
\usepackage{tk_beamer1}
\input{tk_packages}
\input{tk_macros}
\input{tk_environ}
\input{tk_ximera}
\usepackage{wasysym}            % for smiley
\newcommand{\zoz}{$\mathbf{ \frac{0}{0} }$}

%%%% META DATA
\newcommand{\semester}{Autumn 2021}
\newcommand{\course}{Math 1151}
\newcommand{\lecTitle}{Lecture 8: Definition of the Derivative (DOTD)}

%%%% TITLE PAGE
\title[\course]{\lecTitle}
\institute[Ohio State]
{
  \medskip
}
\date[\week]{\semester}
\author{Tae Eun Kim, Ph.D.}

\begin{document}
\begin{frame}
  \titlepage
\end{frame}

\begin{frame}
  \frametitle{Rates of change}
  The rate of change of
  \begin{itemize}
  \item \textbf{position} of an object in time: \textbf{velocity}
  \item \textbf{velocity} of an object in time: \textbf{acceleration}
  \item \textbf{revenue} generated by selling objects: \textbf{marginal revenue} % (additional revenue generated by selling one additional unit)
  \item \textbf{cost} to produce objects: \textbf{marginal cost} % (additional cost required to produce one additional unit)
  \item \textbf{profit}  gained by selling objects: \textbf{marginal
      profit} % (additional profit gained by selling one additional unit)
  \end{itemize}
\end{frame}

\begin{frame}
  \frametitle{From slopes of secant lines \ldots}
  The general formula for average rate of change is given by
  \[
    \frac{ \text{change in the function} }{ \text{change in the input to the function} } \,.
  \]
  \begin{itemize}
  \item In order to produce this rate of change, we need two distinct
    input values, e.g., two distinct points in time, and their corresponding outputs.
  \item  On the graph of the function $f$ representing the quantity of
    interest, this rate is exactly the slope of the straight line
    connecting two points $(a, f(a))$ and $(b, f(b))$.
  \item Such a line is called a \textbf{secant line} whose slope is given by
    \[
      m_{\rm sec}
      = \frac{f(b) - f(a)}{b - a} \,.
    \]
  \end{itemize}
\end{frame}

\begin{frame}
  \vs
  \begin{question}
    If $f(x) = 2x^2 + 3$, find the slope of the secant line through $(2, f(2))$ and $(x, f(x))$ in terms of $x$. Do the same when $x$ is expressed as $2+h$. The answer must be written in terms of $h$.
  \end{question}
\end{frame}

\begin{frame}
  \frametitle{\ldots to slopes of tangent lines}
  Now, an important question is: how do we get an instantaneous
  rate of change out of this?
  \begin{itemize}
  \item The slope of so-called \textbf{tangent line} represents this rate.
  \item It is given in terms of limit of slope of secant lines~\footnote{\textbf{Note.} We have two
      equivalent characterizations of this instantaneous rate of change
      depending on how we solved the previous problem.}:
    \[
      m_{\rm tan}
      = \lim_{x \to a} \frac{f(x)-f(a)}{x-a}
      = \lim_{h \to 0} \frac{f(a+h)-f(a)}{h} \,.
    \]
  \end{itemize}
\end{frame}

\begin{frame}
  \frametitle{Illustration}

  \begin{image}
    \begin{tikzpicture}
      \begin{axis}[
        domain=0:2, range=0:6,ymax=6,ymin=0,
        axis lines =left, xlabel=$x$, ylabel=$y$,
        every axis y label/.style={at=(current axis.above origin),anchor=south},
        every axis x label/.style={at=(current axis.right of origin),anchor=west},
        xtick={1,1.666}, ytick={1,3},
        xticklabels={$a$,$a+h$}, yticklabels={$f(a)$,$f(a+h)$},
        axis on top,
        ]
        \addplot [penColor2!15!background, domain=(0:2)] {-3.348+4.348*x};
        \addplot [penColor2!32!background, domain=(0:2)] {-2.704+3.704*x};
        \addplot [penColor2!49!background, domain=(0:2)] {-1.994+2.994*x};
        \addplot [penColor2!66!background, domain=(0:2)] {-1.326+2.326*x};
        \addplot [penColor2!83!background, domain=(0:2)] {-0.666+1.666*x};
        \addplot [textColor,dashed] plot coordinates {(1,0) (1,1)};
        \addplot [textColor,dashed] plot coordinates {(0,1) (1,1)};
        \addplot [textColor,dashed] plot coordinates {(0,3) (1.666,3)};
        \addplot [textColor,dashed] plot coordinates {(1.666,0) (1.666,3)};
        \addplot [very thick,penColor, smooth,domain=(0:1.833)] {-1/(x-2)};
        \addplot[color=penColor,fill=penColor,only marks,mark=*] coordinates{(1.666,3)};  %% closed hole
        \addplot[color=penColor,fill=penColor,only marks,mark=*] coordinates{(1,1)};  %% closed hole
        \addplot [very thick,penColor2, smooth,domain=(0:2)] {x};
      \end{axis}
    \end{tikzpicture}
  \end{image}
\end{frame}

\begin{frame}
  \frametitle{Definition of derivative}
  % The instantaneous rate of change of a function is called its \textbf{derivative}. Let's formalize our previous discussion:
  \begin{defn}
    The \textbf{derivative} of $f$ at $a$ is
    \begin{align*}
      \biggl[ \frac{d}{dx} f(x) \biggr]_{x=a}
      & = \lim_{h\to 0} \frac{f(a+h) - f(a)}{h}
        \qquad \text{($h \to 0$ characterization)} \\
      & = \lim_{x \to a} \frac{f(x)-f(a)}{x-a}
        \qquad \text{($x \to a$ characterization)} \,.
    \end{align*}
    If this limit exists, then we say that $f$ is \textbf{differentiable} at $a$. If this limit does not exist for a given value of $a$, then $f$ is \textbf{non-differentiable} at $a$.
  \end{defn}

  \begin{notn}
    The following are equivalent notations for derivatives:
    \[
      \biggl[ \frac{d}{dx} f(x) \biggr]_{x=a}
      = f'(a) \,.
    \]
  \end{notn}
\end{frame}

\begin{frame}
  \vs
  \begin{example}
    If $f(x) = x^2 - 2x$, find the derivative of $f$ at $2$ using the $h \to 0$ characterization.
  \end{example}
\end{frame}

\begin{frame}
  \vs
  \begin{example}
    Find the derivative of $f(x) = x^2+x+1$ at $x=-1$ using the $x \to a$ characterization.
  \end{example}
\end{frame}

\begin{frame}
  \vs
  \begin{example}
    Find an equation for the line tangent to the curve $y = f(x) = 1/(3-x)$ at the point $(2,1)$.
  \end{example}
\end{frame}

\begin{frame}
  \vs
  \begin{example}
    The position of an object moving along a straight line is given by $s(t) = \sqrt{t+3}$. Find its (instanteneous) velocity at time $t = 6$.
  \end{example}
\end{frame}

\end{document}
%%% Local Variables:
%%% mode: latex
%%% TeX-master: t
%%% End:
