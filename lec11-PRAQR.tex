\documentclass[10pt,t,presentation,ignorenonframetext,aspectratio=169]{beamer}
% \documentclass[10pt,t,handout,ignorenonframetext,aspectratio=169]{beamer}
\usepackage[default]{lato}
\usepackage{tk_beamer1}
\input{tk_packages}
\input{tk_macros}
\input{tk_environ}
\input{tk_ximera}
\usepackage{wasysym}            % for smiley
\newcommand{\zoz}{$\mathbf{ \frac{0}{0} }$}

%%%% META DATA
\newcommand{\semester}{Autumn 2021}
\newcommand{\course}{Math 1151}
\newcommand{\lecTitle}{Lecture 11: Product Rule and Quotient Rule (PRAQR)}

%%%% TITLE PAGE
\title[\course]{\lecTitle}
\institute[Ohio State]
{
  \medskip
}
\date[\week]{\semester}
\author{Tae Eun Kim, Ph.D.}

\begin{document}
\begin{frame}
  \titlepage
\end{frame}

\begin{frame}
  \frametitle{Motivation}
  \vs
  \question{} Let $f(x)=(x^2+1)$ and $g(x)=(x^3-3x)$. Suppose that
  you want to compute
  \[
    \ddx [f(x)g(x)] \,.
  \]
  \begin{itemize}
  \item We can proceed by expanding $f(x)g(x)$ then differentiating
    the result using the sum rule and power rule.
  \item This can get very tedious.
  \item At times, the strategy may not even be applicable.
  \end{itemize}
\end{frame}

\begin{frame}
  \frametitle{Product Rule}
  \begin{thm}[Product Rule]
    If $f$ and $g$ are differentiable, then
    \[
      \ddx [f(x)g(x)] = f(x)g'(x)+f'(x)g(x) \,.
    \]
  \end{thm}
\end{frame}

\begin{frame}
  \vs
  Let's revisit the opening example: \\ \vspace{0.5em}
  \question{}  Let $f(x)=(x^2+1)$ and $g(x)=(x^3-3x)$. Using the
  product rule, compute
  \[
    \ddx [f(x)g(x)] \,.
  \]
\end{frame}

\begin{frame}
  \vs
  \question{} Compute
  \[
    \ddx(xe^x-e^x) \,.
  \]
\end{frame}

\begin{frame}
  \frametitle{Quotient Rule}
  \begin{thm}[Quotient Rule]
    If $f$ and $g$ are differentiable, then
    \[
      \ddx \frac{f(x)}{g(x)} = \frac{f'(x)g(x)-f(x)g'(x)}{g(x)^2}.
    \]
  \end{thm}

  \begin{itemize}
  \item Viewing the quotient as a product $f(x) (1/g(x))$, we can use the
    product rule to derive the above.
  \item But in order to do that, we need to know what $\ddx (1/g(x))$
    is.
  \end{itemize}

  \note{
  \begin{align*}
    \lim_{h \to 0} \left\{
    \frac{1}{h} \left( \frac{1}{g(x+h)} - \frac{1}{g(x)} \right)
    \right\}
    & = \lim_{h \to 0} \left\{
      \frac{1}{h} \frac{g(x) - g(x+h)}{g(x+h)g(x)}
      \right\} \\
    & = -\lim_{h \to 0} \left\{
      \frac{g(x+h)-g(x)}{h} \frac{1}{g(x+h)g(x)}
      \right\} \\
    & = -\frac{g'(x)}{g(x)^2} \,.
  \end{align*}
  }
\end{frame}

\begin{frame}
  \vs
  \question{} Compute:
  \[
    \ddx \frac{x^2+1}{x^3-3x} \,.
  \]
\end{frame}

\begin{frame}
  \vs
  \question{} Compute:
  \[
    \ddx \frac{625-x^2}{\sqrt{x}}
  \]
  in two ways. First using the quotient rule and then using the product
  rule.
\end{frame}


\end{document}
%%% Local Variables:
%%% mode: latex
%%% TeX-master: t
%%% End:
