% \documentclass[10pt,t,presentation,ignorenonframetext,aspectratio=169]{beamer}
\documentclass[10pt,t,handout,ignorenonframetext,aspectratio=169]{beamer}
\usepackage[default]{lato}
\usepackage{tk_beamer1}
\input{tk_packages}
\input{tk_macros}
\input{tk_environ}
\input{tk_ximera}
\usepackage{wasysym}            % for smiley
\newcommand{\zoz}{$\mathbf{ \frac{0}{0} }$}

% some inverse trigs
\DeclareMathOperator{\arcsec}{arcsec}
\DeclareMathOperator{\arccot}{arccot}
\DeclareMathOperator{\arccsc}{arccsc}

%%%% META DATA
\newcommand{\semester}{Autumn 2021}
\newcommand{\course}{Math 1151}
\newcommand{\lecTitle}{Lecture 35: Antiderivatives and Area (AAA)}

%%%% TITLE PAGE
\title[\course]{\lecTitle}
\institute[Ohio State]
{
  \medskip
}
\date[\week]{\semester}
\author{Tae Eun Kim, Ph.D.}

\begin{document}
\begin{frame}
  \titlepage
\end{frame}

\begin{frame}
  \frametitle{Relating antiderivatives and
    areas}

  In investigating connection between antiderivatives and area, we
  will use our favorite position-velocity-acceleration triple for
  illustration.

  \vs

  \textbf{Displacement vs. distance.}\\
  \vspace{0.5em}
  Consider a moving object (in 1-D) from time $t = a$ to $t = b$. The
  \textit{displacement} measures the difference in position. In other
  words,
  \[
    (\text{displacement})
    = (\text{terminal position}) - (\text{initial position})
    = s(b) - s(a) \,.
  \]

  \textbf{Note:}
  \begin{itemize}
  \item When an object moves without changing directions, the (traveled)
    distance equals the absolute value of displacement.
  \item However, when it changes directions along the course of
    movement, they are going to be different.
  \item In particular, distance is always going to be positive, but
    displacement may be negative.
  \end{itemize}
\end{frame}

\begin{frame}
  \frametitle{Simple case:  uniform velocity}
  Now consider a simple situation where an object is moving at a
  constant velocity $v_{0}$ for $a \le t \le b$. Then the displacement
  is simply the velocity multiplied by the time traveled, i.e.,

  \[
    (\text{displacement}) = v_0 (b-a)\,. \tag{constant velocity}
  \]

  \begin{itemize}
  \item The graph of velocity against time is a horizontal line.
  \item The displacement is exactly equal to the (signed) area of rectangle
    between the velocity curve (i.e., the straight line) and the
    horizontal time axis on $[a, b]$.
  \end{itemize}
\end{frame}

\begin{frame}
  \frametitle{Motion with changing velocity}
  Then how would we calculate the displacement when the
  object is moving at a varying velocity?

  \begin{itemize}
  \item Assuming that it moves at a constant velocity over a small
    interval of time, we can approximate displacement using Riemann
    sums;
  \item The quality of approximation improves as we increase the number of
    approximating rectangles;
  \item We obtain the exact displacement once
    we take the limit of general Riemann sum as the number $n$ of
    rectangles approaches infinity, that is
    \[
      (\text{displacement})
      =  \int_a^b v(t) \; dt \,. \tag{variable velocity}
    \]
  \end{itemize}
\end{frame}

\begin{frame}
  \frametitle{The connection}
  \begin{itemize}
  \item   But recall that the displacement is the difference between the
    terminal and initial positions, i.e., $s(b)-s(a)$. Thus
    \[
      \int_a^b v(t) \; dt = s(b) - s(a) \,. \tag{$\star$}
    \]
  \item   Noting that $s'(t) = v(t)$, i.e., $s(t)$ is an antiderivative of
    $v(t)$, we may interpret the equation ($\star$) in a general setting
    as:
    \vspace{0.5em}
    \begin{center}
      \begin{minipage}{0.9\linewidth}
        \textit{
          The net \blue{\textbf{area}} between the curve $y = f(x)$ and the $x$-axis on $[a,
          b]$ is the difference of values of its \blue{\textbf{antiderivative}} at the endpoints.
        }
      \end{minipage}
    \end{center}
    \vspace{0.5em}
  \item The statement above can be written as
    \[
      \int_a^b f(x) \; dx = F(b) - F(a) \,,
    \]
    where $F$ is an antiderivative of $f$. This is the celebrated
    Fundamental Theorem of Calculus.
  \end{itemize}
\end{frame}

\begin{frame}
  \frametitle{Example}
  \begin{question}
    Assume an object is moving along a straight line with the velocity
    $v(t) = 3 - 3t^2$ for $0 \le t \le 2$. Find the displacement of the
    object over the time interval $[0, 4]$.
  \end{question}
\end{frame}

\end{document}
%%% Local Variables:
%%% mode: latex
%%% TeX-master: t
%%% End:
