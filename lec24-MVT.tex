% \documentclass[10pt,t,presentation,ignorenonframetext,aspectratio=169]{beamer}
\documentclass[10pt,t,handout,ignorenonframetext,aspectratio=169]{beamer}
\usepackage[default]{lato}
\usepackage{tk_beamer1}
\input{tk_packages}
\input{tk_macros}
\input{tk_environ}
\input{tk_ximera}
\usepackage{wasysym}            % for smiley
\newcommand{\zoz}{$\mathbf{ \frac{0}{0} }$}

% some inverse trigs
\DeclareMathOperator{\arcsec}{arcsec}
\DeclareMathOperator{\arccot}{arccot}
\DeclareMathOperator{\arccsc}{arccsc}

%%%% META DATA
\newcommand{\semester}{Autumn 2021}
\newcommand{\course}{Math 1151}
\newcommand{\lecTitle}{Lecture 24: Extreme and Mean Value Theorems (MVT)}

%%%% TITLE PAGE
\title[\course]{\lecTitle}
\institute[Ohio State]
{
  \medskip
}
\date[\week]{\semester}
\author{Tae Eun Kim, Ph.D.}

\begin{document}
\begin{frame}
  \titlepage
\end{frame}


\begin{frame}
  \frametitle{Extreme Values of a Function}
  \begin{defn}
    \begin{itemize}
    \item A function $f$ has a \dfn{global maximum} at $a$ if $f(a)\ge
      f(x)$ for every $x$ in the domain of the function.
    \item A function $f$ has a \dfn{global minimum} at $a$ if $f(a)\le
      f(x)$ for every $x$ in the domain of the function.
    \end{itemize}
    A \dfn{global extremum}\index{extremum!global} is either a
    global maximum or a global minimum.
  \end{defn}
\end{frame}

\begin{frame}
  \vs{}
  \question{} Let $f$ be the function given by the graph below.
  \begin{image}[0.35\linewidth]
    \begin{tikzpicture}[baseline=(current bounding box.north)]
      \begin{axis}[
        xmin=0, xmax=5, ymin=-1,ymax=2.2,
        unit vector ratio*=1 1 1,
        axis lines =middle, xlabel=$x$, ylabel=$y$,
        every axis y label/.style={at=(current axis.above origin),anchor=south},
        every axis x label/.style={at=(current axis.right of origin),anchor=west},
        xtick={0,...,5}, ytick={-1,...,2},
        ]
        \addplot[ultra thick, color=penColor, smooth, domain=(0:1)] {(x-1)^2};
        \addplot[ultra thick, color=penColor, smooth, domain=(1:2)] {2*(x-1)^2};
        \addplot[ultra thick, color=penColor, smooth, domain=(2:5)] {4-x};
        \addplot [color=penColor,fill=penColor,only marks,mark=*] coordinates{(0,1)};
        \addplot [color=penColor,fill=penColor,only marks,mark=*] coordinates{(5,-1)};
      \end{axis}
    \end{tikzpicture}
  \end{image}
  \begin{enumerate}
  \item Find the $x$-coordinate of the point where the function $f$
    has a global maximum and a global minimum.
    \vfill
  \item Find the $x$-coordinate(s) of the point(s) where the function
    $f$ has a local minimum and a local maximum.
    \vfill
  \end{enumerate}
\end{frame}


\begin{frame}
  \vs{}

  \begin{alertblock}{Caution}
    A function may not attain a global extremum on its
    domain. Consider the following graph.
  \end{alertblock}
  \vfill
  \begin{image}[0.95\linewidth]
    \begin{tikzpicture}
      \begin{axis}[
        xmin=-1, xmax=6, ymin=-2.5,ymax=1.5,
        unit vector ratio*=1 1 1,
        axis lines =middle, xlabel=$x$, ylabel=$y$,
        every axis y label/.style={at=(current axis.above origin),anchor=south},
        every axis x label/.style={at=(current axis.right of origin),anchor=west},
        xtick={-1,...,6}, ytick={-2,...,1},
        ]
        \addplot[ultra thick, color=penColor, smooth, domain=(-1:1)] {-2*x^2};
        \addplot[ultra thick, color=penColor, smooth, samples=500, domain=(1:3)] {-2*2^(-1/3)*(3-x)^(1/3)};
        \addplot[ultra thick, color=penColor, smooth, samples=100, domain=(3:5)] {2^(-1/3)*(x-3)^(1/3)};
        \addplot[ultra thick, color=penColor, smooth, domain=(5:6)] {x-6};
        \addplot[ultra thick, color=penColor, smooth]plot coordinates{(3,-.1) (3,0)};
        \addplot [color=penColor,fill=background,only marks,mark=*] coordinates{(-1,-2)};
        \addplot [color=penColor,fill=background,only marks,mark=*] coordinates{(5,1)};
        \addplot [color=penColor,fill=background, only marks, mark=*] coordinates{(5,-1)};
        \addplot[color=penColor,fill=background,only marks, mark=*] coordinates{(6,0)};
        \addplot[color=penColor,fill=background,only marks, mark=*] coordinates{(1,-2)};
      \end{axis}
    \end{tikzpicture}
    \hspace{0.5in}
    \begin{tikzpicture}
      \begin{axis}[
        xmin=-1, xmax=6, ymin=-2.5,ymax=1.5,
        unit vector ratio*=1 1 1,
        axis lines =middle, xlabel=$x$, ylabel=$y$,
        every axis y label/.style={at=(current axis.above origin),anchor=south},
        every axis x label/.style={at=(current axis.right of origin),anchor=west},
        xtick={-1,...,6}, ytick={-2,...,1},
        ]
        \addplot[ultra thick, color=penColor, smooth, domain=(-1:1)] {-2*x^2};
        \addplot[ultra thick, color=penColor, smooth, samples=500, domain=(1:3)] {-2*2^(-1/3)*(3-x)^(1/3)};
        \addplot[ultra thick, color=penColor, smooth, samples=100, domain=(3:5)] {2^(-1/3)*(x-3)^(1/3)};
        \addplot[ultra thick, color=penColor, smooth, domain=(5:6)] {x-6};
        \addplot[ultra thick, color=penColor, smooth]plot coordinates{(3,-.1) (3,0)};
        \addplot [color=penColor,fill=background,only marks,mark=*] coordinates{(-1,-2)};
        \addplot [color=penColor,fill=penColor,only marks,mark=*] coordinates{(5,1)};
        \addplot [color=penColor,fill=background, only marks, mark=*] coordinates{(5,-1)};
        \addplot[color=penColor,fill=background,only marks, mark=*] coordinates{(6,0)};
        \addplot[color=penColor,fill=background,only marks, mark=*] coordinates{(1,-2)};
      \end{axis}
    \end{tikzpicture}
  \end{image}
  \vfill
\end{frame}

\begin{frame}
  \frametitle{Extreme Value Theorem}
  So when do we know for sure that a function attains a global extremum?
  \begin{thm}[The Extreme Value Theorem]
    If $f$ is continuous on the closed interval $[a,b]$, then there
    are points $c$ and $d$ in $[a,b]$, such that $(c,f(c))$ is a
    global maximum and $(d,f(d))$ is a global minimum on $[a, b]$.
  \end{thm}

  \vs{}

  \textbf{Remark.}
  \begin{itemize}
  \item The theorem does not hold if we work on an open interval
    $(a,b)$. Can you come up with an example?
  \item The theorem does not hold if we work on a closed interval
    $[a,b]$ but $f$ is not continuous. Can you come up with an
    example?
  \item Do the previous examples invalidate the EVT?
  \end{itemize}
\end{frame}



\begin{frame}
  \frametitle{Remarks}
  In finding global extrema:
  \begin{itemize}
  \item The EVT guarantees the existence of global extrema when we work with a function $f$ that is continuous on a closed interval.
  \item The global extrema may occur either at the end points of the interval or in the interior.
  \item If a global extremum occurs at an interior points, then it is also a local extremum thus a critical point.
  \item Hence, all \textbf{interior critical points} as well as the \textbf{end points} are candidates for global extrema.
  \item In sum, in order to locate global extrema, we evaluate the
    function at these candidate points and compare them to determine
    global maxima and global minima.
  \end{itemize}
\end{frame}

\begin{frame}
  \vs{}

  \question{} Let $f(x)=x^2e^{-x}$, for $ -2\le x\le1$. Locate the global
  extrema of $f$ on the closed interval $[-2,1]$.
\end{frame}


\begin{frame}
  \frametitle{Rolle's Theorem}

  \begin{itemize}
  \item We now explore an intricate relation between \textbf{average
      rate} and \textbf{instantaneous rate of change}.
  \item In some context, this can be translated as relation between
    \textit{average velocity} and \textit{instantaneous velocity} or
    \textit{slope of secant line} vs. \textit{slope of tangent line}.
  \end{itemize}

  \begin{thm}[Rolle's Theorem]
    Suppose that $f$ is differentiable on the interval $(a,b)$, is
    continuous on the interval $[a,b]$, and $f(a)=f(b)$. Then
    \[
      f'(c)=0
    \]
    for some $a<c<b$.
  \end{thm}
\end{frame}

\begin{frame}
  \frametitle{Illustration of Rolle's Theorem}
  \begin{image}
    \begin{tikzpicture}
      \begin{axis}[
        xmin=0, xmax=4.5,ymin=1,ymax=5,
        axis lines =left, xlabel=$x$, ylabel=$y$,
        every axis y label/.style={at=(current axis.above origin),anchor=south},
        every axis x label/.style={at=(current axis.right of origin),anchor=west},
        xtick={1,2,3}, xticklabels={$a$,$c$,$b$},
        ytickmin=1, ytickmax=0,
        axis on top,
        ]
        \addplot [draw=none, fill=fill1,domain=(1:3)] {5} \closedcycle;
        \addplot [very thick,penColor, smooth] {-(x-2)^2+4};
        \addplot [very thick,penColor2, smooth] {4};
        \node at (axis cs:.4,2.5) [penColor] {$f$};
        \addplot [textColor,dashed] plot coordinates {(2,0) (2,4)};
        \addplot [textColor,dashed] plot coordinates {(1,3) (3,3)};
        \addplot[color=penColor3,fill=penColor3,only marks,mark=*] coordinates{(2,4)};
        \addplot[color=penColor,fill=penColor,only marks,mark=*] coordinates{(1,3)};
        \addplot[color=penColor,fill=penColor,only marks,mark=*] coordinates{(3,3)};
      \end{axis}
    \end{tikzpicture}
  \end{image}
\end{frame}
\begin{frame}
  \frametitle{The Mean Value Theorem}
  A generalization of Rolle's theorem is coined as the mean value
  theorem.
  \begin{thm}[The Mean Value Theorem]
    Suppose that $f$ has a derivative on the interval $(a,b)$ and is
    continuous on the interval $[a,b]$. Then
    \[
      f'(c)=\frac{f(b)-f(a)}{b-a}
    \]
    for some $a<c<b$.
  \end{thm}

  \vs{}

  \begin{itemize}
  \item The mean value theorem says that when a function is continuous
    on a closed interval and differentiable in its interior, then its
    average rate of change must be achieved at some point as an
    instantaneous rate of change.
  \end{itemize}
\end{frame}

\begin{frame}
  \frametitle{Illustration of the Mean Value Theorem}
  \begin{image}
    \begin{tikzpicture}
      \begin{axis}[
        xmin=.5, xmax=5.5,ymin=0,ymax=3.1,
        axis lines =center, xlabel=$x$, ylabel=$y$,
        every axis y label/.style={at=(current axis.above origin),anchor=south},
        every axis x label/.style={at=(current axis.right of origin),anchor=west},
        xtick={1,2.04,5}, xticklabels={$a$,$c$,$b$},
        ytickmin=1, ytickmax=0,
        axis on top,
        ]
        \addplot [draw=none, fill=fill1,domain=(1:5)] {3.1} \closedcycle;
        \addplot [penColor2!40!background,very thick,dashed] plot coordinates {(1,.84+1.5) (5,1.5-.96)};
        \addplot [textColor,dashed] plot coordinates {(2.04,0) (2.04,1.5+.89)};
        \addplot [very thick,penColor, smooth,domain=(1:5)] {sin(deg(x))+1.5};
        \addplot [very thick,penColor2,domain=(.5:5.5)] {-.45*(x-2.04)+.89+1.5};
        % \node at (axis cs:.4,2.5) [penColor] {$f(x)$};
        \addplot[color=penColor,fill=penColor,only marks,mark=*] coordinates{(1,.84+1.5)};  %% closed hole
        \addplot[color=penColor,fill=penColor,only marks,mark=*] coordinates{(5,-.96+1.5)};  %% closed hole
        \addplot[color=penColor3,fill=penColor3,only marks,mark=*] coordinates{(2.04,.89+1.5)};  %% closed hole
      \end{axis}
    \end{tikzpicture}
  \end{image}
\end{frame}

\begin{frame}
  \vs{}
  There are many interesting questions that can be answered using the
  mean value theorem.
  \begin{example}
    Suppose you toss a ball into the air and then catch it. Must the
    ball's vertical velocity have been zero at some point?
  \end{example}

  \note{
    \textbf{Answer.} Let $s(t)$ be the vertical location of the ball at
    time $t$. Suppose the ball was tossed at $t = 0$ and caught at
    $t = t_f$. It is plausible to assume that $s$ is continuous on
    $[0, t_f]$ and differentiable on $(0, t_f)$. Moreover,
    $s(0) = s(t_f) = 0$ and so
    \[
      \frac{s(t_f) - s(0)}{t_f - 0} = 0 \,.
    \]
    So by Rolle's theorem, there must be a time $c$ in $(0, t_f)$ at
    which $s'(c) = v(c) = 0$.
  }
\end{frame}


\begin{frame}
  \vs{}

  \begin{example}
    Suppose you drive a car from toll booth on a toll road to another
    toll booth $30$ miles away in half of an hour. Was there a moment
    that you violated the speed limit of 55 mph?
  \end{example}

  \note{
    \textbf{Answer.} Let $s(t)$ be the distance droven from the first
    toll booth at time $t$ with $t = 0$ corresponding to the time you
    left the first booth. It is reasonable to assume that $s$ is
    continuous on $[0, 1/2]$ and differentiable on $(0, 1/2)$. So the
    mean value theorem is applicable and it tells us that there is a
    time $c$ in $(0, 1/2)$ such that
    \[
      s'(c) = \frac{s(1/2) - s(0)}{1/2} = 60 \,,
    \]
    meaning that you drove at instantaneous velocity of 60 mph, 5 mph over
    the speed limit, at that particular moment.
  }
\end{frame}


\begin{frame}
  \vs{}
  \begin{example}
    Suppose the derivative of a function is 0 on an open interval
    $I$. What can we say about $f$?
  \end{example}

  \note{
    \textbf{Answer.} Let $a<b$ be two distinct points in $I$. Since $f$ is
    continuous on $[a,b]$ and differentiable on $(a,b)$, by the mean value
    theorem, we know that
    \[
      \frac{f(b)-f(a)}{b-a} = f'(c) \,,
    \]
    for some $c$ in $(a,b)$. But since $f' = 0$ on $I$, it must be the
    case that $f(a) = f(b)$. Since $a$ and $b$ were arbitrary, it must be
    the case that $f$ is a constant function.
  }

\end{frame}


\begin{frame}
  \vs{}
  \begin{example}
    Suppose two different functions have the same derivative. What can
    you say about the relationship between the two functions?
  \end{example}

  \note{
    \textbf{Answer.} Let $f$ and $g$ be two functions with the same
    derivative on $(a,b)$. If we define $h(x) = f(x)-g(x)$, then
    $h' = 0$ on the interval $(a,b)$. By the previous example,
    $h(x) = C$ for some constant $C$. This implies that $f$ and $g$
    differ by a constant.
  }

\end{frame}

\end{document}
%%% Local Variables:
%%% mode: latex
%%% TeX-master: t
%%% End:
