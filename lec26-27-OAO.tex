% \documentclass[10pt,t,presentation,ignorenonframetext,aspectratio=169]{beamer}
\documentclass[10pt,t,handout,ignorenonframetext,aspectratio=169]{beamer}
\usepackage[default]{lato}
\usepackage{tk_beamer1}
\input{tk_packages}
\input{tk_macros}
\input{tk_environ}
\input{tk_ximera}
\usepackage{wasysym}            % for smiley
\newcommand{\zoz}{$\mathbf{ \frac{0}{0} }$}

% some inverse trigs
\DeclareMathOperator{\arcsec}{arcsec}
\DeclareMathOperator{\arccot}{arccot}
\DeclareMathOperator{\arccsc}{arccsc}

%%%% META DATA
\newcommand{\semester}{Autumn 2021}
\newcommand{\course}{Math 1151}
\newcommand{\lecTitle}{Lecture 26-27: Optimization (O \& AO)}

%%%% TITLE PAGE
\title[\course]{\lecTitle}
\institute[Ohio State]
{
  \medskip
}
\date[\week]{\semester}
\author{Tae Eun Kim, Ph.D.}

\begin{document}
\begin{frame}
  \titlepage
\end{frame}

\begin{frame}
  \frametitle{Basic Idea and Terminology}
  An \dfn{optimization problem} is a problem where you need to maximize
  or minimize some quantity under some constraints. This can be
  accomplished using the tools of differential calculus that we have
  already developed.

  \vs

  \textbf{Terminology.}
  \begin{itemize}
  \item \textbf{constraints:} conditions imposed on variables
  \item \textbf{objective functions:} the quantities desired to be optimized
  \end{itemize}
\end{frame}

\begin{frame}
  \frametitle{A Solitary Local Extremum}
  \begin{itemize}
  \item The extreme value theorem guarantees the existence of global
    extrema only on a closed interval.
  \item On intervals that are not closed, the theorem is not
    applicable. Yet, when there is only one local extreme value, we
    can say something about global extrema.
  \end{itemize}

  \begin{thm}
    Suppose $f$ is continuous on an interval $I$ that contains exactly
    one local extremum at $c$.
    \begin{itemize}
    \item If a local maximum occurs at $c$, \\
      then $f(c)$ is the global maximum of $f$ on $I$.
    \item If a local minimum occurs at $c$, \\
      then $f(c)$ is the global minimum of $f$ on $I$.
    \end{itemize}
  \end{thm}
\end{frame}

\begin{frame}
  \vs
  \begin{example}[Maximum area rectangles]
    Of all rectangles of perimeter 12, which side lengths give the
    greatest area?
  \end{example}
\end{frame}


\begin{frame}
  \vs
  \begin{example}[Minimum perimeter rectangles]
    Of all rectangles of area 100, which has the smallest perimeter?
  \end{example}
\end{frame}


\begin{frame}
  \vs
  \begin{example}[Rectangles beneath a semicircle]
    A rectangle is constructed with its base on the diameter of a
    semicircle with radius 5 and its two other vertices on the
    semicircle. What are the dimensions of the rectangle with maximum
    area?
  \end{example}
\end{frame}

\begin{frame}
  \vs
  \begin{example}[Minimum distance]
    Find the point $P$ on the curve $y = x^2$ that is closest to the
    point $(18,0)$. What is the least distance between $P$ and $(18,0)$?
  \end{example}
\end{frame}

% \begin{frame}
%   \vs
%   \begin{example}[Airline regulations]
%     Suppose an airline policy states that all baggage must be box-shaped
%     with a sum of length, width, and height not exceeding 64 in. What
%     are the dimensions and volume of a square-based box with the
%     greatest volume under these conditions?
%   \end{example}
% \end{frame}


% \begingroup
% \small
% \begin{frame}
%   \vs
%   The following examples are more involved and require careful thinking.
%   \begin{example}
%     You are making cylindrical containers to contain a given volume.
%     Suppose that the top and bottom are made of a material that is $N$
%     times as expensive (cost per unit area) as the material used for the
%     lateral side of the cylinder.  Find (in terms of $N$) the ratio of
%     height to base radius of the cylinder that minimizes the cost of
%     making the containers.
%   \end{example}
% \end{frame}

% \begin{frame}
%   \vs
%   \begin{example}
%     You want to sell a certain number $n$ of items in order to maximize
%     your profit.  Market research tells you that if you set the price at
%     \$$1.50$, you will be able to sell $5000$ items, and for every
%     $10$ cents you lower the price below \$$1.50$ you will be able to
%     sell another $1000$ items.  Suppose that your fixed costs
%     (``start-up costs'') total
%     \$$2000$, and the per item cost of production (``marginal cost'') is
%     \$$0.50$.  Find the price to set per item and the number of items
%     sold in order to maximize profit, and also determine the maximum
%     profit you can get.
%   \end{example}
% \end{frame}


\begin{frame}
  \vs
  \begin{example}
    If you fit the largest possible cone inside a sphere, what fraction of the
    volume of the sphere is occupied by the cone?  (Here by ``cone'' we mean a
    right circular cone, i.e., a cone for which the base is perpendicular to
    the axis of symmetry, and for which the cross-section cut perpendicular to
    the axis of symmetry at any point is a circle.)
  \end{example}
  \vs
  \hfill
  \begin{minipage}{0.4\linewidth}
    \begin{image}[0.9\linewidth]
      \begin{tikzpicture}
        \draw[very thick,penColor2!20!background] (2,0) arc (0:180:2 and .7);% top half of ellipse
        \draw [penColor, very thick] (0,1) ellipse (1.7 and .5);
        \draw[penColor, very thick] (1.7,.95) -- (0,-2);
        \draw[penColor, very thick] (-1.7,.95) -- (0,-2);
        \draw[very thick,penColor2] (-2,0) arc (180:360:2 and .7);% bottom half of ellipse


        \draw [penColor2, very thick] (0,0) circle [radius=2];

        \draw[penColor2, dashed, very thick] (0,0) -- (2,0);
        \draw[penColor, dashed, very thick] (0,1) -- (1.7,1);
        \draw[penColor, dashed, very thick] (0,1) -- (0,-2);

        \node [below,penColor2] at (1.5,0) {$R$};
        \node [above,penColor] at (.85,1) {$r$};
        \node [left,penColor] at (0,-.33) {$h$};

        \node [penColor,left] at (-1.5,1.42) {$V_{\text{c}} = \frac{\pi r^2h}{3}$};
        \node [penColor2, right] at (1.5,-1.42) {$V_{s} = \frac{4\pi R^3}{3}$};
      \end{tikzpicture}
    \end{image}
  \end{minipage}
\end{frame}

\begin{frame}
  \vs
  \begingroup
  \footnotesize
  \begin{example}
    Suppose you want to reach a point $A$ that is located across the
    sand from a nearby road. Suppose that the road is straight, and $b$
    is the distance from $A$ to the closest point $C$ on the road.  Let
    $v$ be your speed on the road, and let $w$, which is less than $v$,
    be your speed on the sand.  Right now you are at the point $D$,
    which is a distance $a$ from $C$.  At what point $B$ should you turn
    off the road and head across the sand in order to minimize your
    travel time to $A$?
  \end{example}
  \endgroup
  \vs
  \hfill
  \begin{minipage}{0.4\linewidth}
    \begin{image}[0.9\linewidth]
      \begin{tikzpicture}
        \draw[fill2, fill=fill1!60!background] (0,-.2) rectangle (6,3.5);
        \draw[fill1, fill=fill2] (0,.4) rectangle (6,1);

        % \node[penColor] at (.5,.75) {\scalebox{-2}[2] \Bicycle};

        \draw [penColor, fill] (1,.75) circle [radius=.07];
        \draw [penColor, fill] (2.5,.75) circle [radius=.07];
        \draw [penColor, fill] (5,.75) circle [radius=.07];
        \draw [penColor, fill] (5,3) circle [radius=.07];

        \draw[penColor2, very thick, ->] (1.2,.75) -- (2.3,.75);
        \draw[penColor2, very thick, ->] (2.6,.85) -- (4.9,2.9);

        \draw[penColor, very thick, dashed] (5,.75) -- (5,3);
        \draw[penColor, very thick, dashed] (5,.75) -- (2.5,.75);
        % \draw[penColor, very thick, dashed] (1,.4) -- (5,.4);
        \draw[decoration={brace,mirror,raise=.2cm},decorate,thin] (1,.5)--(5,.5);

        \node [right,penColor] at (5,3) {$A$};
        \node [below,penColor] at (1,.75) {$D$};
        \node [below,penColor] at (2.5,.75) {$B$};
        \node [below,penColor] at (5,.75) {$C$};
        \node [right,penColor] at (5,2) {$b$};
        \node [above,penColor] at (4,.75) {$x$};
        \node [left,penColor2] at (3.8,2) {$w$};
        \node [above,penColor2] at (1.75,.75) {$v$};
        \node [below,penColor] at (3,.2) {$a$};
      \end{tikzpicture}
    \end{image}
  \end{minipage}
\end{frame}
%\endgroup



\end{document}
%%% Local Variables:
%%% mode: latex
%%% TeX-master: t
%%% End:
