\documentclass[10pt,t,presentation,ignorenonframetext,aspectratio=169]{beamer}
% \documentclass[10pt,t,handout,ignorenonframetext,aspectratio=169]{beamer}
\usepackage[default]{lato}
\usepackage{tk_beamer1}
\input{tk_packages}
\input{tk_macros}
\input{tk_environ}
\input{tk_ximera}
\usepackage{wasysym}            % for smiley
\newcommand{\zoz}{$\mathbf{ \frac{0}{0} }$}

%%%% META DATA
\newcommand{\semester}{Autumn 2021}
\newcommand{\course}{Math 1151}
\newcommand{\lecTitle}{Lecture 12: Chain Rule (CR)}

%%%% TITLE PAGE
\title[\course]{\lecTitle}
\institute[Ohio State]
{
  \medskip
}
\date[\week]{\semester}
\author{Tae Eun Kim, Ph.D.}

\begin{document}
\begin{frame}
  \titlepage
\end{frame}

\begin{frame}
  \frametitle{Chain Rule}
  % \begin{itemize}
  % \item The \textbf{chain rule} spells out the method of differentiating composite
  %   functions, that is, functions in the form
  %   \[
  %     (f \circ g) (x) = f( g(x) ) \,.
  %   \]
  %   % \item The idea is that the derivative of the composite function can
  %   %   be written in terms of derivatives of individual constituent
  %   %   functions.
  % \end{itemize}
  The \textbf{chain rule} spells out the method of differentiating composite functions.
  \begin{thm}[Chain Rule]
    If $f$ and $g$ are differentiable, then
    \[
      \ddx f(g(x)) = f'(g(x))g'(x) \,.
    \]
  \end{thm}
\end{frame}

\begin{frame}
  \vs
  \question{} Compute $\ds \ddx \sin(1+2x)$.
\end{frame}

\begin{frame}
  \vs
  \question{} Compute $\ds \ddx \sqrt{1+\sqrt{x}}$.
\end{frame}

\begin{frame}
  \vs
  \question{} Compute $\ds \ddx e^{\sin(x^2)}$.
\end{frame}

\begin{frame}
  \vs
  \question{} Derive the quotient rule using the power rule, the
  product rule, and the chain rule.
\end{frame}

\begin{frame}
  \frametitle{Derivatives of trigonometric functions}
  \begin{itemize}
  \item At the moment, we only know that $\ddx \sin(x) = \cos(x)$.
    % \[
    %   \ddx \sin(x) = \cos(x) \,.
    % \]
  \item This one fact along with other derivative {``shortcuts''} will
    give us the derivative formulas for all other standard trigonometric
    functions.
  \end{itemize}

  \begin{thm}[Derivatives of Trigonometric Functions] \hfill
    \begin{multicols}{2}
      \begin{itemize}
      \item $\ds \ddx \sin(x) = \cos(x)$.
      \item $\ds \ddx \sec(x) = \sec(x)\tan(x)$.
      \item $\ds \ddx \tan(x) = \sec^2(x)$.
      \item $\ds \ddx \cos(x) = -\sin(x)$.
      \item $\ds \ddx \csc(x) = -\csc(x)\cot(x)$.
      \item $\ds \ddx \cot(x) = -\csc^2(x)$.
      \end{itemize}
    \end{multicols}
    \vspace{0.5em}
  \end{thm}

  \note{
    \begin{itemize}
    \item For cosine, apply the chain rule to the identity $\cos(x) = \sin(\pi/2-x)$.
    \item For tangent, use the quotient rule to the identity $\tan(x) = \sin(x)/\cos(x)$.
    \item For secant, write $\sec(x)$ as $(\cos(x))^{-1}$ and use the power rule and the chain rule.
    \item Cosecant and cotangent functions are handled analogously.
    \end{itemize}
  }
\end{frame}

\begin{frame}
  \vs
  \question{} Compute $\ds \eval{\ddx \cos \left( \frac{x^3}{2}
      \right)}_{x=\sqrt[3]{\pi}}$.
\end{frame}

\begin{frame}
  \vs
  \question{} Compute $\ds \ddx \left( \frac{5x \tan(x)}{x^2-3} \right)$.
\end{frame}

\begin{frame}
  \vs
  \question{} Compute $\ds \eval{\ddx \left( \csc(x)\cot(x) \right)}_{x=\pi/3}$.
\end{frame}

\end{document}
%%% Local Variables:
%%% mode: latex
%%% TeX-master: t
%%% End:
