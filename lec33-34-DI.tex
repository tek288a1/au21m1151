% \documentclass[10pt,t,presentation,ignorenonframetext,aspectratio=169]{beamer}
\documentclass[10pt,t,handout,ignorenonframetext,aspectratio=169]{beamer}
\usepackage[default]{lato}
\usepackage{tk_beamer1}
\input{tk_packages}
\input{tk_macros}
\input{tk_environ}
\input{tk_ximera}
\usepackage{wasysym}            % for smiley
\newcommand{\zoz}{$\mathbf{ \frac{0}{0} }$}

% some inverse trigs
\DeclareMathOperator{\arcsec}{arcsec}
\DeclareMathOperator{\arccot}{arccot}
\DeclareMathOperator{\arccsc}{arccsc}

%%%% META DATA
\newcommand{\semester}{Autumn 2021}
\newcommand{\course}{Math 1151}
\newcommand{\lecTitle}{Lecture 33-34: Definite Integrals (DI)}

%%%% TITLE PAGE
\title[\course]{\lecTitle}
\institute[Ohio State]
{
  \medskip
}
\date[\week]{\semester}
\author{Tae Eun Kim, Ph.D.}

\begin{document}
\begin{frame}
  \titlepage
\end{frame}


\begin{frame}
  \frametitle{Definite Integrals}
  \begin{defn}
    Let $f$ be a function which is continuous on the interval
    $[a,b]$. We define the \textbf{definite integral} of $f$ on $[a, b]$
    by
    \[
      \int_a^b f(x) \; dx = \lim_{n\to\infty} \sum_{k=1}^{n} f(x_k^*) \Delta x \,.
    \]
    The definite integral is a number that gives the \textbf{net area} of the
    region between the curve $y = f(x)$ and the $x$-axis on the interval
    $[a,b]$.
  \end{defn}
\end{frame}

\begin{frame}
  \frametitle{Basic Properties}
  \begin{thm}[Properties of the definite integral]
    Let $f$ and $g$ be defined on a closed interval $[a,b]$ that
    contains the value $c$, and let $k$ be a constant. The following
    hold:
    \begin{enumerate}
    \item $\ds \int_a^a f(x)\d x = 0$
    \item $\ds \int_a^c f(x)\d x + \int_c^b f(x)\d x = \int_a^b f(x)\d x$
    \item $\ds \int_a^bf(x)\d x = -\int_b^a f(x)\d x$
    \item $\ds \int_a^bk f(x)\d x = k \int_a^bf(x)\d x$
    \item $\ds \int_a^b \left\{ f(x)\pm g(x) \right\}  \d x = \int_a^bf(x)\d x \pm \int_a^bg(x)\d x$
    \end{enumerate}
  \end{thm}
\end{frame}


\begin{frame}
  \frametitle{Definite Integrals Using Geometry
    vs. Definition}
  \begin{question}
    Compute the integral
    \[
      \int_{0}^{10} (4 - x)\; dx
    \]
    in two ways:
    \begin{enumerate}
    \item by interpreting the integral as the net area of the region
      between the curve $y=4-x$ and the interval $[0,10]$ on  the
      $x$-axis;
    \item using the definition of the definite integral, i.e. by
      computing the limit of Riemann sums.
    \end{enumerate}
  \end{question}
\end{frame}

\begin{frame}
  \vs
  \begin{question}
    Compute the integral
    \[
      \int_{0}^{10} |4 - x| \; dx \,.
    \]
  \end{question}
\end{frame}

\begin{frame}
  \frametitle{Note: Net Areas vs. Geometric Areas}
  We know that the net area of the region between a curve $y=f(x)$ and the $x$-axis
  on $[a,b]$ is given by
  \[
    \int_a^b f(x) \d x.
  \]
  On the other hand, if we want to know the \textit{geometric area},
  meaning the ``actual'' area, we compute
  \[
    \int_a^b |f(x)| \d x.
  \]
\end{frame}


\begin{frame}
  \vs
  \begingroup
  \small
  \begin{question}
    The graph of a function $f$ is given in the figure.
    \begin{enumerate}
    \item Express the geometric area of the region between the curve
      $y=f(x)$ and the $x-$axis on the interval $[0,9]$ as a definite
      integral.
    \item Express the geometric area of the region between the curve
      $y=f(x)$ and the $x-$axis on the interval $[0,9]$ in terms of
      definite integrals of $f$.
    \item Express the geometric area of the region between the curve
      $y=f(x)$ and the $x-$axis on the interval $[0,9]$ in terms of
      areas $A_1$, $A_2$, $A_3$ and $A_4$.
    \end{enumerate}
  \end{question}
  \endgroup

  \hfill
  \begin{minipage}[t]{0.5\linewidth}
    \begin{image}[0.98\linewidth]
      \begin{tikzpicture}[
        declare function = {f(\x) = (1/4)*(x-1)*(x-5)* (x-8);}, baseline=(current bounding box.north)]
        \begin{axis}[
          domain=0:9, xmin =-1,xmax=9.1,ymax=10,ymin=-10,
          width=6in,
          height=3in,xtick={0,1,...,9},
          xticklabels={0,1,...,9},
          %% ytick style={draw=none},
          %% yticklabels={},
          axis lines=center, xlabel=$x$, ylabel=$y$,
          every axis y label/.style={at=(current axis.above origin),anchor=south},
          every axis x label/.style={at=(current axis.right of origin),anchor=west},
          axis on top,
          ]
          \addplot [draw=none,fill=fillp,domain=0:1, smooth] {f(x)}
          \closedcycle;
          \addplot [draw=none,fill=fillp,domain=1:5, smooth] {f(x)}
          \closedcycle;
          \addplot [draw=none,fill=fillp,domain=5:8, smooth] {f(x)}
          \closedcycle;
          \addplot [draw=none,fill=fillp,domain=8:9, smooth] {f(x)}
          \closedcycle;
          \addplot [very thick,penColor, smooth] {f(x)};
          \node at (axis cs:0.5,-1.2) {\large$A_1$};
          \node at (axis cs:3,1.2) {\large$A_2$};
          \node at (axis cs:6.5,-1.2) {\large$A_3$};
          \node at (axis cs:8.5,1.2) {\large$A_4$};
          \node at (axis cs:6,6.2) {\large$y=f(x)$};
        \end{axis}
      \end{tikzpicture}
    \end{image}
  \end{minipage}
\end{frame}


\begin{frame}
  \frametitle{From Riemann Sums to Definite Integrals}
  \begin{question}
    Compute the limit:
    \[
      \lim_{n\to \infty} \sum_{k=1}^n \left(\sqrt{1-\left(-1+\frac{2k}{n}\right)^2}\right)
      \left(\frac{2}{n}\right)
    \]
  \end{question}
\end{frame}

\begin{frame}
  \vs
  \begin{question}
    Express the following limit of Riemann sum as a definite integral:
    \[
      \lim_{n\to\infty} \sum_{k=1}^{n} \left( \frac{k\pi}{n} + \cos \frac{k\pi}{n}
      \right) \frac{\pi}{n} \,.
    \]
  \end{question}
\end{frame}

\begin{frame}
  \frametitle{Definite Integrals of Symmetric Functions}
  Recall that a function $f$ is
  \begin{itemize}
  \item an \textbf{odd} function if $f(-x) = -f(x)$;
  \item an \textbf{even} function if $f(-x) = f(x)$.
  \end{itemize}

  \begin{thm}
    Let $f$ be a symmetric function on a symmetric interval $[-a, a]$
    where $a > 0$. Then
    \begin{equation*}
      \int_{-a}^a f(x) \; dx =
      \begin{dcases*}
        2 \int_0^{a} f(x) \; dx & if $f$ is even \\
        0 & if $f$ is odd.
      \end{dcases*}
    \end{equation*}
  \end{thm}
\end{frame}

\begin{frame}
  \vs
  \begin{question}
    \begin{enumerate}
    \item Find the following definite integral:
      \[
        \int_{-4}^{4} \frac{x^2 \sin^3(x)}{\sqrt{x^4+1}} \; dx \,.
      \]
      \vfill
    \item Suppose that $f$ is an even function. Given that $\ds \int_0^6
      f(x) \; dx = 13$, find $\ds \int_{-6}^{6} (5f(x)+14) \;dx$.
      \vfill
    \end{enumerate}
  \end{question}
\end{frame}

\end{document}
%%% Local Variables:
%%% mode: latex
%%% TeX-master: t
%%% End:
