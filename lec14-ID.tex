\documentclass[10pt,t,presentation,ignorenonframetext,aspectratio=169]{beamer}
% \documentclass[10pt,t,handout,ignorenonframetext,aspectratio=169]{beamer}
\usepackage[default]{lato}
\usepackage{tk_beamer1}
\input{tk_packages}
\input{tk_macros}
\input{tk_environ}
\input{tk_ximera}
\usepackage{wasysym}            % for smiley
\newcommand{\zoz}{$\mathbf{ \frac{0}{0} }$}

%%%% META DATA
\newcommand{\semester}{Autumn 2021}
\newcommand{\course}{Math 1151}
\newcommand{\lecTitle}{Lecture 14: Implicit Differentiation (ID)}

%%%% TITLE PAGE
\title[\course]{\lecTitle}
\institute[Ohio State]
{
  \medskip
}
\date[\week]{\semester}
\author{Tae Eun Kim, Ph.D.}

\begin{document}
\begin{frame}
  \titlepage
\end{frame}

\begin{frame}
  \frametitle{Motivation}
  Depending on whether or not the dependent variable is written explicitly in
  terms of the independent variable, a function can be classified as
  an \textbf{explicit function} or an \textbf{implicit function}. For
  example:
  \begin{itemize}
  \item Explicit functions ($y = f(x)$ form)
    \[
      y = 3x^2 - 2x + 1 \,, \qquad
      y = e^{3x} \,, \qquad
      y = \frac{x-2}{x^2-3x+2} \,,\ldots
    \]
  \item Implicit functions ($F(x,y) = 0$ form)
    \[
      x^2 + y^2 = 4 \,, \qquad
      x^3 + y^3 = 9xy \,, \qquad
      x^4 + 3x^2 = x^{2/3} + y^{2/3} + 1 \,,\ldots
    \]
  \end{itemize}
  \vfill
  Today's goals are:
  \begin{itemize}
  \item to learn how to differentiate implicit functions
  \item to derive more differentiation shortcuts using the new technique
  \end{itemize}
  % \begin{itemize}
  % \item So far, we learned how to differentiate explicit functions.
  % \item Today, we will learn how to differentiate implicit functions
  %   which display subtle relations between $x$ and $y$
  % \end{itemize}
\end{frame}

\begin{frame}
  \frametitle{Implicit Differentiation}
  \begin{block}{Procedures}
    In order to differentiate an implicit function:
    \begin{enumerate}
    \item Differentiate the entire equation with respect to $x$.
    \item Solve for $\frac{dy}{dx}$.
    \end{enumerate}
  \end{block}

  \textbf{Note.}
  \begin{itemize}
  \item In Step 1, keep in mind that $y$ is actually a
    function of $x$.
  \item This inevitably requires an application of the
    chain rule.
  \end{itemize}
\end{frame}

\begin{frame}
  \vs{}
  \question{} Consider the curve defined by:
  \[
    x^2 + y^2 = 1 \,.
  \]
  \begin{enumerate}
  \item Compute $\ds \frac{dy}{dx}$.\vfill
  \item Find the slope of the tangent line at
    $\ds \left(\frac{\sqrt{2}}{2},\frac{\sqrt{2}}{2}\right)$. \vfill
  \end{enumerate}
\end{frame}

\begin{frame}
  \vs
  \question{}
  Consider the curve defined by:
  \[
    x^3+y^3 = 9xy \,.
  \]
  \begin{enumerate}
  \item Compute $\ds \frac{dy}{dx}$.\vfill
  \item Find the slope of the tangent line at $(4,2)$.\vfill
  \end{enumerate}
\end{frame}

\begin{frame}
  \vs
  Graph for the previous problem.

  \begin{image}[0.7\textwidth]
    \begin{tikzpicture}
      \begin{axis}[
        xmin=-6,xmax=6,ymin=-6,ymax=6,
        axis lines=center,
        xlabel=$x$, ylabel=$y$,
        every axis y label/.style={at=(current axis.above origin),anchor=south},
        every axis x label/.style={at=(current axis.right of origin),anchor=west},
        ]
        \addplot [very thick, penColor, smooth, samples=100, domain=(-.99:0)] ({9*x/(1+x^3)},{9*x^2/(1+x^3)});
        \addplot [very thick, penColor, smooth, samples=100, domain=(-.99:0)] ({9*x^2/(1+x^3)},{9*x/(1+x^3)});
        \addplot [very thick, penColor, smooth, samples=100, domain=(0:1)] ({9*x/(1+x^3)},{9*x^2/(1+x^3)});
        \addplot [very thick, penColor, smooth, samples=100, domain=(0:1)] ({9*x^2/(1+x^3)},{9*x/(1+x^3)});

        \addplot [very thick, penColor2, domain=(-6:6)] {(5/4)*(x-4)+2};

        \addplot[color=penColor,fill=penColor,only marks,mark=*] coordinates{(4,2)};  %% closed hole
      \end{axis}
    \end{tikzpicture}
  \end{image}
\end{frame}

\begin{frame}
  \frametitle{Derivatives of Logarithmic Functions}
  \begin{minipage}[t]{0.5\textwidth}
    \begin{itemize}
    \item Recall that $e^x$ and $\ln(x)$ are inverses and thus their
      graphs are reflections of each other about $y = x$.
    \item In particular, we can write $y = \ln (x)$ as $x = e^y$.
    \item And we know that $\ds \ddx e^x = e^x$.
    \end{itemize}
  \end{minipage}
  \hfill
  \begin{minipage}[t]{0.48\textwidth}
    \begin{image}[0.98\textwidth]
      \begin{tikzpicture}[baseline=(current bounding box.north)]
        \begin{axis}[
          xmin=-6,xmax=6,ymin=-6,ymax=6,
          axis lines=center,
          xlabel=$x$, ylabel=$y$,
          every axis y label/.style={at=(current axis.above origin),anchor=south},
          every axis x label/.style={at=(current axis.right of origin),anchor=west},
          ]
          \addplot [very thick, penColor, smooth, domain=(-6:6)] {e^x};
          \addplot [very thick, penColor2, samples=100, smooth, domain=(.002:6)] {ln(x)};
          \addplot [dashed, textColor, domain=(-6:6)] {x};
          \node at (axis cs:-2,1) [penColor] {$e^x$};
          \node at (axis cs:1,-2) [penColor2] {$\ln(x)$};
        \end{axis}
      \end{tikzpicture}
    \end{image}
  \end{minipage}

  \textbf{Question:} What is $\ddx \ln (x)$?
\end{frame}

\begin{frame}
  \vs
  \begin{thm}[The derivative of logarithm]
    Let $b>0$ and $b \neq 1$. Then
    \[
      \ddx \log_b(x) = \frac{1}{x\ln(b)}.
    \]
    In particular, when $b = e$, the formula reduces to
    \[
      \ddx \ln(x) = \frac{1}{x}.
    \]
  \end{thm}
  \vs
  \textbf{Explanation.}
  \note{
    Recall that $y = \ln (x)$ exctly when
    $e^y = x$ and $x > 0$. On implicit differentiation of the latter
    with respect to $x$, we obtain
    \[
      e^y \frac{dy}{dx} = 1
      \qquad \Longrightarrow \qquad
      \frac{dy}{dx} = \frac{1}{e^y} = \frac{1}{x} \,,
    \]
    which proves the second result. To derive the first result, utilize
    \[
      \log_b(x) = \frac{\ln(x)}{\ln(b)}.
    \]
  }
\end{frame}

\begin{frame}
  \vs
  \question{} Compute:
  \begin{enumerate}
  \item $\ds \ddx \left(-\ln(\cos(x))\right)$ \vfill
  \item $\ds \ddx \log_7 (x)$ \vfill
  \end{enumerate}
\end{frame}

\begin{frame}
  \frametitle{The Derivative of an Exponential Function}
  \begin{thm}[The derivative of an exponential function]
    Let $a$ be a positive real number. Then
    \[
      \ddx a^x = a^x\cdot \ln(a) \,.
    \]
  \end{thm}
  \vs
  \textbf{Explanation.}
  \note{
    Note that
    \[
      a^x = e^{\ln(a^x)} = e^{x\ln(a)}\,.
    \]
    So we may write
    \begin{align*}
      \ddx a^x
      &= \ddx e^{x\ln(a)}\\
      &= e^{x\ln(a)} \cdot \ln(a) \,.
    \end{align*}
  }
\end{frame}

\begin{frame}
  \vs
  \question{} Compute
  \[
    \ddx 7^x.
  \]
\end{frame}


\end{document}
%%% Local Variables:
%%% mode: latex
%%% TeX-master: t
%%% End:
