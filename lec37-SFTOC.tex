% \documentclass[10pt,t,presentation,ignorenonframetext,aspectratio=169]{beamer}
\documentclass[10pt,t,handout,ignorenonframetext,aspectratio=169]{beamer}
\usepackage[default]{lato}
\usepackage{tk_beamer1}
\input{tk_packages}
\input{tk_macros}
\input{tk_environ}
\input{tk_ximera}
\usepackage{wasysym}            % for smiley
\newcommand{\zoz}{$\mathbf{ \frac{0}{0} }$}

% some inverse trigs
\DeclareMathOperator{\arcsec}{arcsec}
\DeclareMathOperator{\arccot}{arccot}
\DeclareMathOperator{\arccsc}{arccsc}

%%%% META DATA
\newcommand{\semester}{Autumn 2021}
\newcommand{\course}{Math 1151}
\newcommand{\lecTitle}{Lecture 37: Second Fundamental Theorem \\ of Calculus (SFTOC)}
%%%% TITLE PAGE
\title[\course]{\lecTitle}
\institute[Ohio State]
{
  \medskip
}
\date[\week]{\semester}
\author{Tae Eun Kim, Ph.D.}

\begin{document}
\begin{frame}
  \titlepage
\end{frame}


\begin{frame}
  \frametitle{The Second Fundamental Theorem of Calculus}
  Here comes the second form of the Fundamental Theorem of
  \begin{thm}[Second Fundamental Theorem of Calculus, FTC2]
    Let $f$ be continuous on $[a,b]$. If $F$ is \textbf{any}
    antiderivative of $f$, then
    \[
      \int_a^b f(x)\d x = F(b)-F(a)\,. \tag{FTC2}
    \]
  \end{thm}

  \begin{itemize}
  \item An alternate interpretation of (FTC2) is to write it as
    \begin{image}[0.5\linewidth]
      \begin{tikzpicture}[scale=2,every node/.style={transform shape}]
        \node at (0,0) {
          $\ds \color{green!70!black!70!blue}\int_a^b\color{blue!70!green} \ddx f(x)\color{green!70!black!70!blue}\d x\color{black} =
          \color{purple!50!blue!90!black}f(b) - f(a).$
        };
      \end{tikzpicture}
    \end{image}
  \item The above reads as
    \vspace{0.5em}
    \begin{center}
      \begin{minipage}{0.95\linewidth}
        \begin{quote}
          \textbf{The \textcolor{green!70!black!70!blue}{accumulation}
            of a \textcolor{blue!70!green}{rate} is given by the
            \textcolor{purple!50!blue!90!black}{change in the amount}.}
        \end{quote}
      \end{minipage}
    \end{center}
  \end{itemize}
\end{frame}


\begin{frame}
  \frametitle{Notation}
  \begin{itemize}
  \item FTC2 is useful in computing a definite integral:
    \begin{enumerate}
    \item find an antiderivative of the integrand;
    \item evaluate it at the limits of integration;
    \item take the difference.
    \end{enumerate}
  \item In the differencing process, you may find the following notation convenient:
    \[
      \eval{F(x)}_a^b = F(x) \bigg|_a^b = F(b) - F(a) \,.
    \]
  \end{itemize}
\end{frame}

\begin{frame}
  \vs
  \red{\textit{Proof.}}
  Let $a\le c\le b$ and write
  \begin{align*}
    \int_a^b f(x) \d x &= \int_a^c f(x) \d x + \int_c^b f(x) \d x \\
                       &= \int_c^b f(x) \d x - \int_c^a f(x) \d x.
  \end{align*}
  By the First Fundamental Theorem of Calculus, we have
  \[
    F(b) = \int_c^b f(x) \d x\qquad\text{and}\qquad F(a) = \int_c^a f(x) \d x
  \]
  for some antiderivative $F$ of $f$. So
  \[
    \int_a^b f(x) \d x = F(b)-F(a)
  \]
  for this antiderivative. However, \textbf{any} antiderivative
  could have be chosen, as antiderivatives of a given function
  differ only by a constant, and this constant \textit{always}
  cancels out of the expression when evaluating $F(b)-F(a)$.
  \qed{}
\end{frame}

\begin{frame}
  \vs
  \begin{question}
    Compute:
    \begin{enumerate}
    \item $\ds \int_{-2}^2 x^3\d x$
      \vfill
    \item $\ds \int_0^1 \frac{\pi}{3} \sin \frac{\pi}{3}\theta \d
      \theta$
      \vfill
    \end{enumerate}
  \end{question}
\end{frame}

\begin{frame}
  \vs
  \begin{question}
    Compute:
    \begin{enumerate}
    \item $\ds \int_0^5 e^t\d t$
      \vfill
    \item $\ds \int_1^2\left(x^9 + \frac{1}{x}\right) \d x$
      \vfill
    \end{enumerate}
  \end{question}
\end{frame}


\subsection{Net Change and Future Value}
\begin{frame}
  \frametitle{Displacement and net change}
  Let's recall that
  \begin{itemize}
  \item The derivative of a position function $s$ is a velocity function $v$.
  \item The derivative of a velocity function $v$ is an acceleration function $a$.
  \end{itemize}

  In other words,
  \begin{itemize}
  \item A velocity function $v$ is an antiderivative of an acceleration function $a$.
  \item A position function $s$ is an antiderivative of a velocity function $v$.
  \end{itemize}

  In particular, by FTC2,
  \[
    \int_a^b v(t) \d t = s(b) - s(a) \,,
  \]
  which measures a \textbf{change in position}, or \textbf{displacement} as already introduced on Monday.
\end{frame}


\begin{frame}
  \frametitle{Net change and future value}
  \begin{itemize}
  \item In general, FTC2 states that the definite integral of a rate
    of change of a certain quantity $Q$ is the \textbf{net change} in
    its amount between two limits of integration:
    \[
      \int_a^b Q'(s) \d s = Q(b) - Q(a) \,. \tag{Net change}
    \]
  \item If we replace $a = 0$ and $b = t$, we have a formula for \textbf{future value}:
    \[
      Q(t) = Q(0) + \int_0^t Q'(s) \d s \,. \tag{Future value}
    \]
  \end{itemize}
\end{frame}


\begin{frame}
  \vs
  \question{}
  A book publisher estimates that the marginal cost of a particular
  title (in dollars/book) is given by
  \[
    C'(x) = 12 - 0.0002x \,,
  \]
  where $0 \le x \le 50,000$ is the number of books printed. What is the
  cost of producing the 12,001st through 15,000th book?
\end{frame}

\begin{frame}
  \frametitle{Summary of three different integrals}
  \begingroup
  \small
  \begin{enumerate}
  \item An \textbf{indefinite integral}, a.k.a. an antiderivative computes a
    family of functions:
    \note{
      \[
        \int f(x) \d x = \text{``a class of functions whose derivative is $f$''}
      \]
    }
    \begin{center}
      \begin{tcolorbox}[width=0.5\linewidth]
        \vspace{-1em}
        \begin{image}[0.95\linewidth]
          \begin{tikzpicture}[scale=2,every node/.style={transform shape}]
            \node at (0,0) {
              $\ds \int f(x) \d x = F(x) + C$
            };
          \end{tikzpicture}
        \end{image}
        \vspace{-1em}
      \end{tcolorbox}
    \end{center}
    where $F'(x) = f(x)$.
  \item An \textbf{accumulation function} computes an accumulated area:
    \note{
      \[
        \int_a^x f(t) \d t = \text{``a function $F$ whose derivative is $f$''}
      \]
    }
    \begin{center}
      \begin{tcolorbox}[width=0.5\linewidth]
        \vspace{-1em}
        \begin{image}[0.95\linewidth]
          \begin{tikzpicture}[scale=2,every node/.style={transform shape}]
            \node at (0,0) {
              $\ds F(x)=\int_a^x f(t) \d t$
            };
          \end{tikzpicture}
        \end{image}
        \vspace{-1em}
      \end{tcolorbox}
    \end{center}
    FTC1 says that $F'(x) = f(x)$.
  \item A \textbf{definite integral} computes a signed area:
    \note{
      \[
        \int_a^b f(x) \d x = \text{``the signed area between the $x$-axis and $f$''}
      \]
    }
    \begin{center}
      \begin{tcolorbox}[width=0.5\linewidth]
        \vspace{-1em}
        \begin{image}[0.99\linewidth]
          \begin{tikzpicture}[scale=2,every node/.style={transform shape}]
            \node at (0,0) {
              $\ds \int_a^b f(x) \d x = F(b)-F(a)$
            };
          \end{tikzpicture}
        \end{image}
        \vspace{-1em}
      \end{tcolorbox}
    \end{center}
  \end{enumerate}
  \endgroup
\end{frame}

\end{document}
%%% Local Variables:
%%% mode: latex
%%% TeX-master: t
%%% End:
