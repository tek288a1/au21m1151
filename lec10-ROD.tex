\documentclass[10pt,t,presentation,ignorenonframetext,aspectratio=169]{beamer}
% \documentclass[10pt,t,handout,ignorenonframetext,aspectratio=169]{beamer}
\usepackage[default]{lato}
\usepackage{tk_beamer1}
\input{tk_packages}
\input{tk_macros}
\input{tk_environ}
\input{tk_ximera}
\usepackage{wasysym}            % for smiley
\newcommand{\zoz}{$\mathbf{ \frac{0}{0} }$}

%%%% META DATA
\newcommand{\semester}{Autumn 2021}
\newcommand{\course}{Math 1151}
\newcommand{\lecTitle}{Lecture 10: Rules of Differentiation (ROD)}

%%%% TITLE PAGE
\title[\course]{\lecTitle}
\institute[Ohio State]
{
  \medskip
}
\date[\week]{\semester}
\author{Tae Eun Kim, Ph.D.}

\begin{document}
\begin{frame}
  \titlepage
\end{frame}

\begin{frame}
  \frametitle{Basic rules of differentiation}
  \begin{itemize}
  \item We have learned the definition of derivative;
  \item We understand that the derivative of a function can be
    interpreted as another function;
  \item Today, we will learn a
    bunch of differentiation shortcuts which will help us to avoid tedious calculations and focus on more important issues.
  \end{itemize}

  % \begin{itemize}
  % \item the constant rule
  % \item the power rule
  % \item the sum rule
  % \end{itemize}
\end{frame}

\begin{frame}
  \vs
  \begin{thm}[Constant Rule]
    Given a constant $c$,
    \[
      \ddx c = 0.
    \]
  \end{thm}

  \vs
  This can be confirmed easily using the definition. However, it is
  equally important to understand this with good intuition:
  \begin{itemize}
  \item The constant function plots a horizontal line, so the slope of
    the tangent line at any point is $0$.
  \item If $s(t)$ represents the position of an object with respect to
    time and $s(t)$ is constant, then the object is not moving, so its
    velocity is zero. Hence $\dfrac{d}{d t} s(t) = 0$.
  \item If $v(t)$ represents the velocity of an object with respect to
    time and $v(t)$ is constant, then the object's acceleration is
    zero. Hence $\dfrac{d}{dt} v(t) = 0$.
  \end{itemize}
\end{frame}

\begin{frame}
  \vs
  We simply state the result here. For derivation, please read the textbook.
  \begin{thm}[Power Rule]
    For any real number $n$,
    \[
      \ddx x^n = n x^{n-1} \,.
    \]
  \end{thm}

  \vs
  \textbf{Remark.} Note that the power rule holds for \textbf{any real
    number} $n$. This allows us to differentiate such functions as
  \begin{itemize}
  \item $\ds f(x) = x^{13}$
  \item $\ds f(x) = 1/x^4$
  \item $\ds f(x) = \sqrt[5]{x}$
  \end{itemize}
\end{frame}

\begin{frame}
  \vs
  \begin{thm}[Sum Rule]
    If $f(x)$ and $g(x)$ are differentiable and $c$ is a constant, then
    \begin{enumerate}
    \item\label{SR:1} $\ds \ddx \big( f(x) + g(x)\big) = f'(x) + g'(x)$,
    \item $\ds \ddx \big( f(x) - g(x)\big) = f'(x) - g'(x)$,
    \item $\ds \ddx \big(c\cdot f(x)\big) = c\cdot f'(x)$.
    \end{enumerate}
  \end{thm}

  \vs
  \begin{question}
    Compute:
    \[
      \ddx \left(
        \frac{3}{\sqrt[3]{x}} - 2\sqrt{x} + \frac{1}{x^7}
      \right) \,.
    \]
  \end{question}
\end{frame}

\begin{frame}
  \frametitle{The derivative of the natural exponential function}
  \begin{itemize}
  \item The slope of the line tangent to the  curve $y =
    a^x$, $a > 1$, is positive.
  \item It is easy to observe that this slope increases as $a$
    increases.
  \item Moreover, we can speculate that for a suitably chosen $a$, the
    slope at $x = 0$ can be precisely 1.
  \item Such a number does exist and is one of the most important constants in mathematics.
  \end{itemize}

  \begin{defn}
    The number denoted by $e$, called \textbf{Euler's number}, is defined to be the number satisfying the following relation
    \[
      \lim_{h \to 0} \frac{e^h - 1}{h} = 1 \,.
    \]
    Its approximate value is $e = 2.718281828459045 \dots$.
  \end{defn}
\end{frame}

\begin{frame}
  \vs
  With this definition, we can derive the following important result.
  \begin{thm}[Derivative of the Natural Exponential Function]
    The derivative of the natural exponential function is the natural exponential function itself.  In other words,
    \[
      \ddx e^x = e^x \,.
    \]
  \end{thm}
\end{frame}

\begin{frame}
  \vs
  \begin{question}
    Find the slope of the tangent line to the graph of the function $f(x) = e^x$ at $x=5$.
  \end{question}
\end{frame}


\begin{frame}
  \frametitle{The derivative of sine}
  \begin{itemize}
  \item In order to derive the derivative of sine function, we need the following results:
    \[
      \lim_{\theta \to 0} \frac{\sin \theta}{\theta} = 1
      \qquad \text{and} \qquad
      \lim_{\theta \to 0} \frac{\cos \theta - 1}{\theta} = 0 \,.
    \]
  \item We derived the first one using the squeeze theorem; the second
    one follows from the first one. Let's derive it here.
    \note{
      \begin{align*}
        \lim_{\theta\to 0}\frac{\cos(\theta)-1}{\theta}
        &= \lim_{\theta\to 0} \left(
          \frac{\cos(\theta)-1}{\theta} \cdot
          \frac{\cos(\theta)+1}{\cos(\theta)+1}
          \right)\\
        &= \lim_{\theta\to 0}\frac{\cos^2(\theta)-1}{\theta(\cos(\theta)+1)}\\
        &= \lim_{\theta\to 0}\frac{-\sin^2(\theta)}{\theta(\cos(\theta)+1)}\\
        &= -\lim_{\theta\to 0}\left(
          \frac{\sin(\theta)}{\theta}\cdot
          \frac{\sin(\theta)}{\cos(\theta)+1}
          \right)\\
        &= -\lim_{\theta\to 0} \frac{\sin(\theta)}{\theta} \cdot
          \lim_{\theta\to 0}\frac{\sin(\theta)}{\cos(\theta)+1}\\
        &= -1 \cdot \frac{0}{2} = 0.
      \end{align*}
    }
  \item In addition, let's recall the addition formula for sine:
    \[
      \sin(\alpha+\beta)
      = \sin(\alpha)\cos(\beta)+\sin(\beta)\cos(\alpha) \,.
    \]
  \end{itemize}
\end{frame}

\begin{frame}
  \vs
  Now, we are ready to prove the following result:
  \begin{thm}[Derivative of Sine]
    For any angle $\theta$ measured in radians, the derivative of $\sin(\theta)$ with respect to $\theta$ is $\cos(\theta)$.  In other words,
    \[
      \frac{d}{d\theta} \sin(\theta) = \cos(\theta) \,.
    \]
  \end{thm}

  \note{
    \begin{align*}
      &= \lim_{h\to0} \frac{\sin(\theta)\cos(h)+\sin(h)\cos(\theta)-\sin(\theta)}{h}  \\
      &= \lim_{h\to0} \left(\frac{\sin(\theta)\cos(h)-\sin(\theta)}{h} + \frac{\sin(h)\cos{\theta}}{h} \right)\\
      &=\lim_{h\to0} \left(\sin (\theta)\frac{\cos(h) - 1}{h}+\cos(\theta)\frac{\sin(h)}{h}\right) \\
      &=\lim_{h\to0} \sin (\theta)\frac{\cos(h) - 1}{h}+\lim_{h\to0}\cos(\theta)\frac{\sin(h)}{h} \\
      &= \sin (\theta)\lim_{h\to0}\frac{\cos(h) - 1}{h}+\cos(\theta)\lim_{h\to0}\frac{\sin(h)}{h} \\
      &=\sin(\theta) \cdot 0 + \cos(\theta) \cdot 1\\
      &= \cos(\theta).
    \end{align*}
  }
\end{frame}

\begin{frame}
  \vs
  \begin{question}
    What is the value of $x$ in the interval $[0, \pi]$ where the tangent to the graph of $f(x) = \sin(x)$ has slope $-1/2$?
  \end{question}
\end{frame}


\begin{frame}
  \vs
  For those interested, the following diagram gives a visual interpretation of sine-differentiation.

  \begin{image}[0.6\textwidth]
    \begin{tikzpicture}
      \begin{axis}[
        xmin=-.1,xmax=1.1,ymin=-.1,ymax=1.1,
        axis lines=center,
        ticks=none,
        width=5in,
        unit vector ratio*=1 1 1,
        xlabel=$x$, ylabel=$y$,
        every axis y label/.style={at=(current axis.above origin),anchor=south},
        every axis x label/.style={at=(current axis.right of origin),anchor=west},
        ]
        \addplot [very thick, textColor!30!background, smooth, domain=(-.2:.2+pi/2)] ({cos(deg(x))},{sin(deg(x))});
        \addplot [textColor,very thick] plot coordinates {(0,0) (.766,.643)}; %% 40 degrees
        \addplot [textColor,very thick] plot coordinates {(0,0) (.766,0)}; %% bottom
        \addplot [very thick, penColor2!30!background] {(x-.766)*(-.766/.643)+.643};
        \addplot [textColor,dashed] plot coordinates {(0,0) (.766-.196,.643+1-.766)}; %% 40+16.98 degrees

        %% \addplot [textColor!20!background] plot coordinates {(.766,.643) (1,.839)}; %% hyp
        %% \addplot [textColor!20!background] plot coordinates {(1,.643) (1,.839)}; %% side
        %% \addplot [textColor!20!background] plot coordinates {(.766,.643) (1,.643)}; %% bottom
        %% \addplot [textColor!20!background,smooth, domain=(0:40)] ({.05*cos(x)+.766},{.05*sin(x)+.643}); %% angle
        %% \node at (axis cs:.84,.670) [textColor!20!background] {\footnotesize$\theta$};

        %% \addplot [textColor!20!background] plot coordinates {(.766,.643) (.766,.839)}; %% side
        %% \addplot [textColor!20!background] plot coordinates {(.766,.839) (1,.839)}; %% bottom
        %% \addplot [textColor!20!background,smooth, domain=(180:220)] ({.05*cos(x)+1},{.05*sin(x)+.839}); %% angle
        %% \node at (axis cs:.926,.812) [textColor!20!background] {\footnotesize$\theta$};

        \draw[rotate around={30:(.5,.5)}] (.7,.7) rectangle (.25,.25);

        % \draw[textColor, rotate around={45:(.5,.5)}] (.5,.5) rectangle (.2,.2);

        \addplot [penColor4,very thick] plot coordinates {(.766,.643) (.766,.643+1-.766)}; %% side
        \addplot [textColor,very thick] plot coordinates {(.766,.643+1-.766) (.766-.196,.643+1-.766)}; %% top
        \addplot [textColor,smooth, domain=(90:130)] ({.05*cos(x)+.766},{.05*sin(x)+.643}); %% angle
        \addplot [very thick, textColor] plot coordinates {(.766-.196,.643+1-.766) (.766,.643)}; %% hyp
        \node at (axis cs:.739,.717) [textColor] {\footnotesize$\theta$};

        \node at (axis cs:.668,.877) [anchor=south] {\footnotesize$h\sin(\theta)$};
        \node at (axis cs:.766,.76) [anchor=west] {\footnotesize$h\cos(\theta)$};
        \node at (axis cs:.65,.78) [anchor=west] {\footnotesize$\approx h$};

        \addplot [very thick, penColor] plot coordinates {(.766,0) (.766,.643)}; %% sin theta

        \addplot [textColor, smooth, domain=(0:40)] ({.15*cos(x)},{.15*sin(x)});
        \addplot [textColor, smooth, domain=(40:56.90)] ({.17*cos(x)},{.17*sin(x)});
        \addplot [textColor, smooth, domain=(40:56.90)] ({.185*cos(x)},{.185*sin(x)});
        \node at (axis cs:.15,.07) [anchor=west] {$\theta$};
        \node at (axis cs:.15,.17) {$h$};
        \node at (axis cs:.766,.322) [anchor=east] {$\sin(\theta)$};
        \node at (axis cs:.383,0) [anchor=north] {$\cos(\theta)$};
      \end{axis}
    \end{tikzpicture}
  \end{image}
\end{frame}

\end{document}
%%% Local Variables:
%%% mode: latex
%%% TeX-master: t
%%% End:
