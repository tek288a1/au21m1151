\documentclass[10pt,t,presentation,ignorenonframetext,aspectratio=169]{beamer}
% \documentclass[10pt,t,handout,ignorenonframetext,aspectratio=169]{beamer}
\usepackage[default]{lato}
\usepackage{tk_beamer1}
\input{tk_packages}
\input{tk_macros}
\input{tk_environ}
\input{tk_ximera}
\usepackage{wasysym}            % for smiley
\newcommand{\zoz}{$\mathbf{ \frac{0}{0} }$}

%%%% META DATA
\newcommand{\semester}{Autumn 2021}
\newcommand{\course}{Math 1151}
\newcommand{\lecTitle}{Lecture 15: Logarithmic Differentiation (LD)}

%%%% TITLE PAGE
\title[\course]{\lecTitle}
\institute[Ohio State]
{
  \medskip
}
\date[\week]{\semester}
\author{Tae Eun Kim, Ph.D.}

\begin{document}
\begin{frame}
  \titlepage
\end{frame}

\begin{frame}
  \frametitle{Introduction}
  Let's recall:

  \begin{block}{Properties of logarithms}
    Let $b>0$ and $b \neq 1$; let $x, y > 0$.
    \begin{itemize}
    \item $\log_b(xy) = \log_b(x) + \log_b(y)$
    \item $\log_b(x/y) = \log_b(x) - \log_b(y)$
    \item $\log_b(x^y) = y\log_b(x)$
    \end{itemize}
  \end{block}
  % In words, a logarithm turns products into sums and a power into a
  % scalar multiple.
\end{frame}

\begin{frame}
  \frametitle{Logarithmic differentiation}
  A key point of the logarithmic differentiation is the following
  application of the chain rule:
  \[
    \ddx \ln ( f(x) ) = \frac{f'(x)}{f(x)} \,.
  \]

  \begin{block}{Illustration of method}
    To differentiate $y = f(x)$, i.e., to find $\frac{dy}{dx} = y'$:
    \begin{enumerate}
    \item Take the logarithm of $y = f(x)$: $\ln y = \ln f(x)$
    \item Differentiate implicitly: $y'/y =  f'(x)/f(x)$
    \item Solve for $y'$.
    \end{enumerate}
  \end{block}
\end{frame}

\begin{frame}
  \vs
  \begin{question}
    Compute
    \[
      \ddx \frac{x^9e^{4x}}{\sqrt{x-4}}.
    \]
  \end{question}
\end{frame}

\begin{frame}
  \vs
  \begin{question}
    For  $x>0$, compute
    \[
      \ddx x^x \,.
    \]
    This function is an example of a \textit{tower
      function}.~\footnote{
      \textbf{Note.} Make sure you are able to distinguish the following functions:
      \[
        a^x\,, \qquad x^a\,, \qquad x^x \,.
      \]
    }
  \end{question}
\end{frame}

\begin{frame}
  \vs
  \begin{question}
    Compute the derivative
    \[
      \ddx\ln{(|x|)} \,.
    \]
    Use the result to compute the derivative
    \[
      \ddx\ln{(|f(x)|)} \,.
    \]
  \end{question}
\end{frame}

\begin{frame}
  \vs
  \begin{question}
    Using logarithmic differentiation, compute the derivative.
    \[
      \ddx\Bigl(\dfrac{\sin{x}}{x}\Bigr) \,.
    \]
  \end{question}
\end{frame}

\begin{frame}
  \frametitle{The Power Rule Revisited}
  \begin{thm}[The power rule]
    For any real number $n$ and a positive real number $x$,
    \[
      \ddx x^n = n x^{n-1}.
    \]
  \end{thm}

  \note{
    \begin{proof}
      We will use logarithmic differentiation. Set $f(x) = x^n$. Write
      \begin{align*}
        \ln(|f(x)|) &= \ln\left(|x|^n\right) , x\ne0\\
                    &= n\ln(|x|).
      \end{align*}
      Now differentiate both sides, and solve for $f'(x)$
      \begin{align*}
        \frac{f'(x)}{f(x)} &= \frac{n}{x}\\
        f'(x) &=\frac{n f(x)}{x}\\
                           &= {n x^{n-1}}.
      \end{align*}
      Thus we see that the power rule holds for all real-valued exponents.
    \end{proof}
  }
\end{frame}


\end{document}
%%% Local Variables:
%%% mode: latex
%%% TeX-master: t
%%% End:
