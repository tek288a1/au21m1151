\documentclass[10pt,t,presentation,ignorenonframetext,aspectratio=169]{beamer}
% \documentclass[10pt,t,handout,ignorenonframetext,aspectratio=169]{beamer}
\usepackage[default]{lato}
\usepackage{tk_beamer1}
\input{tk_packages}
\input{tk_macros}
\input{tk_environ}
\input{tk_ximera}
\usepackage{wasysym}            % for smiley
\newcommand{\zoz}{$\mathbf{ \frac{0}{0} }$}

%%%% META DATA
\newcommand{\semester}{Autumn 2021}
\newcommand{\course}{Math 1151}
\newcommand{\lecTitle}{Lecture 4: (In)determinate Forms (IF)}

%%%% TITLE PAGE
\title[\course]{\lecTitle}
\institute[Ohio State]
{
  \medskip
}
\date[\week]{\semester}
\author{Tae Eun Kim, Ph.D.}

\begin{document}
\begin{frame}
  \titlepage
\end{frame}

% \begin{frame}
%   \frametitle{Weekly Overview}
%   \tableofcontents
% \end{frame}

% \section{Indeterminate Forms (IF)}
\begin{frame}
  \frametitle{Limits of the form zero over zero -- indeterminate form}
  \begin{defn}
    A limit
    \[
      \lim_{x \to a} \frac{f(x)}{g(x)}
    \]
    is said to be of the form $\ds \mathbf{ \frac{0}{0} }$ if
    \[
      \lim_{x \to a} f(x) = 0
      \quad \text{and} \quad
      \lim_{x \to a} g(x) = 0 \,.
    \]
  \end{defn}

  \begin{itemize}
  \item \textbf{Warning!} The symbol \zoz{} is NOT the number 0 divided by 0.
  \item A key trick to handle limits in \zoz{} form is to
    cancel out vanishing factors.
  \end{itemize}
\end{frame}

\begin{frame}
  \vs
  \question{} Compute the following limits:
  \[
    \lim_{x \to 2} \frac{ x^2-3x+2 }{ x-2 }
  \]
\end{frame}

\begin{frame}
  \vs
  \question{} Compute the following limits:
  \[
    \lim_{x \to 1} \frac{ \frac{1}{x+1} - \frac{3}{x+5} }{ x-1 }
  \]
\end{frame}

\begin{frame}
  \vs
  \question{} Compute the following limits:
  \[
    \lim_{x \to -1} \frac{ \sqrt{x+5}-2 }{ x+1 }
  \]
\end{frame}

\begin{frame}
  \vs
  \begin{rmk}
    \begin{itemize}
    \item Limits of the form \zoz{} can take any value!
    \item Having this particular form does not give us enough information to determine
      whether a function has a limit or not;
    \item Even if the limit exists, the value of the limit is not apparent
      without further manipulation.
    \item That is why such a limit is said to be in an \textbf{indeterminate form}.
    \end{itemize}
  \end{rmk}
\end{frame}

\begin{frame}
  \frametitle{Limits of the form nonzero over zero -- determinate form}
  \begin{defn}
    A limit
    \[
      \lim_{x \to a} \frac{f(x)}{g(x)}
    \]
    is said to be of the form $\mathbf{ \frac{\#}{0} }$ if
    \[
      \lim_{x \to a} f(x) = k
      \quad \text{and} \quad
      \lim_{x \to a} g(x) = 0 \,,
    \]
    where $k$ is some nonzero constant.
  \end{defn}
  \begin{itemize}
  \item When a fixed nonzero number is divided by a small number, the
    quotient is generally large.
  \item As the denominator get smaller and smaller, the quotient gets
    larger and larger.
  \end{itemize}
\end{frame}

\begin{frame}
  \vs
  \textbf{Illustration.} The following graph of
  $f(x) = 1/(x-1)^2$ near $x = 1$ displays the behavior of limits of
  the form $\mathbf{ \frac{\#}{0} }$.

  \begin{image}[2.3in]
    \begin{tikzpicture}
      \begin{axis}[
        domain=-1:2,
        ymax=100,
        samples=100,
        axis lines =middle, xlabel=$x$, ylabel=$y$,
        every axis y label/.style={at=(current axis.above origin),anchor=south},
        every axis x label/.style={at=(current axis.right of origin),anchor=west}
        ]
        \addplot [thick, penColor, smooth, domain=(-1:0.9)] {1/(x-1)^2};
        \addplot [thick, penColor, smooth, domain=(1.1:2)] {1/(x-1)^2};
        \addplot [textColor, dashed] plot coordinates {(1,0) (1,100)};
      \end{axis}
    \end{tikzpicture}
  \end{image}
\end{frame}

\begin{frame}
  \vs
  \begin{defn}
    \begin{itemize}
    \item If $f(x)$ grows arbitrarily large for all $x$ sufficiently
      close, but not equal, to $a$, we write
      \[
        \lim_{x \to a} f(x) = \infty
      \]
      and say that the limit of $f(x)$ as $x$ approaches $a$
      \textbf{is infinity}.
    \item If $f(x) < 0$ and $\left| f(x) \right|$ grows arbitrarily
      large for all $x$ sufficiently close, but not equal, to $a$, we
      write
      \[
        \lim_{x \to a} f(x) = -\infty
      \]
      and say that the limit of $f(x)$ as $x$ approaches $a$
      \textbf{is negative infinity}.
    \end{itemize}
  \end{defn}

  \vspace{1.5ex}
  \textbf{Note.} We can analogously define one-sided infinite limits, e.g.,
  \[
    \lim_{x \to a^+} f(x) = \pm \infty
    \quad \text{or} \quad
    \lim_{x \to a^-} f(x) = \pm \infty \,.
  \]
\end{frame}

\begin{frame}
  \vs
  \question{} Compute
  \[
    \lim_{x \to 0} \frac{ e^x }{ 1-\cos(x) } \,.
  \]
\end{frame}

\begin{frame}
  \vs
  \question{} Compute
  \[
    \lim_{x \to 3} \frac{ x^2-9x+14 }{ x^2-5x+6 } \,.
  \]
\end{frame}
\end{document}


%%% Local Variables:
%%% mode: latex
%%% TeX-master: t
%%% End:
