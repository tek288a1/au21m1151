\documentclass[10pt,t,presentation,ignorenonframetext,aspectratio=169]{beamer}
% \documentclass[10pt,t,handout,ignorenonframetext,aspectratio=169]{beamer}
\usepackage[default]{lato}
\usepackage{tk_beamer1}
\input{tk_packages}
\input{tk_macros}
\input{tk_environ}
\input{tk_ximera}
\usepackage{wasysym}            % for smiley
\newcommand{\zoz}{$\mathbf{ \frac{0}{0} }$}

%%%% META DATA
\newcommand{\semester}{Autumn 2021}
\newcommand{\course}{Math 1151}
\newcommand{\lecTitle}{Lecture 7: An Application of Limits (AAOL)}

%%%% TITLE PAGE
\title[\course]{\lecTitle}
\institute[Ohio State]
{
  \medskip
}
\date[\week]{\semester}
\author{Tae Eun Kim, Ph.D.}

\begin{document}
\begin{frame}
  \titlepage
\end{frame}



\begin{frame}
  \frametitle{Average velocity}
  \begin{itemize}
  \item Let $s(t)$ denote the position of an object moving along a vertical
    (or a horizontal) line at time $t$.
  \item The \textbf{average velocity} of the object on the time interval $[a, b]$ is given by
    \[
      v_{\rm avg}
      = \frac{\text{change in position}}{\text{change in time}}
      = \frac{s(b)-s(a)}{b-a} \,.
    \]
  \end{itemize}
\end{frame}

\begin{frame}
  \vs
  \begin{example}
    Suppose you are throwing a ball straight upward into the air with velocity $64\ \text{ft}/\text{sec}$. Its height (in feet) after $t$ seconds is given by
    \[
      s(t) = 64t - 16t^2 \,.
    \]
    Answer the following questions.
  \end{example}
  % The relationship between $t$ and $s(t)$ can be represented via a graph. Note the parabolic relation.
  % \begin{image}[0.4\textwidth]
  %   \begin{tikzpicture}
  %     \begin{axis}[
  %       domain=0:4,
  %       ymax=70,
  %       ymin=0,
  %       samples=100,
  %       axis lines =middle, xlabel=$t$, ylabel=$s(t)$,
  %       every axis y label/.style={at=(current axis.above origin),anchor=south},
  %       every axis x label/.style={at=(current axis.right of origin),anchor=west}
  %       ]
  %       \addplot [very thick, penColor, smooth] {64*x - 16*x^2};
  %       \addplot [textColor, dashed] plot coordinates {(2,0) (2,65)};
  %     \end{axis}
  %   \end{tikzpicture}
  % \end{image}
\end{frame}

\begin{frame}
  \vs
  \textbf{Questions.}
  \begin{enumerate}
  \item Sketch the graph of $s(t)$.
    \begin{image}[0.4\textwidth]
      \begin{tikzpicture}
        \begin{axis}[
          xmin=0, xmax=4.2, ymin=0, ymax=75,
          unit vector ratio*=20 1 1,
          axis lines=middle, xlabel=$t$, ylabel=$s(t)$,
          every axis y label/.style={at=(current axis.above origin),anchor=south},
          every axis x label/.style={at=(current axis.right of origin),anchor=west},
          xtick={1, 2, 3, 4},
          ytick={20, 40, 60},
          grid=major,
          width=3in,
          grid style={dashed, gridColor},
          ]
        \end{axis}
      \end{tikzpicture}
    \end{image}
  \item When will it hit the ground?
    \seti
  \end{enumerate}
\end{frame}

\begin{frame}
  \vs
  \begin{enumerate}
    \conti
  \item Compute the average velocity of the ball on the time interval
    $[1.5, 3]$.
    \seti
  \end{enumerate}
\end{frame}

\begin{frame}
  \vs
  \begin{enumerate}
    \conti
  \item \label{v_avg1} Compute the average velocity of the ball on the time interval
    $[t, 3]$ for $0 < t < 3$. \vfill
  \item \label{v_avg2} Finally, do the same with $[3, t]$ for $3 < t < 4$.
    \seti
  \end{enumerate}
\end{frame}

\begin{frame}
  \frametitle{Instantaneous velocity}
  \begin{itemize}
  \item An average velocity over a shorter time interval yields a better
    approximation of the \textbf{instantaneous velocity} at a moment
    contained in the time interval.
  \item This statement can be made precise
    using limits:
    \[
      \underbrace{v(a)}_{\text{inst. vel.}}
      = \lim_{t \to a}
      \underbrace{\frac{s(t)-s(a)}{t-a}}_{\text{avg. vel.}}
    \]
    where $v(a)$ is the (instantaneous) velocity at time $a$.
  \end{itemize}
\end{frame}

\begin{frame}
  \vs
  \begin{enumerate}
    \conti
  \item Using the results of parts~\ref{v_avg1} and~\ref{v_avg2}, compute the
    velocity of the ball at $t = 3$.
  \end{enumerate}
\end{frame}
\end{document}
%%% Local Variables:
%%% mode: latex
%%% TeX-master: t
%%% End:
